%
% If the calculus has an acronym, define it.
% (e.g. \newcommand{\LK}{\ensuremath{\mathbf{LK}}\xspace})

\calculusName{Hilbert's Axiomatic Calculus}         % The name of the calculus
\calculusAcronym{}          % The acronym if defined above, or empty otherwise. 
\calculusLogic{Classical Logic}        % Specify the logic (e.g. Classical Logic, Intuitionistic Logic, ...) for which this calculus is intended.
\calculusLogicOrder{First-Order}   % Specify the order of the logic (e.g. Propositional, Quantified Propositional, First-Order, Higher-Order, ...).
\calculusType{Axiomatic}         % Specify the calculus type (e.g. Tableau, Sequent Calculus, Hyper-Sequent Calculus, Natural Deduction, ...)
\calculusYear{1917}         % The year when the calculus was published.

\calculusAuthor{David Hilbert}       % The name(s) of the author(s) of the calculus.
%\calculusAuthor{ToDo:FullNameAuthor2}
%\calculusAuthor{ToDo:FullNameAuthor3}


\entryTitle{Hilbert's Axiomatic Calculus}     % Title of the entry (usually coincides with the name of the calculus).
\entryAuthor{Richard Zach}     
%\entryAuthor{ToDo:FullNameAuthor2}
%\entryAuthor{ToDo:FullNameAuthor3}

% The encyclopedia's peer-reviewing policy is described here: 
% http://proofsystem.github.io/Encyclopedia/
%
% Reviewers of this entry will be acknowledged in the following lines:
% \entryReviewer{Reviewer 1's name}
% \entryReviewer{Reviewer 2's name}
% \entryReviewer{Reviewer 3's name}
%
% The lines above will be filled by the coordinators. 
% If you would like to indicate people 
% who could review your entry, contact the coordinators.


% If you wish, use tags to give any other information 
% that might be helpful for classifying and grouping this entry:
% e.g. \etag{Two-Sided Sequents}
% e.g. \etag{Multiset Cedents}
% e.g. \etag{List Cedents}
% You are free to invent your own tags. 
% The Encyclopedia's coordinator will take care of 
% merging semantically similar tags in the future.


\maketitle

\begin{entry}{Hilbert}  

\begin{calculus}
Axioms:
\begin{align*}
\text{I.\quad} 1)\quad & X \lor X \rightarrow X &
\text{II.}\quad 1)\quad & (x)Z \rightarrow Z\\
2) \quad & X \rightarrow X \lor Y &
2) \quad & (x)F(x) \rightarrow (Ex)F(x)\\
3) \quad & X \lor Y \rightarrow Y \lor X &
3) \quad & (x)(Z \lor F(x)) \rightarrow ((x)Z \lor (x)F(x))\\
4) \quad & X \lor (Y \lor Z) \rightarrow (X \lor Y) \lor Z &
4) \quad & (x)(F(x) \rightarrow G(x)) \rightarrow \\
&&& \qquad ((x)F(x) \rightarrow (x)G(x))\\
5) \quad & (X \rightarrow Y) \rightarrow (Z \lor X \rightarrow Z \lor Y) &
5) \quad & (x)(y)F(x, y) \rightarrow (y)(x)F(x, y)\\
&& 6) \quad & (x)(y)F(x, y) \rightarrow (x)F(x, x)
\end{align*}
Rules:
\begin{enumerate}
\item Renaming of bound variables
\item Substitution for propositional and predicate variables
\item Universal instantiation, i.e.,
\[
\infer{\alpha(a)}{(x)\alpha(x)}
\]
\item Universal closure of additional argument places, i.e.,
\[
\infer{(x)\alpha(F(x))}{\alpha(X)}
\qquad
\infer{(y)\alpha(F(x_1, \dots, x_n, y))}{\alpha(F(x_1, \dots, x_n))}
\]
\item Modus ponens, i.e.,
\[
\infer{\beta}{\alpha & \alpha \rightarrow \beta}
\]
\end{enumerate}

\end{calculus}

\begin{clarifications}
Hilbert used the symbol $\times$ for disjunction $\lor$, $+$ for
conjunction $\land$, and overlining $\overline{\phantom{X}}$ for
negation. $\alpha \rightarrow \beta$ is an abbreviation for
$\overline{\alpha} \lor \beta$, and $\alpha \land \beta$ for
$\overline{\overline{\alpha} \lor \overline{\beta}}$. The usual
conventions about operator precendence apply. $(x)$ is a universal
quantifier and $(Ex)$ an existential quantifier. The system is
formulated in a language that distinguishes between propositional and
predicate constants and variables. In particular, the axioms
are \emph{not} understood as schemas. The substitution rule allows the
replacement of propositional variables by a formula that contains no
object variable free, and a predicate variable with $n$ arguments
$x_1$, \dots, $x_n$ by an expression in which all and only $x_1$, \dots, $x_n$
occur free.
\end{clarifications}

\begin{history}
The system was first presented in lecture notes by Paul Bernays to Hilbert's
course ``Principles of Mathematics,'' taught in the
Winter term 1917--18; see \cite{Hilbert1917}.  The system
is based on the axiomatic proof system of Whitehead and
Russell's \emph{Principia
Mathematica} \irefmissing{PrincipiaMathematica}, but restricts the
language to first order. It adds the explicit rules of substitution
and renaming, and avoids the use of free variables in its axioms and
theorems.
\end{history}

% \begin{technicalities}
% ToDo: write here remarks about soundness, completeness, decidability...
% \end{technicalities}

\end{entry}
