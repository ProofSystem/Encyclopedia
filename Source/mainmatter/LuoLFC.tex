\newcommand{\LFC}{\ensuremath{\mathbf{LF_{c}}}\xspace}
\calculusName{LF with Coercive Subtyping} 
\calculusAcronym{LFc} 
\calculusLogic{Type Theory} 
\calculusLogicOrder{Higher-Order}
%\calculusType{Logical Framework}
\calculusType{Natural Deduction} 
\calculusYear{1996} 
\calculusAuthor{Zhaohui Luo} 


\entryTitle{T[C] (LF with Coercive Subtyping)} 

\entryAuthor{Zhaohui Luo} 
\entryAuthor{Sergei Soloviev}  


\maketitle


\begin{entry}{LuoLFC}

\begin{calculus}
        
\centering
Basic subkinding rule
$$
\infer{\Gamma\seq El(A)<_cEl(B)}{\Gamma\seq A<_cB:{\bf Type}}
$$
Subkinding for dependent product kinds
$$\infer{\Gamma\seq (x:K_1)K_2<_{[f:(x:K_1)K_2][x:K'_1]c(fx)}(x:K'_1)K'_2}
{\Gamma\seq K'_1=K_1\!\quad\!\Gamma, x:K'_1\seq K_2<_cK'_2
\!\quad\!\Gamma, x:K_1\seq K_2:{\bf kind}}$$
$$\infer{\Gamma\seq (x:K_1)K_2<_{[f:(x:K_1)K_2][x:K'_1]f(cx)}(x:K'_1)K'_2}
{\Gamma\seq K'_1<_cK_1\!\quad\!\Gamma, x:K'_1\seq [cx/x]K_2=K'_2
\!\quad\!\Gamma, x:K_1\seq K_2:{\bf kind}}$$
$$\infer{\Gamma\seq (x:K_1)K_2<_{[f:(x:K_1)K_2][x:K'_1]c_2(f(c_1x))}(x:K'_1)K'_2}
{\Gamma\seq K'_1<_{c_1}K_1\!\quad\!\Gamma, x:K'_1\seq [c_1x/x]K_2<_{c_2}K'_2
\!\quad\!\Gamma, x:K_1\seq K_2:{\bf kind}}$$
Coercive application rules
$$\infer{\Gamma\seq f(k_0):[c(k_0)/x]K'}{\Gamma\seq f:(x:K)K'\!\quad\!\Gamma\seq k_0:K_0\!\quad\!
\Gamma\seq K_0<_cK}$$
$$\infer{\Gamma\seq f(k_0)=f'(k'_0):[c(k_0)/x]K'}
{\Gamma\seq f(k_0)=f(ck_0):(x:K)K'\!\quad\!\Gamma\seq k_0:K_0\!\quad\!
\Gamma\seq K_0<_cK}$$
Coercive definition rule
$$\infer{\Gamma\seq f(k_0):[c(k_0)/x]K'}
{\Gamma\seq f:(x:K)K'\!\quad\!\Gamma\seq k_0:K_0\!\quad\!\Gamma\seq K_0<_cK}$$
Structural rules
$$
\infer{\Gamma\seq A'<_{c'}B'}{\Gamma\seq A<_cB\!\quad\!\Gamma\seq A=A':Type
\!\quad\!\Gamma\seq B=B'\!\quad\!\Gamma\seq c=c':(El(A))El(B)}
$$
$$\infer{\Gamma\seq A_{c'\circ c}A''}{\Gamma\seq A<_cA'\!\quad\!\Gamma\seq A'<_{c'}A''}$$
$$\infer{\Gamma, [k/x]\Gamma'\seq [k/x]A<_{[k/x]c}[k/x]B}{\Gamma, x:K, \Gamma'\seq A<_c B
\!\quad\!\Gamma\seq k:K}\,\,\,\,\,\infer{\Gamma, \Gamma'',\Gamma'\seq A<_cB}{\Gamma, \Gamma'\seq A<_cB\!\quad\!\Gamma, 
\Gamma''\seq {\bf valid}}$$
$$\infer{\Gamma, x:K', \Gamma'\seq A<_c B}{\Gamma, x:K, \Gamma'\seq A<_c B
\!\quad\!\Gamma \seq K=K'}$$
        
\end{calculus}

\begin{clarifications}
We follow~\cite{LuoSolXue:13}. In this entry the extensions $T[C]$ of the
logical framework $\LuoLF$~\iref{LuoLF}
are considered. Here $T$ is a type theory specified in $\LuoLF$ (formally, an extension
of $\LuoLF$) and
 $C$ is a (possibly infinite)
set of subtyping judgements of the form $\Gamma\seq A<_cB:Type$. 
The set $C$ itself
may be generated by some user-defined rules. As coercive definition rule
above shows, coercive subtyping is considered as an 
{\em abbreviation mechanism}, the expressions without coercions are 
considered as ``abbreviations'' of the expressions where coercions are 
inserted. For corcive subtyping as an abbreviation mechanism, one of central
questions is the concervativity of the extension $T[C]$ over $T$.

The system $T[C]$ is built by ``layers''
and in this sense may be considered as {\em hybrid}. Above the rules
(except structural rules) of the {\em subkinding} level are given. 
The structure (and rules)
of the subtyping level, as well as its connection with the 
subkinding level, are explained below. 

First the intermediate system $T[C]_0$ is defined. 
The syntax of $T[C]_0$ is the same
as the syntax of $T$ ({\em i.e.}, type theory specified in $\LuoLF$). The rule
 $$\infer{\Gamma\seq A<_cB:Type}{\Gamma\seq A<_cB:Type\in C}$$ is added, 
and the structural subtyping rules
given below.  They state that the subtyping relation $<$ (annotated
by coercion terms $c$) is congruent, transitive, and closed under
substitution, and satisfies the rules of weakening and contextual equality. Similar structural rules are included in the subkinding level above. 

Main requirement to the set $C$ (expressed in terms of $T[C]_0$) is {\em coherence}:
\begin{itemize}
\item If $\Gamma\seq A<_cB:Type$ then $\Gamma\seq A:Type$, $\Gamma\seq B:Type$ and $\Gamma\seq c: (El(A))El(B)$.
\item $\Gamma\not\seq A<_cA:Type$ for any $\Gamma, A, c$.
\item If $\Gamma\seq A<_c B:Type$ and $\Gamma\seq A<_{c'}B:Type$ then $\Gamma\seq c=c':(El(A))El(B)$.
\end{itemize}
\end{clarifications}

\begin{history}
  Coercive subtiping as an abbreviation mechanism
  was introduced in a conference paper~\cite{Luo96:CSL}.  
  It was described for type theories specified in Z. Luo's typed $\LuoLF$
  (extensions of $\LuoLF$)~\iref{LuoLF}, but
  the idea itself is much more general and may apply
  to other type theories. The approach was further
  developed in~\cite{jls:TYPES96, Luo:99, SolLuo:02, LuoSolXue:13}.
\end{history}

\begin{technicalities}
  The main theorem (justifying the view of coercive subtyping as an abbreviation
  mechanism) is the conservativity of $T[C]$ w.r.t. the type theory $T$.
  %(cf.~\cite{Luo:99, SolLuo:02, LuoSolXue:13}).  
\end{technicalities}    


\end{entry}
