\calculusName{Proper Multi-type Display Calculus for Lattice Logic MtD.LatL}
\calculusAcronym{MtDLatL}
\calculusLogic{Lattice Logic}
\calculusLogicOrder{Propositional}
\calculusType{Multi-type Sequent Calculus}
\calculusYear{2017}
\calculusAuthor{Giuseppe Greco}
\calculusAuthor{Alessandra Palmigiano}

\entryTitle{Proper Multi-type Display Calculus for Lattice Logic MtD.LatL}
\entryAuthor{Giuseppe Greco}

\etag{Two-Sided Sequents}
\etag{Multi-type Cedents}
\etag{Multi-Premise Cedent}
\etag{Multi-Conclusion Cedent}

%%%Definition and abbreviations

\def\fCenter{{\mbox{$\ \vdash\ $}}}
\def\ffCenter{{\mbox{$\overline{\vdash}$}}}
\def\fcenter{{\mbox{\ }}}
\EnableBpAbbreviations

\def\fns{\footnotesize}
\def\mc{\multicolumn}

\def\aol{\rule[0.5865ex]{1.38ex}{0.1ex}}
\def\pdra{\mbox{$\,>\mkern-8mu\raisebox{-0.065ex}{\aol}\,$}}
\def\pdla{\mbox{\rotatebox[origin=c]{180}{$\,>\mkern-8mu\raisebox{-0.065ex}{\aol}\,$}}}

\newcommand{\TOPBOTL}{\ensuremath{\textrm{I}}\xspace}

%Modalities
\newcommand{\WCIRCW}{\ensuremath{\circ}\xspace}
\newcommand{\BCIRCB}{\ensuremath{\bullet}\xspace}
%white
\newcommand{\wboxw}{\ensuremath{\Box}\xspace}
\newcommand{\wdiaw}{\ensuremath{\Diamond}\xspace}
%black
\newcommand{\bboxb}{\ensuremath{\blacksquare}\xspace}
%\newcommand{\bdiab}{\ensuremath{\blacklozenge}\xspace}
\newcommand{\bdiab}{\ensuremath{\Diamondblack}\xspace}

%Power-set Lattice Connectives
%operational
\newcommand{\pnegp}{\ensuremath{-}\xspace}
\newcommand{\ptopp}{\ensuremath{\wp}\xspace}
\newcommand{\pbotp}{\ensuremath{\varnothing}\xspace}
\newcommand{\pandp}{\ensuremath{\cap}\xspace}
\newcommand{\prarrp}{\ensuremath{\,{\raisebox{-0.065ex}{\rule[0.5865ex]{1.38ex}{0.1ex}}\mkern-5mu\supset}\,}\xspace}

\newcommand{\porp}{\ensuremath{\cup}\xspace}
\newcommand{\pdrarrp}{\ensuremath{\,{\supset\mkern-5.5mu\raisebox{-0.065ex}{\rule[0.5865ex]{1.38ex}{0.1ex}}}\,}\xspace}

%structural Connectives
\newcommand{\PNEGP}{\ensuremath{\ominus\,}\xspace}
\newcommand{\PTOPBOTP}{\ensuremath{\circledS}\xspace}
\newcommand{\PANDORP}{\ensuremath{\centerdot}\xspace}
\newcommand{\PARRP}{\ensuremath{\supset}\xspace}

%Power-set Lattice Modalities
%operational
\newcommand{\pwdiawp}{\ensuremath{\Diamond}\xspace}
\newcommand{\pbdiabp}{\ensuremath{\Diamondblack}\xspace}
\newcommand{\pbboxbp}{\ensuremath{\blacksquare}\xspace}
\newcommand{\pwboxwp}{\ensuremath{\Box}\xspace}
%structural
\newcommand{\PWCIRCWP}{\ensuremath{\circ}\xspace}
\newcommand{\PBCIRCBP}{\ensuremath{\bullet}\xspace}


%Left Power-set Lattice Connectives
%operational
\newcommand{\pneg}{\ensuremath{-}\xspace}
\newcommand{\ptop}{\ensuremath{\wp}\xspace}
\newcommand{\pbot}{\ensuremath{\varnothing}\xspace}
\newcommand{\pand}{\ensuremath{\cap}\xspace}
\newcommand{\prarr}{\ensuremath{\,{\raisebox{-0.065ex}{\rule[0.5865ex]{1.38ex}{0.1ex}}\mkern-5mu\supset}\,}\xspace}
\newcommand{\por}{\ensuremath{\cup}\xspace}
\newcommand{\pdrarr}{\ensuremath{\,{\supset\mkern-5.5mu\raisebox{-0.065ex}{\rule[0.5865ex]{1.38ex}{0.1ex}}}\,}\xspace}

%structural Connectives
\newcommand{\PNEG}{\ensuremath{\ominus\,}\xspace}
\newcommand{\PTOPBOT}{\ensuremath{\circledS}\xspace}
\newcommand{\PANDOR}{\ensuremath{\centerdot}\xspace}
\newcommand{\PARR}{\ensuremath{\supset}\xspace}

%Left Power-set Lattice Modalities
%operational
\newcommand{\pwbox}{\ensuremath{\Box}\xspace}
\newcommand{\pwdia}{\ensuremath{\Diamond}\xspace}
\newcommand{\pbbox}{\ensuremath{\blacksquare}\xspace}
\newcommand{\pbdia}{\ensuremath{\Diamondblack}\xspace}
%structural
\newcommand{\PWCIRC}{\ensuremath{\circ}\xspace}
\newcommand{\PBCIRC}{\ensuremath{\bullet}\xspace}

%Right Power-set Lattice Connectives
%%operational
\newcommand{\negp}{\ensuremath{-^\mathrm{op}}\xspace}
\newcommand{\topp}{\ensuremath{\wp^\mathrm{op}}\xspace}
\newcommand{\botp}{\ensuremath{\varnothing^\mathrm{op}}\xspace}
\newcommand{\andp}{\ensuremath{\cap^\mathrm{op}}\xspace}
\newcommand{\rarrp}{\ensuremath{\,{\raisebox{-0.065ex}{\rule[0.5865ex]{1.38ex}{0.1ex}}\mkern-5mu\supset^\mathrm{op}}\,}\xspace}
\newcommand{\orp}{\ensuremath{\cup^\mathrm{op}}\xspace}
\newcommand{\drarrp}{\ensuremath{\,{\supset\mkern-5.5mu\raisebox{-0.065ex}{\rule[0.5865ex]{1.38ex}{0.1ex}}^\mathrm{op}}\,}\xspace}

%%structural Connectives
\newcommand{\NEGP}{\ensuremath{\ominus^\mathrm{op}\,}\xspace}
\newcommand{\TOPBOTP}{\ensuremath{\circledS^\mathrm{op}}\xspace}
\newcommand{\ANDORP}{\ensuremath{\centerdot^\mathrm{op}}\xspace}
\newcommand{\ARRP}{\ensuremath{\supset^\mathrm{op}}\xspace}

%Right Power-set Lattice Modalities
%operational
\newcommand{\diawp}{\ensuremath{\Diamond^\mathrm{op}}\xspace}
\newcommand{\diabp}{\ensuremath{\Diamondblack^\mathrm{op}}\xspace}
\newcommand{\boxbp}{\ensuremath{\blacksquare^\mathrm{op}}\xspace}
\newcommand{\boxwp}{\ensuremath{\Box^\mathrm{op}}\xspace}
%structural
\newcommand{\CIRCWP}{\ensuremath{\circ^\mathrm{op}}\xspace}
\newcommand{\CIRCBP}{\ensuremath{\bullet^\mathrm{op}}\xspace}

%%%

\maketitle

\begin{entry}{MtDLatL}

\begin{calculus}
{%\small
Identity and Cut rules:
\vspace{0.2cm}	
\begin{center}
\AXC{\fns Id}
\noLine
\UI$p \fCenter p$
\DisplayProof
\qquad
\AX$X \fCenter A$
\AX$A \fCenter Y$
\RightLabel{\fns Cut}
\BI$X \fCenter Y$
\DisplayProof
\qquad
\AX$\Gamma \fCenter \alpha$
\AX$\alpha \fCenter \Delta$
\RightLabel{\fns Cut}
\BI$\Gamma \fCenter \Delta$
\DisplayProof
\qquad
\AX$\Pi \fCenter \xi$
\AX$\xi \fCenter \Sigma$
\RightLabel{\fns Cut}
\BI$\Pi \fCenter \Sigma$
\DisplayProof
\end{center}
\vspace{0.2cm}	




Multi-type display rules
\begin{center}
\AX$\Gamma \fCenter \PWCIRCWP  X $
\RightLabel{\fns adj}
\doubleLine
\UI$ \PBCIRCBP \Gamma \fCenter X $
\DisplayProof
\qquad
\AX$\CIRCWP  X  \fCenter \Pi$
\RightLabel{\fns adj}
\doubleLine
\UI$ X \fCenter \CIRCBP \Pi$
\DisplayProof
\end{center}
 }
\end{calculus}

\begin{clarifications}
  The language $\mathcal{L}_\mathrm{MT}(\mathcal{F}, \mathcal{G})$ of  MtD.LatL
  consists of  {\em logical}   and {\em structural terms} in the types
  %
  $\mathsf{T}_1:  = \mathsf{Lattice}$, $\mathsf{T}_2: = \mathsf{Left}$ 
  %
  and
  %
  $\mathsf{T}_3: = \mathsf{Right}$. Following the notation of~\iref{MtSC}, the set
  of logical terms takes as parameters: 1) a denumerable set of atomic terms
  $\mathsf{At}(\mathsf{Lattice})$, elements of which are denoted $p$, possibly
  with indexes; 2) disjoint sets of connectives 
  %
  $\mathcal{F}: = \mathcal{F}_{\mathsf{Lattice}}\uplus\mathcal{F}_{\mathsf{Left}}\uplus\mathcal{F}_{\mathsf{Right}}\uplus\mathcal{F}_{\mathrm{MT}}$
  %
  and 
  %
  $\mathcal{G}: = \mathcal{G}_{\mathsf{Lattice}}\uplus\mathcal{G}_{\mathsf{Left}}\uplus\mathcal{G}_{\mathsf{Right}}\uplus\mathcal{G}_{\mathrm{MT}}$
  %
  defined as follows: 
  %
  $\mathcal{F}_{\mathsf{Lattice}}: = \{\top\}$,
  $\mathcal{F}_{\mathsf{Left}}: = \{\pand \}$, 
  $\mathcal{F}_{\mathsf{Right}}: = \{\andp \}$, 
  $\mathcal{F}_{\mathrm{MT}}: = \{\pbdiabp, \diawp\}$, 
  %
  where 
  $n_\top = 0$, 
  $n_{\pbdiabp} =n_{\diawp} = 1$, 
  $n_{\pand} = n_{\andp} = 2$, and
  $\varepsilon_{\pbdiabp}(1) = \varepsilon_{\diawp}(1) = \varepsilon_{\pand}(i) = \varepsilon_{\andp}(i) = 1$ 
  for every $i\in \{1, 2\}$, and
  %
  $\mathcal{G}_{\mathsf{Lattice}}: = \{\bot\}$, 
  $\mathcal{G}_{\mathsf{Left}}: = \{\por \}$, 
  $\mathcal{G}_{\mathsf{Right}}: = \{\orp \}$, 
  $\mathcal{G}_{\mathrm{MT}}: = \{\boxbp, \pwboxwp\}$, 
  %
  where 
  $n_\bot = 0$, 
  $n_\boxbp =n_\pwboxwp = 1$, 
  $n_\por = n_\orp = 2$, and 
  $\varepsilon_{\boxbp}(1) = \varepsilon_{\pwboxwp}(1) = \varepsilon_{\por}(i) = \varepsilon_{\orp}(i) = 1$ 
  for every $i\in \{1, 2\}$.
  %
  The functional types of the heterogeneous connectives $\pwboxwp$, $\diawp$,
  $\pbdiabp$ and $\boxbp$ are $\mathsf{Lattice}\rightarrow \mathsf{Left}$,
  $\mathsf{Lattice}\rightarrow \mathsf{Right}$, $\mathsf{Left}\rightarrow
  \mathsf{Lattice}$, and  $\mathsf{Right}\rightarrow \mathsf{Lattice}$
  respectively.
  
  The structural terms are built by means of structural connectives, taking
  logical terms as atomic structures. The set of structural connectives includes
  $\PWCIRC, \CIRCWP, \PBCIRC, \CIRCBP$   which are the structural counterparts of
  $\pwbox, \diawp, \pbdiabp, \boxbp$, respectively.  It also includes $\TOPBOTL$
  which is the structural counterpart of $\top$ (when occurring in precedent
  position) and $\bot$ (when occurring in succedent position); $\PANDOR$ which is
  the structural counterpart of $\pand$ (when occurring in precedent position) and
  $\por $ (when occurring in succedent position); $\ANDORP$ which is the
  structural counterpart of $\andp$ (when occurring in precedent position) and
  $\orp $ (when occurring in succedent position). Finally, it includes $\PARR$ and
  $\ARRP$ which, when occurring in precedent position, correspond to the left
  residuals of $\PANDOR$ and $\ANDORP$ respectively,  and, when occurring in
  succedent position, correspond to the right residuals of $\PANDOR$ and $\ANDORP$
  respectively, and the structural constants  $\PTOPBOT$ and $\TOPBOTP$,
  corresponding to the top (when occurring in precedent position) and bottom (when
  occurring in succedent position) of $\mathsf{Left}$ and $\mathsf{Right}$,
  respectively.
  
  Summing up, the well formed terms of MtD.LatL are generated by simultaneous
  induction as follows:
  \begin{center}
  \begin{tabular}{lll}
  $\mathsf{Lattice}$ \ \ \ &
  $\mathsf{Left}$ \ \ \ &
  $\mathsf{Right}$ \\
  
  $A ::= \,p \mid \pbdiabp \alpha \mid \bboxb \xi$ \ \ \ &
  $\alpha     ::= \, \pwbox A \mid \alpha \pand \alpha \mid \alpha \por \alpha$ \ \ \ &
  $\xi     ::= \, \diawp A \mid \xi \andp \xi \mid \xi \orp \xi$ \\
  
  $X ::= \, p \mid \TOPBOTL \mid \PBCIRCBP \Gamma \mid \PBCIRCBP^{\mathrm{op}} \Pi$ \ \ \ &
  $\Gamma ::= \, \PWCIRC X \mid \PTOPBOT \mid \Gamma \PANDOR \Gamma \mid \Gamma\PARR \Gamma$ \ \ \ &
  $\Pi ::= \, \CIRCWP X \mid \TOPBOTP \mid \Pi \ANDORP \Pi \mid \Pi \ARRP \Pi$ \\
  \end{tabular}
  \end{center}
  
  The introduction rules instantiate the general template described
  in~\iref{MtSC}, and hence are omitted. Also, the pure-type structural rules are
  the standard ones capturing distributive lattices (for the types $\mathsf{Left}$
  and $\mathsf{Right}$), and are also omitted.
\end{clarifications}

\begin{history} 
  Lattice logic is the $\{\wedge, \vee, \top, \bot\}$-fragment of classical
  propositional logic without distributivity. In~\cite{Belnap1990}, a display
  calculus is introduced for lattice logic, regarded as the `additive fragment' of
  linear logic. In this calculus, no  structural counterparts are assigned  to
  $\wedge$ and $\vee$, and the introduction rules of these connectives are given
  in the so called {\em additive} form, which subsumes the usual weakening,
  associativity, exchange and contraction rules. In~\cite{GrecoPalmigiano2017},
  Birkhoff's representation theorem for complete lattices provides the guidelines
  for the design of the proper multi-type display calculus MtD.LatL.
\end{history}

\begin{technicalities}
  In MtD.LatL, all connectives are introduced by means of standard rules as
  discussed in~\iref{MtSC}, which occur in the so called {\em multiplicative} form
  and do not subsume weakening, associativity, exchange and contraction. The
  calculus above is sound and complete w.r.t.~complete lattices, equivalently
  presented as heterogeneous algebras as discussed  in~\cite{GrecoPalmigiano2017};
  it is conservative, and enjoys the cut elimination and subformula property as
  immediate consequences of the general theory of multi-type calculi.
\end{technicalities}

\end{entry}
