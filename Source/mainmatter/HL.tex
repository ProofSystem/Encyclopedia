\calculusName{Hybrid Logic}   
\calculusAcronym{\HL}     
\calculusLogic{Modal Logics} 
\calculusLogicOrder{Propositional}
\calculusType{Natural Deduction}   
\calculusYear{2001}   
\calculusAuthor{Torben Bra\"uner}

\entryTitle{Hybrid Logic (HL)}
\entryAuthor{Torben Bra\"uner}

\maketitle

\begin{entry}{HL}  
  
\begin{calculus}
\[ 
\begin{array}{ccc}
\prooftree
@_{a}\phi
\hspace{5 mm}
@_{a}\psi
\justifies
@_{a}(\phi \wedge \psi)
\using
(\wedge I)
\endprooftree
\hspace{6 mm} & \hspace{6 mm}
\prooftree
@_{a}(\phi \wedge \psi)
\justifies
@_{a} \phi
\using
(\wedge E1)
\endprooftree
\hspace{5 mm}
\prooftree
@_{a}(\phi \wedge \psi)
\justifies
@_{a} \psi
\using
(\wedge E2)
\endprooftree
\hspace{6 mm} & \hspace{6 mm}
\\ \vspace{-2mm} & &\\
\prooftree
\prooftree
{[} @_{a} \phi {]}
\leadsto
@_{a} \psi
\endprooftree
\justifies
@_{a}( \phi \rightarrow \psi)
\using
(\rightarrow I)
\endprooftree
\hspace{6 mm} & \hspace{6 mm}
\prooftree
@_{a}( \phi \rightarrow \psi)
\hspace{5mm}
@_{a} \phi
\justifies
@_{a}\psi
\using
(\rightarrow E)
\endprooftree
\hspace{6 mm} & \hspace{6 mm}
\prooftree
\prooftree
{[} @_{a} \neg \phi {]}
\leadsto
@_{a} \bot
\endprooftree
\justifies
@_{a} \phi
\using
(\bot 1)^{\ast}
\endprooftree
\\ \vspace{-2mm} & & \\
\prooftree
@_{a}\phi
\justifies
@_{c} @_{a} \phi
\using
(@ I)
\endprooftree
\hspace{6 mm} & \hspace{6 mm}
\prooftree
@_{c} @_{a} \phi
\justifies
@_{a}\phi
\using
(@ E)
\endprooftree
\hspace{6 mm} & \hspace{6 mm}
\prooftree
@_{a} \bot
\justifies
@_{c} \bot
\using
(\bot 2)
\endprooftree
\\ \vspace{-2mm} & &\\
\prooftree
\prooftree
{[} @_{a} \lozenge c {]}
\leadsto
@_{c} \phi
\endprooftree
\justifies
@_{a}\square \phi
\using
(\square I)^{\star}
\endprooftree
\hspace{6 mm} & \hspace{6 mm}
\prooftree
@_{a}\square \phi
\hspace{5mm}
@_{a} \lozenge e
\justifies
@_{e}\phi
\using
(\square E)
\endprooftree
\hspace{6 mm} & \hspace{6 mm}\\
\prooftree
\justifies
@_{a}a
\using
({\it Ref})
\endprooftree
\hspace{6 mm} & \hspace{6 mm}
\prooftree
@_{a}c \hspace{5 mm} @_{a} \phi
\justifies
@_{c} \phi
\using
({\it Nom}1)^{\ast}
\endprooftree
\hspace{6 mm} & \hspace{6 mm}
\prooftree
@_{a}c \hspace{5 mm} @_{a} \lozenge b
\justifies
@_{c} \lozenge b
\using
({\it Nom}2)
\endprooftree
\end{array}
\vspace{-2mm}
\]
$\ast$ $\phi$ is a propositional symbol (ordinary or a nominal).\\
$\star$ $c$ does not occur in $@_{a}\square \phi $ or
in any undischarged assumptions other than the occurrences
of $@_{a} \lozenge c$.

\end{calculus}

\begin{clarifications}
  Hybrid logic is an extension of ordinary modal logic which allows explicit
  reference to individual points in a Kripke model. Formulas of HL are defined by
  %
  $S \hspace{.5mm} ::= \hspace{.5mm} p \hspace{.5mm} | \hspace{.5mm} a
  \hspace{.5mm} | \hspace{.5mm} S \wedge S \hspace{.5mm} | \hspace{.5mm}
  S \rightarrow S \hspace{.5mm} | \hspace{.5mm} \bot \hspace{.5mm} |
  \hspace{.5mm} \square S \hspace{.5mm} | \hspace{.5mm} @_{a} S$
  %
  where $p$ ranges over ordinary propositional symbols and $a$ ranges over
  nominals (a second sort of propositional symbols that refer to points in the
  model). As usual, $\neg \phi$ stands for $ \phi \rightarrow \bot$ and $\lozenge
  \phi$ stands for $\neg \square \neg \phi$.
\end{clarifications}

\begin{history}
  This natural deduction system for classical HL was originally suggested
  in~\cite{Brauner01c} and developed in~\cite{Brauner01b}. A natural deduction
  system for intuitionistic hybrid logic can be found in the entry~\iref{IHL}.
  These and other proof systems are included in the book~\cite{Brauner11a}, which
  considers a spectrum of different hybrid logics (propositional, first-order,
  intensional first-order, and intuitionistic) and different types of proof
  systems for hybrid logic (natural deduction, Gentzen, tableau, and axiom
  systems). See~\cite{AC06} for a general introduction to hybrid logic.
\end{history}

\begin{technicalities}
  The system satisfies normalization, and normal derivations satisfy a version of
  the subformula property. Completeness is preserved when the system is extended
  with additional rules corresponding to first-order conditions on Kripke frames
  expressed by geometric theories.
\end{technicalities}

\end{entry}
