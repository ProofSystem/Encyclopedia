\calculusName{Resolution for Modal Logic K}   % The name of the calculus
\calculusAcronym{\RK}    % The acronym if defined above, or empty otherwise. 
\calculusLogic{Modal Logics}  % Specify the logic (e.g. classical, intuitionistic, ...) for which this calculus is intended.
\calculusLogicOrder{Propositional}
\calculusType{Resolution}   % Specify the calculus type (e.g. Frege-Hilbert style, tableau, sequent calculus, hypersequent calculus, natural deduction, ...)
\calculusYear{1982}   % The year when the calculus was invented.
\calculusAuthor{Luis Fari\~nas del Cerro} % The name(s) of the author(s) of the calculus.


\entryTitle{Resolution for Modal Logic K (\RK)}     % Title of the entry (usually coincides with the name of the calculus).
\entryAuthor{Joseph Boudou}
\entryAuthor{Luis Fari\~nas del Cerro}     


\maketitle


\begin{entry}{RK}  

\newcommand{\sepproof}{\hskip 2em plus 6em\relax}
\newcommand{\sepline}{\]\[}

\def\fCenter{\longrightarrow}
\newenvironment{infruleset}[1]{%
  \sc{#1} \[
}{%
  \]
}

\begin{calculus}

\begin{infruleset}{Rules for computing resolvents}
    \AxiomC{}
    \LeftLabel{(A1)}
    \UnaryInfC{$\Sigma(p, \neg p) \fCenter \bot$}
    \DisplayProof
  \sepproof
    \AxiomC{}
    \LeftLabel{(A2)}
    \UnaryInfC{$\Sigma(\bot, A) \fCenter \bot$}
    \DisplayProof
  \sepline
    \Axiom$\Sigma(A, B) \fCenter C$
    \LeftLabel{($\Sigma\vee$)}
    \UnaryInf$\Sigma(A \vee D_1, B \vee D_2) \fCenter C \vee D_1 \vee D_2$
    \DisplayProof
  \sepline
    \Axiom$\Sigma(A, B) \fCenter C$
    \LeftLabel{($\Sigma\Box\Diamond$)}
    \UnaryInf$\Sigma(\Box A, \Diamond(B \wedge E)) \fCenter \Diamond(B \wedge C \wedge E)$
    \DisplayProof
  \sepproof
    \Axiom$\Sigma(A, B) \fCenter C$
    \LeftLabel{($\Sigma\Box\Box$)}
    \UnaryInf$\Sigma(\Box A, \Box B) \fCenter \Box C$
    \DisplayProof
  \sepline
    \Axiom$\Gamma(A) \fCenter B$
    \LeftLabel{($\Gamma\vee$)}
    \UnaryInf$\Gamma(A \vee C) \fCenter B \vee C$
    \DisplayProof
  \sepproof
    \Axiom$\Gamma(A) \fCenter B$
    \LeftLabel{($\Gamma\Box$)}
    \UnaryInf$\Gamma(\Box A) \fCenter \Box B$
    \DisplayProof
  \sepline
    \Axiom$\Sigma(A, B) \fCenter C$
    \LeftLabel{($\Gamma\Diamond1$)}
    \UnaryInf$\Gamma(\Diamond(A \wedge B \wedge E)) \fCenter \Diamond(A \wedge B \wedge C \wedge E)$
    \DisplayProof
  \sepline
    \Axiom$\Gamma(A) \fCenter B$
    \LeftLabel{($\Gamma\Diamond2$)}
    \UnaryInf$\Gamma(\Diamond(A \wedge E)) \fCenter \Diamond(A \wedge B \wedge E)$
    \DisplayProof
\end{infruleset}

\sc{Simplification rules}

\begin{center}
\begin{tabular}{rlrl}
  $(S_1)$ &~ $\Diamond\bot \approx \bot       $ ~&\quad
  $(S_3)$ &~ $\bot \wedge E \approx \bot      $ \\
  $(S_2)$ &~ $\bot \vee A \approx A           $ ~&\quad
  $(S_4)$ &~ $A \vee A \vee B \approx A \vee B$ 
\end{tabular}
\end{center}
\vspace{2ex}

\begin{infruleset}{Inference rules}
    \AxiomC{$C$}
    \LeftLabel{(R1)}
    \RightLabel{if $\Gamma(C) \Rightarrow D$}
    \UnaryInfC{$D$}
    \DisplayProof
  \sepline
    \AxiomC{$C_1$}
    \AxiomC{$C_2$}
    \LeftLabel{(R2)}
    \RightLabel{if $\Sigma(C_1, C_2) \Rightarrow D$}
    \BinaryInfC{$D$}
    \DisplayProof
\end{infruleset}

\end{calculus}

\begin{clarifications}
  $A, B, C$ and $D$ denote formulas in disjunctive normal form (DNF)
  whereas $E$ denotes a formula in conjunctive normal form (CNF).
  A formula is in DNF if it has the general form
  $ L_1 \vee \ldots \vee L_n \vee
    \Box A_1 \vee \ldots \vee \Box A_p \vee
    \Diamond E_1 \vee \ldots \vee \Diamond E_q $,
  where $L_i$ are literals, $A_i$ are in DNF and $E_i$ are in CNF.
  A formula is in CNF if it is a conjunction of formulas in DNF.
  The relation $\approx$ is the least congruence satisfying all simplification rules.
  The normal form $A$ of a formula $A'$ is the least formula such that $A' \approx A$.
  We write $\Sigma(A,B) \Rightarrow C$ (respectively $\Gamma(A) \Rightarrow C$)
  if there exist $C'$ such that $\Sigma(A,B) \fCenter C'$ (resp. $\Gamma(A) \fCenter C'$)
  and $C$ is the normal form of $C'$.
\end{clarifications}

\begin{history}
  Introduced in~\cite{farinas.1982}.
  The current presentation is inspired by~\cite{enjalbert-farinas.1989}.
  The method is at the core of the MOLOG language~\cite{bieber-farinas-herzig.1988}.
  With slight variations of the rules, some other modal logics like S4 or S5 can be obtained~\cite{enjalbert-farinas.1989}.
  The method has been adaptated to first-order modal logic~\cite{cialdea.1991}.
  An alternative non-clausal resolution method is presented in~\cite{abadi-manna.1985} (for LTL).
\end{history}

\begin{technicalities}
  The method is sound and complete with respect to the classical modal logic K.
\end{technicalities}

\end{entry}
