
% If the calculus has an acronym, define it.
% (e.g. \newcommand{\LK}{\ensuremath{\mathbf{LK}}\xspace})

\calculusName{Entailment for Structured Specifications}         % The name of the calculus
\calculusAcronym{\Pi_{SPEC}}          % The acronym if defined above, or empty otherwise. 
\calculusLogic{Institution}        % Specify the logic (e.g. Classical Logic, Intuitionistic Logic, ...) for which this calculus is intended.
\calculusLogicOrder{General}   % Specify the order of the logic (e.g. Propositional, Quantified Propositional, First-Order, Higher-Order, ...).
\calculusType{Sequent-style or Hilbert-style calculus}         % Specify the calculus type (e.g. Tableau, Sequent Calculus, Hyper-Sequent Calculus, Natural Deduction, ...)
\calculusYear{1988}         % The year when the calculus was published.

\calculusAuthor{Don Sannella}       % The name(s) of the author(s) of the calculus.
\calculusAuthor{Andrzej Tarlecki}
\calculusAuthor{Martin Wirsing}


\entryTitle{Entailment for Structured Specifications}     % Title of the entry (usually coincides with the name of the calculus).
\entryAuthor{Rolf Hennicker}     
\entryAuthor{Donald Sannella}     
\entryAuthor{Andrzej Tarlecki}
\entryAuthor{Martin Wirsing}

% The encyclopedia's peer-reviewing policy is described here: 
% http://proofsystem.github.io/Encyclopedia/
%
% Reviewers of this entry will be acknowledged in the following lines:
% \entryReviewer{Reviewer 1's name}
% \entryReviewer{Reviewer 2's name}
% \entryReviewer{Reviewer 3's name}
%
% The lines above will be filled by the coordinators. 
% If you would like to indicate people 
% who could review your entry, contact the coordinators.

\maketitle

% If your files are called "MyProofSystem.tex" and "MyProofSystem.bib", 
% then you should write "\begin{entry}{MyProofSystem}" in the line below
\begin{entry}{StructSpec}  

% Define here any newcommands you may need:
% e.g. \newcommand{\necessarily}{\Box}
% e.g. \newcommand{\possibly}{\Diamond}

\newcommand{\ang}[1]{\langle{#1}\rangle}
\newcommand{\Family}[2]{\ang{#1}_{#2}}
\newcommand{\SP}{\mathit{SP}}
\newcommand{\Sig}{\mathit{Sig}}
\newcommand{\SPSig}[1]{\Sig[#1]}
\newcommand{\INS}{{\bf INS}}
\newcommand{\Sign}{{\bf Sign}}
\newcommand{\Sen}{{\bf Sen}}
\newcommand{\Set}{{\bf Set}}
\newcommand{\Mod}{{\bf Mod}}
\newcommand{\Cat}{{\bf Cat}}
\newcommand{\op}{\mathit{op}}
%\newcommand{\pow}{{\mathscr{P}}}
\newcommand{\pow}{{\mathit{Pow}}}
\newcommand{\translate}[2]{#1\textbf{\upshape{} with }#2}
\newcommand{\derive}[2]{#1\textbf{\upshape{} hide via }#2}
\newcommand{\union}[2]{#1 \cup #2}

\begin{calculus}

% Add the inference rules of your proof system here.
% The "proof.sty" and "bussproofs.sty" packages are available.
% If you need any other package, please contact the coordinator (Bruno Woltzenlogel Paleo <bruno.wp@gmail.com>)

\[
\infer{\SP \vdash \varphi}
      {\SP \vdash \varphi_1 \quad \cdots \quad \SP \vdash \varphi_n
         \qquad
         \{\varphi_1,\ldots,\varphi_n\} \vdash_{\SPSig{\SP}} \varphi}
\]
\[
\infer[\varphi \in \Phi]{\ang{\Sigma,\Phi} \vdash \varphi}
                        {}
\]
\[
\infer{\union{\SP_1}{\SP_2} \vdash \varphi}
      {\SP_1 \vdash \varphi}
\qquad
\infer{\union{\SP_1}{\SP_2} \vdash \varphi}
      {\SP_2 \vdash \varphi}
\]
\[
\infer{\translate{\SP}{\sigma} \vdash \sigma(\varphi)}
      {\SP \vdash \varphi}
\qquad
\infer{\derive{\SP}{\sigma} \vdash \varphi}
      {\SP \vdash \sigma(\varphi)}
\]

\end{calculus}

 \begin{clarifications}
$\INS = \ang{\Sign,\; \Sen\colon\Sign\to\Set,\; \Mod\colon\Sign^\op\to\Cat,\;
      \ang{{\models_\Sigma}\subseteq|\Mod(\Sigma)|\times\Sen(\Sigma)}_{\Sigma\in|\Sign|}}$
is an institution that defines the logical system used for specifications,
$\SP$, $\SP_1$ and $\SP_2$ are structured $\Sigma$-specifications over $\INS$,
where $\Sigma$ is a signature in the category $\Sign$,
$\varphi, \varphi_1, \ldots, \varphi_n$ are $\Sigma$-sentences,
i.e.\ elements in $\Sen(\Sigma)$,
$\Phi$ is a set of $\Sigma$-sentences, and
$\sigma(\varphi)$ denotes $\Sen(\sigma)(\varphi)$, the translation of the sentence $\varphi$ along $\sigma\colon\Sigma\to\Sigma'$.
Structured specifications in $\INS$ are built from basic specifications
$\ang{\Sigma, \Phi}$,
the union of $\Sigma$-specifications $\union{\SP_1}{\SP_2}$,
the translation ``$\translate{\SP}{\sigma}$'' of $\SP$
    along a signature morphism $\sigma\colon\Sigma\to\Sigma'$,
and hiding ``$\derive{\SP}{\sigma}$'' for hiding the symbols in $\SP$
    not occurring in the image of $\sigma\colon\Sigma'\to\Sigma$.
$\SPSig{\SP}$ is the signature of $\SP$.
Translations of $\Sigma$-sentences and $\Sigma'$-models
along $\sigma \colon \Sigma \rightarrow \Sigma'$
are required to preserve satisfaction:
for any $\varphi \in \Sen(\Sigma)$ and $M' \in |\Mod(\Sigma')|$,
$M' \models_{\Sigma'} \Sen(\sigma)(\varphi)
     \Leftrightarrow
  \Mod(\sigma)(M') \models_{\Sigma} \varphi$.
Finally, $\ang{{\vdash_\Sigma}\subseteq\pow(\Sen(\Sigma))\times\Sen(\Sigma)}_{\Sigma\in|\Sign|}$
is a sound entailment relation
for the satisfaction relation $\ang{{\models_\Sigma}}_{\Sigma\in|\Sign|}$.

The judgement $\SP \vdash \varphi$ is meant to capture the property that $\varphi$ is satisfied
in all models of $\SP$.
\end{clarifications}

\begin{history}
The first systems for proving entailment in structured specifications were
given by Sannella and Burstall \cite{SB83}, Sannella and Tarlecki \cite{ST88},
and Wirsing \cite{Wir91}.
The above presentation can be found in \cite{ST12}, Sect.~9.2.
\end{history}

 \begin{technicalities}
The system is sound; completeness is shown in \cite{Wir91} for the first-order logic instance and in \cite{Bor02,ST12}
for an institution $\INS$ which is finitely exact, admits propositional operators, satisfies Craig interpolation,
and has a complete entailment relation $\ang{{\vdash_\Sigma}}_{\Sigma\in|\Sign|}$.
\cite{ST14} shows that this is the most powerful sound proof system that is compositional in the structure of specifications.
 \cite{RH97} provides additional rules for observability operators.


\end{technicalities}
%References: [SB83], [ST88], [Wir91], [Bor02], [ST12], [ST14]
\end{entry}
