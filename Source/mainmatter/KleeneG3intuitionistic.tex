
\calculusName{Kleene's Intuitionistic $\Gthree$}   
\calculusAcronym{\Gthree}     
\calculusLogic{Intuitionistic Logic}  
\calculusLogicOrder{First-Order}
\calculusType{Sequent Calculus}   
\calculusYear{1952}   
\calculusAuthor{Stephen Cole Kleene} 

\entryTitle{Kleene's Intuitionistic $\Gthree$ System}     
\entryAuthor{Bj{\"o}rn Lellmann}
\entryAuthor{Valeria de Paiva}     

\etag{Two-Sided Sequents}
\etag{Asymmetric Sequents}
\etag{Single-succedent Sequents}

\maketitle

\begin{entry}{KleeneG3intuitionistic}

\begin{calculus}

% Add the inference rules of your proof system here.
% The "proof.sty" and "bussproofs.sty" packages are available.
% If you need any other package, please contact the editor (bruno@logic.at)
\[
\infer[]{A, \Gamma \seq A}{}
\]
\[
\begin{array}{c@{\qquad}c}
  \infer[\to\seq]{A \to B, \Gamma \seq \Theta}{A \to B, \Gamma \seq A
  & B, A \to B, \Gamma \seq \Theta} &
  \infer[\seq\to]{\Gamma \seq A \to B}{A,\Gamma \seq B}\medskip\\
  \infer[\lor\seq]{A \lor B, \Gamma\seq \Theta}{A, A \lor B, \Gamma
  \seq \Theta & B, A \lor B, \Gamma \seq \Theta}
  &
  \infer[\seq\lor_1]{\Gamma \seq A \lor B}{\Gamma \seq A} \quad 
  \infer[\seq\lor_2]{\Gamma \seq A \lor B}{\Gamma \seq B}\medskip\\
  \infer[\land\seq_1]{A \land B, \Gamma \seq \Theta}{A, A \land B,
  \Gamma \seq \Theta}
  \quad
  \infer[\land\seq_2]{A \land B, \Gamma \seq \Theta}{B, A \land B,
  \Gamma \seq \Theta}
  &
  \infer[\seq\land]{\Gamma \seq A \land B}{\Gamma \seq A & \Gamma \seq
                                                           B}\medskip\\
  \infer[\neg\seq]{\neg A, \Gamma\seq \Theta}{\neg A, \Gamma \seq
  A}
  &
  \infer[\seq\neg]{\Gamma \seq \neg A}{A, \Gamma \seq \neg
    A}\medskip\\
  \infer[\forall\seq]{\forall x A(x),\Gamma\seq \Theta}{A(t), \forall
  x A(x), \Gamma \seq \Theta}
  &
  \infer[\seq\forall]{\Gamma \seq \forall x A(x)}{\Gamma \seq
    A(b)}\medskip\\
  \infer[\exists\seq]{\exists x A(x), \Gamma \seq \Theta}{A(b), \exists x
  A(x), \Gamma \seq \Theta}
  &
  \infer[\seq\exists]{\Gamma \seq \exists x A(x)}{\Gamma \seq A(t)}
\end{array}
\]
\begin{center}
The term $t$ is free for $x$ in $A(x)$.\\
The variable $b$ is free for $x$ in $A(x)$ and (unless $b$ is $x$)
does not occur in $\Gamma,\Theta,A(x)$.
\end{center}
\end{calculus}


\begin{clarifications}
  $A,B$ are formulae; $\Gamma$ and $\Theta$ are a finite (possibly
  empty) sequences of formulae with $\Theta$ containing at most one
  formula; $x$ is a variable; $A(x)$ is a formula. In applications of
  the rules every sequent $\Gamma \seq \Theta$ can be replaced with a
  \emph{cognate} one, i.e., a sequent $\Gamma' \seq \Theta'$ such that
  the sets of formulae occurring in $\Gamma$ and $\Gamma'$ resp.\
  $\Theta$ and $\Theta'$ are the same (respecting the restriction to
  at most one formula on the right hand side).
\end{clarifications}

\begin{history}
  Kleene's systems, introduced in his~1952 monograph, were the staple of
  generations of logicians, who learned about sequent calculus from
  his textbooks~\cite{Kleene:1952} and~\cite{Kleene:1967}.
\end{history}

\begin{technicalities}
  Based on Gentzen's sequent calculus $\LJ$~\iref{GentzenLJ}
  (corresponding to intuitionistic $\mathbf{G1}$
  in~\cite{Kleene:1952}). Seems to be the first system
  (with~\iref{KleeneG3classical}) in which admissibility of
  contraction is obtained by copying the principal formulae into the
  premisses (accordingly, this is sometimes called \emph{Kleene's
  Method}). Used together with its multi-conclusion version for
  classical logic~\iref{KleeneG3classical} to uniformly obtain
  decidability of propositional classical and intuitionistic logics via
  backwards proof search in~\cite{Kleene:1952}.
\end{technicalities}

\nocite{Kleene:1952}

\end{entry}
