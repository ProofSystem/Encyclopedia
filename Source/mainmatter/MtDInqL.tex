
\calculusName{Proper Multi-type Display Calculus for Inquisitive Logic MtD.InqL}
\calculusAcronym{MtDInqL}
\calculusLogic{Inquisitive Logic}
\calculusLogicOrder{Propositional}
\calculusType{Multi-type Sequent Calculus}
\calculusYear{2016}
\calculusAuthor{Giuseppe Greco} \calculusAuthor{Alessandra Palmigiano} \calculusAuthor{Fan Yang} \calculusAuthor{Sabine Frittella} 

\entryTitle{Proper Multi-type Display Calculus for Inquisitive Logic MtD.InqL}
\entryAuthor{Giuseppe Greco}

\etag{Two-Sided Sequents}
\etag{Multi-type Cedents}
\etag{Multi-Premise Cedent}
\etag{Multi-Conclusion Cedent}

%%%Definition and abbreviations

\def\fCenter{{\mbox{$\ \vdash\ $}}}
\def\ffCenter{{\mbox{$\overline{\vdash}$}}}
\def\fcenter{{\mbox{\ }}}
\EnableBpAbbreviations

\def\fns{\footnotesize}
\def\mc{\multicolumn}

%operational classical connectives
\newcommand{\cbot}{\ensuremath{0}\xspace}
\newcommand{\cand}{\ensuremath{\sqcap}\xspace}
\newcommand{\cor}{\ensuremath{\sqcup}\xspace}
\newcommand{\bigcor}{\ensuremath{\bigotimes}\xspace}
%\newcommand{\cra}{\ensuremath{\mapsto}\xspace}
\newcommand{\cra}{\ensuremath{\rightarrowtriangle}\xspace}
\newcommand{\cdra}{\ensuremath{\mapsto}\xspace}

\newcommand{\clra}{\ensuremath{\leftrightarrowtriangle}\xspace}
%\newcommand{\cla}{\ensuremath{\leftarrow}\xspace}
\newcommand{\cla}{\ensuremath{\leftarrowtriangle}\xspace}
%\newcommand{\cdla}{\ensuremath{-\mkern-10mu\prec\,}\xspace}

\newcommand{\cneg}{\ensuremath{{\sim}}\xspace}

%operational intuitionistic connectives
\newcommand{\iand}{\ensuremath{\wedge}\xspace}
\newcommand{\ior}{\ensuremath{\vee}\xspace}
\newcommand{\bigior}{\ensuremath{\bigvee}\xspace}
\newcommand{\ira}{\ensuremath{\to}\xspace}
%\newcommand{\idra}{\ensuremath{\succ\mkern-10mu-\,}\xspace}
\newcommand{\idra}{\ensuremath{\rightarrowtail}\xspace}
\newcommand{\ila}{\ensuremath{\leftarrow}\xspace}
\newcommand{\idla}{\ensuremath{-\mkern-10mu\prec\,}\xspace}
\newcommand{\ineg}{\ensuremath{\neg}\xspace}


%structural classical connectives
\newcommand{\CBOT}{\ensuremath{\Phi}\xspace}
\newcommand{\CRA}{\ensuremath{\sqsupset}\xspace}

%multi-type connectives
\newcommand{\bh}{\ensuremath{{\downarrow}}\xspace}
\newcommand{\BH}{\ensuremath{{\Downarrow}}\xspace}
\newcommand{\hb}{\ensuremath{{\mathrm{f}}}\xspace}
\newcommand{\HB}{\ensuremath{\mathrm{F}}\xspace}
\newcommand{\ilra}{\ensuremath{\leftrightarrow}\xspace}

\maketitle

\begin{entry}{MtDInqL}

\begin{calculus}
{%\small
Identity and Cut rules:
\vspace{0.2cm}	
\begin{center}
\AXC{Id}
\noLine
\UI$p \fCenter p$
\DisplayProof
\qquad
\AX$X \fCenter A$
\AX$A \fCenter Y$
\RightLabel{Cut}
\BI$X \fCenter Y$
\DisplayProof
\qquad
\AX$\Gamma \fCenter \alpha$
\AX$\alpha \fCenter \Delta$
\RightLabel{Cut}
\BI$\Gamma \fCenter \Delta$
\DisplayProof
\end{center}
\vspace{0.2cm}		
Structural rules governing the interaction between the types $\mathsf{Flat}$ and $\mathsf{General}$:
\begin{center}
\AX$\HB^\ast \Gamma \fCenter \Delta$
\RightLabel{f adj}
\doubleLine
\UI$\Gamma \fCenter \HB \Delta$
\DisplayProof
\qquad
\AX$\HB X \fCenter \Gamma$
\RightLabel{d adj}
\doubleLine
\UI$X \fCenter \BH \Gamma$
\DisplayProof
\qquad
\AX$X \fCenter \BH \HB  Y$
\RightLabel{d-f elim}
\UI$X \fCenter Y$
\DisplayProof
\end{center}

\begin{center}
\AX$\Gamma \fCenter \Delta$
\RightLabel{bal}
\UI$\HB^\ast \Gamma \fCenter \BH \Delta$
\DisplayProof
\qquad
\AX$X \fCenter Y$
\RightLabel{f mon}
\UI$\HB X \fCenter \HB Y$
\DisplayProof
\qquad
\AX$X \fCenter \BH (\Gamma \CRA \Delta)$
\RightLabel{d dis}
\doubleLine
\UI$X \fCenter \HB^\ast \Gamma > \BH \Delta$
\DisplayProof
\qquad
\AX$\HB X \,, \HB Y \fCenter Z$
\RightLabel{f dis}
\doubleLine
\UI$\HB (X \,; Y) \fCenter Z$
\DisplayProof
\end{center}

\begin{center}
\AX$X \fCenter \HB^\ast \Gamma > (Y \,; Z)$
\AX$X \fCenter \HB^\ast \Gamma > (Y \,; Z)$
\RightLabel{KP}
\BI$X \fCenter (\HB^\ast \Gamma > Y) \,; (\HB^\ast \Gamma > Z)$
\DisplayProof
\end{center}

 }
\end{calculus}

\begin{clarifications}
The language $\mathcal{L}_\mathrm{MT}(\mathcal{F}, \mathcal{G})$ of  MtD.InqL consists of  {\em logical}   and {\em structural terms} in the types $\mathsf{T}_1:  = \mathsf{Flat}$ and $\mathsf{T}_2: = \mathsf{General}$. Following the notation of \iref{MtSC}, the set  of logical terms takes as parameters: 1) a denumerable  set of atomic terms $\mathsf{At}(\mathsf{Flat})$, elements of which are denoted $p$, possibly with indexes; 2) disjoint sets of connectives $\mathcal{F}: = \mathcal{F}_{\mathsf{Flat}}\uplus\mathcal{F}_{\mathsf{General}}\uplus\mathcal{F}_{\mathrm{MT}}$ and $\mathcal{G}: = \mathcal{G}_{\mathsf{Flat}}\uplus\mathcal{G}_{\mathsf{General}}\uplus\mathcal{G}_{\mathrm{MT}}$ defined as follows: 
$\mathcal{F}_{\mathsf{Flat}}: = \{\cand\}$, $\mathcal{F}_{\mathsf{General}}: = \{\iand \}$, $\mathcal{F}_{\mathrm{MT}}: = \emptyset$, where  $n_\cand = n_\iand = 2$,   and $\varepsilon_{\cand}(i) = \varepsilon_{\iand}(i) = 1$ for every $i\in \{1, 2\}$, and
$\mathcal{G}_{\mathsf{Flat}}: = \{\cbot, \cra\}$, $\mathcal{G}_{\mathsf{General}}: = \{\ior,\ira \}$, $\mathcal{G}_{\mathrm{MT}}: = \{\bh\}$ where  $n_\cbot = 0$, $n_\cra = n_\ior = n_{\ira} =  2$  and $n_{\bh} = 1$,  and $\varepsilon_{\cra}(1) = \varepsilon_{\ira}(1) = \partial$, and $\varepsilon_{\cra}(2) = \varepsilon_{\ira}(2) = \varepsilon_{\cor}(i) = \varepsilon_{\ior}(i) = 1$ for every $i\in \{1, 2\}$, and $\varepsilon_{\bh}(1) = 1$. The functional type of the heterogeneous connective $\bh$ is $\mathsf{Flat}\rightarrow \mathsf{General}$.

The structural terms are built by means of structural connectives, taking logical terms as atomic structures. The set of structural connectives includes  $\CRA,\, \CBOT,\, \BH$   which are the structural counterparts of $\cra, \cbot, \bh$, respectively.  It also includes $;$  as the structural counterpart of both $\iand$ (when occurring in antecedent position) and $\ior$ (when occurring in succedent position), and $>$ as structural counterpart of  $\ira$ (when occurring in succedent position) and left residual of $;$ (when occurring in antecedent position).
 Finally, it includes $\mathrm{F}$ and $\mathrm{F}^\ast$, where $\mathrm{F}$ is the left adjoint of $\BH$ and  $\mathrm{F}^\ast$ is the left  adjoint of $\mathrm{F}$. Hence, the functional type of $\mathrm{F}$ is $\mathsf{General}\rightarrow \mathsf{Flat}$ and of $\mathrm{F}^\ast$ is $\mathsf{Flat}\rightarrow \mathsf{General}$, and  $\mathrm{F}^\ast\in \mathcal{F}$, while $\mathrm{F}\in \mathcal{F}\cap \mathcal{G}$. 

Summing up, the well formed terms of MtD.InqL are generated by simultaneous induction as follows: 
\begin{center}
\setlength{\tabcolsep}{0.4em}
\begin{tabular}{lcl}
$\mathsf{Flat}$ & \ \ \ \ \ \ \ \ \ & $\mathsf{General}$\\
%&&\vspace{-4pt}\\
$\alpha ::= \,p \mid 0 \mid \alpha \cand \alpha \mid \alpha \cra \alpha$ & & $A ::= \,\bh \alpha \mid A \iand A \mid A \ior A \mid A \ira A$ \\
% & &\vspace{-4pt} \\
$\Gamma ::= \,\alpha \mid \Phi \mid \Gamma \,,\, \Gamma \mid \Gamma \CRA \Gamma\mid \HB X$ & & $X ::= \,A\mid\BH \Gamma\mid \HB^*\Gamma \mid X \,; X \mid X > X$ \\
\end{tabular}
\end{center}

The introduction rules instantiate the general template described in \iref{MtSC}, and hence are omitted. Also, the pure-type structural rules are the standard ones capturing classical logic (for $\mathsf{Flat}$) and intuitionistic logic (for $\mathsf{General}$) and are also omitted. 
\end{clarifications}


\begin{history}
Inquisitive logic is the logic of inquisitive semantics, developed by Ciardelli, Groenendijk and Roelofsen \cite{GroenendijkRoelofsen2009,CiardelliRoelofsen2011} for capturing both assertions and questions in natural language. In \cite{Yang2014}, systematic connections are developed between inquisitive semantics and the so called team semantics for dependence logic \cite{AbramskyVaananen2009}. Building on the algebraic analysis of the team semantics of \cite{AbramskyVaananen2009}, in  
 \cite{FrittellaGrecoPalmigianoYang2016}  the team semantics for inquisitive logic is recast in a multi-type algebraic framework which provides the guidelines for the design of a multi-type sequent calculus for InqL, of which the proper multi-type display calculus  above is a straightforward refinement. 
\end{history}

\begin{technicalities}
Every known axiomatization of inquisitive logic is neither analytic nor closed under uniform substitution. Hence, inquisitive logic cannot be captured by a single-type proper display calculus on the basis of any known axiomatization. The calculus above is sound and complete w.r.t. the team semantics for inquisitive logic, reformulated algebraically as discussed in \cite{FrittellaGrecoPalmigianoYang2016}; it is conservative, and enjoys the cut elimination and subformula property as immediate consequences of the general theory of multi-type calculi. 
\end{technicalities}

\end{entry}
