


\calculusName{Resolution}   
\calculusAcronym{}     
\calculusLogic{Classical Logic}  
\calculusType{Resolution}   
\calculusYear{1965}   
\calculusAuthor{John Alan Robinson} 


\entryTitle{Resolution}     
\entryAuthor{Uwe Waldmann}     





\maketitle



\begin{entry}{Resolution}  




\begin{calculus}

% Add the inference rules of your proof system here.
% The "proof.sty" and "bussproofs.sty" packages are available.
% If you need any other package, please contact the editor (bruno@logic.at)

\[
\infer[\textit{Resolution}]
{(D \lor C)\sigma}{D \lor B_1 \lor \dots \lor B_m
& C \lor \neg A_1 \lor \dots \lor \neg A_n}
\]
$C,D$ are (possibly empty) clauses,
$A_i,B_j$ are atoms.

$A_1,\dots,A_n,B_1,\dots,B_m$ are unifiable with most general unifier $\sigma$.

% \bigskip
% 
% \textbf{Saturation}
% \[
% \infer
% {N \cup \{C\}}{N & C \textrm{~is the conclusion of a Resolution inference
% from clauses in~} N}
% \]
% \mbox{}\quad $N$ is a finite set of clauses,
% $C$ is a clause.
\end{calculus}



\begin{clarifications}
Resolution is a refutational saturation calculus for
first-order clauses (disjunctions of possibly negated atoms).
It works on a set $N$ of clauses that is saturated
by successively computing \textit{Resolution} inferences
with premises in $N$ and adding the conclusion of the inference to $N$,
until the empty clause (i.\,e., false) is derived.
\end{clarifications}

\begin{history}
The ground version of the \textit{Resolution} rule appeared already
as ``Rule for Eliminating Atomic Formulas'' in
\cite{DavisPutnam1960JACM}.
To refute a set of non-ground clauses,
the rule was combined with a na{\"i}ve enumeration of ground instances.
Robinson's fundamental achievement~\cite{Robinson1965JACM}
was to extend the inference rule
to non-ground clauses in such a way that
the computation of useful instances became a by-product of the rule
application.
It was later detected that resolution can also be described
as the dual form of a special case of Maslov's
\textit{inverse method}~\cite{Maslov1964,Maslov1971}\irefmissing{Maslov}.

Many refinements of resolution were developed in the sequel,
aiming on the one hand at reducing the number of possible inferences
(e.\,g., using atom orderings~\iref{OrderedRes},
selection functions, set-of-support strategies)
and on the other hand at integrating particular axioms
into the calculus
(e.\,g., the equality axioms, yielding paramodulation~\iref{Paramodulation}).
Note that the factoring step (i.\,e., unification of literals
within the same clause) that is built into Robinson's original
\textit{Resolution} rule is usually given as a separate inference
rule in later publications, e.\,g.,~\iref{Paramodulation}.

\end{history}

\begin{technicalities}
The resolution calculus is refutationally complete for
sets of first-order clauses.
\end{technicalities}













\end{entry}
