\calculusName{First-Order Unification}   % The name of the calculus
\calculusAcronym{}    % The acronym if defined above, or empty otherwise.
\calculusLogic{Unification}  % Specify the logic (e.g. classical, intuitionistic, ...) for which this calculus is intended.
\calculusLogicOrder{First-Order}
\calculusType{Unification}   % Specify the calculus type (e.g. Frege-Hilbert style, tableau, sequent calculus, hypersequent calculus, natural deduction, ...)
\calculusYear{1965}   % The year when the calculus was invented.
\calculusAuthor{John Alan Robinson} % The name(s) of the author(s) of the calculus.

\entryTitle{First-Order Unification}     % Title of the entry (usually coincides with the name of the calculus).
\entryAuthor{Tomer Libal}    % Your name(s). Separate multiple names with "\and".

\entryReviewer{Bruno Woltzenlogel Paleo}

\maketitle

\begin{entry}{FolUnif}

\newcommand{\upair}[2]{\langle#1, #2\rangle}
\newcommand{\spair}[2]{\langle\langle#1, #2\rangle\rangle}

\begin{calculus}
\[
\infer[delete] {S}
               {\{\upair u u\} \cup S}
\qquad
\infer[decomp] {\{\upair {v_1} {u_1} ,\ldots, \upair {v_n} {u_n}\}  \cup S}
               {\{\upair {f(v_1,\ldots,v_n)} {f(u_1,\ldots,u_n)}\} \cup S}
\qquad
\infer[varelim]{\{\spair x v\} \cup \sigma(S)}
               {\{\upair x v\} \cup S}
\]
Where $x$ does not occur in $v$ and $\sigma = [v/x]$.
\end{calculus}

\begin{clarifications}
  $x$ is a variable. $v_1,\ldots,v_n, u_1,\ldots, u_n$ are terms.
  $\upair v u$ and $\spair v u$ are unsolved and solved, respectively, pairs of first-order terms.
  $S$ is a set of such pairs and $\sigma$ is a substitution.
\end{clarifications}

\begin{history}
   The unification principle was first described by Herbrand in his thesis \cite{herbrand1930recherches}
   but was overlooked until rediscovered independently by Prawitz \cite{Prawitz1960}
   and Guard \cite{guard1964automated} (where it is called matching, not to
   be confused with the modern notion of matching - the unification of a term
   with a ground term).
   These findings helped pave the way for Robinson's seminal work on Resolution \cite{Robinson1965JACM} (see \iref{Resolution}).
   The above set of rules is taken from Snyder and Gallier \cite{Snyder1989101}.
\end{history}

\begin{technicalities}
   The set $S$ is considered
   solved if it contains only solved pairs. The application of the above rules always terminates on a given set of pairs of terms
   and if, in addition, the set is unifiable, then it terminates in a set $S'$ containing only solved pairs.
   The set $S'$ contains the substitution components \cite{Robinson1965JACM} of a most general unifier of $S$.
   The choice of which equation to process is a "don't-care" non-determinism, which means
   that the resulting substitutions, if they exist, are identical up to the renaming of free variables.
 \end{technicalities}

% Leave an empty line above "\end{entry}".

\end{entry}
