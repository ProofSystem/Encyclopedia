\calculusName{Sequent Calculus TC}
\calculusAcronym{}
\calculusLogic{Ancestral Logic}
\calculusLogicOrder{Higher-Order}
\calculusType{Sequent Calculus}
\calculusYear{2002/2014}

\calculusAuthor{Arnon Avron}
\calculusAuthor{Liron Cohen}


\entryTitle{Sequent Calculus TC}
\entryAuthor{Arnon Avron}     
\entryAuthor{Liron Cohen}


\maketitle

\begin{entry}{TC}  

\begin{calculus}

\[
\infer[(sub)]{\Gamma\vdash\Delta,\left(TC_{x,y}\varphi\right)\left(s,t\right)}{\Gamma\vdash\Delta,\varphi\left\{ \frac{s}{x},\frac{t}{y}\right\} }
\]

\[
\infer[(trans)]{\Gamma\vdash\Delta,\left(TC_{x,y}\varphi\right)\left(s,t\right)}
  {\Gamma\vdash\Delta,\left(TC_{x,y}\varphi\right)\left(s,r\right)\,\,\,\,\,\,\,
   \Gamma\vdash\Delta,\left(TC_{x,y}\varphi\right)\left(r,t\right)}
\]

\[
\infer[(min)]{\Gamma,\left(TC_{x,y}\varphi\right)\left(s,t\right)\vdash\Delta,\psi\left\{ \frac{s}{x},\frac{t}{y}\right\} }
  {\Gamma,\varphi\left(x,y\right)\vdash\Delta,\psi\left(x,y\right)\,\,\,\,\,\,\,\,
   \Gamma,\psi\left\{ \frac{u}{x},\frac{v}{y}\right\} ,\psi\left\{ \frac{v}{x},\frac{w}{y}\right\} \vdash
       \Delta,\psi\left\{ \frac{u}{x},\frac{w}{y}\right\} }
\]


\end{calculus}


\begin{clarifications}
  The system is an extension of the sequent calculus for classical first-order
  logic, $\LK$~\iref{GentzenLK}. The letters $\Gamma,\Delta$ represent finite
  multisets of formulas, $\varphi,\psi$ arbitrary formulas, $x,y,u,v,w$ variables,
  and $r,s,t$ terms. 
  $\varphi\left\{ \frac{t_{1}}{x_{1}},...,\frac{t_{n}}{x_{n}}\right\}$ 
  stands for the result of simultaneously substituting $t_{i}$ for $x_{i}$ in
  $\varphi$ $(i=1,...,n)$. In all three rules the terms which are substituted
  should be free for substitution, and no forbidden capturing should occur. In
  Rule $(min)$ $x,y$ should not occur free in $\Gamma$ and $\Delta$, and $u,v,w$
  should not occur free in $\Gamma,\Delta,\phi$ and $\psi$.
\end{clarifications}

\begin{history}
  The sequent calculus was presented in~\cite{Cohen2014AL}, based on suggestions
  made in~\cite{AvronTC03}. A similar sequent calculus for the reflexive
  transitive closure operator was also presented in~\cite{Cohen2014AL}. Equivalent
  Hilbert-style systems for the reflexive transitive closure operator were
  suggested in~\cite{martin1943homogeneous, martin1949note, Myhill52}.
\end{history}

\begin{technicalities}
  %The transitive closure logic is inherently incomplete, i.e., any formal
  %deductive system which is sound for it is incomplete. It 
  The sequent calculus generalizes Gentzen's calculus for Peano's Arithmetic. It
  is sound with respect to the intended semantics of the transitive closure
  operator. It is also sound and complete with respect to generalized Henkin-style
  semantics of the operator.
\end{technicalities}

\end{entry}
