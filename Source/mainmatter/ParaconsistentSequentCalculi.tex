\calculusName{Sequent Calculi for Paraconsistent Logics}
\calculusAcronym{}
\calculusLogic{Paraconsistent Logics}
\calculusLogicOrder{Propositional}
\calculusType{Sequent Calculus}
\calculusYear{2012}

\calculusAuthor{Arnon Avron}
\calculusAuthor{Beata Konikowska}
\calculusAuthor{Anna Zamansky}


\entryTitle{Sequent Calculi for Paraconsistent Logics}
\entryAuthor{Anna Zamansky}
\entryAuthor{Yoni Zohar}

\maketitle

\begin{entry}{ParaconsistentSequentCalculi}

\newcommand{\G}{{\bf G}}
\newcommand{\GBK}{{\G_{\bf BK}}}
\newcommand{\name}[1]{{\bf (#1)}}
\newcommand{\vfi}{\varphi}
\newcommand{\su}{\supset}
\newcommand{\ssrul}[2]{\begin{array}{c}#1\\ \hline #2\end{array}}
\newcommand{\ddrul}[3]{\begin{array}{c}#1\hspace{2em}#2\\
\hline #3\end{array}}
\newcommand{\Ga}{\Gamma}
\newcommand{\De}{\Delta}
\newcommand{\Ra}{\Rightarrow}
\newcommand{\srul}[4]{\ssrul{\Ga #1\Ra\De #2}{\Ga
#3\Ra\De #4}}
\newcommand{\drul}[6]{\ddrul{\Ga #1\Ra\De #2}{\Ga #3\Ra\De #4}{\Ga
#5\Ra\De #6}}
\newcommand{\w}{\wedge}

\begin{calculus}

\renewcommand*{\arraystretch}{1.5}
\footnotesize
\[\begin{tabular}{ccc@{\hspace{2em}}ccc@{\hspace{2em}}cc}%\hline
% & { R}($ax$) &  & &  { R}($ax$)\\
%
${\name{c}}$ &     $\srul{,\vfi}{}{,\neg\neg\vfi}{}$  &&
${\name{e}}$    & $\srul{}{,\vfi}{}{,\neg\neg\vfi}$  &&
${\name{i}}$ &    $\srul{,\vfi,\neg\vfi}{}{,\neg\! \circ \vfi}{}$ \\\hline
${\name{n^r_\w}}$ &
$\srul{}{,\neg\psi,\neg\vfi}{}{,\neg (\vfi\w\psi)}$ &&
${\name{n^r_\vee}}$ &       $\drul{}{,\neg\vfi}{}{,\neg\psi}{}{,\neg(\vfi\vee\psi)}$
&&
${\name{n^r_\su}}$ &   $\drul{}{,\vfi}{}{,\neg\psi}{}{,\neg(\vfi\su\psi)}$
 \\

${\name{n^l_\w}}$ &   $\drul{,\neg \vfi}{}{,\neg \psi}{}{,\neg(\vfi\w\psi)}{}$ &&
${\name{n^l_\vee}}$ & $\srul{,\neg\vfi,\neg\psi}{}{,\neg(\vfi\vee\psi)}{}$ &&
${\name{n^l_\su}}$  &  $\srul{,\vfi,\neg\psi}{}{,\neg(\vfi\su \psi)}{}$  \\\hline

%begin a_\w, a_\vee and a_\su
\vspace{-1.5em}
    & &&  &  $\drul{,\neg\vfi}{}{,\neg\psi,\psi}{}{,\neg     (\vfi\vee\psi)}{}$ && &  $\drul{,\vfi}{}{,\neg\psi,\psi}{}{,\neg    (\vfi\su\psi)}{}$ \ \  \\
\vspace{-1.5em}
\phantom{${\name{a_\wedge}}$ }
  &
  \phantom{$\drul{,\neg \vfi}{}{,\neg\psi}{}{,\neg  (\vfi\w\psi)}{}$ }
  &&
 ${\name{a_\vee}}$ & &&   ${\name{a_\su}}$  & \ \ \\
  &&&&  $\drul{,\neg\psi}{}{,\neg\vfi,\vfi}{}{,\neg(\vfi\vee\psi)}{}$ &&   &    $\drul{,\neg\vfi,\vfi}{}{,\neg\psi}{}{,\neg      (\vfi\su\psi)}{}$ \\\hline
%end a_\vee and a_\su

%begin o_\w^1 \vee and \su
\vspace{-1.5em}
&&&&   $\srul{,\neg \vfi}{}{,\neg (\vfi\vee\psi)}{}$   &&& $\drul{,\neg \vfi}{}{}{,\psi}{,\neg (\vfi\su\psi)}{}$  \ \ \\
\vspace{-1.5em}
 ${\name{o^{1}_\w}}$ &  $\drul{,\neg \vfi}{}{}{,\psi}{,\neg (\vfi\w\psi)}{}$ &&
 ${\name{o^{1}_\vee}}$ &&& ${\name{o^{1}_\su}}$ & \ \ \\
 &&&&  $\drul{,\vfi}{}{}{,\psi}{,\neg (\vfi\vee\psi)}{}$ &&&
 $\srul{,\vfi}{}{,\neg (\vfi\su\psi)}{}$ \\\hline
%end o_\w^1 \vee and \su

%begin o_\w^2 \vee and \su
\vspace{-1.5em}
&&&& $\srul{,\neg \psi}{}{,\neg (\vfi\vee\psi)}{}$ &&&   $\srul{,\neg \psi}{}{,\neg (\vfi\su\psi)}{}$ \ \ \\
\vspace{-1.5em}
 ${\name{o^{2}_\w}}$ &  $\drul{,\neg \psi}{}{}{,\vfi}{,\neg (\vfi\w\psi)}{}$ &&
 ${\name{o^{2}_\vee}}$ &&&
 ${\name{o^{2}_\su}}$ & \ \ \\
 &&&& $\drul{,\psi}{}{}{,\vfi}{,\neg (\vfi\vee\psi)}{}$ &&&$\drul{,\vfi}{}{,\psi}{}{,\neg (\vfi\su\psi)}{}$%\\\hline
%end o_\w^2 \vee and \su

\end{tabular}
\]
\normalsize
\end{calculus}

\begin{clarifications}
  Let $B$ be a subset of the above rules, that does not contain any of the
  pairs:
  ${\bf (o^1_\w)}$, ${\bf (n^r_\w)}$;
  ${\bf (o^2_\w)}$, ${\bf (n^r_\w)}$;
  ${\bf (o^1_\vee)}$, ${\bf (n^r_\vee)}$;
  ${\bf (o^2_\vee)}$,${\bf (n^r_\vee)}$; and
  ${\bf (o^1_\su)}$, ${\bf (n^r_\su)}$.
  $\GBK\left[B\right]$ is obtained from  propositional
   $\LK$~\iref{GentzenLK} by deleting
   %the left rule of negation,
  $\neg l$,
  and
  adding the rules $R(x)$ for every $x\in B$, as well as the rules:
  \footnotesize
  $(\circ\Ra)~\dfrac{\Gamma\Ra\vfi,\Delta~~~~ \Gamma\Ra\neg\vfi,\Delta}{\Gamma,\circ\vfi\Ra\De}$
  \normalsize and 
  \footnotesize
  $(\Ra\circ)~\dfrac{\Gamma,\vfi,\neg\vfi\Ra\Delta}{\Gamma\Ra\circ\vfi,\Delta}$.
\end{clarifications}

\begin{history}
  One of the most important families of da Costa's Brazilian School of
  paraconsistency is that of C-systems~\cite{car:jmar:Taxonomy}.
  %the so-called {\em Logics of Formal Inconsistency (LFIs)}~\cite{Carnielli2007}.
  The family of logics induced by the calculi above includes every C-system ever
  studied in the literature.
  %that includes the most important class of LFIs, called C-systems
  %\cite{car:jmar:Taxonomy}.
  All calculi were given in~\cite{Avron2015}, where well-known axioms for
  paraconsistent logics were translated to sequent rules.
\end{history}

\begin{technicalities}
  All calculi above enjoy cut-admissibility.
\end{technicalities}


\end{entry}
