% If the calculus has an acronym, define it.
% (e.g. \newcommand{\LK}{\ensuremath{\mathbf{LK}}\xspace})

\calculusName{Sequent Calculi for Paraconsistent Logics}         % The name of the calculus
\calculusAcronym{}          % The acronym if defined above, or empty otherwise.
\calculusLogic{Paraconsistent Logics}        % Specify the logic (e.g. Classical Logic, Intuitionistic Logic, ...) for which this calculus is intended.
\calculusLogicOrder{Propositional}   % Specify the order of the logic (e.g. Propositional, Quantified Propositional, First-Order, Higher-Order, ...).
\calculusType{Sequent Calculus}         % Specify the calculus type (e.g. Tableau, Sequent Calculus, Hyper-Sequent Calculus, Natural Deduction, ...)
\calculusYear{2012}         % The year when the calculus was published.

\calculusAuthor{Arnon Avron}       % The name(s) of the author(s) of the calculus.
\calculusAuthor{Beata Konikowska}
\calculusAuthor{Anna Zamansky}


\entryTitle{Sequent Calculi for Paraconsistent Logics}     % Title of the entry (usually coincides with the name of the calculus).
\entryAuthor{Anna Zamansky}
\entryAuthor{Yoni Zohar}
%\entryAuthor{ToDo:FullNameAuthor3}

% The encyclopedia's peer-reviewing policy is described here:
% http://proofsystem.github.io/Encyclopedia/
%
% Reviewers of this entry will be acknowledged in the following lines:
% \entryReviewer{Reviewer 1's name}
% \entryReviewer{Reviewer 2's name}
% \entryReviewer{Reviewer 3's name}
%
% The lines above will be filled by the coordinators.
% If you would like to indicate people
% who could review your entry, contact the coordinators.


% If you wish, use tags to give any other information
% that might be helpful for classifying and grouping this entry:
% e.g. \etag{Two-Sided Sequents}
% e.g. \etag{Multiset Cedents}
% e.g. \etag{List Cedents}
% You are free to invent your own tags.
% The Encyclopedia's coordinator will take care of
% merging semantically similar tags in the future.


\maketitle


% If your files are called "MyProofSystem.tex" and "MyProofSystem.bib",
% then you should write "\begin{entry}{MyProofSystem}" in the line below
\begin{entry}{ParaconsistentSequentCalculi}

% Define here any newcommands you may need:
% e.g. \newcommand{\necessarily}{\Box}
% e.g. \newcommand{\possibly}{\Diamond}
\newcommand{\G}{{\bf G}}
\newcommand{\GBK}{{\G_{\bf BK}}}
\newcommand{\name}[1]{{\bf (#1)}}
\newcommand{\vfi}{\varphi}
\newcommand{\su}{\supset}
\newcommand{\ssrul}[2]{\begin{array}{c}#1\\ \hline #2\end{array}}
\newcommand{\ddrul}[3]{\begin{array}{c}#1\hspace{2em}#2\\
\hline #3\end{array}}
\newcommand{\Ga}{\Gamma}
\newcommand{\De}{\Delta}
\newcommand{\Ra}{\Rightarrow}
\newcommand{\srul}[4]{\ssrul{\Ga #1\Ra\De #2}{\Ga
#3\Ra\De #4}}
\newcommand{\drul}[6]{\ddrul{\Ga #1\Ra\De #2}{\Ga #3\Ra\De #4}{\Ga
#5\Ra\De #6}}
\newcommand{\w}{\wedge}

\begin{calculus}

% Add the inference rules of your proof system here.
% The "proof.sty" and "bussproofs.sty" packages are available.
% If you need any other package, please contact the coordinator (Bruno Woltzenlogel Paleo <bruno.wp@gmail.com>)
\renewcommand*{\arraystretch}{1.5}
\footnotesize
\[\begin{tabular}{ccc@{\hspace{2em}}ccc@{\hspace{2em}}cc}%\hline
% & { R}($ax$) &  & &  { R}($ax$)\\
%
${\name{c}}$ &     $\srul{,\vfi}{}{,\neg\neg\vfi}{}$  &&
${\name{e}}$    & $\srul{}{,\vfi}{}{,\neg\neg\vfi}$  &&
${\name{i}}$ &    $\srul{,\vfi,\neg\vfi}{}{,\neg\! \circ \vfi}{}$ \\\hline
${\name{n^r_\w}}$ &
$\srul{}{,\neg\psi,\neg\vfi}{}{,\neg (\vfi\w\psi)}$ &&
${\name{n^r_\vee}}$ &       $\drul{}{,\neg\vfi}{}{,\neg\psi}{}{,\neg(\vfi\vee\psi)}$
&&
${\name{n^r_\su}}$ &   $\drul{}{,\vfi}{}{,\neg\psi}{}{,\neg(\vfi\su\psi)}$
 \\

${\name{n^l_\w}}$ &   $\drul{,\neg \vfi}{}{,\neg \psi}{}{,\neg(\vfi\w\psi)}{}$ &&
${\name{n^l_\vee}}$ & $\srul{,\neg\vfi,\neg\psi}{}{,\neg(\vfi\vee\psi)}{}$ &&
${\name{n^l_\su}}$  &  $\srul{,\vfi,\neg\psi}{}{,\neg(\vfi\su \psi)}{}$  \\\hline

%begin a_\w, a_\vee and a_\su
\vspace{-1.5em}
    & &&  &  $\drul{,\neg\vfi}{}{,\neg\psi,\psi}{}{,\neg     (\vfi\vee\psi)}{}$ && &  $\drul{,\vfi}{}{,\neg\psi,\psi}{}{,\neg    (\vfi\su\psi)}{}$ \ \  \\
\vspace{-1.5em}
\phantom{${\name{a_\wedge}}$ }
  &
  \phantom{$\drul{,\neg \vfi}{}{,\neg\psi}{}{,\neg  (\vfi\w\psi)}{}$ }
  &&
 ${\name{a_\vee}}$ & &&   ${\name{a_\su}}$  & \ \ \\
  &&&&  $\drul{,\neg\psi}{}{,\neg\vfi,\vfi}{}{,\neg(\vfi\vee\psi)}{}$ &&   &    $\drul{,\neg\vfi,\vfi}{}{,\neg\psi}{}{,\neg      (\vfi\su\psi)}{}$ \\\hline
%end a_\vee and a_\su

%begin o_\w^1 \vee and \su
\vspace{-1.5em}
&&&&   $\srul{,\neg \vfi}{}{,\neg (\vfi\vee\psi)}{}$   &&& $\drul{,\neg \vfi}{}{}{,\psi}{,\neg (\vfi\su\psi)}{}$  \ \ \\
\vspace{-1.5em}
 ${\name{o^{1}_\w}}$ &  $\drul{,\neg \vfi}{}{}{,\psi}{,\neg (\vfi\w\psi)}{}$ &&
 ${\name{o^{1}_\vee}}$ &&& ${\name{o^{1}_\su}}$ & \ \ \\
 &&&&  $\drul{,\vfi}{}{}{,\psi}{,\neg (\vfi\vee\psi)}{}$ &&&
 $\srul{,\vfi}{}{,\neg (\vfi\su\psi)}{}$ \\\hline
%end o_\w^1 \vee and \su



%begin o_\w^2 \vee and \su
\vspace{-1.5em}
&&&& $\srul{,\neg \psi}{}{,\neg (\vfi\vee\psi)}{}$ &&&   $\srul{,\neg \psi}{}{,\neg (\vfi\su\psi)}{}$ \ \ \\
\vspace{-1.5em}
 ${\name{o^{2}_\w}}$ &  $\drul{,\neg \psi}{}{}{,\vfi}{,\neg (\vfi\w\psi)}{}$ &&
 ${\name{o^{2}_\vee}}$ &&&
 ${\name{o^{2}_\su}}$ & \ \ \\
 &&&& $\drul{,\psi}{}{}{,\vfi}{,\neg (\vfi\vee\psi)}{}$ &&&$\drul{,\vfi}{}{,\psi}{}{,\neg (\vfi\su\psi)}{}$%\\\hline
%end o_\w^2 \vee and \su



 \end{tabular}\]
\normalsize
\end{calculus}

% The following sections ("clarifications", "history",
% "technicalities") are optional. If you use them,
% be very concise and objective. Nevertheless, do write full sentences.
% Try to have at most one paragraph per section, because line breaks
% do not look nice in a short entry.

 \begin{clarifications}
Let $B$ be a subset of the above rules, that does not contain any of the
 pairs:
${\bf (o^1_\w)}$, ${\bf (n^r_\w)}$;
${\bf (o^2_\w)}$, ${\bf (n^r_\w)}$;
${\bf (o^1_\vee)}$, ${\bf (n^r_\vee)}$;
${\bf (o^2_\vee)}$,${\bf (n^r_\vee)}$; and
${\bf (o^1_\su)}$, ${\bf (n^r_\su)}$.
$\GBK\left[B\right]$ is obtained from  propositional
 $\LK$ (\iref{GentzenLK}) by deleting
 %the left rule of negation,
$\neg l$,
 and
 adding the rules $R(x)$ for every $x\in B$, as well as the rules:
\footnotesize$
(\circ\Ra)~\dfrac{\Gamma\Ra\vfi,\Delta~~~~ \Gamma\Ra\neg\vfi,\Delta}{\Gamma,\circ\vfi\Ra\De} $
\normalsize and \footnotesize
$(\Ra\circ)~\dfrac{\Gamma,\vfi,\neg\vfi\Ra\Delta}{\Gamma\Ra\circ\vfi,\Delta}
$.
 \end{clarifications}

 \begin{history}
One of the most important families of da Costa's Brazilian School
of paraconsistency is that of
C-systems \cite{car:jmar:Taxonomy}.
%the so-called {\em Logics of Formal Inconsistency (LFIs)}
%\cite{Carnielli2007}.
The family of logics induced by the calculi above includes
every C-system ever studied in the literature.
%that includes the most important class of LFIs,
%called C-systems \cite{car:jmar:Taxonomy}.
All calculi were given in \cite{Avron2015},
where well-known axioms for paraconsistent logics were translated to sequent rules.
 \end{history}

 \begin{technicalities}
All calculi above enjoy cut-admissibility.
 \end{technicalities}


% General Instructions:
% =====================

% The preferred length of an entry is 1 page.
% Do the best you can to fit your proof system in one page.
%
% If you are finding it hard to fit what you want in one page, remember:
%
%   * Your entry needs to be neither self-contained nor fully understandable
%     (the interested reader may consult the cited full paper for details)
%
%   * If you are describing several proof systems in one entry,
%     consider splitting your entry.
%
%   * You may reduce the size of your entry by ommitting inference rules
%     that are already described in other entries.
%
%   * Cite parsimoniously (see detailed citation instructions below).
%
%
% If you do not manage to fit everything in one page,
% it is acceptable for an entry to have 2 pages.
%
% For aesthetic reasons, it is preferable for an entry to have
% 1 full page or 2 full pages, in order to avoid unused blank space.



% Citation Instructions:
% ======================

% Please cite the original paper where the proof system was defined.
% To do so, you may use the \cite command within
% one of the optional environments above,
% or use the \nocite command otherwise.

% You may also cite a modern paper or book where the
% proof system is explained in greater depth or clarity.
% Cite parsimoniously.

% Do not cite related work. Instead, use the "\iref" or "\irefmissing"
% commands to make an internal reference to another entry,
% as explained within the "history" environment above.

% You do not need to create the "References" section yourself.
% This is done automatically.


% Remove all instruction comments before submitting.


% Leave an empty line above "\end{entry}".

\end{entry}
