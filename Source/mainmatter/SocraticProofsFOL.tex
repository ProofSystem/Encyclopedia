
\newcommand{\EPQ}{\mathbf{E}^\mathsf{PQ}}

\calculusName{Socratic Proofs for FOL}   % The name of the calculus
\calculusAcronym{\EPQ}    % The acronym if defined above, or empty otherwise. 
\calculusLogic{Classical Logic}  % Specify the logic (e.g. classical, intuitionistic, ...) for which this calculus is intended.
\calculusLogicOrder{First-Order}
\calculusType{Socratic Proof System}   % Specify the calculus type (e.g. Frege-Hilbert style, tableau, sequent calculus, hypersequent calculus, natural deduction, ...)
\calculusYear{2004}   % The year when the calculus was invented.
\calculusAuthor{Andrzej Wi\'{s}niewski, Vasilyi Shangin} % The name(s) of the author(s) of the calculus.


\entryTitle{Socratic Proofs for FOL}     % Title of the entry (usually coincides with the name of the calculus).
\entryAuthor{Dorota Leszczy\'nska-Jasion}    % Your name(s). Separate multiple names with "\and".


\tag{Two-Sided Sequents}
\tag{Single Conclusion}
\tag{Sequence Cedents}
\tag{Invertible Rules}


\maketitle


\begin{entry}{SocraticProofsFOL}  

\begin{calculus}

The rules of calculus $\ESTAR$ (see \iref{SocraticProofsCPL}) and the quantifier rules:

\[
\begin{array}{ccc}

\infer[\textbf{L}_{\forall}]{?(\Phi ~;~ S ~'~ \forall x_{i}A ~'~ A(x_{i}/\tau) ~'~ T\:\vdash\: C ~;~ \Psi)}{?(\Phi ~;~  S ~'~ \forall x_{i}A ~'~ T\:\vdash\: C ~;~ \Psi)}

&~~&

\infer[\textbf{R}_{\forall}]{?(\Phi ~;~  S\:\vdash\: A(x_{i}/\tau) ~;~ \Psi)}{?(\Phi ~;~  S\:\vdash\:\forall x_{i}A ~;~ \Psi)}

\\

&& \\

\infer[\textbf{L}_{\exists}]{?(\Phi ~;~  S ~'~ A(x_{i}/\tau) ~'~ T\:\vdash\: C ~;~ \Psi)}{?(\Phi ~;~  S ~'~ \exists x_{i}A ~'~ T\:\vdash\: C ~;~ \Psi)}

&~~&

\infer[\textbf{R}_{\exists}]{?(\Phi ~;~  S ~'~ \forall x_{i}\neg A\:\vdash\: A(x_{i}/\tau) ~;~ \Psi)}{?(\Phi ~;~  S\:\vdash\:\exists x_{i}A ~;~ \Psi)}

\\
\end{array}
\]

In $\textbf{L}_{\forall}$, $\textbf{R}_{\exists}$: $x_i$ is free in $A$ and $\tau$ is any parameter. In $\textbf{L}_{\exists}$, $\textbf{R}_{\forall}$: $x_i$ is free in $A$, $\tau$ is a parameter which does not occur in the sequent distinguished in the premise.

\[
\begin{array}{ccc}
\infer[\textbf{L}_{\kappa}]{?(\Phi ~;~  S ~'~\kappa^{*}\:~'~ T\:\vdash\: C ~;~ \Psi)}{?(\Phi ~;~  S ~'~ \kappa ~'~ T\:\vdash\: C ~;~ \Psi)}

&~~~~~~~~&

\infer[\textbf{R}_{\kappa}]{?(\Phi ~;~  S\:\vdash\:\kappa^{*} ~;~ \Psi)}{?(\Phi ~;~  S\:\vdash\:\kappa ~;~ \Psi)}
\end{array}
\]

Where:

\begin{center}

	\begin{tabular}{c|c}
			\hline
$\kappa$ & $\kappa^{*}$   \\ 
			\hline
$\neg\exists x_{i}A$ & $\forall x_{i}\neg A$  \\
			
$\neg\forall x_{i}A$ & $\exists x_{i}\neg A$ \\
			
$\forall x_{i}A$, provided that $x_{i}$ is not free in $A$ & $A$ \\
			
$\exists x_{i}A$, provided that $x_{i}$ is not free in $A$ & $A$ \\
	\end{tabular}

\end{center}
\end{calculus}

\begin{clarifications}
For notational conventions see entry \ref{SocraticProofsCPL}. Socratic proofs for FOL start with questions concerning \textit{pure sequents}, i.e. sequents formed with sentences only and containing no parameters. A Socratic proof of a pure sequent `$S \vdash A$' in $\EPQ$ is a finite sequence of questions guided by the rules of $\EPQ$, starting with `$? (S \vdash A)$' and ending with a question based on a sequence of basic sequents, where a \textit{basic sequent} is a sequent containing the same formula in both of its cedents or containing a formula and its negation in the antecedent.
\end{clarifications}

 \begin{history}
The method has been first presented in \cite{AW:2006SPQ}, together with a constructive completeness proof. All the rules of $\EPQ$ are invertible and there are no structural rules. The erotetic calculus $\EPQ$ may be reconstructed into a sequent calculus $\mathbf{G}^\mathsf{PQ}$ for FOL with invertible rules and no structural rules. $\mathbf{G}^\mathsf{PQ}$ has been first described in \cite{AW:2006SPQ} and later examined in \cite{LJUW:2013}.
\end{history}

\begin{technicalities}
A pure sequent `$S \vdash A$' has a Socratic proof in $\EPQ$ iff $A$ is FOL-entailed by the set of terms of $S$.
\end{technicalities}


% General Instructions:
% =====================

% The preferred length of an entry is 1 page. 
% Do the best you can to fit your proof system in one page.
%
% If you are finding it hard to fit what you want in one page, remember:
%
%   * Your entry needs to be neither self-contained nor fully understandable
%     (the interested reader may consult the cited full paper for details)
%
%   * If you are describing several proof systems in one entry, 
%     consider splitting your entry.
%
%   * You may reduce the size of your entry by ommitting inference rules
%     that are already described in other entries.
%
%   * Cite parsimoniously (see detailed citation instructions below).
%
% 
% If you do not manage to fit everything in one page, 
% it is acceptable for an entry to have 2 pages.
%
% For aesthetical reasons, it is preferable for an entry to have
% 1 full page or 2 full pages, in order to avoid unused blank space.



% Citation Instructions:
% ======================

% Please cite the original paper where the proof system was defined.
% To do so, you may use the \cite command within 
% one of the optional environments above,
% or use the \nocite command otherwise.

% You may also cite a modern paper or book where the 
% proof system is explained in greater depth or clarity.
% Cite parsimoniously.

% Do not cite related work. Instead, use the "\iref" or "\irefmissing" 
% commands to make an internal reference to another entry, 
% as explained within the "history" environment above.

% You do not need to create the "References" section yourself. 
% This is done automatically.




% Leave an empty line above "\end{entry}".

\end{entry}
