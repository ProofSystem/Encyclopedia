
% If the calculus has an acronym, define it.
% (e.g. \newcommand{\LK}{\ensuremath{\mathbf{LK}}\xspace})

\calculusName{Logic of Epistemic Inconsistency}         % The name of the calculus
\calculusAcronym{LEI}          % The acronym if defined above, or empty otherwise. 
\calculusLogic{Paraconsistent}        % Specify the logic (e.g. Classical Logic, Intuitionistic Logic, ...) for which this calculus is intended.
\calculusLogicOrder{First-Order}   % Specify the order of the logic (e.g. Propositional, Quantified Propositional, First-Order, Higher-Order, ...).
\calculusType{Natural Deduction}         % Specify the calculus type (e.g. Tableau, Sequent Calculus, Hyper-Sequent Calculus, Natural Deduction, ...)
\calculusYear{1993}         % The year when the calculus was published.

\calculusAuthor{Ana Teresa Martins}       % The name(s) of the author(s) of the calculus.
\calculusAuthor{Lilia Ramalho Martins}
\calculusAuthor{Felipe Ferreira de Morais}


\entryTitle{Logic of Epistemic Inconsistency}     % Title of the entry (usually coincides with the name of the calculus).
\entryAuthor{Ana Teresa Martins}     
\entryAuthor{Francicleber Ferreira}
%\entryAuthor{ToDo:FullNameAuthor3}

% The encyclopedia's peer-reviewing policy is described here: 
% http://proofsystem.github.io/Encyclopedia/
%
% Reviewers of this entry will be acknowledged in the following lines:
% \entryReviewer{Reviewer 1's name}
% \entryReviewer{Reviewer 2's name}
% \entryReviewer{Reviewer 3's name}
%
% The lines above will be filled by the coordinators. 
% If you would like to indicate people 
% who could review your entry, contact the coordinators.


% If you wish, use tags to give any other information 
% that might be helpful for classifying and grouping this entry:
% e.g. \etag{Two-Sided Sequents}
% e.g. \etag{Multiset Cedents}
% e.g. \etag{List Cedents}
% You are free to invent your own tags. 
% The Encyclopedia's coordinator will take care of 
% merging semantically similar tags in the future.


\maketitle


% If your files are called "MyProofSystem.tex" and "MyProofSystem.bib", 
% then you should write "\begin{entry}{MyProofSystem}" in the line below
\begin{entry}{bibliographies/LEI}  

% Define here any newcommands you may need:
% e.g. \newcommand{\necessarily}{\Box}
% e.g. \newcommand{\possibly}{\Diamond}

%ToDo




\begin{calculus}

% Add the inference rules of your proof system here.
% The "proof.sty" and "bussproofs.sty" packages are available.
% If you need any other package, please contact the coordinator (Bruno Woltzenlogel Paleo <bruno.wp@gmail.com>)

\[
\begin{array}{ccccccccccccccc}

  
  \infer[!I]{\alpha !}{\deduce{\alpha}{\Pi}}
  
  &
  \quad
  &
  
  \infer[!E]{\alpha}{\alpha !}
  
  &
  \quad
  &
  
  \infer[?I]{\alpha ?}{\alpha}
  
  &
  \quad
  &
  
  \infer[?E]{\beta}{\alpha ? & \deduce{\beta}{\deduce{\Pi}{[\alpha]}}}
  
  &
  \quad
  &
  
  \infer[\neg I]{\neg \alpha}{\infer*{\bot}{[\alpha]}}
  
  
  

\\
\\
  

  
  \infer[\neg E]{\bot}{A & \neg A}
  
  &
  \quad
  &
  
  \infer[\neg ? I]{\neg(\alpha ?)}{(\neg\alpha) ?}
  
  &
  \quad
  &
  
  \infer[\neg ? E]{(\neg\alpha) ?}{\neg (\alpha ?)}
  
%  \infer[\sim I]{\sim \alpha}{\infer*{\bot}{[\alpha]}}
%  
%  &
%  \quad
%  &
%  
%  \infer[\sim E]{\bot}{\alpha & \sim \alpha}
%  
  &
  \quad
  &
  
  \infer[\bot\sim]{\alpha}{\infer*{\bot}{[\sim\alpha]}}
  
  &
  \quad
  &
  
  \infer[\bot\neg]{\alpha}{\infer*{\bot}{[\neg\alpha]}}
  
 
  
  \\
  \\
  
  
  \infer[\neg\land I]{\neg(\alpha\land\beta)}{\neg\alpha \lor \neg\beta}

  &
  \quad
  &
  
  \infer[\neg\land E]{\neg\alpha \land \neg\beta}{\neg(\alpha \lor \beta)}

  
  &
  \quad
  &
  
  \infer[\neg\lor I]{\neg(\alpha \lor \beta)}{\neg\alpha \land \neg\beta}
  
  &
  \quad
  &
  
  \infer[\neg\lor E]{\neg\alpha \lor \neg\beta}{\neg(\alpha\land\beta)}
  

  &
  \quad
  &
  
  \infer[\neg\neg I]{\neg\neg \alpha}{\alpha}
  
  
  
  \\
  \\
  
  
  \infer[\neg\neg E]{\alpha}{\neg\neg \alpha}
  
  &
  \quad
  &
  
  \infer[\neg\forall I]{\neg\forall x \alpha}{\exists x \neg\alpha}
  
  &
  \quad
  &
  
  \infer[\neg\forall E]{\exists x \neg\alpha}{\neg\forall x \alpha}
  
  &
  \quad
  &
  
  
  \infer[\neg\exists I]{\neg\exists x \alpha}{\forall x \neg\alpha}
  
  &
  \quad
  &
  
  \infer[\neg\exists E]{\neg\forall x \alpha}{\exists x \neg\alpha}
  
  \\
  \\
  
  
  %\infer[\neg\neg E]{\alpha}{\neg\neg \alpha}
  
  &
  \quad
  &
  
  \infer[\neg\to I]{\neg(\alpha \to \beta)}{\alpha \land \neg\beta}
  
  &
  \quad
  &
  
  %\infer[\neg\forall E]{\exists x \neg\alpha}{\neg\forall x \alpha}
  
  &
  \quad
  &
  
  
  \infer[\neg\to E]{\alpha \land \neg\beta}{\neg(\alpha \to \beta)}
  
  &
  \quad
  &
  
  %\infer[\neg\exists E]{\neg\forall x \alpha}{\exists x \neg\alpha}
 % 
%  \\
%  \\
%  
%  \infer[\neg ? I]{\neg(\alpha ?)}{(\neg\alpha) ?}
%  
%  &
%  \quad
%  &
%  
%  \infer[\neg ? E]{(\neg\alpha) ?}{\neg (\alpha ?)}
  
  \end{array}
\]
\vspace{-1em}

\end{calculus}

% The following sections ("clarifications", "history", 
% "technicalities") are optional. If you use them, 
% be very concise and objective. Nevertheless, do write full sentences. 
% Try to have at most one paragraph per section, because line breaks 
% do not look nice in a short entry.

% \begin{clarifications}
% ToDo: write here short remarks that may help the reader to understand 
% the inference rules of the proof system.
% \end{clarifications}


\begin{clarifications}
  The syntax of LEI is given by the following BNF rule
  \begin{center} 
    \begin{math}
      %\begin{array}{lll}
        \phi ::= p \in Atomic \mid \bot \mid \phi \land \phi \mid \phi \lor \phi \mid \phi \rightarrow \phi \mid \neg \phi \mid \sim \phi \mid \exists x\phi \mid \forall x\phi.
      %\end{array}
    \end{math}
  \end{center}
  
  LEI is intended to deal with the concept of plausibility. The $?$ and $!$ post-fixed operators symbolize credulous and skeptical plausibility. The skeptical plausibility can be defined as the dual of $?$, that is $\alpha ! := \sim ((\sim \alpha)?)$ LEI is a paraconsistent logic. It has two negation symbols $\sim$ (the classical negation) and $\neg$ (the paraconsistent negation). The $\sim$ can be defined as $\sim \alpha := \alpha \to \bot$. The intuitionistic absurd rule $\neg E$ is restricted to $?$-free formulas, which are represented by Roman capital letters. Introduction and elimination rules for $\land$, $\lor$, $\to$, $\forall$ and $\exists$ are the usual ones. The $?E$ and $!I$ rule has a quite involved restriction for its application. We need the following definitions.
  
  %\begin{definition} [Connection]
By a \emph{connection} in a deduction $\Pi$ between two formula occurrences $\alpha$ 
and
$\beta$, we understand a sequence $\alpha_{1},\ldots,\alpha_{n}$ of formula 
occurrences
in $\Pi$ such that $\alpha_{1}=\alpha, \alpha_{n}=\beta$, and one of the 
following
conditions holds for each $i \leq n$:


\begin{enumerate}

\item
 $\alpha_{i}$ is not the major premise of an application of $\vee E$, 
$\exists E$ and
$?E$, and $\alpha_{i+1}$ stands immediately below $\alpha_{i}$; or vice 
versa;

\item 
$\alpha_{i}$ is a premise of an application of $\rightarrow E$, $\neg E$ or $\sim E$, and $\alpha_{i+1}$ is side connected with  $\alpha_{i}$;

\item 
$\alpha_{i}$ is  the major premise of an application of $\vee E$, $\exists E$ and
$?E$, and $\alpha_{i+1}$ is a hypothesis discharged by this application;  or 
vice versa;

\item 
$\alpha_{i}$ is a consequence of an application of $\rightarrow I$, 
$\neg I$, $\sim I$,
$\bot\neg$, $\bot\!\!\sim$, and $\alpha_{i+1}$ is a hypothesis discharged by 
this application;
 or vice versa;
\end{enumerate}
%\end{definition}

Two formula occurrences $\alpha$
and $\beta$  are said to be \emph{modally independent} in a deduction $\Pi$ iff every
connection in $\Pi$ between $\alpha$ and $\beta$ contains an occurrence of a 
?-closed formula.

The $?E$ can only be applied when each occurrence of $\alpha$ is modally independent in $\Pi$ from $\beta$ and any (open) assumption of $\Pi$. The $!I$ requires that $\alpha$ is modally independent from any (open) assumption of $\Pi$.

\end{clarifications}


% \begin{history}
% ToDo: write here short historical remarks about this proof system,
% especially if they relate to other proof systems. 
% Use "\iref{OtherProofSystem}" to refer to another proof system 
% in the Encyclopedia (where "OtherProofSystem" is its ID). 
% Use "\irefmissing{SuggestedIDForOtherProofSystem}" to refer to 
% another proof system that is not yet available in the encyclopedia.
% \end{history}

\begin{history}
The Logic of Epistemic Inconsistency was proposed by Pequeno and Buchsbaum \cite{pequeno1991}. The natural deduction system for LEI was created by Martins and Pequeno \cite{martins1993}.
\end{history}


% \begin{technicalities}
% ToDo: write here remarks about soundness, completeness, decidability...
% \end{technicalities}


\begin{technicalities}
Completeness and correctness for LEI has been proved in \cite{martins1997}. Normalization for the system above has been proved in \cite{martins2007}.
\end{technicalities}

% General Instructions:
% =====================

% The preferred length of an entry is 1 page. 
% Do the best you can to fit your proof system in one page.
%
% If you are finding it hard to fit what you want in one page, remember:
%
%   * Your entry needs to be neither self-contained nor fully understandable
%     (the interested reader may consult the cited full paper for details)
%
%   * If you are describing several proof systems in one entry, 
%     consider splitting your entry.
%
%   * You may reduce the size of your entry by ommitting inference rules
%     that are already described in other entries.
%
%   * Cite parsimoniously (see detailed citation instructions below).
%
% 
% If you do not manage to fit everything in one page, 
% it is acceptable for an entry to have 2 pages.
%
% For aesthetic reasons, it is preferable for an entry to have
% 1 full page or 2 full pages, in order to avoid unused blank space.



% Citation Instructions:
% ======================

% Please cite the original paper where the proof system was defined.
% To do so, you may use the \cite command within 
% one of the optional environments above,
% or use the \nocite command otherwise.

% You may also cite a modern paper or book where the 
% proof system is explained in greater depth or clarity.
% Cite parsimoniously.

% Do not cite related work. Instead, use the "\iref" or "\irefmissing" 
% commands to make an internal reference to another entry, 
% as explained within the "history" environment above.

% You do not need to create the "References" section yourself. 
% This is done automatically.


% Remove all instruction comments before submitting.


% Leave an empty line above "\end{entry}".

\end{entry}
