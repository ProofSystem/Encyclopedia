\calculusName{Full Intuitionistic Linear Logic}   
\calculusAcronym{\FILL}     
%\calculusLogic{Intuitionistic Linear Logic} 
\calculusLogic{Linear Logics} 
\calculusLogicOrder{Propositional}
\calculusType{Sequent Calculus}   
\calculusYear{1990}   
\calculusAuthor{Martin Hyland}
\calculusAuthor{Valeria de Paiva} 

\entryTitle{Full Intuitionistic Linear Logic (FILL)}        
\entryAuthor{Harley Eades III}
\entryAuthor{Valeria de Paiva}  


\etag{Two-Sided Sequents}
\etag{Term Assignment}


\maketitle



\begin{entry}{FILL}  


\newcommand{\letbe}[3]{\mathsf{let}\,#1\,\mathsf{be}\,#2\,\mathsf{in}\,#3}
\newcommand{\letpat}[3]{\mathsf{let}\mbox{-}\mathsf{pat}\,#1\,#2\,#3}

\begin{calculus}
\[
\begin{array}{ccc}
\infer[Ax]{x : A \vdash x : A}{}
&
\quad
&
\infer[Cut]{\Gamma, \Gamma' \vdash \Delta \mid [t/y]\Delta'}{\Gamma \vdash t : A \mid \Delta & \Gamma', y : A \vdash \Delta'}
\\[8pt]
\infer[\top_L]{\Gamma,x : \top \vdash \letbe{x}{*}{\Delta}}{\Gamma \vdash \Delta}
&
\quad
&
\infer[\top_R]{\cdot \vdash * : \top}{}
\\[8pt]
\infer[\perp_L]{x : \perp \vdash \cdot}{}
& \quad & 
\infer[\perp_R]{\Gamma \vdash \circ : \perp \mid \Delta}{\Gamma \vdash \Delta}
\\[8pt]
\infer[\otimes_L]{\Gamma,z : A \otimes B \vdash \letbe{z}{x \otimes y}{\Delta}}{\Gamma,x : A,y : B \vdash \Delta}
&
\quad
&
\infer[\otimes_R]{\Gamma,\Gamma' \vdash t_1 \otimes t_2 : A \otimes B \mid \Delta \mid \Delta'}{\Gamma \vdash t_1 : A \mid \Delta & \Gamma' \vdash t_2 : B \mid \Delta'}
\\[8pt]
  \infer[\multimap_L]{\Gamma,y:A \multimap B,\Gamma' \vdash [y\,t/x]t_i : C_i \mid \Delta}{\Gamma \vdash t : A \mid \Delta & \Gamma',x : B \vdash t_i : C_i}
  &
  \quad
  &
  \infer[\multimap_R]{\Gamma \vdash \lambda x.t : A \multimap B \mid \Delta}{\Gamma, x : A \vdash t : B & x \not\in \mathsf{FV}(\Delta)}
  \\
\end{array}
\]
\[
\begin{array}{ccc}
  \infer[\parr_L]{\Gamma,\Gamma',z : A \parr B \vdash \letpat{z}{(x \parr -)}{t_i} : C_i \mid \letpat{z}{(- \parr y)}{t_j} : D_j}{\Gamma,x : A \vdash t_i : C_i & \Gamma', y : B \vdash t_j : D_j}\\[8pt]

  \infer[\parr_R]{\Gamma \vdash \Delta \mid t_1 \parr t_2 : A \parr B \mid \Delta'}{\Gamma \vdash \Delta \mid t_1 : A \mid t_2 : B \mid \Delta'}
\end{array}
\]


\end{calculus}



\begin{clarifications}
Both the left-hand and right-hand sides of sequents above are
multisets of formulas, denoted $\Gamma$ and $\Delta$.  The terms
annotating formulas are standard terms used in the simply typed
$\lambda$-calculus.  Capture avoiding substitution is denoted by
$[t/x]t'$, and uniformly replaces every occurrence of $x$ in $t'$ with
$t$.  The definition of the let-pattern function used in the rule
$\parr_L$ is defined as follows:
\begin{center}
    \begin{math}
      \begin{array}{lll}      
        \begin{array}{lll}
          \letpat{z}{(x \parr -)}{t} = t\\
          \,\,\,\,\,\,\mbox{where } x \not\in \mathsf{FV}(t)\\
        \end{array}
        & 
          \begin{array}{lll}
            \letpat{z}{(- \parr y)}{t} = t\\
        \,\,\,\,\,\,\mbox{where } y \not\in \mathsf{FV}(t)\\
          \end{array}
        & 
          \begin{array}{lll}
            \letpat{z}{p}{t} = \letbe{z}{p}{t}\\
            & \\
          \end{array}
      \end{array}
    \end{math}
\end{center}
We denote vectors of terms (resp. types) by $t_i$ (resp. $A_j$).  The
function $\mathsf{FV}(\Delta)$ constructs the set of all free
variables in each term found in $\Delta$.
\end{clarifications}

\begin{history}
The original formulation of FILL by Valeria de Paiva in her
thesis~\cite{dePaiva:1990} did not satisfy cut-elimination, as shown
by Schellinx.  Martin Hyland and Valeria de Paiva~\cite{Hyland:1993}
added a term assignment system to cope with the notion of dependency
in the right implication rule and obtain cut-elimination. However,
there was still a mistake in the par rule in~\cite{Hyland:1993}, which
was corrected independently, with different proof methods, by
Bierman~\cite{Bierman:1996}, Bellin~\cite{Bellin:1997},
Brauner/dePaiva~\cite{Brauner:1998},
dePaiva/Ritter~\cite{dePaiva:2006}. The version here is the minimal
modification suggested by Bellin, (who used proofnets), but using a
traditional term assignment, as described in
Eades/dePaiva~\cite{Eades:2015}.
\end{history}

%% \begin{technicalities}
%%   $\FILL$ enjoys cut elimination.  It also has a categorical model in
%%   dialectica categories~\cite{dePaiva:1990}.
%% \end{technicalities}

\end{entry}
