
\calculusName{Cancellative Superposition}   
\calculusAcronym{}     
\calculusLogic{Classical Logic}  
\calculusLogicOrder{First-Order}
\calculusType{Superposition}   
\calculusYear{1996}   
\calculusAuthor{Harald Ganzinger} \calculusAuthor{Uwe Waldmann} 


\entryTitle{Cancellative Superposition}     
\entryAuthor{Uwe Waldmann}     





\maketitle



\begin{entry}{CancellativeSup}




\begin{calculus}

% Add the inference rules of your proof system here.
% The "proof.sty" and "bussproofs.sty" packages are available.
% If you need any other package, please contact the editor (bruno@logic.at)

Cancellative rules (for simplicity, the ground versions
are given; the non-ground rules are obtained by lifting):
\[
\infer[\textit{Equality Resolution}]
{C}{C \lor \neg t \approx t}
\]
\[
\infer[\textit{Cancellation}]
{C \lor [\neg] (n{-}m)u + t \approx s}
{C \lor [\neg] nu + t \approx mu + s}
\]
\[
\infer[\textit{Cancellative Superposition}]
{D \lor C \lor [\neg] (n{-}m)u + t + s' \approx t' + s}
{D \lor mu + s \approx s' & C \lor [\neg] nu + t \approx t'}
\]
\[
\infer[\textit{Cancellative Equality Factoring}]
{C \lor \neg s + t' \approx s' + t \lor nu + t \approx t'}
{C \lor nu + s \approx s' \lor nu + t \approx t'}
\]
plus, if there are any non-constant function symbols besides $+$,
the rules of the standard superposition calculus~\iref{Superposition}
and
\[
\infer[\textit{Abstraction}]
{C \lor \neg x \approx nu + t \lor [\neg] w[x] \approx w'}
{C \lor [\neg] w[nu + t] \approx w'}
\]
$C,D$ are (possibly empty) equational clauses,
$s,s',t,t'$ are terms,
$u$ is an atomic term,
$n,m$ are positive integers.
Every literal involved in some inference
is maximal in the respective premise
(except for the last but one literal in
\textit{Equality Factoring} inferences).
A positive literal involved in a
\textit{Superposition} inference
is strictly maximal in the respective clause.
In every literal involved in a cancellative inference
(except \textit{Equality Resolution}),
the term $u$ is the maximal atomic term.

\end{calculus}



\begin{clarifications}
Cancellative superposition is a refutational saturation calculus for
first-order clauses containing the axioms of
cancellative abelian monoids or abelian groups.
The inference rules are supplemented by a redundancy criterion
that permits to delete clauses that are unnecessary for
deriving a contradiction during the saturation, see \iref{SaturationWithRed}.
\end{clarifications}

\begin{history}
As a na{\"\i}ve handling of axioms like commutativity or associativity
in an automated theorem prover
leads to an explosion of the search space,
there has been a lot of interest in
incorporating specialized techniques into general proof systems
to work efficiently within standard algebraic theories.
The cancellative superposition calculus~\cite{GanzingerWaldmann1996CADE}
shown above
is one example of a saturation calculus with a built-in algebraic theory.
By using dedicated inference rules,
explicit inferences with the theory axioms become superfluous;
moreover variable elimination techniques
and strengthened ordering restrictions and redundancy criteria
lead to a significant reduction of the search space.
The cancellative superposition calculus is refutationally complete for
first-order logic modulo cancellative abelian monoids.

Other examples for ``white-box'' theory integration include
calculi for dealing with associativity and
commutativity~\cite{Plotkin1972,Slagle1974JACM,RusinowitchVigneron1995,BachmairGanzinger1994CTRS},
% \cite{PetersonStickel1981}
% \cite{Paul1992}
superposition modulo abelian groups~\cite{GodoyNieuwenhuis2004},
chaining calculi~\cite{Slagle1972JACM,Hines1992JAR,BachmairGanzinger1994LICS,BachmairGanzinger1994CADE},
or superposition modulo
divisible torsion-free abelian groups
or ordered divisible abelian groups~\cite{Waldmann2002abJSC,Waldmann2001IJCAR}.

\end{history}

% \begin{technicalities}
% \end{technicalities}













\end{entry}
