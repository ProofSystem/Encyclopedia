\calculusName{Principia Mathematica}       
\calculusAcronym{}         
\calculusLogic{Classical Logic}      
\calculusLogicOrder{First-Order}
\calculusType{Hilbert-Style}
\calculusYear{1910}     

\calculusAuthor{Bertrand A. Russel}
\calculusAuthor{Alfred N. Whitehead}


\entryTitle{Principia Mathematica}    
\entryAuthor{Bernard Linsky}  


\maketitle


\begin{entry}{PrincipiaMathematica}  




\begin{calculus}

\begin{center}

\textbf{Axioms:}

(1) \ $( p\; {\scriptstyle \vee} \;  p )  \supset  \;  p $ 
\qquad
(2) \ $\; q \;  \supset (  p \;  {\scriptstyle \vee} \; q ) $ 
\qquad
(3) \ $( p \;  {\scriptstyle \vee} \; q  ) \supset ( q \;  {\scriptstyle \vee} \; p )  $ 
\qquad
(4) \ $p \;  {\scriptstyle \vee} \; (q \;  {\scriptstyle \vee} \;r ) \;  \supset . \; q \;  {\scriptstyle \vee} \; (p \;  {\scriptstyle \vee} \;r  )   $

\smallskip

(5) \ $( q \supset r    )  \supset . \; (p \;  {\scriptstyle \vee} \; q \; ) \supset (\; p \;  {\scriptstyle \vee} \; r ) $ 

\smallskip

(6) Universal Instantiation: 
$ \forall \nu_{\tau} \; \psi \; \supset \;  \psi '$ \\ where $\psi '$ is like $\psi$ except for having a term $\nu$ of r - type $\tau$ substituted for $\nu_{\tau}$ in $\psi$.

\smallskip

(7) Comprehension: $\exists \phi \: \forall x_1 \: \ldots \forall x_m \: [ \phi (x_1, \ldots , x_m) \equiv \psi], (\phi \mathrm{\; not \; free \; in\; } \psi)$

\smallskip

(8) Axiom of Reducibility: $ \forall \psi \: \exists \phi \: \forall x_1 \ldots \forall x_m \: [ \phi ! (x_1, \ldots , x_m) \equiv \psi  (x_1, \ldots , x_m)]$


\medskip

\textbf{Rules:}


(1) Modus Ponens: \[ \frac{\phi \; \; \; \phi \supset \psi}{\psi} \]

(2) Substitution for individual, functional and propositional variables of each type. 

\medskip

(3) Universal Generalization: \[ \frac{\phi}{\forall \nu_{\tau} \: \phi}  \]



\textbf{Definition of Identity:}

$ x = y \;\; =_{df} \;\; \forall \phi \: [\phi ! (x) \equiv \phi ! (y) ]$, (for $\phi !$ a  predicative function)


\end{center}
\end{calculus}


\begin{clarifications} The
\emph{primitive connectives} are $ \vee $ and $\sim$. The connective $p \: \supset \: q$ is defined as $ \sim p \: \vee \: q$ and the conjunction $p\: . \: q$ as $\sim \! (\sim \! p \: \vee \: \sim \! q)$.

{\bf {\em r-types:}}
The system of symbols for {\em r-types} ({\em ramified types}) and the assignment of  r-types to variables for different entities (individuals and functions) is as follows:
$\iota$ is the r-type for an {\em individual}. Where $\tau_1 
\ldots, \tau_m$ are any r-types, then $(\tau_1 
\ldots, \tau_m) / n$ is  the r-type of any $m$-ary propositional function of {\em level} $n$, which has arguments of r- types $\tau_1 
\ldots, \tau_m$, respectively. 
 The {\em order} of an individual  is 0.
The {\em order} of a function of r-type $\tau_1 
\ldots, \tau_m / n$ is $n+N$ where $N$ is the greatest of the order of the arguments $\tau_1 
\ldots, \tau_m$.

{\bf Typical ambiguity:} All statements of axioms and rules apply in each r-type.

{\bf Restriction on comprehension principle:} The comprehension principle avoid paradox by imposing the following restriction on the function $\phi$: $\phi$ is a functional variable of r-type $(\beta_1, \beta_2, \ldots, \beta_m )/ n$ and $x_1, \ldots , x_m$ are distinct variables of r -types $\beta_1, \beta_2, \ldots, \beta_m$, and the bound variables of $A$ are all of order less than the order of $\phi$ and the free variables of $A$ and constants occurring in $A$ are all of order not greater than the order of $\phi$.


{\bf Predicative functions:} The notation $\phi !$ indicates that the function $\phi$ is {\em predicative}, that is,  the variables $x_1, \ldots , x_m$ are of r-types $\beta_1, \beta_2, \ldots, \beta_m$ respectively and $\phi$ is of r-type $(\beta_1, \beta_2, \ldots, \beta_m )/1 $ and $\psi$ is of r-type $(\beta_1, \beta_2, \ldots, \beta_m) /n $. 

{\bf The ``Multiplicative Axiom'' (Axiom of Choice)  and ``Axiom of Infinity'':}. These sentences are not axioms of the formal system, but rather appear in theorems as the antecedents of conditional theorems when used to derive the consequent. 

{\bf Adequacy of the Definition of Identity:} The Axiom of Reducibility guarantees that if for any $\psi$ of any r-type,  $\exists \psi \sim \! [\psi (x) \equiv \psi (y)] $, then $x \neq y$. 
\end{clarifications}

\begin{history}
The system was published in \cite{PrincipiaMathematica}. This formulation, and most notably, the system of {\em r-types} follows that of Church \cite{Church}. Church also adds a Comprehension principle, and rule of Substitution, neither of which are explicit in \cite{PrincipiaMathematica}. See \cite{LinskySEPNotation} for an account of Whitehead and Russell's notation and \cite{LinskySEP} for a survey of the contents of \cite{PrincipiaMathematica}.
\end{history}

\begin{technicalities}
In 1926 Paul Bernays \cite{Bernays} showed that the axioms can be reduced by one, as axiom 4 can be proved from 1,2,3 and 5.
\end{technicalities}

\end{entry}
