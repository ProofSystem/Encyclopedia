\calculusName{Conflict Resolution}
\calculusAcronym{\CR} 
\calculusLogic{Classical Logic}
\calculusLogicOrder{First-Order}
\calculusType{Resolution} 
\calculusType{Natural Deduction}
\calculusYear{2016} 
\calculusAuthor{Bruno Woltzenlogel Paleo}

\implementation{Scavenger}{https://gitlab.com/aossie/Scavenger}

\entryTitle{Conflict Resolution}  
\entryAuthor{Bruno Woltzenlogel Paleo}     


% Reviewers of this entry will be acknowledged in the following lines:
% \entryReviewer{Reviewer 1's name}
% \entryReviewer{Reviewer 2's name}
% \entryReviewer{Reviewer 3's name}

\maketitle


\begin{entry}{ConflictResolution}  


\newcommand{\upr}[1]{\mathbf{u}(#1)}
\newcommand{\con}[1]{\mathbf{c}(#1)}
\newcommand{\cdcl}{\mathbf{cl}}
\newcommand{\dual}[1]{\overline{#1}}
\newcommand{\dls}[2]{\begin{tiny}\overset{#1}{#2}\end{tiny}}

\begin{calculus}
\centering

\begin{multicols}{2}
\textbf{Unit-Propagating Resolution:}
\[
\infer[\upr{\sigma}]{\ell~\sigma}{\ell_1 & \ldots & \ell_n & \dual{\ell'_1} \vee \ldots \vee \dual{\ell'_n} \vee \ell}
\]
$\sigma$ is a unifier of $\ell_k$ and $\ell'_k$, for all $k \in \{1, \ldots, n \}$

\columnbreak

\textbf{Conflict:}
$$
\infer[\con{\sigma}]{\bot}{\ell & \dual{\ell'}}
$$
$\sigma$ is a unifier of $\ell$ and $\ell'$.

\end{multicols}

%\bigskip
\bigskip



\textbf{Conflict-Driven Clause Learning:}
$$
\infer[\cdcl^{i}]{
  (\dual{\ell_1} \sigma^1_1 \vee \ldots \vee \dual{\ell_1} \sigma^1_{m_1}) \vee 
  \ldots \vee 
  (\dual{\ell_n} \sigma^n_1 \vee \ldots \vee \dual{\ell_n} \sigma^n_{m_n})
}{
  \infer*{\bot}{\infer*[(\sigma_1^1,\ldots,\sigma_{m_1}^1)]{}{[\ell_1]^{\dls{i}{1} } } 
  &  & 
  \infer*[(\sigma_1^n,\ldots,\sigma_{m_n}^n)]{}{[\ell_n]^{\dls{i}{n} } } }
}
$$

$\sigma^k_j$ %(for $1 \leq k \leq n$ and $1 \leq j \leq m_k$) 
is the composition of all substitutions used on the $j$-th path from $\ell_k$ to $\bot$.
\end{calculus}


\begin{clarifications}
Proofs are directed acyclic graphs and not necessarily tree-like. Therefore,
there may be several paths connecting $\ell_k$ to $\bot$. The conflict-driven
clause learning rule takes all paths into account.
When restricted to propositional logic, \CR-derivations ending in $\bot$ are
isomorphic to \emph{conflict graphs} of \textsc{Sat}-solvers.
\end{clarifications}

\begin{history}
Unit-propagating resolution is a restriction of resolution~\iref{Resolution} 
also known as unit-resulting resolution. The conflict-driven clause learning rule is,
at the proof-theoretical level, a first-order lifting of a procedure implemented by 
\textsc{Sat}-solvers. This rule generalizes natural deduction's implication
introduction rule to the case with unification and multiple assumptions, as
Robinson's resolution rule generalizes implication elimination (modus
ponens)~\cite{UniversalityOfProofs}. \texttt{Scavenger}~\cite{AITP,CADE} was the
first theorem prover based on Conflict Resolution.
\end{history}

\begin{technicalities}
Soundness was proven by simulation by a clausal variant of natural
deduction~\iref{GentzenNJ} (i.e. every $\CR$-proof can be translated into a
natural deduction proof) and refutational completeness was proven by simulating
resolution~\iref{Resolution} (i.e. every resolution refutation can be translated
into a $\CR$-refutation)~\cite{ConflictResolution}.  
\end{technicalities}


\end{entry}
