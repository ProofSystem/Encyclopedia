\newcommand{\dest}[4]{\mathsf{dest}~#1~\mathsf{as}~(#2,#3)~\mathsf{in}~#4}
\newcommand{\Case}[5]{\mathsf{case}~#1~\mathsf{of}~[#2\,\Rightarrow\,#3~|~#4\,\Rightarrow\,#5]}
\newcommand{\efq}[1]{\texttt{efq}~{#1}}
\newcommand{\lmu}{\lambda\mu}

\calculusName{LambdaMu} 
\calculusAcronym{$\lmu$} 
\calculusLogic{Classical Logic} 
\calculusLogicOrder{First-Order}
\calculusType{Natural Deduction} 
\calculusYear{1992} 
\calculusAuthor{Parigot} 
\entryTitle{Classical Natural Deduction ($\lmu$-calculus)} 
\entryAuthor{Hugo Herbelin}  

\etag{Multi-Conclusion Sequents}
\etag{Term-annotated}

\maketitle

\begin{entry}{LambdaMu}


\begin{calculus}

{\sc Structural subsystem}
\[
\infer[\mathit{Ax}]
      {a : \Gamma \vdash A ~|~ \Delta}
      {A^a \in \Gamma}
\]
\[
\infer[\mathit{Focus}]
      {\mu\alpha.c : \Gamma \vdash A ~|~ \Delta}
      {c : \Gamma \vdash A^\alpha, \Delta}
\qquad
\infer[\mathit{Unfocus}]
      {[\alpha]p : \Gamma \vdash \Delta}
      {p : \Gamma \vdash A ~|~ \Delta \qquad A^\alpha \in \Delta}
\]
{\sc Introduction rules}
\[
\infer[\wedge_E^i]
      {\pi_1(p) : \Gamma \vdash A_i ~|~ \Delta}
      {p : \Gamma \vdash A_1 \wedge A_2 ~|~ \Delta}
\qquad
\infer[\wedge_I]
      {(p_1,p_2) : \Gamma \vdash A_1 \wedge A_2 ~|~ \Delta}
      {p_1 : \Gamma \vdash A_1 ~|~ \Delta & p_2 : \Gamma \vdash A_2 ~|~ \Delta}
\]
\[
\infer[\vee_E]
      {\Case{p}{a_1}{p_1}{a_2}{p_2} : \Gamma \vdash C ~|~ \Delta}
      {p : \Gamma \vdash A_1 \vee A_2 ~|~ \Delta & p_1 : \Gamma, A_1^{a_1} \vdash C ~|~ \Delta & p_2 : \Gamma, A_2^{a_2} \vdash C ~|~ \Delta}
\]
\[
\infer[\vee_I^i]
      {\iota_i(q) : \Gamma \vdash A_1 \vee A_2 ~|~ \Delta}
      {q : \Gamma \vdash A_i ~|~ \Delta}
\]
\[
\infer[\rightarrow_E]
      {p\,q: \Gamma \vdash B ~|~ \Delta}
      {p : \Gamma \vdash A\rightarrow B ~|~ \Delta & q : \Gamma \vdash A ~|~ \Delta}
\qquad
\infer[\rightarrow_I]
      {\lambda a.p : \Gamma \vdash A \rightarrow B ~|~\Delta}
      {p : \Gamma, A^a \vdash B ~|~ \Delta}
\]
\[
\infer[\exists_E]
      {\dest{p}{y}{a}{q} : \Gamma \vdash C ~|~ \Delta}
      {p : \Gamma \vdash \exists x\,A ~|~ \Delta & q : \Gamma, A[y/x]^a \vdash C~|~ \Delta}
\qquad
\infer[\exists_I]
      {(t,p) : \Gamma \vdash \exists x\,A ~|~ \Delta}
      {p :  \Gamma \vdash A[t/x] ~|~ \Delta}
\]
\[
\infer[\forall_E]
      {p\,t : \Gamma \vdash A[t/x] ~|~ \Delta}
      {p: \Gamma \vdash \forall x\, A ~|~ \Delta}
\qquad
\infer[\forall_I]
      {\lambda y.p : \Gamma \vdash \forall x\,A ~|~ \Delta}
      {p : \Gamma \vdash A[y/x] ~|~ \Delta}
\]
\[
\infer[\bot_E]
      {\efq{p} : \Gamma \vdash C ~|~ \Delta}
      {p : \Gamma \vdash \bot ~|~ \Delta}
\qquad
\infer[\top_I]
      {\Gamma \vdash () : \top ~|~ \Delta}
      {}
\]

\end{calculus}

\begin{clarifications}
There are two kinds of sequents: first $p: \Gamma \vdash A ~|~
\Delta$ with a distinguished formula on the right for typing the
so-called {\em unnamed} term $p$, second $c : \Gamma \vdash \Delta$ with no
distinguished formula for typing the so-called {\em named} term $c$.
The syntax of the underlying $\lambda\mu$-calculus 
is:
$$
\begin{array}{lll}
c & ::= & [\alpha]p\\
p,q  & ::= & a ~|~ \mu\alpha.c ~|~ (p,p) ~|~ \pi_i(p) ~|~ \iota_i(p) ~|~ \Case{p}{a_1}{p_1}{a_2}{p_2} \\
& ~|~ & \lambda a.p ~|~ p\,q 
~|~ \lambda x.p ~|~ p\,t 
~|~ (t,p) ~|~ \dest{p}{x}{a}{q}
 ~|~ () ~|~ \efq{p}\\
\end{array}
$$ The variables used for referring to assumptions in $\Gamma$ and to
conclusions in $\Delta$ range over distinct classes (denoted by Latin
and Greek letters respectively). In the rules
$\exists_E$ (resp. $\forall_I$), $y$ is assumed fresh in $\Gamma$,
$\Delta$ and $\exists x\, A$ (resp. $\forall x\, A$).
\end{clarifications}

\begin{history}
This system, defined in Parigot~\cite{Parigot92},
% and derived from free deduction~\cite{Parigot91},
highlights that classical logic in natural deduction can be obtained
from allowing several conclusions with contraction and weakening on
the right of the sequent, as in Gentzen's LK.  Additionally, the
system assigns to this form of classical reasoning a computational
content, based on the $\mu$ and bracket operator which provides with a
fine-grained decomposition
%~\cite{AriolaHerbelin07}
of the operators {\tt call-cc} (from Scheme/ML) or ${\cal C}$
(from~\cite{FelFriKohDub86}) that were known at this time to provide
computational content to classical logic~\cite{Griffin90}, as well as
a decomposition of Prawitz's classical elimination rule of
negation~\cite{Prawitz65}.

The original presentation~\cite{Parigot92} only contains implication
as well as first-order and second-order universal quantification \`a
la Curry (i.e. without leaving trace of the quantification in the
proof-term, what corresponds to computationally interpreting
quantification as an intersection type). The presentation above has
quantification \`a la Church (i.e. with an explicit trace in the proof
term) what makes the calculus compatible with several reduction
strategies such as both call-by-name or call-by-value (see
e.g.~\cite{HerbelinHdR}). Variants with multiplicative disjunctions
can be found in~\cite{Selinger01} or \cite{PymRitter01}, or
multiplicative conjunctions in \cite{HerbelinHdR}.

A standard variant originating in~\cite{deGroote94} uses only one kind
of sequents, interpreting $c : \Gamma \vdash \Delta$ as $c : \Gamma
\vdash \bot ~|~ \Delta$ (and hence removing $\bot_E$ and merging the
syntactic categories $c$ and $p$ into one). This variant is logically
equivalent to the original presentation (in the presence of $\bot$),
but not computationally equivalent~\cite{HerbelinSaurin10}.

\end{history}


\end{entry}
