


\newcommand{\Gthree}{\ensuremath{\mathbf{G3}}\xspace}

\calculusName{Kleene's Classical $\Gthree$}   
\calculusAcronym{\Gthree}     
\calculusLogic{Classical Logic}  
\calculusLogicOrder{First-Order}
\calculusType{Sequent Calculus}   
\calculusYear{1952}   
\calculusAuthor{Stephen Cole Kleene} 


\entryTitle{Kleene's Classical G3 System}     
\entryAuthor{Bj{\"o}rn Lellmann}
\entryAuthor{Valeria de Paiva}     




\etag{Two-Sided Sequents}
\etag{Symmetric Sequents}
\etag{Multi-succedent Sequents}

\maketitle



\begin{entry}{KleeneG3classical}  




\begin{calculus}

% Add the inference rules of your proof system here.
% The "proof.sty" and "bussproofs.sty" packages are available.
% If you need any other package, please contact the editor (bruno@logic.at)
\[
\infer[]{A, \Gamma \seq \Theta,A}{}
\]
\[
\begin{array}{c@{\qquad}c}
  \infer[\to\seq]{A \to B, \Gamma \seq \Theta}{A \to B, \Gamma \seq \Theta, A
  & B, A \to B, \Gamma \seq \Theta} &
  \infer[\seq\to]{\Gamma \seq \Theta, A \to B}{A,\Gamma \seq
                                      \Theta,A\to B, B}\medskip\\
  \infer[\lor\seq]{A \lor B, \Gamma\seq \Theta}{A, A \lor B, \Gamma
  \seq \Theta & B, A \lor B, \Gamma \seq \Theta}
  &
  \infer[\seq\lor_1]{\Gamma \seq \Theta, A \lor B}{\Gamma \seq \Theta, A \lor B,
  A} \quad   \infer[\seq\lor_2]{\Gamma \seq \Theta, A \lor B}{\Gamma \seq \Theta, A \lor B,
      B}\medskip\medskip\\
  \infer[\land\seq_1]{A \land B, \Gamma \seq \Theta}{A, A \land B,
  \Gamma \seq \Theta}
  \quad
  \infer[\land\seq_2]{A \land B, \Gamma \seq \Theta}{B, A \land B,
  \Gamma \seq \Theta}
  &
  \infer[\seq\land]{\Gamma \seq \Theta, A \land B}{\Gamma \seq \Theta,
    A \land B, A & \Gamma \seq \Theta, A \land B, B}\medskip\\
  \infer[\neg\seq]{\neg A, \Gamma\seq \Theta}{\neg A, \Gamma \seq
  \Theta, A}
  &
  \infer[\seq\neg]{\Gamma \seq \Theta, \neg A}{A, \Gamma \seq \Theta,
    \neg A}\medskip\\
  \infer[\forall\seq]{\forall x A(x),\Gamma\seq \Theta}{A(t), \forall
  x A(x), \Gamma \seq \Theta}
  &
  \infer[\seq\forall]{\Gamma \seq \Theta, \forall x A(x)}{\Gamma \seq
    \Theta, \forall x A(x), A(b)}\medskip\\
  \infer[\exists\seq]{\exists x A(x), \Gamma \seq \Theta}{A(b), \exists x
  A(x), \Gamma \seq \Theta}
  &
  \infer[\seq\exists]{\Gamma \seq \Theta, \exists x A(x)}{\Gamma \seq
    \Theta, \exists x A(x), A(t)}
\end{array}
\]
\begin{center}
The term $t$ is free for $x$ in $A(x)$.\\
The variable $b$ is free for $x$ in $A(x)$ and (unless $b$ is $x$)
does not occur in $\Gamma,\Theta,A(x)$.
\end{center}
\end{calculus}



\begin{clarifications}
  $A,B$ are formulae; $\Gamma,\Theta$ are finite (possibly empty)
  sequences of formulae; $x$ is a variable; $A(x)$ is a formula. In
  applications of the rules every sequent $\Gamma \seq \Theta$ can be
  replaced with a \emph{cognate} one, i.e., a sequent
  $\Gamma' \seq \Theta'$ such that the sets of formulae occurring in
  $\Gamma$ and $\Gamma'$ resp.\ $\Theta$ and $\Theta'$ are the same.
\end{clarifications}

\begin{history}
  Kleene's systems, introduced in his~1952 monograph, were the staple of
  generations of logicians, who learned about sequent calculus from
  his textbooks~\cite{Kleene:1952} and~\cite{Kleene:1967}.
\end{history}

\begin{technicalities}
  Based on Gentzen's sequent calculus $\LK$~\iref{GentzenLK} (called
  classical $\mathbf{G1}$ in~\cite{Kleene:1952}). Seems to be the
  first system (with~\iref{KleeneG3intuitionistic}) in which
  admissibility of contraction is obtained by copying the principal
  formulae into the premisses (accordingly, this is sometimes called
  \emph{Kleene's Method}). Used together with its single-conclusion
  version for intuitionistic logic~\iref{KleeneG3intuitionistic} to
  uniformly obtain decidability of propositional classical and
  intuitionistic logics via backwards proof search
  in~\cite{Kleene:1952}.
\end{technicalities}


\nocite{Kleene:1952}


\end{entry}
