
\calculusName{Superposition}   
\calculusAcronym{}     
\calculusLogic{Classical Logic}  
\calculusType{Superposition}   
\calculusYear{1990/1994}   
\calculusAuthor{Leo Bachmair} \calculusAuthor{Harald Ganzinger} 


\entryTitle{Superposition}     
\entryAuthor{Uwe Waldmann}     





\maketitle



\begin{entry}{Superposition}  




\begin{calculus}

% Add the inference rules of your proof system here.
% The "proof.sty" and "bussproofs.sty" packages are available.
% If you need any other package, please contact the editor (bruno@logic.at)

\newcommand{\myspace}{10pt}
\vspace{\myspace}
\[
\infer[\textit{Equality Resolution}]
{C\sigma}{C \lor \neg u \approx v\vphantom{[]}}
\]
\vspace{\myspace}
\[
\infer[\textit{Negative Superposition}]
{(D \lor C \lor \neg t[u'] \approx t')\sigma}{D \lor u \approx u'
& C \lor \neg t[v] \approx t'}
\]
\vspace{\myspace}
\[
\infer[\textit{Positive Superposition}]
{(D \lor C \lor t[u'] \approx t')\sigma}{D \lor u \approx u'
& C \lor t[v] \approx t'}
\]
\vspace{\myspace}

plus either
\[
\infer[\textit{Merging Paramodulation}]
{(D \lor C \lor s \approx s' \lor t \approx t'[u'])\sigma}{D \lor u \approx u'
& C \lor s \approx s' \lor t \approx t'[v]}
\]
\vspace{\myspace}
\[
\infer[\textit{Ordered Factoring}]
{(C \lor s \approx u)\sigma}{C \lor s \approx u \lor t \approx v}
\]
\vspace{\myspace}

or
\vspace{\myspace}
\[
\infer[\textit{Equality Factoring}]
{(C \lor \neg u' \approx v' \lor v \approx v')\sigma}{C \lor v \approx v' \lor u \approx u'\vphantom{[]}}
\]
\vspace{\myspace}

$C,D$ are (possibly empty) equational clauses,
$s,s',t,t',u,u',v,v'$ are terms,
$u$ and $v$ (and, for \textit{Ordered Factoring}
and \textit{Merging Paramodulation}, $s$ and $t$)
are unifiable with most general unifier~$\sigma$.
In all binary inferences, $v$ is not a variable.

\vspace{\myspace}

Except for the last but one literal in
\textit{Equality Factoring} and \textit{Merging Paramodulation} inferences,
every literal involved in some inference
is maximal in the respective premise
(strictly maximal, if the literal is positive and the inference is binary).
In every literal involved in some inference
(except \textit{Equality Resolution}),
the lhs is strictly maximal.
Optionally,
ordering restrictions can be overridden by \emph{selection functions}.

\vspace{\myspace}

For simplicity, it is assumed that the equality predicate $\approx$
is the only predicate symbol in the signature.
Non-equational atoms $P(t_1,\dots,t_n)$ can be encoded as
equations $P(t_1,\dots,t_n) \approx \mathit{true}$.

% \bigskip
% 
% \textbf{Saturation}
% \[
% \infer
% {N \cup \{C\}}{N & C \textrm{~is the conclusion of a Resolution inference
% from clauses in~} N}
% \]
% \mbox{}\quad $N$ is a finite set of clauses,
% $C$ is a clause.
\end{calculus}



\begin{clarifications}
Superposition is a refutational saturation calculus for
first-order clauses (disjunctions of possibly negated atoms)
with equality.
The inference rules are supplemented by a redundancy criterion
that permits to delete clauses that are unnecessary for
deriving a contradiction during the saturation, see \iref{SaturationWithRed}.
\end{clarifications}

\begin{history}
The superposition
calculus~\cite{BachmairGanzinger1990CTRS,BachmairGanzinger1994JLC}
by Bachmair and Ganzinger
refines the paramodulation calculus~\iref{Paramodulation}.
It uses a syntactic ordering on terms and
literals to restrict the paramodulation inference rules
in such a way that only (strictly) maximal sides of (strictly) maximal
literals participate in inferences,
thus combining the restrictions of
ordered resolution~\iref{OrderedRes} and
unfailing completion~\iref{Completion}.
In order to preserve refutational completeness, one new
inference rule must be added -- either
the \textit{Merging Paramodulation} rule~\cite{BachmairGanzinger1990CTRS}
or the \textit{Equality Factoring} rule originally due to Nieuwenhuis
(which then subsumes \textit{Ordered Factoring}).

The superposition calculus is the basis of most current automated
theorem provers for full first-order logic with equality,
such as E, SPASS, or Vampire.
The calculus and the \emph{model construction} technique used
to prove its refutational completeness have been a prototype
for numerous refinements, such as
constraint superposition~\iref{ConstraintSup},
theory superposition~\iref{CancellativeSup},
% ordered chaining~\irefmissing{OrderedChaining},
or hierarchic superposition~\iref{HierarchicSup}.

\end{history}

\begin{technicalities}
The superposition calculus is refutationally complete for
first-order logic with equality.
For certain fragments of first-order logic with equality,
there exist strategies that guarantee termination of the calculus,
turning superposition into a decision procedure for these fragments.
\end{technicalities}



\end{entry}
