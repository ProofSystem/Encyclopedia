
\calculusName{Multi type Sequent Calculi}
\calculusAcronym{MtSC}
\calculusLogic{Classical Logic}
\calculusLogicOrder{Modal Propositional Logics}
\calculusType{Multi type Sequent Calculus}
\calculusYear{2014, 2016}
\calculusAuthor{Giuseppe Greco} \calculusAuthor{Alessandra Palmigiano} \calculusAuthor{Alexander Kurz} \calculusAuthor{Sabine Frittella} \calculusAuthor{Vlasta Sikimi\'{c}} 

\entryTitle{Multi type Sequent Calculi MtSC}
\entryAuthor{Giuseppe Greco}

\etag{Two-Sided Sequents}
\etag{Multi-type Cedents}
\etag{Multi-Premise Cedent}
\etag{Multi-Conclusion Cedent}

%%%Definition and abbreviations
\def\fCenter{{\mbox{$\ \vdash\ $}}}
\def\ffCenter{{\mbox{$\overline{\vdash}$}}}
\def\fcenter{{\mbox{\ }}}
\EnableBpAbbreviations

\def\fns{\footnotesize}
\def\mc{\multicolumn}

\def\aol{\rule[0.5865ex]{1.38ex}{0.1ex}}
\def\pdra{\mbox{$\,>\mkern-8mu\raisebox{-0.065ex}{\aol}\,$}}
\def\pdla{\mbox{\rotatebox[origin=c]{180}{$\,>\mkern-8mu\raisebox{-0.065ex}{\aol}\,$}}}


\maketitle

\begin{entry}{MtSC}

\begin{calculus}
{\small
%\begin{enumerate}			
Identity and Cut rules:	
\vspace{0.2cm}
\begin{center}
\AXC{\fns LId}
\noLine
\UIC{$p^{\texttt{i}} \fCenter X^{\texttt{i}}[p^{\texttt{i}}]^{suc}$}
\DisplayProof 
\AXC{\fns RId}
\noLine
\UIC{$X^{\texttt{i}}[p^{\texttt{i}}]^{pre} \fCenter p^{\texttt{i}}$}
\DisplayProof 
\qquad
\AX$Z^{\texttt{i}} \fCenter A^{\texttt{i}}$
\AX$(X \fCenter Y)[A^{\texttt{i}}]^{pre}$
\LeftLabel{\fns LCut}
\BI$(X \fCenter Y)[Z/A]^{pre}$
\DisplayProof 
\ 
\AX$(X \fCenter Y)[A^{\texttt{i}}]^{suc}$
\AX$A^{\texttt{i}} \fCenter Z^{\texttt{i}}$
\RightLabel{\fns RCut}
\BI$(X \fCenter Y)[Z/A]^{suc}$						
\DisplayProof
\end{center}
\vspace{0.2cm}		
Introduction rules: for any $f\in\mathcal{F}$ and $g\in\mathcal{G}$,
\begin{center}
\AxiomC{$H (A_1,\ldots, A_{n_f})\vdash X$}
\LeftLabel{\fns$f_L$}
\UnaryInfC{$f(A_1,\ldots, A_{n_f})\vdash X$}
\DisplayProof
\qquad
\AxiomC{$\Big(X_i \vdash A_i \quad A_j \vdash X_j \mid 1\leq i, j\leq n_f, \varepsilon_f(i) = 1\mbox{ and } \varepsilon_f(j) = \partial \Big)$}
\RightLabel{\fns$f_R$}
\UnaryInfC{$H(X_1,\ldots, X_{n_f})\vdash f(A_1,\ldots, A_{n})$}
\DisplayProof
\end{center}

%\vspace{-0.4cm}

\begin{center}
\AxiomC{$\Big( A_i \vdash X_i \quad X_j \vdash A_j \,\mid\, 1\leq i, j\leq n_g, \varepsilon_g(i) = 1\mbox{ and } \varepsilon_g(j) = \partial \Big)$}
\LeftLabel{\fns$g_L$}
\UnaryInfC{$g(A_1,\ldots, A_{n_g}) \vdash K (X_1,\ldots, X_{n})$}
\DisplayProof						
\qquad
\AxiomC{$X \vdash K(A_1,\ldots, A_{n_g})$}
\RightLabel{\fns$g_R$}
\UnaryInfC{$X \vdash g(A_1,\ldots, A_{n_g})$}
\DisplayProof
\end{center}
				
Display postulates: for any $f\in \mathcal{F}$ and $g\in \mathcal{G}$, for some/any $1\leq i\leq n_f$ and $1\leq h\leq n_g$,
\begin{center}
\AXC{$H\, (X_1, \ldots, X_i, \ldots, X_{n_f}) \fCenter Y$}
\doubleLine
\LeftLabel{$(\varepsilon_f(i) = 1)$}
\UIC{$X_i \fCenter H_i\, (X_1, \ldots, Y, \ldots, X_{n_f})$}
\DisplayProof
\qquad
\AXC{$H\, (X_1, \ldots, X_i, \ldots, X_{n_f}) \fCenter Y$}
\doubleLine
\RightLabel{$(\varepsilon_f(i) = \partial)$}
\UIC{$H_i\, (X_1, \ldots, Y, \ldots, X_{n_f}) \fCenter X_i $}
\DisplayProof
\end{center}

%\vspace{-0.4cm}

\begin{center}						
\AXC{$Y \fCenter K\, (X_1 \ldots, X_h, \ldots X_{n_g})$}
\doubleLine
\LeftLabel{$(\varepsilon_g(h) = 1)$}
\UIC{$K_h\, (X_1, \ldots, Y, \ldots, X_{n_g}) \fCenter X_h$}
\DisplayProof 
\qquad
\AXC{$Y \fCenter K\, (X_1, \ldots, X_h, \ldots, X_{n_g})$}
\doubleLine
\RightLabel{$(\varepsilon_g(h) = \partial)$}
\UIC{$ X_h\fCenter K_h\, (X_1, \ldots, Y, \ldots, X_{n_g})$}
\DisplayProof 
\end{center}
		
%\end{enumerate}
 }
\end{calculus}

\begin{clarifications}
The language $\mathcal{L}_\mathrm{MT}(\mathcal{F}, \mathcal{G})$ of any MtSC consists of  {\em logical} (or {\em operational})  and {\em structural terms} in each (pairwise disjoint) type $\mathsf{T}_1,\ldots,\mathsf{T}_n$. 

The set  of logical terms takes as parameters: 1) denumerable (possibly empty) sets of atomic terms $\mathsf{At}(\mathsf{T}_i)$ for {\em some} $1\leq i\leq m$, elements of which are denoted $p^{\texttt{i}}$, possibly with indexes; 2) disjoint sets of connectives $\mathcal{F}$ and $\mathcal{G}$. Each $f\in \mathcal{F}$ and $g\in \mathcal{G}$ has arity $n_f\in \mathbb{N}$ (resp.\ $n_g\in \mathbb{N}$), and is associated with some {\em functional type} $f: \mathsf{T}_{i_1}\times\cdots \times \mathsf{T}_{i_{n_f}}\to \mathsf{T}_{f}$ (resp.~$g: \mathsf{T}_{i_1}\times\cdots \times \mathsf{T}_{i_{n_g}}\to \mathsf{T}_{g}$) and with some {\em order-type} $\varepsilon_f$ over $n_f$ (resp.~$\varepsilon_g$ over $n_g$), where an {\em order-type} over $m\in \mathbb{N}$ is an $m$-tuple $\varepsilon\in \{1, \partial\}^m$. 
The functional type of each connective uniquely determines which type $\mathsf{T}_i$ is taken as argument in each coordinate  and the type of the output of the connective. The order-type   of each connective $f$ or $g$ determines which of its coordinates is monotone ($\varepsilon_f(i) = 1$ and $\varepsilon_g(i) = 1$ respectively) or antitone ($\varepsilon_f(i) = \partial$ and $\varepsilon_g(i) = \partial$ respectively). If $\mathsf{T}_{i_1} = \cdots  =  \mathsf{T}_{i_{n_f}} =  \mathsf{T}_{f} = \mathsf{T}$ (resp.\ $\mathsf{T}_{i_1} = \cdots  =  \mathsf{T}_{i_{n_g}} =  \mathsf{T}_{g} = \mathsf{T}$) then $f$ (resp.\ $g$) is {\em homogeneous} of type $\mathsf{T}$ ($f\in \mathcal{F}_{\mathsf{T}}$ and $g\in \mathcal{G}_{\mathsf{T}}$); otherwise, $f$ (resp.\ $g$) is {\em heterogeneous} ($f\in \mathcal{F}_{\mathrm{MT}}$ and $g\in \mathcal{G}_{\mathrm{MT}}$). 

The structural terms are built by means of structural connectives, taking logical terms as atomic structures. The set of structural connectives includes the structural counterpart $H$ (resp.~$K$) of each $f\in \mathcal{F}$ ($g\in \mathcal{G}$), and possibly connectives $H_i$ (resp.~$K_j$) corresponding to the residual of $H$ (resp.~$K$) in the $i$th (resp.~$j$th) coordinate, for $1\leq i\leq n_f$ (resp.~$1\leq j\leq n_g$).

Summing up, for each $1\leq i\leq n$, the logical and structural terms of type $\mathsf{T}_i$ are generated by simultaneous induction as follows (where $\overline{A}$ and $\overline{X}$ are vectors of formulas and structures, respectively of suitable length): 
\begin{center}
$A^{\texttt{i}} ::= (p^{\texttt{i}} \mid)\, f(\overline{A}) \mid g(\overline{A})\,$ \qquad and \qquad $X^{\texttt{i}} ::= (A^{\texttt{i}} \mid)\, H(\overline{X}) \mid K(\overline{X}) \, (\mid H'_i(\overline{X}) \mid K'_j(\overline{X})$.
\end{center}

In the Identity rules, the notation $X[p^{\texttt{i}}]^{pre}$ (resp.~$X[p^{\texttt{i}}]^{suc}$) indicates that $p^{\texttt{i}}$ occurs in precedent (resp.~succedent) position in the structure $X$. The principal formulas $p^{\texttt{i}}$ occurs in display either in precedent or in succedent position. In the Cut rules, the notation $(X \fCenter Y)[A^{\texttt{i}}]^{pre}$ (resp.~$(X \fCenter Y)[A^{\texttt{i}}]^{suc}$) indicates that $A^{\texttt{i}}$ occurs in precedent (resp.~succedent) position in the sequent $X \fCenter Y$.  For  each type $\texttt{T}_i$, the calculus defines a deduction relation $\vdash^{\texttt{i}}$ (the superscript is usually dropped unambiguously). 

Introduction rules for $f\in\mathcal{F}$ and $g\in\mathcal{G}$ encode the information relative to their arity, functional type and order-type, and cover also zero-ary connectives (e.g.~$\top\in \mathcal{F}$ and $\bot\in \mathcal{G}$) and classical conjunction and disjunction ($\wedge\in \mathcal{F}$ and $\vee\in \mathcal{G}$). Notice that the auxiliary formulas in the premises and the principal formula in the conclusion of each rule occur in isolation (this feature is sometimes referred to as the {\em visibility property}). 

If each operator has an adjoint or a residual in each coordinate, then the calculus includes all display postulates, and hence enjoys the so-called {\em display property}, which makes it a multi-type {\em display} calculus. Notice that the display property implies the visibility property but not conversely. If each rule is {\em closed under uniform substitution within each type}, the multi-type (display) calculus is called {\em proper}. \end{clarifications}


\begin{history}
Belnap-style cut elimination for multi-type sequent calculi was proved in \cite{FrittellaGrecoKurzPalmigianoSikimic2014}, where the principal atomic formulas in the Identity rules are required to be displayable. The restricted meta-theorem for proper multi-type display calculi is stated in \cite[Sec.~A]{GrecoPalmigiano} (cf.~Thm.~A2). Display calculi are introduced by Belnap \cite{Belnap1982}, and proper display calculi by Wansing \cite{Wansing1998}. The visibility property is identified by Sambin et al.~(cf.~\cite{SambinBattilottiFaggian2014}). Properly displayable logics are characterized in a purely proof-theoretic way in \cite{CiabattoniRamanayake2016} and in an algebraic way in \cite{GrecoMaPalmigianoTzimoulisZhao2016}, building on the theory of unified correspondence \cite{ConradieGhilardiPalmigiano2014}.

\end{history}

\begin{technicalities}
(Proper) Multi-type calculi are endowed by design with a natural algebraic semantics, thanks to which soundness, completeness and conservativity can be proved in a uniform way for the basic calculi reported on above, and are preserved when extending the calculi with {\em analytic} structural rules (cf.~\cite[Definition 4]{GrecoMaPalmigianoTzimoulisZhao2016}). 
Cut elimination and subformula property for MtSC can be inferred from a meta-theorem, following the strategy introduced by Belnap for display calculi. These properties are preserved when adding or removing analytic structural rules. 

Logics which are not properly displayable have been presented as proper multi-type display calculi, see for instance semi De Morgan \iref{MtDSDM}, Inquisitive logic \iref{MtDInqL}, Lattice Logic \iref{MtDLatL}, Dynamic Epistemic Logic \iref{MtDDEL}, Linear Logic \irefmissing{MtDLL}, Propositional Dynamic Logic \irefmissing{MtDPDL}. 


\end{technicalities}

\end{entry}
