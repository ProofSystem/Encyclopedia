
% If the calculus has an acronym, define it.
% (e.g. \newcommand{\LK}{\ensuremath{\mathbf{LK}}\xspace})
\calculusName{Resolution for Computational Tree Logic}         % The name of the calculus
\calculusAcronym{}          % The acronym if defined above, or empty otherwise. 
\calculusLogic{Computational Tree Logic}        % Specify the logic (e.g. Classical Logic, Intuitionistic Logic, ...) for which this calculus is intended.
\calculusLogicOrder{Propositional}   % Specify the order of the logic (e.g. Propositional, Quantified Propositional, First-Order, Higher-Order, ...).
\calculusType{Resolution}         % Specify the calculus type (e.g. Tableau, Sequent Calculus, Hyper-Sequent Calculus, Natural Deduction, ...)
\calculusYear{2009}         % The year when the calculus was published.

\calculusAuthor{Lan Zhang}       % The name(s) of the author(s) of the calculus.
\calculusAuthor{Ullrich Hustadt}
\calculusAuthor{Clare Dixon}




\entryTitle{Resolution for Computation Tree Logic (CTL)}     % Title of the entry (usually coincides with the name of the calculus).
\entryAuthor{Clare Dixon}     
\entryAuthor{Ullrich Hustadt}



% The encyclopedia's peer-reviewing policy is described here: 
% http://proofsystem.github.io/Encyclopedia/
%
% Reviewers of this entry will be acknowledged in the following lines:
% \entryReviewer{Reviewer 1's name}
% \entryReviewer{Reviewer 2's name}
% \entryReviewer{Reviewer 3's name}
%
% The lines above will be filled by the coordinators. 
% If you would like to indicate people 
% who could review your entry, contact the coordinators.


% If you wish, use tags to give any other information 
% that might be helpful for classifying and grouping this entry:
% \tag{Branching-Time Temporal Logic}
% e.g. \tag{Multiset Cedents}
% e.g. \tag{List Cedents}
% You are free to invent your own tags. 
% The Encyclopedia's coordinator will take care of 
% merging semantically similar tags in the future.


\maketitle


% If your files are called "MyProofSystem.tex" and "MyProofSystem.bib", 
% then you should write "\begin{entry}{MyProofSystem}" in the line below
\begin{entry}{CTLClausalResolution}  

% Define here any newcommands you may need:
% e.g. \newcommand{\necessarily}{\Box}
% e.g. \newcommand{\possibly}{\Diamond}

%\newcommand{\Next}[0]{\mathop{\raisebox{0.2ex}{\scalebox{0.9}{\ensuremath{\ocircle}}}}}
\newcommand{\Always}[0]{\mathop{\scalebox{1.25}{\ensuremath{\Box}}}}
\newcommand{\Sometimes}[0]{\mathop{\Diamond}}
\newcommand{\Eventually}[0]{\Diamond}

\newcommand{\Next}{\!\raisebox{0.1ex}{ %possibly add a little space before
                        \mbox{\unitlength=0.62ex
                        \begin{picture}(2,2)
                        \linethickness{0.1ex}
                        \put(1,1){\circle{2}} % Draws circle with
                        \end{picture}}}       % diameter 2 at centre 1,1
                        \,}

\newcommand{\TAlways}{\raisebox{-.2ex}{
                           \mbox{\unitlength=0.9ex
                           \begin{picture}(2,2)
                           \linethickness{0.06ex}
                           \put(0,0){\line(1,0){2}}
                           \put(0,2){\line(1,0){2}}
                           \put(0,0){\line(0,1){2}}
                           \put(2,0){\line(0,1){2}}
                           \end{picture}}}
                      \,}
\newcommand{\TSometime}{\hbox{$\,\Diamond \,$}}

\newcommand{\lstart}{{\mathbf{start}}}
\newcommand{\ltrue}{{\mathbf{true}}}
\newcommand{\lfalse}{{\mathbf{false}}}

\def\unless{\hbox{$\,\mathcal W \,$}}

\def\allpaths{\ensuremath{\mathbf A}}

\def\somepath{\ensuremath{\mathbf{E}}}

\def\somepathind{\ensuremath{\mathbf{E}_{\langle ind \rangle}}}

\def\eitherpath{\ensuremath{\mathbf{P}}}


\begin{calculus}

% Add the inference rules of your proof system here.
% The "proof.sty" and "bussproofs.sty" packages are available.
% If you need any other package, please contact the coordinator (Bruno
% Woltzenlogel Paleo <bruno.wp@gmail.com>) 

%

\newcommand{\myspace}{-5pt}

\begin{small}
%\vspace{\myspace}
\[
\infer %[\textit{Initial Resolution}]
{\allpaths \Always (\lstart \rightarrow (C \lor D))}
{\allpaths\Always( \lstart \rightarrow (C \lor l))  &
\allpaths\Always(\lstart \rightarrow (D \lor \neg l))} 
%
\qquad
%
\infer %[\textit{Step Resolution}]
{\allpaths\Always(P \land Q  \rightarrow  \somepathind 
  \Next (C \lor D))}
{\allpaths\Always(P \rightarrow \somepathind\Next (C \lor l))  &
\allpaths\Always (Q \rightarrow  \somepathind\Next (D \lor \neg l))} 
\]
%
\vspace{\myspace}
%
\[
\infer %[\textit{Initial Resolution}]
{\allpaths\Always (\lstart \rightarrow (C \lor D))}
{\allpaths\Always( \ltrue \rightarrow (C \lor l))  &
\allpaths\Always(\lstart \rightarrow (D \lor \neg l))} 
%
\qquad
%
\infer %[\textit{Step Resolution}]
{\allpaths\Always(P \land Q  \rightarrow  \allpaths \Next (C \lor D))}
{\allpaths\Always(P \rightarrow \allpaths\Next (C \lor l))  &
\allpaths\Always (Q \rightarrow  \allpaths\Next (D \lor \neg l))} 
\]
%
\vspace{\myspace}
\[
%
 \infer %[\textit{Initial Resolution}]
 {\allpaths\Always (\ltrue \rightarrow (C \lor D))}
 {\allpaths\Always( \ltrue \rightarrow (C \lor l))  &
 \allpaths\Always(\ltrue \rightarrow (D \lor \neg l))} 
%
\qquad
%
\infer %[\textit{Step Resolution}]
{\allpaths\Always(P \land Q  \rightarrow  \somepathind \Next (C \lor D))}
{\allpaths\Always(P \rightarrow \allpaths\Next (C \lor l))  &
\allpaths\Always (Q \rightarrow  \somepathind\Next (D \lor \neg l))} 
%
\]
%
\vspace{\myspace}
%
\[
\infer %[\textit{Step Resolution}]
{\allpaths\Always(P \rightarrow  \allpaths \Next (C \lor D))}
{\allpaths\Always(\ltrue \rightarrow  (C \lor l))  &
\allpaths\Always (P \rightarrow  \allpaths \Next (D \lor \neg l))} 
%
\qquad
%
\infer %[\textit{Step Resolution}]
{\allpaths\Always(P \rightarrow  \somepathind \Next (C \lor D))}
{\allpaths\Always(\ltrue \rightarrow  (C \lor l))  &
\allpaths\Always (P \rightarrow  \somepathind \Next (D \lor \neg l))} 
\]
\vspace{\myspace}
\[
\infer[\textit{(ERES1)} ]
{\textstyle \allpaths\Always (Q \rightarrow  \allpaths (\neg P^\dagger  \unless l))}
{\allpaths\Always (P^\dagger \rightarrow  \somepath \Next \somepath \Always l) &
\allpaths\Always  (D   \rightarrow  \allpaths \Eventually \neg l)}
%
\quad
%
\infer[\textit{(ERES2)} ]
{\textstyle \allpaths\Always  (Q \rightarrow  \somepathind (\neg
  P^\dagger  \unless \neg l))}
{\allpaths\Always  (P^\dagger  \rightarrow  \somepathind \Next \somepathind \Always l) &
\allpaths \Always  (Q   \rightarrow  \somepathind \Eventually \neg l)}
\]
%
\vspace{\myspace}
%
\[
\allpaths\Always(P  \rightarrow  \allpaths \Next \lfalse)
\Rightarrow
\allpaths\Always(\ltrue  \rightarrow  \neg P)
%
\qquad
%
\allpaths\Always(P  \rightarrow  \somepathind \Next \lfalse)
\Rightarrow
\allpaths\Always(\ltrue  \rightarrow  \neg P)
\]
%
where $l$ is a literal, 
$C$ and $D$ are (possibly empty) disjunctions of literals,
$P$ and $Q$ are (possibly empty) conjunctions of literals,
$P^\dagger \rightarrow  \varphi$, 
$\varphi\in\{\somepath \Next \somepath \Always l,
             \somepathind \Next \somepathind \Always l\}$, 
represents a set of clauses that together imply $\varphi$. 
%$\{\Always(P^i_j\rightarrow\star^{i_j} C^i_j\}$
%with $\star^{i_j}$ either being empty or being an operator
%in $\{\Always\Next\}\cup\{\somepathind\Next|\mathit{ind}\in\mathsf{Ind}\}$.

Derivations terminate if either $\allpaths\Always(\lstart\rightarrow\lfalse)$ or
$\allpaths\Always(\ltrue\rightarrow\lfalse)$  is derived (unsatisfiable) or no new
clause can be derived (satisfiable).    
\end{small}

\end{calculus}

% The following sections ("clarifications", "history", 
% "technicalities") are optional. If you use them, 
% be very concise and objective. Nevertheless, do write full sentences. 
% Try to have at most one paragraph per section, because line breaks 
% do not look nice in a short entry.

 \begin{clarifications}
% ToDo: write here short remarks that may help the reader to understand 
% the inference rules of the proof system.
% but has less temporal resolution rules, incorporates an ordering and selection function and has been
% implemented. 
This calculus is for propositional computation tree
logic~\cite{Clarke+Emerson@LP1981}.
Formulae are first translated, in a satisfiability preserving way,
into the following normal form: 
$\allpaths \Always (\lstart \rightarrow A)$, an \emph{initial clause},
$\allpaths  \Always (\ltrue \rightarrow A)$, a \emph{global clause},
$\allpaths \Always (P \rightarrow  \eitherpath \Next C)$,  
a $P$-\emph{step clause}, and
$\allpaths  \Always  (P   \rightarrow  \eitherpath \Eventually l)$,
a $\eitherpath$-\emph{eventuality clause} 
where $\eitherpath$ represents  either $\allpaths$ or $\somepathind$.
%
Indices $\mathit{ind}$ attached to existential path operators 
$\somepathind$ are elements of an arbitrary enumerable set
$\mathit{Ind}$; they are introduced during the transformation to normal
form to represent a particular path and are used to
preserve satisfiability.
% 
The logical constant $\lstart$ only holds in the first moment in time. 
%
%Several global or step clauses are needed to be found to satisfy the
%first premise of the ERES rules. 
%
In the rules, $\ltrue$ stands for an empty conjunction of literals,  
$\lfalse$ stands for an empty disjunction of literals.
The resolvents of the ERES rules need transformation
into the normal form. 
%
\end{clarifications}

\begin{history}
% ToDo: write here short historical remarks about this proof system,
% especially if they relate to other proof systems. 
% Use "\iref{OtherProofSystem}" to refer to another proof system 
% in the Encyclopedia (where "OtherProofSystem" is its ID). 
% Use "\irefmissing{SuggestedIDForOtherProofSystem}" to refer to 
% another proof system that is not yet available in the encyclopedia.
The calculus was first presented
in~\cite{Zhang+Hustadt+Dixon@CADE2009}.
It removes two redundant rules from an earlier resolution calculus for
CTL~\cite{BF99:jetai}.
Full details, including a normal form transformation and algorithms
for the application of ERES1 and ERES2, are provided in~\cite{Zhang+Hustadt+Dixon@ToCL2014}. 
The calculus has been implemented in the prover
CTL-RP~\cite{Zhang+Hustadt+Dixon@AIC2010}. 
\end{history}

\begin{technicalities}
% ToDo: write here remarks about soundness, completeness,
% decidability...
Soundness, completeness and termination are shown in~\cite{Zhang+Hustadt+Dixon@ToCL2014}.
\end{technicalities}


% General Instructions:
% =====================

% The preferred length of an entry is 1 page. 
% Do the best you can to fit your proof system in one page.
%
% If you are finding it hard to fit what you want in one page, remember:
%
%   * Your entry needs to be neither self-contained nor fully understandable
%     (the interested reader may consult the cited full paper for details)
%
%   * If you are describing several proof systems in one entry, 
%     consider splitting your entry.
%
%   * You may reduce the size of your entry by ommitting inference rules
%     that are already described in other entries.
%
%   * Cite parsimoniously (see detailed citation instructions below).
%
% 
% If you do not manage to fit everything in one page, 
% it is acceptable for an entry to have 2 pages.
%
% For aesthetic reasons, it is preferable for an entry to have
% 1 full page or 2 full pages, in order to avoid unused blank space.



% Citation Instructions:
% ======================

% Please cite the original paper where the proof system was defined.
% To do so, you may use the \cite command within 
% one of the optional environments above,
% or use the \nocite command otherwise.

% You may also cite a modern paper or book where the 
% proof system is explained in greater depth or clarity.
% Cite parsimoniously.

% Do not cite related work. Instead, use the "\iref" or "\irefmissing" 
% commands to make an internal reference to another entry, 
% as explained within the "history" environment above.

% You do not need to create the "References" section yourself. 
% This is done automatically.


% Remove all instruction comments before submitting.


% Leave an empty line above "\end{entry}".

\end{entry}
