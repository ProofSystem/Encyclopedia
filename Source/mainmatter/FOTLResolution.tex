\calculusName{Resolution for First-Order Temporal Logic}
\calculusAcronym{}
\calculusLogic{Monodic First Order Temporal Logic}
\calculusLogicOrder{First-Order}
\calculusType{Resolution}
\calculusYear{2003}

\calculusAuthor{Boris Konev}
\calculusAuthor{Anatoli Degtyarev}
\calculusAuthor{Michael Fisher}
\calculusAuthor{Ullrich Hustadt}
%\calculusAuthor{Clare Dixon}

\entryTitle{Resolution for Monodic First-Order Temporal Logic}
\entryAuthor{Clare Dixon}  
\entryAuthor{Michael Fisher}   
\entryAuthor{Ullrich Hustadt}
\entryAuthor{Boris Konev}
\implementation{TeMP}{http://cgi.csc.liv.ac.uk/~konev/software/TeMP/}


\maketitle

\begin{entry}{FOTLResolution}  

%\newcommand{\Next}[0]{\mathop{\raisebox{0.2ex}{\scalebox{0.9}{\ensuremath{\circle}}}}}
\newcommand{\Always}[0]{\mathop{\scalebox{1.25}{\ensuremath{\Box}}}}
\newcommand{\Sometimes}[0]{\Diamond}
\newcommand{\Eventually}[0]{\Diamond}

\newcommand{\Next}{\!\raisebox{0.1ex}{ %possibly add a little space before
                        \mbox{\unitlength=0.62ex
                        \begin{picture}(2,2)
                        \linethickness{0.1ex}
                        \put(1,1){\circle{2}} % Draws circle with
                        \end{picture}}}       % diameter 2 at centre 1,1
                        \,}

\newcommand{\TAlways}{\raisebox{-.2ex}{
                           \mbox{\unitlength=0.9ex
                           \begin{picture}(2,2)
                           \linethickness{0.06ex}
                           \put(0,0){\line(1,0){2}}
                           \put(0,2){\line(1,0){2}}
                           \put(0,0){\line(0,1){2}}
                           \put(2,0){\line(0,1){2}}
                           \end{picture}}}
                       \,}
\newcommand{\TSometime}[0]{\hbox{$\,\Diamond \,$}}


% \newcommand{\ltrue}{{\bf true}}
% \newcommand{\lfalse}{{\bf false}}


%\newcommannd{\mC}[1]{\mathcal{#1}}

\begin{calculus}


\newcommand{\myspace}{6pt}


%\vspace{\myspace}
\[
%\begin{array}{cc}
\infer[\textit{Step Resolution} \qquad 
\begin{array}{@{}l@{}}
\text{where }\mathcal{A}\rightarrow\Next\mathcal{B}\text{ is a merged
  derived step clause}\\
\text{and }\mathcal{U} \cup \{\mathcal{B}\} \models \bot
\end{array}]
{\neg \mathcal{A} }
{ \mathcal{A} \rightarrow \Next \mathcal{B}}
%&
% \end{array}
\]
%
\[
 \infer[\begin{tabular}{@{}l@{}}
 \textit{Eventuality}\\
 \textit{Resolution}
 \end{tabular}]
 {\forall x \bigwedge_{i=1}^{n}(\neg \mathcal{A}_i\lor\neg A_i(x))}
 {\forall x ((\mathcal{A}_1\land A_1(x))\rightarrow
   \Next(\mathcal{B}_1\land B_1(x)))  &
\cdots  &
\forall x ((\mathcal{A}_n\land A_n(x)) \rightarrow 
   \Next(\mathcal{B}_n\land B_n(x)))  &
\Eventually L(x)}\vspace*{-1ex}
 \]
where 
$\forall x ((\mathcal{A}_i \land A_1(x)) \rightarrow \Next
(\mathcal{B}_i\land B_i(x)))$, $1\leq i\leq n$,
are full merged step clauses such that for all $i$, $1\leq i\leq n$,
$\forall x ((\mathcal{U} \land \mathcal{B}_i \land B_i(x)) \rightarrow
\neg L(x))$ and
$\forall x ((\mathcal{U} \land \mathcal{B}_i \land B_i(x)) \rightarrow
\bigvee_{j=1}^{n} (\mathcal{A}_j\land A_j(x))$ are valid.

\[
 \infer[\begin{tabular}{@{}l@{}}
  \textit{Ground Eventuality}\\
  \textit{Resolution}
  \end{tabular}]
{\bigwedge_{i=1}^{n}\neg \mathcal{A}_i}
{\mathcal{A}_1\rightarrow \Next\mathcal{B}_1  &
\cdots  &
\mathcal{A}_n \rightarrow \Next\mathcal{B}_n  &
\Eventually L}\vspace*{-1ex}
 \]
where $\Eventually L\in\mathcal{E}$, for a proposition or ground
literal $L$, and
$\mathcal{A}_i \rightarrow \Next B_i$, $1\leq i\leq n$,
are merged derived step clauses such that for all $i$, $1\leq i\leq n$,
$(\mathcal{U} \land \mathcal{B}_i) \rightarrow\neg L$ and
$(\mathcal{U} \land \mathcal{B}_i) \rightarrow \bigvee_{j=1}^{n} \mathcal{A}_j$ are valid.\vspace*{3pt}

Derivations terminate if 
$\mathcal{U} \cup \mathcal{I} \models \bot$,
$\mathcal{U}  \models \forall x \neg L(x)$ where 
$\Eventually L(x) \in \mathcal{E}$, or 
$\mathcal{U}  \models \neg L$ where 
$\Eventually L \in \mathcal{E}$ for a proposition or ground literal
$L$.
% or no new clause can be derived (satisfiable).    
\end{calculus}

\begin{clarifications}
  To determine the satisfiability of a formula of monodic first-order
  discrete linear time temporal logic over expanding domains, the
  formula is transformed into a \emph{Monodic Temporal Problem
  in Divided Separated Normal Form}
  $\mathcal{P} = \langle \mathcal{U, I, S, E} \rangle$ 
  where $\mathcal{U}$
  % (the universal part) 
  and $\mathcal{I}$ 
  % (the step part) 
  are finite sets of arbitrary closed first-order formulae;
  $\mathcal{S}$ 
  % (the step part) 
  is a finite set of formulae of the form
  $p\rightarrow\Next q$ (\emph{original ground step clause}), 
  where $p$ and $q$ are propositions, or
  $P(x)\rightarrow\Next Q(x)$ (\emph{original non-ground step clauses}), 
  where $P$ and $Q$ are unary predicates and $x$ is variable; and
  $\mathcal{E}$  
  % (the eventuality part) 
  is a set of \emph{eventuality clauses}
  of the form $\Eventually l$, where $l$ is a propositional literal,
  $\Eventually L(c)$, where $L(c)$ is a unary ground literal,
  or
  $\Eventually L(x)$, where $L(x)$ is a unary non-ground literal.
  
  If $P_1(x)\rightarrow\Next Q_1(x)$, \dots, $P_k(x)\rightarrow\Next Q_k(x)$ are
  original non-ground step clauses, then
  $\forall x (P_1(x)\lor\dots\lor P_k(x)\rightarrow\Next
   \forall x (Q_1(x)\lor\dots\lor Q_k(x))$,
  $\exists x (P_1(x)\land\dots\land P_k(x)\rightarrow\Next
   \exists x (Q_1(x)\land\dots\land Q_k(x))$, and
  $P_i(c)\rightarrow Q_i(c)$, $1\leq i\leq k$, where $c$ is a constant
  occurring in $\mathcal{P}$, are \emph{derived step clauses}.
  If $\Phi_1\rightarrow\Next\Psi_1$, \dots,$\Phi_n\rightarrow\Psi_n$
  are derived step clauses or original ground step clauses, then
  $\bigwedge_{i=1}^n\Phi_i\rightarrow\Next\bigwedge_{i=1}^n\Psi_i$ is a
  \emph{merged derived step clause}. 
  If $\mathcal{A}\rightarrow\Next\mathcal{B}$ is a merged derived step
  clause and $P_1(x)\rightarrow\Next Q_1(x)$, \dots,
  $P_k(x)\rightarrow\Next Q_k(x)$ are original step clauses,
  then $\forall x (\mathcal{A}\land\bigwedge_{i=1}^k P_i(x))
  \rightarrow\Next (\mathcal{B}\land\bigwedge_{i=1}^k Q_i(x))$
  is a \emph{full merged step clause}.
\end{clarifications}

\begin{history}
  Related to~\iref{LTLClausalResolution}, this calculus was introduced in
  \cite{Degtyarev+Fisher+Konev@CADE2003},
  a comprehensive description is given
  in~\cite{DBLP:journals/tocl/DegtyarevFK06}. It was 
  extended to handling equality in~\cite{Konev+Degtyarev+Fisher@LPAR2003} 
  and a calculus closer to an implementable form was provided 
  in~\cite{KDDFH05}. It has been implemented in the prover
  TeMP~\cite{DBLP:conf/cade/HustadtKRV04}.
  %Satisfiability equivalence of the transformation, 
  Soundness and completeness 
  %of the calculus 
  are shown 
  in~\cite{DBLP:journals/tocl/DegtyarevFK06}. 
  %A later
  %implementation focusing on fair derivations is provided in
  %TSPASS~\cite{Ludwig+Hustadt@CADE2009,Ludwig+Hustadt@AIC2010}.
\end{history}

\end{entry}
