\calculusName{Resolution for Propositional Linear Time Temporal Logic}
\calculusAcronym{}
\calculusLogic{Temporal Logics}
\calculusLogicOrder{Propositional}
\calculusType{Resolution}
\calculusYear{1991/2001}

\calculusAuthor{Michael Fisher}
\calculusAuthor{Clare Dixon}
\calculusAuthor{Ullrich Hustadt}
\calculusAuthor{Boris Konev}
\calculusAuthor{Martin Peim}

\newcommand{\TRPpp}{\textsf{TRP\protect\raisebox{0.4ex}{\small{+}{+}}}\xspace}
\implementation{\TRPpp}{http://cgi.csc.liv.ac.uk/~konev/software/trp++/}

\entryTitle{Resolution for Propositional Linear Time Temporal Logic (LTL)}
\entryAuthor{Clare Dixon}
\entryAuthor{Michael Fisher}
\entryAuthor{Ullrich Hustadt}
\entryAuthor{Boris Konev}

\maketitle


\begin{entry}{LTLClausalResolution}  

\newcommand{\Next}{\!\raisebox{0.1ex}{ %possibly add a little space before
                        \mbox{\unitlength=0.62ex
                        \begin{picture}(2,2)
                        \linethickness{0.1ex}
                        \put(1,1){\circle{2}} % Draws circle with
                        \end{picture}}}       % diameter 2 at centre 1,1
                        \,}

\newcommand{\TNext}{\!\raisebox{-.2ex}{ %possibly add a little space before
                        \mbox{\unitlength=0.9ex
                        \begin{picture}(2,2)
                        \linethickness{0.06ex}
                        \put(1,1){\circle{2}} % Draws circle with
                        \end{picture}}}       % diameter 2 at centre 1,1
                        \,}

\newcommand{\Always}[0]{\mathop{\scalebox{1.25}{\ensuremath{\Box}}}}
\newcommand{\TAlways}{\raisebox{-.2ex}{
                            \mbox{\unitlength=0.9ex
                            \begin{picture}(2,2)
                            \linethickness{0.06ex}
                            \put(0,0){\line(1,0){2}}
                            \put(0,2){\line(1,0){2}}
                            \put(0,0){\line(0,1){2}}
                            \put(2,0){\line(0,1){2}}
                            \end{picture}}}
                       \,}

\newcommand{\Eventually}[0]{\Diamond}
%\def\sometime{\hbox{$\,\Diamond \,$}}
\newcommand{\sometime}[0]{\Diamond}

\newcommand{\lstart}{{\bf start}}
\newcommand{\ltrue}{{\bf true}}
\newcommand{\lfalse}{{\bf false}}
%{\mathord{\hbox{\large$\mathchar"0\cmsy@7D$}}} % PS \diamondsuit 
%                                                             % is too
%                                                             small
\def\unless{\hbox{$\,\cal W \,$}}

\begin{calculus}
%\textbf{Resolution rules}
\vspace*{-.7em}
\[
\begin{array}{c}
\infer[\textit{(initial)}]
{\Always(\lstart \rightarrow (C \lor D))}
{\Always(\lstart \rightarrow (C \lor A))  &
\Always(\lstart \rightarrow (D \lor \neg A))} 
\qquad
\infer[\textit{(step)}]
{\Always(P \land Q  \rightarrow  \Next (C \lor D))}
{\Always(P \rightarrow \Next (C \lor A))  &
 \Always(Q \rightarrow \Next (D \lor \neg A))} 
\\[4pt]
\infer[\textit{(conversion)}]
{\Always(\lstart \rightarrow \lnot P)  \quad 
 \Always(\ltrue  \rightarrow \Next \lnot P)}
{\Always(P       \rightarrow \Next \lfalse)} 
\\[8pt]
%
%
\begin{array}[b]{ll}
\infer[{\textit{(temporal),}} ]
{\textstyle\Always  (P    \rightarrow  \left[\bigwedge_{i=0}^n (\neg\mathcal{A}_i)\right] \unless L)}
{\Always  (\mathcal{A}_0 \rightarrow  \Next \mathcal{B}_0) &
 \cdots  &\quad
 \Always  (\mathcal{A}_n \rightarrow  \Next \mathcal{B}_n) & 
\Always  (P   \rightarrow  \sometime L)}

& \hspace*{1ex}
% \left[
\begin{array}{l}
\text{where for all } 0 \leq i \leq n \\
\mbox{}\;\vdash \mathcal{B}_i \rightarrow \neg L \; \text{ and }
\vdash \mathcal{B}_i \rightarrow \textstyle\bigvee_{j=0}^n \mathcal{A}_j
\vspace*{0.9em}
\end{array}
\end{array}
\end{array}
\]

\vspace*{-1em}
Here $A$ is a proposition, $L$ is a literal, $C$ and $D$ are (possibly empty) disjunctions of literals,
$P$ and $Q$ are (possibly empty) conjunctions of literals and $\mathcal{A}_i\rightarrow\Next\mathcal{B}_i$  are \emph{merged step clauses}
(conjunctions of step clauses), see~\cite{FDP01} for details.
Derivation terminates if either $\Always(\lstart\rightarrow\lfalse)$
is derived (unsatisfiable) or no new clause can be derived
(satisfiable).
\end{calculus}


\begin{clarifications}
  The clausal temporal resolution calculus is for formulae of discrete propositional
  linear-time temporal logic with finite past and infinite
  future~\cite{Prior@OUP1967,Kamp@PhD1968}.
  Any LTL formula is first translated, in a satisfiability preserving way, into the
  following  normal form: 
  $\Always(\lstart \rightarrow C)$, \emph{an initial clause},
  $\Always(P \rightarrow \Next C)$, \emph{a step clause}, and
  $\Always(P \rightarrow \sometime L)$, \emph{a sometime clause}.
  %
  This removes many of the operators, may add new propositional 
  symbols, for example to rename subformulae.
  %
  The logical constant $\lstart$ only holds in the first moment in time. 
  %
  The resolvent of the temporal resolution rule needs further translation into
  the normal form. 
  %
  In the rules $\ltrue$ stands for an empty conjunction of literals; $\lfalse$
  stands for an empty disjunction of literals.
\end{clarifications}

\begin{history}
  First proposed in~\cite{Fis90-resolve} earlier versions of 
  the normal form and the calculus used clauses with past-time formulae
  on the left-hand sides and present or future formulae on the
  right-hand side. The above version of the resolution rules are
  from~\cite{FDP01} and the calculus has been implemented in the prover
  \TRPpp~\cite{HustadtKonev2003a}. 
  A different presentation is provided in~\cite{DFK02}
  using a {\em Divided Separated Normal Form}.
  The calculus has been extended to the monodic fragment of First-Order 
  temporal logic in~\iref{FOTLResolution}.
  %, implemented in the prover TeMP~\cite{DBLP:conf/cade/HustadtKRV04}.
  % splitting the formule into
  %four sets: the initial and universal parts (both sets of propositional
  %formula), the step part (similar to the above) and the eventuality
  % part (unconditional sometime formula).
\end{history}

\begin{technicalities}
  Soundness, completeness and termination are shown in~~\cite{FDP01}.
\end{technicalities}


\end{entry}
