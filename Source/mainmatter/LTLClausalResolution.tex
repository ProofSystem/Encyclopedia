
% If the calculus has an acronym, define it.
% (e.g. \newcommand{\LK}{\ensuremath{\mathbf{LK}}\xspace})
\calculusName{Clausal Resolution for Propositional Linear Time Temporal Logic}         % The name of the calculus
\calculusAcronym{LTL}          % The acronym if defined above, or empty otherwise. 
\calculusLogic{Linear Time Temporal Logic}        % Specify the logic (e.g. Classical Logic, Intuitionistic Logic, ...) for which this calculus is intended.
\calculusLogicOrder{Propostional}   % Specify the order of the logic (e.g. Propositional, Quantified Propositional, First-Order, Higher-Order, ...).
\calculusType{Resolution}         % Specify the calculus type (e.g. Tableau, Sequent Calculus, Hyper-Sequent Calculus, Natural Deduction, ...)
\calculusYear{1991/2001}         % The year when the calculus was published.

\calculusAuthor{Michael Fisher}       % The name(s) of the author(s) of the calculus.
\calculusAuthor{Clare Dixon}
\calculusAuthor{Martin Peim}




\entryTitle{Propositional Linear Time Temporal Logic (LTL)}     % Title of the entry (usually coincides with the name of the calculus).
\entryAuthor{Clare Dixon}
\entryAuthor{Michael Fisher}
\entryAuthor{Ullrich Hustdat}
\entryAuthor{Boris Konev}


% The encyclopedia's peer-reviewing policy is described here: 
% http://proofsystem.github.io/Encyclopedia/
%
% Reviewers of this entry will be acknowledged in the following lines:
% \entryReviewer{Reviewer 1's name}
% \entryReviewer{Reviewer 2's name}
% \entryReviewer{Reviewer 3's name}
%
% The lines above will be filled by the coordinators. 
% If you would like to indicate people 
% who could review your entry, contact the coordinators.


% If you wish, use tags to give any other information 
% that might be helpful for classifying and grouping this entry:
% e.g. \tag{Two-Sided Sequents}
% e.g. \tag{Multiset Cedents}
% e.g. \tag{List Cedents}
% You are free to invent your own tags. 
% The Encyclopedia's coordinator will take care of 
% merging semantically similar tags in the future.


\maketitle


% If your files are called "MyProofSystem.tex" and "MyProofSystem.bib", 
% then you should write "\begin{entry}{MyProofSystem}" in the line below
\begin{entry}{LTLClausalResolution}  

% Define here any newcommands you may need:
% e.g. \newcommand{\necessarily}{\Box}
% e.g. \newcommand{\possibly}{\Diamond}

\newcommand{\Next}{\!\raisebox{-.2ex}{ %possibly add a little space before
                        \mbox{\unitlength=0.9ex
                        \begin{picture}(2,2)
                        \linethickness{0.06ex}
                        \put(1,1){\circle{2}} % Draws circle with
                        \end{picture}}}       % diameter 2 at centre 1,1
                        \,}

 \newcommand{\Always}{\raisebox{-.2ex}{
                            \mbox{\unitlength=0.9ex
                            \begin{picture}(2,2)
                            \linethickness{0.06ex}
                            \put(0,0){\line(1,0){2}}
                            \put(0,2){\line(1,0){2}}
                            \put(0,0){\line(0,1){2}}
                            \put(2,0){\line(0,1){2}}
                            \end{picture}}}
                       \,}

\def\sometime{\hbox{$\,\Diamond \,$}}

\newcommand{\lstart}{{\bf start}}
\newcommand{\ltrue}{{\bf true}}
\newcommand{\lfalse}{{\bf false}}
%{\mathord{\hbox{\large$\mathchar"0\cmsy@7D$}}} % PS \diamondsuit 
%                                                             % is too
%                                                             small
\def\unless{\hbox{$\,\cal W \,$}}

\begin{calculus}
%\textbf{Resolution rules}
\vspace*{-.7em}
\[
\begin{array}{c}
\infer[\textit{(initial)}]
{\lstart \rightarrow (A \lor B)}
{\lstart \rightarrow (A \lor p)  &
\lstart \rightarrow (B \lor \neg p)} 
\qquad
\infer[\textit{(step)}]
{\Always(C \land D  \rightarrow  \Next (A \lor B))}
{\Always(C \rightarrow \Next (A \lor p))  &
\Always (D	    \rightarrow  \Next (B \lor \neg p))} 
\\[4pt]
\infer[\textit{(conversion)}]
{ \lstart \rightarrow \lnot C  \quad \Always (\ltrue \rightarrow \Next \lnot C)}
{\Always(C  \rightarrow  \Next \lfalse)} 
\\[8pt]
%
%
\begin{array}[b]{ll}
\infer[\textit{(temporal),} ]
{\textstyle\Always  (D    \rightarrow  \left[\bigwedge_{i=0}^n (\neg C_i)\right] \unless l)}
{\Always  (C_0   \rightarrow  \Next A_0) &
\ldots  &\quad
\Always  (C_n   \rightarrow  \Next A_n) & 
\Always  (D   \rightarrow  \sometime l)}

& \quad
% \left[
\begin{array}{l}
\text{where for all } 0 \leq i \leq n \\[-5pt]
\mbox{}\;\vdash A_i \rightarrow \neg l \; \text{ and }
\vdash A_i \rightarrow \displaystyle \bigvee_{j=0}^n C_j
\vspace*{0.9em}
\end{array}
\end{array}
\end{array}
\]

\vspace*{-1em}
Here $p$ is a proposition, $l$ is a literal, $A$ and $B$ are (possibly empty) disjunctions of literals,
$A_i$ are non-empty disjunctions of literals and 
$C$, $C_i$, $D$ are (possibly empty) conjunctions of literals. 
Derivation terminates if either $\lstart\rightarrow\lfalse$ is derived (unsatisfiable) or no new clause can be derived (satisfiable).  
\end{calculus}

% The following sections ("clarifications", "history", 
% "technicalities") are optional. If you use them, 
% be very concise and objective. Nevertheless, do write full sentences. 
% Try to have at most one paragraph per section, because line breaks 
% do not look nice in a short entry.

 \begin{clarifications}
% ToDo: write here short remarks that may help the reader to understand 
% the inference rules of the proof system.
The clausal temporal resolution calculus is for formulae of discrete propositional
linear-time temporal logic with finite past and infinite future.
Any LTL is first translated, in a satisfiability preserving way, into the
following  normal form: 
$\lstart \rightarrow A$, {\em an initial clause},
$ \Always (C \rightarrow  \Next A)$, {\em a step clause}, and
$\Always  (C   \rightarrow  \sometime l)$, {\em a sometime clause}.
%
This removes many of the operators, may add new propositional 
symbols, for example to rename subformulae.
%
The operator $\lstart$ only holds in the first moment in time. 
%
The resolvent of the temporal resolution rule needs further translation into
the normal form. 
%
In the rules $\ltrue$ stands for an empty conjunction of literals; $\lfalse$
stands for an empty disjunction of literals.
 \end{clarifications}

\begin{history}
% ToDo: write here short historical remarks about this proof system,
% especially if they relate to other proof systems. 
% Use "\iref{OtherProofSystem}" to refer to another proof system 
% in the Encyclopedia (where "OtherProofSystem" is its ID). 
% Use "\irefmissing{SuggestedIDForOtherProofSystem}" to refer to 
% another proof system that is not yet available in the encyclopedia.
First proposed in~\cite{Fis90-resolve} earlier versions of 
the normal form and the calculus used clauses with past-time formulae
on the left-hand sides and present or future formulae on the
right-hand side. The above version of the resolution rules are
from~\cite{FDP01} and the calculus has been implemented in the prover
TRP++~\cite{HustadtKonev2003a}. 
A different presentation is provided in~\cite{DFK02}
using a {\em Divided Separated Normal Form}.
The calculus has been extended to the monodic fragment of First-Order 
temporal logic in~\cite{DBLP:journals/tocl/DegtyarevFK06}, implemented in the prover TeMP~\cite{DBLP:conf/cade/HustadtKRV04}.
% splitting the formule into
%four sets: the initial and universal parts (both sets of propositional
%formula), the step part (similar to the above) and the eventuality
% part (unconditional sometime formula).
\end{history}

\begin{technicalities}
% ToDo: write here remarks about soundness, completeness,
% decidability...
Soundness, completeness and termination are shown in~~\cite{FDP01}.
\end{technicalities}


% General Instructions:
% =====================

% The preferred length of an entry is 1 page. 
% Do the best you can to fit your proof system in one page.
%
% If you are finding it hard to fit what you want in one page, remember:
%
%   * Your entry needs to be neither self-contained nor fully understandable
%     (the interested reader may consult the cited full paper for details)
%
%   * If you are describing several proof systems in one entry, 
%     consider splitting your entry.
%
%   * You may reduce the size of your entry by ommitting inference rules
%     that are already described in other entries.
%
%   * Cite parsimoniously (see detailed citation instructions below).
%
% 
% If you do not manage to fit everything in one page, 
% it is acceptable for an entry to have 2 pages.
%
% For aesthetic reasons, it is preferable for an entry to have
% 1 full page or 2 full pages, in order to avoid unused blank space.



% Citation Instructions:
% ======================

% Please cite the original paper where the proof system was defined.
% To do so, you may use the \cite command within 
% one of the optional environments above,
% or use the \nocite command otherwise.

% You may also cite a modern paper or book where the 
% proof system is explained in greater depth or clarity.
% Cite parsimoniously.

% Do not cite related work. Instead, use the "\iref" or "\irefmissing" 
% commands to make an internal reference to another entry, 
% as explained within the "history" environment above.

% You do not need to create the "References" section yourself. 
% This is done automatically.


% Remove all instruction comments before submitting.


% Leave an empty line above "\end{entry}".

\end{entry}
