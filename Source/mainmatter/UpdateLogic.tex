



\calculusName{Update Logic}   
\calculusAcronym{UL}     
\calculusLogic{Substructural Logics} 
\calculusLogicOrder{Propositional}
\calculusType{Display Calculus}   
\calculusYear{2016}   
\calculusAuthor{Guillaume Aucher}

\entryTitle{Update Logic UL}        
%\entryAuthor{Guillaume Aucher} 
\entryAuthor{Guillaume Aucher}  





% % % % % % % % % MACROS % % % % % % %

\newcommand{\distance}{0.5pt}
\newcommand{\distanceb}{2.5pt}
\newcommand{\ddistance}{2.6pt}
\newcommand{\ddistanceb}{3pt}
\newcommand{\WA}{\textsf{K}}
\newcommand{\CA}{\textsf{WI}}
\newcommand{\comA}{\textsf{CI}}
\newcommand{\ass}{\mbox{$\textsf{B}^c$}}
\newcommand{\ulstructruleaddeux}{}
\newcommand{\U}{U}
\newcommand{\V}{V}
\newcommand{\entail}{~{\footnotesize{\sststile[ss]{}{}}}~ }
\newcommand{\com}{\hspace{\distance}{,}\hspace{\distanceb}}
\newcommand{\X}{X}
\newcommand{\Y}{Y}
\newcommand{\Z}{Z}
\newcommand{\commai}{\hspace{\ddistance}\mathord{,}_{\mbox{\tiny{i}}}\hspace{\distanceb}}
\newcommand{\commaj}{\hspace{\ddistance}\mathord{,}_{\mbox{\tiny{j}}}\hspace{\distanceb}}
\newcommand{\commak}{\hspace{\ddistance}\mathord{,}_{\mbox{\tiny{k}}}\hspace{\distanceb}}
\newcommand{\ulstructrulebdun}{} 
\newcommand{\ulstructrulebdunbis}{} 
\newcommand{\ulstructruleaddeuxbis}{} 

\newcommand{\bulletj}{\bullet_{\mbox{\tiny{}}}}

\newcommand{\axun}{Id}
\newcommand{\veeA}{\vee_A}
\newcommand{\veeK}{\vee_K}
\newcommand{\implyA}{\ra_A}
\newcommand{\belj}{\square} 
\newcommand{\feas}{\Diamond}
\newcommand{\backmod}{\feas^{-}} 
\newcommand{\elimfusionid}{\fusions^{\mbox{\tiny{i}}}_A}
\newcommand{\fissionK}{\varoplus^{\mbox{\tiny{i}}}_K}
\newcommand{\fissionA}{\varoplus^{\mbox{\tiny{i}}}_A}

\newcommand{\introfusionid}{\fusions^{\mbox{\tiny{i}}}_K} 
\newcommand{\introfusioni}{\fusions^{\mbox{\tiny{i}}}_K}
\newcommand{\elimfusioni}{\fusions^{\mbox{\tiny{i}}}_A}
\newcommand{\elimentailinii}{\implys^{\mbox{\tiny{j}}}_A}
\newcommand{\introentaili}{\implys^{\mbox{\tiny{j}}}_K}
\newcommand{\introentailbisi}{\implybiss^{\mbox{\tiny{k}}}_K}
\newcommand{\elimentailbisinii}{\implybiss^{\mbox{\tiny{k}}}_A}

\newcommand{\introfusionin}{\fusions^{\mbox{\tiny{i}}}_I}
\newcommand{\elimfusionin}{\fusions^{\mbox{\tiny{i}}}_E}
\newcommand{\elimentailiniin}{\implys^{\mbox{\tiny{j}}}_E}
\newcommand{\introentailin}{\implys^{\mbox{\tiny{j}}}_I}
\newcommand{\introentailbisin}{\implybiss^{\mbox{\tiny{k}}}_I}
\newcommand{\elimentailbisiniin}{\implybiss^{\mbox{\tiny{k}}}_E}


\newcommand{\elimentailiniidb}{\implys^{\tiny{\tau(i)}}_A}
\newcommand{\introentailidb}{\implys^{\tiny{\tau(i)}}_K}

\newcommand{\introentailbisidb}{\implybiss^{\tiny{\rho(i)}}_K}
\newcommand{\elimentailbisiniidb}{\implybiss^{\tiny{\rho(i)}}_A}


\newcommand{\introentailid}{\implys^{\mbox{\tiny{j}}}_K}
\newcommand{\elimentailiniid}{\implys^{\mbox{\tiny{j}}}_A}
\newcommand{\introentailbisid}{\implybiss^{\mbox{\tiny{k}}}_K}
\newcommand{\elimentailbisiniid}{\implybiss^{\mbox{\tiny{k}}}_A}

\newcommand{\fusionzero}{\fusions_{\mbox{\tiny{0}}}}
\newcommand{\implyzero}{\supset_{\mbox{\tiny{0}}}}
\newcommand{\implybiszero}{\subset_{\mbox{\tiny{0}}}}

\newcommand{\ulstructruleb}{\commadeux/\commatrois}
\newcommand{\ulstructrulea}{\commatrois/\comma}

\newcommand{\dr}{\textsf{dr}}

\newcommand{\drquat}{\textsf{dr}_4}

\newcommand{\implys}{\supset}
\newcommand{\imply}{\supset_{\mbox{\tiny{1}}}}
\newcommand{\implydeux}{\subset_{\mbox{\tiny{3}}}}
\newcommand{\implytrois}{\supset_{\mbox{\tiny{3}}}}
\newcommand{\fusions}{\otimes}
\newcommand{\fusion}{\otimes_{\mbox{\tiny{3}}}}
\newcommand{\fusiondeux}{\otimes_{\mbox{\tiny{1}}}}
\newcommand{\fusiontrois}{\otimes_{\mbox{\tiny{2}}}}
\newcommand{\implybiss}{\subset}
\newcommand{\implybis}{\subset_{\mbox{\tiny{2}}}}
\newcommand{\implybisdeux}{\supset_{\mbox{\tiny{2}}}}
\newcommand{\implybistrois}{\subset_{\mbox{\tiny{1}}}}

\newcommand{\implya}{\supset_{\mbox{\tiny{\un}}}}
\newcommand{\implyxdeux}{\subset_{\mbox{\tiny{\trois}}}}
\newcommand{\implyxtrois}{\supset_{\mbox{\tiny{\trois}}}}
%\newcommand{\fusions}{\otimes}
\newcommand{\fusionc}{\otimes_{\mbox{\tiny{\trois}}}}
\newcommand{\fusionxdeux}{\otimes_{\mbox{\tiny{\un}}}}
\newcommand{\fusionxtrois}{\otimes_{\mbox{\tiny{\deux}}}}
%\newcommand{\implybiss}{\subset}
\newcommand{\implybisb}{\subset_{\mbox{\tiny{\deux}}}}
\newcommand{\implybisxdeux}{\supset_{\mbox{\tiny{\deux}}}}
\newcommand{\implybisxtrois}{\subset_{\mbox{\tiny{\un}}}}

\newcommand{\elimentailini}{\implys^{\mbox{\tiny{1}}}_A}
\newcommand{\elimentailinideux}{\implybiss^{\mbox{\tiny{3}}}_A}
\newcommand{\elimentailinitrois}{\implys^{\mbox{\tiny{3}}}_A}
\newcommand{\elimfusion}{\fusions^{\mbox{\tiny{3}}}_A}
\newcommand{\elimfusiondeux}{\fusions^{\mbox{\tiny{1}}}_A}
\newcommand{\elimfusiontrois}{\fusions^{\mbox{\tiny{2}}}_A}
\newcommand{\introfusion}{\fusions^{\mbox{\tiny{3}}}_K}
\newcommand{\introfusiondeux}{\fusions^{\mbox{\tiny{1}}}_K}
\newcommand{\introfusiontrois}{\fusions^{\mbox{\tiny{2}}}_K}
%\newcommand{\introentailbisini}{\implybis_K}
\newcommand{\introentailbiss}{\implybiss_K}
\newcommand{\introentailbis}{\implybiss^{\mbox{\tiny{2}}}_K}
\newcommand{\introentailbisdeux}{\implybiss^{\mbox{\tiny{2}}}_K}
\newcommand{\introentailbistrois}{\implybiss^{\mbox{\tiny{1}}}_K}
\newcommand{\elimentailbisinis}{\implybiss_A}
\newcommand{\elimentailbisini}{\implybiss^{\mbox{\tiny{2}}}_A}
\newcommand{\elimentailbisinideux}{\implybiss^{\mbox{\tiny{2}}}_A}
\newcommand{\elimentailbisinitrois}{\implybiss^{\mbox{\tiny{1}}}_A}
\newcommand{\introentail}{\implys^{\mbox{\tiny{1}}}_K}
\newcommand{\introentaildeux}{\implybiss^{\mbox{\tiny{3}}}_K}
\newcommand{\introentailtrois}{\implys^{\mbox{\tiny{3}}}_K}
\newcommand{\fusioni}{\otimes_{\mbox{\tiny{i}}}}
\newcommand{\implyi}{\supset_{\mbox{\tiny{j}}}}
\newcommand{\implybisi}{\subset_{\mbox{\tiny{k}}}}
\newcommand{\implyii}{\supset_{\mbox{\tiny{i}}}}
\newcommand{\implybisii}{\subset_{\mbox{\tiny{i}}}}

\newcommand{\RA}{\Yright}

\newcommand{\RAK}{\Yleft^{\mbox{\tiny{j}}}_K}
\newcommand{\RAA}{\Yleft^{\mbox{\tiny{j}}}_A}
\newcommand{\BRAK}{\Yright^{\mbox{\tiny{k}}}_A}
\newcommand{\BRAA}{\Yright^{\mbox{\tiny{k}}}_K}


\newcommand{\BRAi}{\Yleft_{\mbox{\tiny{i}}}}
\newcommand{\BRAk}{\Yleft_{\mbox{\tiny{k}}}}
\newcommand{\BRAj}{\Yleft_{\mbox{\tiny{j}}}}

\newcommand{\RAi}{\Yright_{\mbox{\tiny{i}}}}
\newcommand{\RAj}{\Yright_{\mbox{\tiny{j}}}}
\newcommand{\RAk}{\Yright_{\mbox{\tiny{k}}}}

\newcommand{\fissioni}{\varoplus_{\mbox{\tiny{i}}}}

\newcommand{\fission}{\varoplus_{\mbox{\tiny{\trois}}}}

\newcommand{\BRA}{\Yleft}

\newcommand{\backbox}{\belj^{-}}



\maketitle

\begin{entry}{UpdateLogic}  

 
  
\begin{calculus}
\scalebox{0.92}{$\begin{array}{@{\quad\quad}l@{\quad\quad}l@{\quad\quad}l@{\quad\quad}l@{\quad\quad}l@{\quad}l@{\quad}l@{\quad}} 
\infer=[\ulstructruleaddeux]{\Y\entail \Z\commak *\X}{\infer=[\ulstructrulebdun]{\X\entail *\Y\commaj\Z}{\X\commai\Y\entail \Z}} & 		
\infer=[\ulstructruleaddeuxbis]{\Z\commak*\X\entail \Y}{\infer=[\ulstructrulebdunbis]{*\Y\commaj\Z\entail\X}{\Z\entail \X\commai\Y}} & 
\infer={\bulletj\X\entail\Y}{\X\entail\bulletj\Y} & \infer[Cut]{\U\entail\V}{\U\entail\phi & \phi\entail\V} \\
\\
\infer[\WA]{\U\com\X\entail\V}{\U\entail\V} & 
\infer[\CA]{\X\entail\U}{\X\com\X\entail\U} & 
\infer[\comA]{\X\com\Y\entail \U}{\Y\com\X\entail \U}  & 
\infer[\ass]{\X\com(\Y\com\Z)\entail\U}{(\X\com\Y)\com\Z\entail\U} \\
\\ 	
\infer[\axun]{p\entail p}{} & 
\infer[\bot_A]{\bot\entail}{}  & 
\infer[\top_K]{\entail\top}{}  \\
\\
\infer[\veeK^1]{\U\entail\phi\vee\psi}{\U\entail\phi} & 		
\infer[\veeK^2]{\U\entail\phi\vee\psi}{\U\entail\psi} & 
\infer[\veeA]{\phi\vee\psi\entail\U}{\phi\entail\U & \psi\entail\U} & 
\infer[\neg_K]{\U\entail\neg\phi}{\U\entail *\phi} \\ %\\
\\
\infer[\land_K]{\U\entail \phi\land\psi}{\U\entail\phi & \U\entail\psi} & 			
\infer[\land_A^1]{\phi\land\psi\entail\U}{\phi\entail\U} & 
\infer[\land_A^2]{\phi\land\psi\entail\U}{\psi\entail\U} & 
\infer[\neg_A]{\neg\phi\entail \U}{*\phi\entail\U} \\
\\
\infer[\introfusionid]{\X\commai\Y\entail\phi\fusioni\psi}{\X\entail\phi &  \Y\entail\psi} & 
\infer[\elimfusionid]{\phi\fusioni\psi\entail\U}{ \phi\commai\psi\entail\U} & 
\infer[\fissionK]{\U\entail\phi\fissioni\psi}{\U\entail\phi \commai\psi} & 
\infer[\fissionA]{\phi\fissioni\psi\entail\X\commai\Y}{ \phi\entail\X & \psi\entail\Y}\\
\\
\infer[\introentailid]{\X\entail\phi\implyi\psi}{\X\commai\phi\entail \psi} & 
\infer[\elimentailiniid]{\phi\implyi\psi\entail*\X\commaj\Y}{\X\entail \phi & \psi\entail\Y} & 
\infer[\RAK]{*\X\commaj\Y\entail\psi\Yleft_{\mbox{\tiny{j}}}\phi}{\Y\entail\psi & \phi\entail\X} & 
\infer[\RAA]{\psi\Yleft_{\mbox{\tiny{j}}}\phi\entail\X}{\psi\entail \phi\commak\X} \\
\\
\infer[\introentailbisid]{\X\entail\psi\implybisi\phi}{\phi\commai\X\entail\psi} & \infer[\elimentailbisiniid]{\psi\implybisi\phi\entail\Y\commak*\X}{\psi\entail\Y & \X\entail\phi} & 		
\infer[\BRAA]{\Y\commak*\X\entail\phi\Yright_{\mbox{\tiny{k}}}\psi}{\phi\entail\X & \Y\entail\psi} & 
\infer[\BRAK]{\phi\Yright_{\mbox{\tiny{k}}}\psi\entail\X}{\psi\entail\phi\commai\X}\\
\\
\multicolumn{3}{l}{(i,j,k)\in\left\{(1,2,3), (2,3,1), (3,1,2)\right\} } \\
\\
\infer[\Box_K]{\X\entail\belj\phi}{\bulletj\X\entail\phi} 	&		\infer[\Box_A]{\belj\phi\entail\bulletj\X}{\phi\entail \X} & \infer[\Box^-_K]{\U\entail\backbox\phi}{\U\entail*\bulletj*\phi} 	&		\infer[\Box^-_A]{\backbox\phi\entail*\bulletj*\X}{\phi\entail \X}\\
\\
\infer[\Diamond^{-}_K]{\bulletj\X\entail\backmod\phi}{\X\entail\phi} &     \infer[\Diamond^{-}_A]{\backmod\phi\entail\X}{\phi\entail\bulletj\X}	 & \infer[\Diamond_K]{*\bulletj*\X\entail\feas\phi}{\X\entail\phi} &     \infer[\Diamond_A]{\feas\phi\entail\U}{*\bulletj*\phi\entail\U} 						
\end{array}$}
\end{calculus}

\begin{clarifications}
Update logic \cite{Aucher2016}  generalizes  the non--associative Lambek calculus \iref{LambekCalc} and includes  modal, propositional and dual substructural connectives. %It contains three symetric triples of connectives
%The substructural connectives are interconnected by means of cyclic permutations. %The usual fusion, implication
%and co-implication connectives form one of these triples.
 %It is a (proper) display calculus and   the cut rule is eliminable. Its validity problem is also decidable.
 Correspondences between frame properties and structural display
calculus rules are used to define novel display sequent calculi for a broad range of
substructural logics (bi--intuitionistic logic is a case study of \cite{Aucher2016}).  \end{clarifications}

%\begin{history}
%%% ToDo: write here short historical remarks about this proof system,
%%% especially if they relate to other proof systems. 
%%% Use "\iref{OtherProofSystem}" to refer to another proof system 
%%% in the Encyclopedia (where "OtherProofSystem" is its ID). 
%%% Use "\irefmissing{SuggestedIDForOtherProofSystem}" to refer to 
%%% another proof system that is not yet available in the encyclopedia.
%The system now known as the basic Lambek Calculus was introduced in
%1958 by Joachim Lambek as the "Syntactic Calculus" \cite{lambek1958}.
%Lambek's motivation was to ``to obtain an effective rule (or
%algorithm) for distinguishing sentences from non-sentences, which
%works not only for the formal languages of interest to the
%mathematical logician, but also for natural languages $[\ldots]$'', as
%explained by Moortgat in \cite{moortgat2010}.  After a long period of
%ostracism, around the middle 1980s the Syntactic Calculus, now called
%the Lambek Calculus was taken up by logicians interested in
%Computational Linguistics, especially van Benthem, Buszkowski and
%Moortgat. They realized that a computational semantics for categorical
%derivations along the lines of the Curry-Howard proofs-as-programs
%interpretation would provide us with a ``parsing-as-deduction"
%paradigm and a powerful tool to study ``logical" derivational
%semantics. Around the same time, the introduction of Linear Logic
%\iref{LL}, by Jean-Yves Girard also gave a new impulse to the work in
%Categorical Grammars. This was because of Linear Logic's insight that
%even if you had a very weak proof system, you could introduce
%structural rules in a controlled fashion and hence obtain more
%expressive systems, by the use of the so called modalities. Since no
%expressivity is lost in this process, this opened the way for various
%types of experiments, trying to make sure that the logical system
%could cope with more phenomena from the language, see discussion of
%examples in \cite{moortgat2010}.
%\end{history}

 \begin{technicalities}
Update logic  is a (proper) display calculus which  enjoys  cut elimination. % and  is decidable. %It is sound and complete 
 \end{technicalities}

\end{entry}
