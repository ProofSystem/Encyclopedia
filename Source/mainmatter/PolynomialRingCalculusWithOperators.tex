

\calculusName{Polynomial Ring Calculus with Operators}        
\calculusAcronym{PRCO}         
\calculusLogic{Modal Logics}     
\calculusLogicOrder{Propositional}  
\calculusType{Polynomial Ring  Calculi}      
\calculusYear{2015}        

\calculusAuthor{Walter Carnielli}       
\calculusAuthor{Juan Carlos Agudelo-Agudelo}



\entryTitle{Polynomial Ring Calculus with Operators}    
\entryAuthor{Walter Carnielli}     



\maketitle


\begin{entry}{PolynomialRingCalculusWithOperators}  

\newcommand{\fn}[3]{#1\negthickspace : \negthickspace #2 \negthickspace\to\negthickspace #3}
\def\n{\textbf{n}}
\newcommand{\vdashp}{\vdash_{\approx}}

\begin{calculus} 
 
 As  basis  for modal logics, formulas of   the  Classical Propositional Logic $CPL$ are translated into  Boolean rings   $BR$   via  the translation $\fn{T_{BR}}{Form_{CPL}}{Term_{BR}}$, recursively defined by:
\begin{align*}   
	T_{BR}(p_i) &= x_i,\\
	T_{BR}(\neg \alpha) &= T_{BR}(\alpha) + 1,\\
	T_{BR}(\alpha \vee \beta) &= T_{BR}(\alpha) \cdot T_{BR}(\beta) + T_{BR}(\alpha) + T_{BR}(\beta),\\
	T_{BR}(\alpha \wedge \beta) &= T_{BR}(\alpha) \cdot T_{BR}(\beta),\\
	T_{BR}(\alpha \to \beta) &= T_{BR}(\alpha) \cdot T_{BR}(\beta) + T_{BR}(\alpha) + 1,\\
	T_{BR}(\alpha \leftrightarrow \beta) &= T_{BR}(\alpha) + T_{BR}(\beta) + 1. 
\end{align*}
This obtains an  immediate  proof  procedure and decision  method for $CPL$, in the sense that
 $\vdash_{CPL} \alpha$ iff  $BR \vDash T_{BR}(\alpha) \approx 1$,
i.e, iff the translated expression reduces to 1 by cancellation rules of  polynomials with  coefficients
in $\mathbb{Z}_{2}$.
 
 The  structure   $\mathcal{R}=\langle R,  \n \rangle$ is a \emph{boolean ring with operators} (BRO, or \emph{modal ring})  if $BR$ is a BR and $\fn{\n}{A}{A}$ satisfies:
	\begin{align*}
		\n(1) &= 1,\\
		\n(x \cdot y) &= \n(x) \cdot \n(y).
	\end{align*}
BRO defines a   \emph{Polynomial Ring Calculus with Operators} (PRCO) for the  normal modal  logic   $K$. Subsequent  proof  systems for  normal modal logics are obtained by adding  new conditions to this basic PRCO, according to the characteristics of the   necessitation operator. For each of the  the following axioms that   define the systems $KD$, $KT= T$, $KT4 = S4$ and $KT5= S5$,  the specific  conditions are:


\begin{enumerate}
\item  For D, $\Box\alpha \to \Diamond \alpha $ defines  PRCO$_{D}$,  the extension of PRCO   by adding:
 \begin{equation*}
	 \n(0) \vdashp 0 . 
 \end{equation*}
\item 	For T, $\Box\alpha \to \alpha$ defines PRCO$_{T}$,  the extension of PRCO  by adding:
\begin{equation*}
	\n(P) \vdashp P \n(P).
\end{equation*}
\item For S4, $\Box\alpha \to \Box\Box\alpha$  defines PRCO$_{S4}$, the extension of  PRCO$_T$   by adding:
\begin{align*}
	&\n(P) \vdashp \n(QR) \mbox{ (if $\mbox{Fact}(P) = Q \n(R)$ and $D(Q) \leq D(R)$)},   \\ 
	&\n(P) \vdashp \n(\n(Q)R) \mbox{ (if $\mbox{Fact}(P) = Q R$ and $D(Q) < D(R)$)}.  
 \end{align*} 
  where $Fact(Q)$ denotes the factorization of a modal polynomial $Q$.
	\item For  S5, $\Diamond\alpha \to \Box\Diamond\alpha$ defines  PRCO$_{S5}$, the extension of  PRCO$_T$   by adding:
\begin{equation*}
	\n(P \n(Q) + R) \vdashp \n(Q R + Q P) + \n(Q R) + \n(R). 
\end{equation*}
\end{enumerate}

\end{calculus}


\begin{clarifications}
The Polynomial Ring Calculus with Operators (PRCO's) are algebraic proof methods,  
here devoted to the modal logics K, KD, T, S4, S5 and to intuitionistic logic. 
They consist of translating formulas of a logical system into polynomials 
over an adequate field, in such a way that algebraic reductions on polynomials
allow us to determine whether a formula is or is not a theorem of the respective
system. The decision procedure works as a proof in algebraic  terms. As shown in  \cite{car:2005}, 
PRC are particularly applicable to any truth-functional finite-valued propositional logic,
and can also be applied to non-truth-functional propositional logics, see
\cite{car-mat:2014} and \cite{car-mat:2015}.
\end{clarifications}

\begin{history}
Polynomial Ring Calculi (PRC) were introduced in \cite{car:2005} and extended  
in \cite{Carnielli2007}, \cite{agude-car:2011} and  {agude-car:2015}.
Conceptual discussions and extensions to non-deterministic semantics appear
in \cite{car-mat:2014} and \cite{car-mat:2015}.
 \end{history}

\begin{technicalities}
Full completeness, several examples and a comparison with other methods are provided in 
\cite{agude-car:2015}.  
\end{technicalities}


\end{entry}
