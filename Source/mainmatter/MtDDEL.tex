\calculusName{Multi-type Sequent Calculus for Dynamic Epistemic Logic MtD.DEL}
\calculusAcronym{MtDDEL}
\calculusLogic{Dynamic Epistemic Logic}
\calculusLogicOrder{Propositional}
\calculusType{Multi-type Sequent Calculus}
\calculusYear{2016}
\calculusAuthor{Giuseppe Greco} 
\calculusAuthor{Alessandra Palmigiano} 
\calculusAuthor{Alexander Kurz} 
\calculusAuthor{Sabine Frittella} 
\calculusAuthor{Vlasta Sikimi\'{c}} 

\entryTitle{Multi-type Sequent Calculus for Dynamic Epistemic Logic MtD.DEL}
\entryAuthor{Giuseppe Greco}

\etag{Two-Sided Sequents}
\etag{Multi-type Cedents}
\etag{Multi-Premise Cedent}
\etag{Multi-Conclusion Cedent}



\maketitle

\begin{entry}{MtDDEL}


%%%Definition and abbreviations

\def\fCenter{{\mbox{$\ \vdash\ $}}}
\def\ffCenter{{\mbox{$\overline{\vdash}$}}}
\def\fcenter{{\mbox{\ }}}
\EnableBpAbbreviations

\def\fns{\footnotesize}
\def\mc{\multicolumn}

\def\aol{\rule[0.5865ex]{1.38ex}{0.1ex}}
\def\aoll{\rule[0.5865ex]{1.222ex}{0.1ex}}
\def\pdra{\mbox{$\,>\mkern-8mu\raisebox{-0.065ex}{\aol}\,$}}
\def\pdla{\mbox{\rotatebox[origin=c]{180}{$\,>\mkern-8mu\raisebox{-0.065ex}{\aol}\,$}}}

\newcommand{\mcal}{\mathcal}
\newcommand{\mf}{\mathfrak}
\newcommand{\mb}{\mathbb}
\newcommand{\mbf}{\mathbf}
\newcommand{\msf}{\mathsf}
\newcommand{\llb}{\llbracket}
\newcommand{\rrb}{\rrbracket}
\newcommand{\lb}{\lbrack}
\newcommand{\rb}{\rbrack}
\newcommand{\W}{\wedge}
\newcommand{\V}{\vee}
\newcommand{\ef}{\langle \exists \forall \rangle}
\newcommand{\fe}{\langle \forall \exists \rangle}
\newcommand{\tiff}{\text{ iff }}
\newcommand{\tor}{\text{ or }}
\newcommand{\tand}{\text{ and }}
\newcommand{\tst}{\text{ s.t. }}
\newcommand{\twith}{\text{ with }}
\newcommand{\raDi}{\longleftarrow_{\Delta_{1}}}
\newcommand{\laDi}{\leftarrowtail_{\Delta_{1}}}


\newcommand{\raDii}{\longleftarrow_{\Delta_{2}}}
\newcommand{\laDii}{\leftarrowtail_{\Delta_{2}}}


%\newcommand{\vCenter}[1]{\; \vdash_{#1} \;}
\newcommand{\mP}{\mathcal{P}}


\newcommand{\red}[1]{\textcolor{red}{#1}}
\newcommand{\bfred}[1]{\textbf{\textcolor{red}{#1}}}

%%%%%%%%%%%%%%%%%%%%%%%%%%%%%%%%%%%%%%%%%%%%%%%%%%%%%%%%%%%%%%%%%%%%%%%%%%%%%%%%%%%%%%%%%%%%%%%%%%%%
%%% Dynamic Epistemic Logic's Macros %%%%%%%%%%%%%%%%%%%%%%%%%%%%%%%%%%%%%%%%%%%%%%%%%%%%%%%%%%%%%%%
%%%%%%%%%%%%%%%%%%%%%%%%%%%%%%%%%%%%%%%%%%%%%%%%%%%%%%%%%%%%%%%%%%%%%%%%%%%%%%%%%%%%%%%%%%%%%%%%%%%%

\def\andol{\rule[-0.4563ex]{1.38ex}{0.1ex}}
\def\AOL{\rule[0.65ex]{1.45ex}{0.1ex}}
\def\aol{\rule[0.5865ex]{1.38ex}{0.1ex}}

%%% Proposition %%%

\def\pRA{\mbox{$\,\,{\AOL{\mkern-0.4mu{\rotatebox[origin=c]{-90}{\raisebox{0.12ex}{$\pAND$}}}}}\,\,$}}

\def\pDRA{\mbox{$\,\,{\rotatebox[origin=c]{-90}{\raisebox{0.12ex}{$\pAND$}}{\mkern-1mu\AOL}}\,\,$}}
\def\pdra{\mbox{$\,{\rotatebox[origin=c]{-90}{\raisebox{0.12ex}{$\wedge$}}{\mkern-2mu\aol}}\,$}}
\def\pdla{\mbox{\rotatebox[origin=c]{180}{$\,{\rotatebox[origin=c]{-90}{\raisebox{0.12ex}{$\wedge$}}{\mkern-2mu\aol}}\,$}}}


%%%%%%%%%%%%%%%%%%%%%%%%%%%%%%%%%%%%%%%%%%%%%%%%%%%%%%%%%%%%%%%%%%%%%%%%%%%%%%%%%%%%%%%%%%%%
%Meta-level%%%%%%%%%%%%%%%%%%%%%%%%%%%%%%%%%%%%%%%%%%%%%%%%%%%%%%%%%%%%%%%%%%%%%%%%%%%%%%%%%

\def\mANDORatom#1{\hbox{\hbox to 0pt{$#1\TriangleUp$\hss}$#1\TriangleDown$}}
\newcommand*{\mANDOR}{\mathrel{\mathchoice{\mANDORatom\displaystyle}
                                 {\mANDORatom\textstyle}
                                 {\mANDORatom\scriptstyle}
                                 {\mANDORatom\scriptscriptstyle}}}

\newcommand{\mcAND}{%
\mathrel{\ooalign{\raisebox{-0.39ex}{$\mbox{\TriangleUp}$}\cr\kern4.2pt{\raisebox{-0.13ex}{$\cdot$}}}}}
\newcommand{\mcand}{%
\mathrel{\ooalign{$\vartriangle$\cr\kern1.99pt{\raisebox{-0.17ex}{$\cdot$}}}}}

\newcommand{\mAND}{\raisebox{-0.39ex}{\mbox{\,\TriangleUp\,}}}
\newcommand{\mand}{\vartriangle}


\newcommand{\nAND}{%
\mathrel{\ooalign{$\mbox{\TriangleUp}$\cr\kern0pt$\mbox{\rotatebox[origin=c]{180}{\TriangleUp}}$}}}

\newcommand{\nand}{%
\mathrel{\ooalign{$\vartriangle$\cr\kern0pt$\triangledown$}}}


\newcommand{\mcBAND}{%
\mathrel{\ooalign{\raisebox{-0.39ex}{$\mbox{\FilledTriangleUp}$}\cr\kern4.2pt{\raisebox{-0.13ex}{${\color{white}\cdot}$}}}}}
\def\mBAND{\raisebox{-0.39ex}{\mbox{\,\FilledTriangleUp\,}}}
\def\mband{\mbox{$\mkern+2mu\blacktriangle\mkern+2mu$}}

\newcommand{\mcband}{%
\mathrel{\ooalign{$\blacktriangle$\cr\kern1.99pt{\raisebox{-0.17ex}{${\color{white}\cdot}$}}}}}

\def\mOR{\mbox{\,\rotatebox[origin=c]{-180}{\TriangleUp}\,}}
\def\mor{\mbox{$\mkern+2mu\triangledown\mkern+2mu$}}

\def\mBOR{\mbox{\,\FilledTriangleDown\,}}
\def\mbor{\mbox{$\blacktriangledown$}}

\def\mbra{\mbox{$\,-{\mkern-3mu\blacktriangleright}\,$}}

\def\mRAline{\mbox{\,\raisebox{0.43ex}{\AOL}$\mkern-2.6mu$\rotatebox[origin=c]{-90}{\TriangleUp}\,}}

\newcommand{\mcRA}{%
\mathrel{\ooalign{
                  \raisebox{-0.3ex}{$\rotatebox[origin=c]{-90}{$\mbox{{\TriangleUp}}$}$}
                                                                            \cr\kern2.7pt{\raisebox{0.2ex}{$\cdot\mkern1.3mu$}}}}}

\def\DRA{\mbox{\,\rotatebox[origin=c]{3.9999}{\TriangleRight}$\mkern-2.6mu$\raisebox{0.43ex}{\AOL}\,}}
\def\mRA{\mbox{\,\raisebox{-0.39ex}{\rotatebox[origin=c]{-90}{\TriangleUp}}\,}}
\newcommand{\mra}{\mbox{$\,-{\mkern-3mu\vartriangleright}\,$}}
\newcommand{\mla}{\mbox{$\,{\vartriangleleft\mkern-8mu-}\,$}}

\def\mbdra{\mbox{$\,\blacktriangleright{\mkern-8mu-}\,$}}

\newcommand{\mcra}{%
\mathrel{\ooalign{$\,{\vartriangleright\,}$\cr\kern3pt{\raisebox{0ex}{$\cdot$}}}}}

\newcommand{\mcraline}{%
-{\mkern-6mu{\mathrel{\ooalign{$\,{\vartriangleright\,}$\cr\kern3pt{\raisebox{0ex}{$\cdot$}}}}}}}

\def\mdra{\mbox{$\,\vartriangleright{\mkern-8mu-}\,$}}
\def\nra{\mbox{$\,\vartriangleright\,$}}

\newcommand{\mdraline}{%
{\mathrel{\ooalign{$\,{\vartriangleright\,}$\cr\kern3pt{\raisebox{0ex}{$\cdot$}}}}}{\mkern-6mu}-}

%

\def\mSRA{\mbox{\,\raisebox{0.43ex}{$\thicksim\mkern-1.3mu$}\rotatebox[origin=c]{3.9999}{\TriangleRight}\,}}
\def\nSRA{\mbox{\,\rotatebox[origin=c]{-90}{\TriangleUp}\,}}
\def\msra{\mbox{$\,\sim{\mkern-8mu\vartriangleright}\,$}}

\def\mBRA{\mbox{\,\raisebox{-0.39ex}{\rotatebox[origin=c]{-90}{\FilledTriangleUp}}\,}}
\newcommand{\mcBRA}{%
\mathrel{\ooalign{
                  \raisebox{-0.3ex}{$\rotatebox[origin=c]{-90}{$\mbox{\FilledTriangleUp}$}$}
                                                                            \cr\kern2.7pt{\raisebox{0.2ex}{${\color{white}\cdot}$}}}}}

\def\mBDRA{\mbox{\,\FilledTriangleRight\raisebox{0.43ex}{\AOL}}\,}
\def\BRA{\mbox{\,\rotatebox[origin=c]{-90}{\FilledTriangleUp}\,}}
\newcommand{\mcbra}{%
\mathrel{\ooalign{$\,-{\mkern-3mu\blacktriangleright\,}$\cr\kern8pt{\raisebox{0ex}{$\cdot$}}}}}
\def\mbdra{\mbox{$\,\blacktriangleright{\mkern-8mu-}\,$}}
\def\mLA{\mbox{\,\raisebox{-0.39ex}{\rotatebox[origin=c]{90}{\TriangleUp}}\,}}
\newcommand{\mcLA}{%
\mathrel{\ooalign{
                  \raisebox{-0.3ex}{$\rotatebox[origin=c]{90}{$\mbox{\TriangleUp}$}$}
                                                                                     \cr\kern5.5pt{\raisebox{0.2ex}{$\cdot$}}
                                                                                                                              }}}
\def\nLA{\mbox{\,\rotatebox[origin=c]{-3.9999}{\TriangleLeft}\,}}
\def\la{\mbox{$\,\vartriangleleft{\mkern-8mu-}\,$}}

\newcommand{\mcla}{%
\mathrel{\ooalign{$\,{\vartriangleleft\,}$\cr\kern5pt{\raisebox{0ex}{$\cdot$}}}}}

\newcommand{\mclaline}{%
-{\mkern-6mu{\mathrel{\ooalign{$\,{\vartriangleleft\,}$\cr\kern5pt{\raisebox{0ex}{$\cdot$}}}}}}}

\def\mdla{\mbox{$\,-{\mkern-3mu\vartriangleleft}\,$}}
\def\nla{\mbox{$\,\vartriangleleft\,$}}

\def\mSLA{\mbox{\,\rotatebox[origin=c]{-3.9999}{\TriangleLeft}\raisebox{0.43ex}{$\mkern-1.3mu\thicksim$}\,}}

\def\SLA{\mbox{\,\rotatebox[origin=c]{90}{\TriangleUp}\raisebox{0.43ex}{$\mkern-2.6mu\thicksim$}\,}}
\def\SDLA{\mbox{\,\raisebox{0.43ex}{$\thicksim\mkern-3.8mu$}\rotatebox[origin=c]{90}{\TriangleUp}\,}}


\def\SBLA{\mbox{\,\rotatebox[origin=c]{90}{\FilledTriangleUp}\raisebox{0.43ex}{$\mkern-2.6mu\thicksim$}\,}}
\def\SDBLA{\mbox{\,\raisebox{0.43ex}{$\thicksim\mkern-3.8mu$}\rotatebox[origin=c]{90}{\FilledTriangleUp}\,}}

\def\BRA{\mbox{\,$\raisebox{0.43ex}{\AOL}\mkern-2.6mu$\rotatebox[origin=c]{-90}{\FilledTriangleUp}\raisebox{0.43ex}\,}}
\def\DBRA{\mbox{\,\rotatebox[origin=c]{-90}{\FilledTriangleUp}\raisebox{0.43ex}{$\mkern-3.8mu\AOL$}\,}}

\def\nSLA{\mbox{\,\rotatebox[origin=c]{-3.9999}{\TriangleLeft}\,}}
\def\msla{\mbox{$\,\vartriangleleft{\mkern-8mu\sim}\,$}}
\def\msdla{\mbox{$\,{\sim\mkern-4.5mu\vartriangleleft}\,$}}

\def\mBLA{\mbox{\,\raisebox{-0.39ex}{\rotatebox[origin=c]{90}{\FilledTriangleUp}}\,}}
\newcommand{\mcBLA}{%
\mathrel{\ooalign{
                  \raisebox{-0.3ex}{$\rotatebox[origin=c]{90}{$\mbox{\FilledTriangleUp}$}$}
                                                                                     \cr\kern5.5pt{\raisebox{0.2ex}{${\color{white}\cdot}$}}
                                                                                                                                            }}}

\def\BLA{\mbox{\,\rotatebox[origin=c]{90}{\FilledTriangleUp}\,}}
\def\mbla{\mbox{$\,\blacktriangleleft\,$}}
\def\bla{\mbox{$\,\blacktriangleleft{\mkern-8mu-}\,$}}
%
\def\mBDLA{\mbox{\,\raisebox{0.43ex}{\AOL}{\FilledTriangleLeft}\,}}

\def\mSBRA{\mbox{\,\raisebox{0.43ex}{$\thicksim\mkern-1.3mu$}\FilledTriangleRight}\,}
\def\nSBRA{\mbox{\,\rotatebox[origin=c]{-90}{\FilledTriangleUp}\,}}
\def\msbra{\mbox{$\,\sim{\mkern-8mu\blacktriangleright}\,$}}

\def\mSBLA{\mbox{\,\FilledTriangleLeft\raisebox{0.43ex}{$\mkern-1.3mu\thicksim$}\,}}
\def\nSBLA{\mbox{\,\rotatebox[origin=c]{90}{\FilledTriangleUp}\,}}
\def\msbla{\mbox{$\,\blacktriangleleft{\mkern-8mu\sim}\,$}}
\def\msbdla{\mbox{$\,{\sim\mkern-4.5mu\blacktriangleleft}\,$}}


%%%%%%%%%%%%%%%%%%%%%%%%%%%%%%%%%%%%%%%%%%%%%%%%%%%%%%%%%%%%%%%%%%%%%%%%%%%%%%%%%%%%%%%%%%%%%%%%%%%%
%%%%%%%%%%%%%%%%%%%%%%%%%%%%%%%%%%%%%%%%%%%%%%%%%%%%%%%%%%%%%%%%%%%%%%%%%%%%%%%%%%%%%%%%%%%%%%%%%%%%
%%%%%%%%%%%%%%%%%%%%%%%%%%%%%%%%%%%%%%%%%%%%%%%%%%%%%%%%%%%%%%%%%%%%%%%%%%%%%%%%%%%%%%%%%%%%%%%%%%%%

%%% Abbreviazioni %%%%%%%%%%%%%%%%%%%%%%%%%%%%%%%%%%%%%%%%%%%%%%%%%%%%%%%%%%%%%%%%%%%%%

\def\odotRA{\mathrel\odot\joinrel\rightarrow}
\def\astRA{\mathrel\ast\!\joinrel\rightarrow}
\def\astLA{\mathrel\leftarrow\!\!\!{\joinrel\!\ast}}
\def\astSRA{\mathrel\ast\!\!\!\joinrel{>}\,}

\newcommand{\bbA}{\mathbb{A}}
\newcommand\val[1]{{\lbrack\!\lbrack} {#1}{\rbrack\!\rbrack}}

\newcommand{\f}{\overline}

\newcommand{\ls}{\lbrack}
\newcommand{\rs}{\rbrack}
\newcommand{\lc}{\langle}
\newcommand{\rc}{\rangle}

\newcommand{\Wbox}{\Box}
\newcommand{\Wdia}{\Diamond}
\newcommand{\Bbox}{\blacksquare}
\newcommand{\Bdia}{\Diamondblack}

\newcommand{\Mod}{\, \Delta \,}

%%%%%%%%%%%%%%%%%%%%%%%%%%%%%%%%%%%%%%%%%%%%%%%%%%%%%%%%%%%%%%%%%%%%%%%%%%%%%%%%%%%%%%%%%%%%%%%%%%

\def\conRA{\mbox{$\,\,{\rotatebox[origin=c]{-90}{\raisebox{0.12ex}{$\pAND$}}}\,\,$}}
\def\conLA{\mbox{$\,\,{\rotatebox[origin=c]{90}{\raisebox{0.12ex}{$\pAND$}}}\,\,$}}

\def\conDRA{\mbox{$\,\,{\rotatebox[origin=c]{90}{\raisebox{0.12ex}{$\pOR$}}}\,\,$}}
\def\conDLA{\mbox{$\,\,{\rotatebox[origin=c]{-90}{\raisebox{0.12ex}{$\pOR$}}}\,\,$}}

%%%%%%%%%%%%%%%%%%%%%%%%%%%%%%%%%%%%%%%%%%%%%%%%%%%%%%%%%%%%%%%%%%%%%%%%%%%%%%%%%%%%%%%%%%%%%%%%%%%%%
\newcommand{\pNEG}{\mbox{\boldmath{${\neg\,}$}}}
\newcommand{\pNEGL}{\mbox{{\boldmath{$\neg$}}$_L$\,}}
\newcommand{\pNEGR}{\mbox{{\boldmath{$\neg$}}$_R$\,}}
%
\newcommand{\pAND}{\mbox{$\,\bigwedge\,$}}
%
\newcommand{\pOR}{\mbox{$\,\bigvee\,$}}
%
\newcommand{\pra}{\rightarrow}
\def\pdra{\mbox{$\,{\rotatebox[origin=c]{-90}{\raisebox{0.12ex}{$\wedge$}}{\mkern-2mu\aol}}\,$}}
\newcommand{\pla}{\leftarrow}
\def\pdla{\mbox{\rotatebox[origin=c]{180}{$\,{\rotatebox[origin=c]{-90}{\raisebox{0.12ex}{$\wedge$}}{\mkern-2mu\aol}}\,$}}}
%
\def\pTOP{{_{\textrm{p\!\!}}\textrm{T}}}
\def\ptop{{_{\textrm{p\!\!\!}}\top}}
\def\pBOT{\textrm{\rotatebox[origin=c]{180}{T}}^{\textrm{\!p}}}
\def\pbot{\bot^{\textrm{\!\!p}}}

%%%%%%%%%%%%%%%%%%%%%%%%%%%%%%%%%%%%%%%%%%%%%%%%%%%%%%%%%%%%%%%%%%%%%%%%%%%%%%%%%%%%%%%%%%%%%%%%%%%%%%%%%%
%AGENTS%%%%%%%%%%%%%%%%%%%%%%%%%%%%%%%%%%%%%%%%%%%%%%%%%%%%%%%%%%%%%%%%%%%%%%%%%%%%%%%%%%%%%%%%%%%%%%%%%%%
%
\newcommand{\agand}{\mbox{$\sqcap$}}
\newcommand{\agAND}{\mbox{$\,\,\bigsqcap\,\,$}}
%
\newcommand{\agor}{\mbox{$\sqcup$}}
\newcommand{\agOR}{\mbox{$\,\,\bigsqcup\,\,$}}
%
\def\agra{\mbox{$\,{\aol{\mkern-1.5mu{\rotatebox[origin=c]{-90}{\raisebox{0.12ex}{$\agand$}}}}}\,$}}
\def\agRA{\mbox{$\,\,{\AOL{\mkern-0.4mu{\rotatebox[origin=c]{-90}{\raisebox{0.12ex}{$\agAND$}}}}}\,\,$}}
\def\agla{\mbox{$\,{\rotatebox[origin=c]{180}{$\aol{\mkern-1.6mu{\rotatebox[origin=c]{-90}{\raisebox{0.12ex}{$\agand$}}}}$}}\,$}}
\def\agLA{\mbox{\rotatebox[origin=c]{180}{$\,\,{\AOL{\mkern-0.4mu{\rotatebox[origin=c]{-90}{\raisebox{0.12ex}{$\agAND$}}}}}\,\,$}}}
%
\def\agdra{\mbox{$\,{\rotatebox[origin=c]{-90}{\raisebox{0.12ex}{$\agand$}}{\mkern-1.1mu\aol}}\,$}}
\def\agDRA{\mbox{$\,\,{\rotatebox[origin=c]{-90}{\raisebox{0.12ex}{$\agAND$}}{\mkern-0.7mu\AOL}}\,\,$}}
\def\agdla{\mbox{\rotatebox[origin=c]{180}{$\,{\rotatebox[origin=c]{-90}{\raisebox{0.12ex}{$\agand$}}{\mkern-1.1mu\aol}}\,$}}}
\def\agDLA{\mbox{\rotatebox[origin=c]{180}{$\,\,{\rotatebox[origin=c]{-90}{\raisebox{0.12ex}{$\agAND$}}{\mkern-0.7mu\AOL}}\,\,$}}}
%
\def\agTOP{{_{\textrm{g\!\!}}\textrm{T}}}
\def\agtop{{_{\textrm{g\!\!\!}}\top}}
\def\agBOT{\textrm{\rotatebox[origin=c]{180}{T}}^{\textrm{\!g}}}
\def\agbot{\bot^{\textrm{\!\!g}}}
\def\agneg{\mbox{$\mkern-0.4mu\sim\mkern-0.4mu$}}

\def\aga{\texttt{a}}
\def\agb{\texttt{b}}
\def\agc{\texttt{c}}
\def\agd{\texttt{d}}
\def\agA{{\Large{\texttt{a}}}}
\def\agB{\Large{\texttt{b}}}
\def\agC{\texttt{C}}
\def\agD{\texttt{D}}

\def\bulletaga{{\bullet_{\!\aga}}}
\def\circaga{{\circ_{\!\aga}}}
\def\Wdiaaga{{\Wdia_{\!\aga}}}
\def\Bdiaaga{{\Bdia_{\!\aga}}}
\def\Wboxaga{{\Wbox_{\aga}}}
\def\Bboxaga{{\Bbox_{\aga}}}
%
\def\andol{\rule[-0.4563ex]{1.38ex}{0.1ex}}
\def\aol{\rule[0.5865ex]{1.38ex}{0.1ex}}
\def\AOL{\rule[0.65ex]{1.45ex}{0.1ex}}
\def\orol{\rule[1.4253ex]{1.38ex}{0.1ex}}
%



%%%


\begin{calculus}
{
Atom rule: where  $F_1,\ldots, F_n, G_1,\ldots, G_m\in \mathsf{FNC}$, $\circ \in
\{\mAND_{\! 0} \ , \mBAND_{\! 0} \ \}$, ${\rhd} \in \{\mRA_{\!\! 0} \ ,
\mBRA_{\!\! 0} \ \}$ and $n, m\in \mathbb{N}$,
\[
F_1 \circ (F_2 \circ \cdots (F_n \circ p) \cdots)\vdash G_1 {\rhd} (G_2 {\rhd} \cdots (G_m {\rhd} p)\cdots)
\]

Balance rule:
\[
\AX$ X\fCenter Y$
\UI$ F \mAND_{\! 0} \  X\fCenter F \mRA_{\!\! 0} \  Y$
\DisplayProof
\]

Necessitation, conjugation, Fisher Servi and monotonicity rules: for $0 \leq i \leq 2$,
\vspace{0.2cm}
\begin{center}
\AX$W \fCenter \textrm{I} $
\RightLabel{\scriptsize{$({nec_i} \mbra)$}}
\UI$W \fCenter x \mBRA_{\!\!i} \ \textrm{I}$
\DisplayProof 
\quad
\AX$x \mAND_{\!\!i\,} ((x \mBAND_{\!\!i\,} Y) \,; Z)\fCenter W$
\LeftLabel{\scriptsize{$({conj_i} \mand)$}}
\UI$Y\ ; (x\mAND_{\!\!i\,} Z) \fCenter W$
\DisplayProof
\quad
\AX$W\fCenter ( x \mBAND_{\! i} \  Y)> (x \mBRA_{\!\! i} \  Z)  $
\RightLabel{\scriptsize{$({FS_{\! i}}\mbra)$}}
\UI$W \fCenter x \mBRA_{\!\! i} \  (Y > Z)$
\DisplayProof
\end{center}
\begin{center}
\AX$( x \mAND_{\! i} \  Y) \ ; (x \mAND_{\! i} \  Z)  \fCenter W $
\LeftLabel{\scriptsize{$({mon_i}\mand)$}}
\UI$ x \mAND_{\! i} \  (Y \  ; Z)  \fCenter W$
\DisplayProof
\quad
\AX$ ( x \mBAND_{\! i} \  Y) \  ; (x \mBAND_{\! i} \  Z)  \fCenter W $
\LeftLabel{\scriptsize{$({mon_i}\mband)$}}
\UI$ x \mBAND_{\! i} \  (Y \  ; Z)  \fCenter W$
\DisplayProof
\quad
\AX$ W \fCenter   ( x \mBRA_{\!\! i} \  Y) \  ; (x \mBRA_{\!\! i} \  Z)  $
\RightLabel{\scriptsize{$({mon_i}\mbra)$}}
\UI$W \fCenter  x \mBRA_{\!\! i} \  (Y \  ; Z)  $
\DisplayProof
\end{center}

Interaction rules between dynamic and epistemic modalities:

\begin{center}
\AX$X\fCenter (\agA \mAND F) \mBRA (\agA \mBRA Y)$
\RightLabel{\scriptsize{swap-out$_R$}}
\UI$X \fCenter \agA \mBRA (F \mBRA Y)$
\DisplayProof
\quad
\AX$X \fCenter \agA \mBRA (F \mBRA Y)$
\RightLabel{\scriptsize{swap-in$_R$}}
\UI$X\fCenter (\agA \mAND F) \mBRA (\agA \mBRA ((F \mAND \textrm{I} ) > Y))$
\DisplayProof
\end{center}
 }
\end{calculus}


\begin{clarifications}
  The language $\mathcal{L}_\mathrm{MT}(\mathcal{F}, \mathcal{G})$ of  MtD.DEL
  consists of  {\em logical}   and {\em structural terms} in the types
  $\mathsf{T}_1:  = \mathsf{Fm}$, 
  $\mathsf{T}_2: = \mathsf{Fnc}$, 
  $\mathsf{T}_3: = \mathsf{Act}$, 
  $\mathsf{T}_4: = \mathsf{Ag}$. 
  Following the notation of~\iref{MtSC}, the set  of logical terms takes as
  parameters:
  %
  1) a denumerable set of atomic terms $\mathsf{At}(\mathsf{Fm})$, elements of
  which are denoted $p$, possibly with indexes; a denumerable  set of atomic terms
  $\mathsf{At}(\mathsf{Fnc})$, elements of which are denoted $\alpha$, possibly
  with indexes; a finite  set of atomic terms $\mathsf{At}(\mathsf{Ag})$, elements
  of which are denoted $\aga$, possibly with indices; 
  %
  2)  sets of connectives
  $\mathcal{F}: = \mathcal{F}_{\mathsf{Fm}} \uplus
                  \mathcal{F}_{\mathsf{Fnc}} \uplus
                  \mathcal{F}_{\mathsf{Act}} \uplus
                  \mathcal{F}_{\mathsf{Ag}} \uplus
                  \mathcal{F}_{\mathrm{MT}}$
  %
  and 
  $\mathcal{G}: = \mathcal{G}_{\mathsf{Fm}} \uplus
                  \mathcal{G}_{\mathsf{Fnc}} \uplus
                  \mathcal{G}_{\mathsf{Act}} \uplus
                  \mathcal{G}_{\mathsf{Ag}} \uplus
                  \mathcal{G}_{\mathrm{MT}}$
  %
  defined as follows: 
  $\mathcal{F}_{\mathsf{Fm}}: = \{\top, \wedge, \pdra\}$,
  $\mathcal{F}_{\mathsf{Fnc}}: = \emptyset$, 
  $\mathcal{F}_{\mathsf{Act}}: = \emptyset$, 
  $\mathcal{F}_{\mathsf{Ag}}: = \emptyset$,
  $\mathcal{F}_{\mathrm{MT}}: = \{\mand_k\mid 0\leq k\leq 3\}$, 
  %
  where  
  $n_\top = 0$, 
  $n_{\wedge} = n_{\,>\mkern-3.8mu\raisebox{-0.2ex}{\aoll}\,} = n_{\mand_k} = 2$ 
  %
  for every 
  $0\leq k\leq 3$, and 
  $\varepsilon_{\pdra}(1) = \partial$ and
  $\varepsilon_{\wedge}(i)=\varepsilon_{\,>\mkern-3.8mu\raisebox{-0.2ex}{\aoll}\,}(2) = \varepsilon_{\mand_k}(i)  = 1$ 
  %
  for every $i\in \{1, 2\}$, and every $0\leq k\leq 3$, and 
  $\mathcal{G}_{\mathsf{Fm}}: = \{\bot, \vee, \pra\}$,
  $\mathcal{G}_{\mathsf{Fnc}}: = \emptyset$, 
  $\mathcal{G}_{\mathsf{Act}}: = \emptyset$, 
  $\mathcal{G}_{\mathsf{Ag}}: = \emptyset$,
  $\mathcal{G}_{\mathrm{MT}}: = \{\mra_k\mid 0\leq k\leq 2\}$, 
  %
  where  
  $n_\bot = 0$, 
  $n_{\vee} = n_{\pra} =n_{\,-{\mkern-3mu\vartriangleright}_k\,} = 2$ 
  %
  for every 
  $0\leq k\leq 3$, and 
  $\varepsilon_{\pra}(1) = \varepsilon_{\,-{\mkern-3mu\vartriangleright}_k\,}(1)  =  \partial$ and
  $\varepsilon_{\vee}(i)=\varepsilon_{\pra}(2) = \varepsilon_{\,-{\mkern-3mu\vartriangleright}_k\,}(2) = 1$ 
  %
  for every $i\in \{1, 2\}$, and every $0\leq k\leq 2$.  
  %  
  The functional types of the heterogeneous connectives are given as follows:
  ${\mand}_0, {\mra}_{\! 0} : \mathsf{Fnc}
  \times \mathsf{Fm} \to \mathsf{Fm}$, ${\mand}_1, {\mra}_{\! 1}  :  \mathsf{Act}
  \times \mathsf{Fm} \to \mathsf{Fm}$, ${\mand}_2, {\mra}_{\! 2}  : \mathsf{Ag}
  \times \mathsf{Fm} \to \mathsf{Fm}$, ${\mand}_3  :  \mathsf{Ag} \times
  \mathsf{Fnc} \to \mathsf{Act}$.
  
  The structural terms are built by means of structural connectives, taking
  logical terms as atomic structures. The set of structural connectives includes
  $\mAND_{\!\!k}$ and $ \mRA_{\!\!\!\!j}$ for each $0\leq k\leq 3$  and  $0\leq
  j\leq 2$ which are the structural counterparts of $\mand_k$ and $ \mra_{\!\!j}$,
  respectively, for each $0\leq k\leq 3$ and  $0\leq j\leq 2$.  It also includes
  $;$ as the structural counterpart of both $\wedge$ (when occurring in antecedent
  position) and $\vee$ (when occurring in succedent position), $\mathrm{I}$ as the
  structural counterpart of both $\top$ (when occurring in antecedent position)
  and $\bot$ (when occurring in succedent position), and $>$ as the structural
  counterpart of both $\pdra$ (when occurring in antecedent position) and $\pra$
  (when occurring in succedent position). Finally, it includes  $\mBLA_{\!\!k}$
  and $\mBRA_{\!\!\!k}$ for each $0\leq k\leq 3$, and $\mBAND_{\!\!\!j}$ and
  $\mLA_{\!\!\!j}$ for each $0\leq j\leq 2$, where $\mBLA_{\!\!k}$ and
  $\mBRA_{\!\!k}$ are the residuals of $\mAND_{\!\!k}$ in its first and  second
  coordinate respectively, and $\mLA_{\!\!j}$ and $\mBAND_{\!\!\!j}$ are the
  residuals of $\mRA_{\!\!\!\!j}$ in its first and second  coordinate
  respectively. Hence, the functional types of these structural connectives are
  given as follows:
  %
  ${\mBAND}_{\!0}, {\mBRA}_{\! 0}   :  \mathsf{Fnc} \times \mathsf{Fm} \to \mathsf{Fm}$, 
  ${\mBAND}_{\!1}, {\mBRA}_{\! 1}  :  \mathsf{Act} \times \mathsf{Fm} \to \mathsf{Fm}$, 
  ${\mBAND}_{\!2}, {\mBRA}_{\! 2}  : \mathsf{Ag} \times \mathsf{Fm} \to \mathsf{Fm}$, 
  ${\mBRA}_{\! 3}  :  \mathsf{Ag} \times \mathsf{Act} \to \mathsf{Fnc}$,
  ${\mLA}_{\! 0} , {\mBLA}_{\! 0} : \mathsf{Fm}  \times \mathsf{Fm} \to \mathsf{Fnc}$,
  ${\mLA}_{\! 1} , {\mBLA}_{\! 1} :  \mathsf{Fm}  \times \mathsf{Fm} \to \mathsf{Act},$
  ${\mLA}_{\! 2} , {\mBLA}_{\! 2} : \mathsf{Fm}\times \mathsf{Fm} \to \mathsf{Ag}$,
  ${\mBLA}_{\! 3} :  \mathsf{Act} \times \mathsf{Fnc} \to \mathsf{Ag}$.
  %
  Summing up, the well formed terms of MtD.SDM are generated by simultaneous
  induction as follows:
  %
  {\small
  \begin{center}
  \begin{tabular}{@{}l@{}}
  $\mathsf{Fm}$ \\
  $A:: = \, p \mid \bot \mid \top \mid A \wedge A \mid A \vee A \mid A \rightarrow A 
         \mid A \pdra A \mid \alpha\mand_0 A \mid \alpha\mra_{\! 0} \  A 
         \mid \gamma\mand_1 A\mid \gamma\mra_{\! 1} \  A \mid \aga \mand_2 A \mid \aga\mra_{\! 2} \  A$ \\
  $X:: = \, A \mid \textrm{I} \mid X~; X\mid X > X\mid F \mAND_{\! 0} \ X
         \mid F \mRA_{\!\! 0} \  X \mid \Gamma\mAND_{\! 1} \  X\mid \Gamma\mRA_{\!\! 1} \  X 
         \mid {\large{\texttt{a}}}\mAND_{\! 2} \  X\mid {\large{\texttt{a}}}\mRA_{\!\! 2} \  X 
         \mid$ \\
  \ \ \ \ \ \ \ \ \ \,$F \mBAND_{\! 0} \  X\mid F \mBRA_{\!\! 0} \  X \mid
    \Gamma\mBAND_{\! 1} \  X\mid \Gamma\mBRA_{\!\! 1} \  X 
    \mid {\large{\texttt{a}}}\mBAND_{\! 2} \  X\mid {\large{\texttt{a}}}\mBRA_{\!\! 2} \  X$ \\
  \end{tabular}
  \end{center}
  
  \begin{center}
  \begin{tabular}{lclcl}
  $\mathsf{Fnc}$ &\ \ \ & $\mathsf{Act}$ &\ \ \ & $\mathsf{Ag}$\\
  $\alpha:: = \, \alpha$ &\ \ \ &  
    $\gamma:: =\, \aga\mand_3 \alpha$ &\ \ & 
    $\aga:: = \aga$ \\
  $F:: = \, \alpha\mid X \mLA_{\!\! 0} \   X \mid X \mBLA_{\!\! 0} \  X
         \mid {\large{\texttt{a}}} \mBRA_{\!\! 3} \  \Gamma$ &\ \ \ & $\Gamma:: =\,{\large{\texttt{a}}}\mAND_{\! 3} \  F 
         \mid  X \mLA_{\! 1} \  X 
         \mid X \mBLA_{\! 1} \  X$ &\ \ \ & ${\large{\texttt{a}}}:: =  \,\aga 
         \mid X \mLA_{\!\! 2} \   X \mid X \mBLA_{\!\! 2} \  X  \mid \Gamma \mBLA_{\!\! 3} \ F$\\
  \end{tabular}
  \end{center}
  }
  
  The Identity, Cut, Display and introduction rules instantiate the general
  template described in~\iref{MtSC}, and hence are omitted. Also, the pure-type
  structural rules for $\mathsf{Fm}$ are the standard ones capturing the usual
  (classical, intuitionistic, substructural...) propositional base, and are also
  omitted. The calculus MtD.DEL is a proper fragment of the calculus introduced
  in~\cite{FrittellaGrecoKurzPalmigianoSikimic2016} and captures the diamond-only
  presentation of dynamic epistemic logic. The rules swap-out$_R$ and swap-in$_R$
  are a notational variant of the corresponding rules introduced
  in~\cite{FrittellaGrecoKurzPalmigianoSikimic2016}.
\end{clarifications}


\begin{history}
  MtD.DEL was introduced in~\cite{FrittellaGrecoKurzPalmigianoSikimic2016} and
  captures Baltag-Moss-Solecki's dynamic epistemic
  logic~\cite{BaltagMossSolecki1999} in its classical version, and the logic
  IEAK~\cite{KurzPalmigiano2013} in its intuitionistic version. An overview of the
  literature about the proof theoretic approaches for dynamic epistemic logics can
  be found in~\cite{SabineGrecoKurzPalmigianoSikimic2016a}.   
\end{history}

\begin{technicalities}
  The axiomatization of Baltag-Moss-Solecki's dynamic epistemic logic is not
  closed under uniform substitution and is given in terms of meta-linguistic
  labels. The multi-type language of MtD.DEL makes it possible to internalize the
  labels into the object language of the calculus so that when recast in this
  language, the axioms are analytic inductive according to the definition
  of~\cite{GrecoMaPalmigianoTzimoulisZhao2016}. The $0$-ary rule Atom captures the
  axioms which are not closed under uniform substitution.  The calculus above is
  sound and complete w.r.t. final coalgebra semantics introduced
  in~\cite{GrecoKurzPalmigiano2013}; it is conservative, and enjoys the cut
  elimination and subformula property as immediate consequences of the general
  theory of multi-type calculi.
\end{technicalities}

\end{entry}
