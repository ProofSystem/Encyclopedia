\calculusName{Untyped Lambda Reduction}
\calculusAcronym{}
\calculusLogic{}
\calculusLogicOrder{Lambda Abstraction, First-Order}
\calculusType{(Term-)Sequent Calculus}
\calculusYear{2011}

\calculusAuthor{Michael Gabbay}

\entryTitle{Untyped $\lambda$ Reduction}
\entryAuthor{Michael Gabbay}

\maketitle

\begin{entry}{LambdaTermSequent}

\newcommand{\pa}[1]{\langle #1\rangle}
\newcommand{\col}{{{:}\!{-}}}
\newcommand{\ci}{\rightarrow}
\newcommand\aeq{{\rightsquigarrow}}
\newcommand\aep{{\aeq}L}
\newcommand\aer{{\aeq}R}

\begin{calculus}

\textbf{Term-Sequent Rules}
$$
\begin{array}{cc}
%
\infer[(Ax_{\col})]{\Gamma \vdash {t} \col{t}, \Psi, \Delta}{} 
&
\infer[(Cut_{\lambda})]{\Gamma \vdash \pa{\dots\Theta\dots}\col{t}, \Psi, \Delta}
                       {\Gamma \vdash \Theta\col{s}, \Psi, \Delta & \Gamma \vdash \pa{\dots{s}\dots}\col{t}, \Psi, \Delta}
\\[0.8em]
\infer[(\cdot L)]{\Gamma \vdash \pa{\dots{t}_1\cdot{t}_2\dots}\col{t}, \Psi, \Delta}
                 {\Gamma \vdash \pa{\dots\pa{{t}_1, {t}_2}\dots}\col{t}, \Psi, \Delta} 
&
\infer[(\cdot R)]{\Gamma \vdash \pa{\Theta_1, \Theta_2}\col{t}_1\cdot{t}_2, \Psi, \Delta}
                 {\Gamma \vdash \Theta_1\col{t}_1, \Psi, \Delta & \Gamma \vdash \Theta_2\col{t}_2, \Psi, \Delta}
\\[0.8em]
\infer[(\aep)]{\Gamma, {t}_1\aeq{t}_2 \vdash \pa{\dots\Theta\dots}\col{t}, \Psi, \Delta}
              {\Gamma \vdash \Theta\col{t}_1, \Psi, \Delta & \Gamma \vdash \pa{\dots{t}_2\dots}\col{t}, \Psi, \Delta}
&
\infer[(\aer)]{\Gamma \vdash {t}_1\aeq{t}_2, \Psi, \Delta}
              {\Gamma \vdash {t}_1\col{t}_2, \Psi, \Delta}
\\[0.8em]
\infer[(\lambda L)]{\Gamma \vdash \pa{\dots\pa{\lambda{x}.{t}_2, \Theta}\dots}\col{t}, \Psi, \Delta}
                   {\Gamma \vdash \Theta\col{t}_1, \Psi, \Delta & \Gamma \vdash \pa{\dots{t}_2[{x}/{t}_1]\dots}\col{t}, \Psi, \Delta}
&
\infer[(\lambda R)]{\Gamma \vdash \Theta\col\lambda{x}.{t}, \Psi, \Delta}
                   {\Gamma \vdash \pa{\Theta, {y}}\col{t}[{x}/{y}], \Psi, \Delta}
\end{array}
$$
\textbf{(Classical) Sequent Rules}
$$
%
\infer[(Cut)]{\Gamma \vdash \Psi, \Delta}
             {\Gamma \vdash  \Psi, {A}, \Delta & \Gamma, {A} \vdash  \Psi, \Delta}
%
$$
$$
%
\infer[(\forall L)]{\Gamma, \forall {x}.{A} \vdash \Psi, \Delta}
                   {\Gamma, {A}[{x}/{t}] \vdash \Psi, \Delta}
%
\qquad
%
\infer[(\forall R)]{\Gamma \vdash \Psi, \forall {x}.{A}, \Delta}
                   {\Gamma \vdash \Psi, {A}[{x}/{y}], \Delta}% \ \
%
%\parbox{11ex}{\smaller ${y}$ not free in$\Gamma, \Delta$ or $\Psi$}
% %\quad
%\infer[(\cn R)]{\Gamma \vdash \cn {A}, \Delta}{\Gamma, {A} \vdash \Delta}
%\quad
%\infer[(\cn L)]{\Gamma, \cn {A} \vdash \Delta}{\Gamma \vdash  {A}, \Delta}
$$
$$
\infer[(\ci L)]{\Gamma, {A}\ci {B} \vdash  \Psi, \Delta}
               {\Gamma \vdash {A}, \Delta&\Gamma, {B} \vdash  \Psi, \Delta} 
\qquad 
\infer[(\ci R)]{\Gamma \vdash \Psi,  {A}\ci {B}, \Delta}
               {\Gamma, {A} \vdash \Psi, {B}, \Delta}
\qquad 
\infer[(\bot)]{\Gamma, \bot \vdash \Psi, \Delta}{}%\Gamma, {A} \vdash \Psi, \Delta}
$$
\end{calculus}

\begin{clarifications}
A {\em term-sequent} is a pair $\Theta \col {t}$ where $\Theta$ is a tree and
${t}$ is a term. Define {\em trees} by: $\Theta::={t} \mid
\pa{\Theta_1, \Theta_2}$. If $\Theta'$ occurs as a {\em subtree} of $\Theta$ then
we write $\Theta$ as $\pa{\dots\Theta'\dots}$. A {\em sequent} has the form
$\Gamma \vdash \Psi, \Delta$ where $\Gamma$ and $\Delta$ are sets of formulae and
$\Psi$ is a set of term-sequents. ${y}$ must not be free in the lower
term-sequent of $(\lambda R)$ nor the lower sequent of $(\forall R)$.
\end{clarifications}

\begin{technicalities}
Cut elimination --- for both $(Cut)$ and $(Cut_{\lambda})$ --- is proved
in~\cite{mgabbay:lambdacut}, as is soundness and completeness of term-sequents
with respect to the calculus of untyped lambda reduction with $\beta$-reduction
and $\eta$-expansion. Soundness and completeness of the full calculus is shown
for an axiomatic presentation of a stronger system in~\cite{gabbay:simcks}
(using a model theory of lambda reduction similar to {\em graph models} which is
expanded further in~\cite{gabbay:simcmt,Gabbay2016}).
\end{technicalities}

\end{entry} 
