
\calculusName{Intuitionistic Sequent Calculus with $\varepsilon$-terms} 
\calculusAcronym{\LJstar}
\calculusLogic{Intuitionistic Logic}
\calculusLogicOrder{First-Order}
\calculusType{Sequent Calculus}
\calculusYear{2016} 
\calculusAuthor{Giselle Reis} 
\calculusAuthor{Bruno Woltzenlogel Paleo} 

\entryTitle{Epsilon-Sound Sequent Calculus \LJstar}
\entryAuthor{Giselle Reis}
\entryAuthor{Bruno Woltzenlogel Paleo}

\etag{Two-Sided Sequents}
\etag{Multiset Cedents}
\etag{Single-Conclusion Succedent}

\maketitle


\begin{entry}{LJstar}  


\begin{calculus}
\vspace{5pt}
\[
\infer[\vee_l]{\Gamma_1, \Gamma_2, A \vee B \seq F}{\Gamma_1 , A \seq F & \Gamma_2, B \seq F}
\quad
\infer[\vee_r^1]{\Gamma \seq A \vee B}{\Gamma \seq A}
\quad
\infer[\vee_r^2]{\Gamma \seq A \vee B}{\Gamma \seq B}
\qquad
\infer[\neg_l]{\Gamma, \neg A \seq }{\Gamma \seq A}
\qquad
\infer[\neg_r]{\Gamma \seq \neg A}{\Gamma, A \seq}
\]\\[2pt]
\[
\infer[\wedge_l]{\Gamma, A \wedge B \seq F}{\Gamma, A, B \seq F}
\quad
\infer[\wedge_r]{\Gamma_1, \Gamma_2 \seq A \wedge B}{\Gamma_1 \seq A & \Gamma_2 \seq B}
\quad
\infer[\imp_l]{\Gamma_1, \Gamma_2,  A \imp B \seq F}{\Gamma_1 \seq A & \Gamma_2, B \seq F}
\qquad
\infer[\imp_r]{\Gamma \seq A \imp B}{\Gamma, A \seq B}
\]\\[2pt]
\[
\infer[w_l]{\Gamma, A \seq F}{\Gamma \seq F}
\quad
\infer[w_r]{\Gamma \seq A}{\Gamma \seq}
\quad
\infer[c_l]{\Gamma, A \seq F}{\Gamma, A, A \seq F}
\quad
\infer[cut]{\Gamma_1, \Gamma_2 \vdash F}{\Gamma_1 \vdash A & \Gamma_2, A \vdash F}
\qquad
\infer[a]{A[\nu^l_x~F] \vdash A[\nu^l_x~F]}{}
\]\\[2pt]
\[
\infer[\all_r]{\Gamma \seq \all x. A[x]}{\Gamma \seq A[\alpha]}
\qquad
\infer[\ex_l]{\Gamma, \ex x. A[x]  \seq F}{\Gamma, A[\alpha] \seq F}
\qquad
\infer[\all_l']{\Gamma, \all x. A[x]  \seq F}{\Gamma, A[t] \seq F}
\qquad
\infer[\ex_r']{\Gamma \seq \ex x. A[x]}{\Gamma \seq A[t]}
\]\\[2pt]
\centering
$\nu$ denotes the binders $\varepsilon$ or $\tau$. 
The term $t$ must be accessible in the conclusion sequent (\emph{accessibility
condition}).
Accessible occurrences of $t$ or any of its $\varepsilon$-subterms in $\Gamma$ and $F$
must have a constant as a label (\emph{label condition}).
$l$ is a constant in $a$ (\emph{initial condition}).
\end{calculus}


\begin{clarifications}
A term $t$ is \emph{accessible} in a formula $F$ iff at least one of the
following two conditions hold: 
\begin{itemize}
  \item for any top-level (i.e., not nested inside another $\varepsilon$-term)
  $\varepsilon$-term $\nu_x~G$ in $t$, it is the case that $F[\nu_x~G
  \rightsquigarrow x]$ is a sub-formula of $G$ ($\rightsquigarrow$ denotes term
  rewriting); or

  \item $t$ contains a nested $\varepsilon$-term $\nu_y~H$ such that $\nu_y~H$
  is accessible in $F$ and $t[\nu_y~H \rightsquigarrow y]$ is accessible in
  $F[\nu_y~H \rightsquigarrow y]$.
\end{itemize}
$t$ is accessible in a sequent $S$ iff all top-level
$\varepsilon$-terms in $t$ are accessible in some formula occurring in $S$.
%
$\varepsilon$-terms replace strong quantifiers in formulas (i.e., the ones
that require eigenvariables).
Intuitively, term accessibility corresponds to the availability of the
eigenvariable in a proof of the non-epsilonized formula.
%
Labels on $\varepsilon$-terms are needed for making proof epsilonization injective, allowing de-epsilonization of proofs.
\end{clarifications}

\begin{history}
Skolemization is known to be unsound in intuitionistic logic. \emph{Epsilonization} is similar to skolemization, but replaces strongly quantified variables by $\varepsilon$-terms, instead of Skolem terms. \LJstar was introduced in~\cite{Reis2016} as a sequent calculus for
intuitionistic logic where \emph{epsilonization} is sound: if the
epsilonization of a sequent $S$ is derivable in \LJstar, then $S$ is derivable in $\LJ$.
%
This is achieved by restricting the rules $\forall_l$, $\exists_r$ and initial
from \LJ~\iref{GentzenLJ} to take into account information available in
$\varepsilon$-terms.
\end{history}

\begin{technicalities}
\LJstar is sound and complete with respect to \LJ~\iref{GentzenLJ} for
$\varepsilon$-free formulas. A procedure for de-epsilonizing \LJstar proofs is
defined in~\cite{Reis2016} resulting in a valid \LJ proof.
\end{technicalities}

\end{entry}
