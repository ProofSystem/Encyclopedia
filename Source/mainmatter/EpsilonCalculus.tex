\newcommand{\EC}{\ensuremath{\mathbf{EC}}\xspace}
\newcommand{\ECeps}{\ensuremath{\EC_\varepsilon}\xspace}

\calculusName{Epsilon Calculus}
\calculusAcronym{}
\calculusLogic{Classical Logic}
\calculusLogicOrder{First-Order}
\calculusType{Axiomatic}
\calculusYear{1923, 1939}

\calculusAuthor{David Hilbert}
%\calculusAuthor{Paul Bernays}

\entryTitle{Epsilon Calculus}
\entryAuthor{Kenji Miyamoto}     
\entryAuthor{Georg Moser}


\maketitle

\begin{entry}{EpsilonCalculus}  


\begin{calculus}
Epsilon calculus is first-order predicate calculus extended by the
epsilon-operator and the critical axiom.  Terms $t$ and Formulas $A,
B$ of epsilon calculus are defined as follows.
\begin{align*}
  & t ::= a ~|~ x ~|~ f(t_0, \ldots, t_{n-1}) ~|~ \varepsilon_x A, 
  && A, B ::= P(t_0, \ldots, t_{n-1}) ~|~ \neg A ~|~ A \land B ~|~ A \lor B ~|~ A \to B ~|~ \exists x. A ~|~ \forall x. A,
\end{align*}
where $a$, $x$, $f$, and $P$ range over free variables, bound
variables, function symbols, and predicate symbols, respectively.
Each $f$, $P$ has an arbitrary arity $n$.
The critical axiom is, for any formula $A(x)$, given as follows.
\begin{align*}
  A(t) \to A(\varepsilon_xA(x)).
\end{align*}
\end{calculus}

\begin{clarifications}
  \emph{Epsilon calculus} is an extension of classical first-order
  predicate
  calculus~\cite{HilbertBernays1939,MoserZach06,AvigadZach13}.  The
  symbol $\varepsilon$ is called the \emph{epsilon-operator}, which
  constructs a term by quantifying a bound variable in a formula.  A
  formula $A$ with occurrences of a variable $x$ which is not
  quantified is written as $A(x)$, and $A(t)$ denotes a formula
  obtained by replacing the corresponding $x$ by a term $t$ in $A$.
  The existential and universal quantifiers are definable due to the
  epsilon-operator.
  \begin{align*}
    & \exists x. A(x) := A(\varepsilon_x A(x)), & \forall x. A(x) := A(\varepsilon_x \neg A(x)).
  \end{align*}
  \emph{Pure epsilon calculus} is elementary calculus extended by the
  epsilon-operator and the critical axiom.
\end{clarifications}

  %% Hilbert formulated the prototype of epsilon
  %% calulus~\cite{Hilbert1923} by means of the \emph{tau-operator} and
  %% the axiom \(A(\tau_x A(x)) \to A(t)\) instead of the epsilon-operator
  %% and the critical axiom.  The standard formulation based on the
  %% epsilon-operator in the book by Hilbert and
  %% Bernays~\cite{HilbertBernays1939} first appeared in Ackermann's
  %% dissertation~\cite{Ackermann1924} under the supervision of
  %% Hilbert.
  %%

\begin{history}
  Epsilon calculus is due to Hilbert.  He formulated the prototype of
  epsilon calculus~\cite{Hilbert1923} by means of the
  \emph{tau-operator} and the axiom \(A(\tau_x A(x)) \to A(t)\) instead
  of the epsilon-operator and the critical axiom.  The formulation based
  on the epsilon-operator first appeared in Ackermann's
  dissertation~\cite{Ackermann1924} under the supervision of Hilbert.
  Hilbert and Bernays gave a comprehensive account of epsilon calculus
  and its applications~\cite{HilbertBernays1939}.
\end{history}

\begin{technicalities}
  \emph{First epsilon theorem} states that if there is a proof in
  epsilon calculus of an $\exists,\forall,\varepsilon$-free formula,
  this formula is provable in elementary calculus.  \emph{Second epsilon
  theorem} states that if there is a proof in epsilon calculus of an
  $\varepsilon$-free formula, this formula is provable in predicate
  calculus.  By means of epsilon calculus Hilbert and Bernays gave the
  first correct proof of \emph{Herbrand's
  theorem}~\cite{HilbertBernays1939,MoserZach06}.
\end{technicalities}

\end{entry}
