\newcommand{\EC}{\ensuremath{\mathbf{EC}}\xspace}
\newcommand{\ECeps}{\ensuremath{\EC_\varepsilon}\xspace}

\calculusName{Epsilon Calculus}         % The name of the calculus
\calculusAcronym{}          % The acronym if defined above, or empty otherwise. 
\calculusLogic{Classical Logic}        % Specify the logic (e.g. Classical Logic, Intuitionistic Logic, ...) for which this calculus is intended.
\calculusLogicOrder{First-Order}   % Specify the order of the logic (e.g. Propositional, Quantified Propositional, First-Order, Higher-Order, ...).
\calculusType{Hilbert Calculus}         % Specify the calculus type (e.g. Tableau, Sequent Calculus, Hyper-Sequent Calculus, Natural Deduction, ...)
\calculusYear{1939}         % The year when the calculus was published.

\calculusAuthor{David Hilbert}       % The name(s) of the author(s) of the calculus.
\calculusAuthor{Paul Bernays}

\entryTitle{Epsilon Calculus}     % Title of the entry (usually coincides with the name of the calculus).
\entryAuthor{Kenji Miyamoto}     
\entryAuthor{Georg Moser}


\maketitle

\begin{entry}{EpsilonCalculus}  


\begin{calculus}

Epsilon calculus is first-order predicate calculus extended by the
epsilon-operator and the critical axiom.  Terms $t$ and Formulas $A,
B$ of epsilon calculus are defined as follows.
\begin{align*}
  & t ::= a ~|~ x ~|~ f(t_0, \ldots, t_{n-1}) ~|~ \varepsilon_x A, 
  && A, B ::= P(t_0, \ldots, t_{n-1}) ~|~ \neg A ~|~ A \land B ~|~ A \lor B ~|~ A \to B ~|~ \exists x. A ~|~ \forall x. A,
\end{align*}
where $a$, $x$, $f$, and $P$ range over free variables, bound
variables, function symbols, and predicate symbols, respectively.
Each $f$, $P$ has an arbitrary arity $n$.
The critical axiom is, for any formula $A(x)$, given as follows.
\begin{align*}
  A(t) \to A(\varepsilon_xA(x)).
\end{align*}
\end{calculus}

\begin{clarifications}
  \emph{Epsilon calculus} is an extension of classical first-order
  predicate
  calculus~\cite{HilbertBernays1939,MoserZach06,AvigadZach13}.  The
  symbol $\varepsilon$ is called \emph{epsilon-operator}, which
  constructs a term by quantifying a bound variable in a formula.  A
  formula $A$ with occurrences of a variable $x$ which is not
  quantified is written as $A(x)$, and $A(t)$ denotes a formula
  obtained by replacing the corresponding $x$ by a term $t$ in $A$.
  The existential and universal quantifiers are definable due to the
  epsilon-operator.
\begin{align*}
  & \exists x. A(x) := A(\varepsilon_x A(x)), & \forall x. A(x) := A(\varepsilon_x \neg A(x)).
\end{align*}
\emph{Pure epsilon calculus} is elementary calculus extended by the
epsilon-operator and the critical axiom.

\end{clarifications}

\begin{history}
Epsilon calculus first appeared in the
dissertation~\cite{Ackermann1924} by Ackermann.
\end{history}

\begin{technicalities}
\emph{First epsilon theorem} states that if there is a proof in
epsilon calculus of an $\exists,\forall,\varepsilon$-free formula,
this formula is provable in elementary calculus.  \emph{Second epsilon
  theorem} states that if there is a proof in epsilon calculus of an
$\varepsilon$-free formula, this formula is provable in predicate
calculus.  By means of epsilon calculus Hilbert and Bernays gave the
first correct proof of \emph{Herbrand's theorem}.
\end{technicalities}


% General Instructions:
% =====================

% The preferred length of an entry is 1 page. 
% Do the best you can to fit your proof system in one page.
%
% If you are finding it hard to fit what you want in one page, remember:
%
%   * Your entry needs to be neither self-contained nor fully understandable
%     (the interested reader may consult the cited full paper for details)
%
%   * If you are describing several proof systems in one entry, 
%     consider splitting your entry.
%
%   * You may reduce the size of your entry by ommitting inference rules
%     that are already described in other entries.
%
%   * Cite parsimoniously (see detailed citation instructions below).
%
% 
% If you do not manage to fit everything in one page, 
% it is acceptable for an entry to have 2 pages.
%
% For aesthetic reasons, it is preferable for an entry to have
% 1 full page or 2 full pages, in order to avoid unused blank space.



% Citation Instructions:
% ======================

% Please cite the original paper where the proof system was defined.
% To do so, you may use the \cite command within 
% one of the optional environments above,
% or use the \nocite command otherwise.

% You may also cite a modern paper or book where the 
% proof system is explained in greater depth or clarity.
% Cite parsimoniously.

% Do not cite related work. Instead, use the "\iref" or "\irefmissing" 
% commands to make an internal reference to another entry, 
% as explained within the "history" environment above.

% You do not need to create the "References" section yourself. 
% This is done automatically.


% Remove all instruction comments before submitting.


% Leave an empty line above "\end{entry}".

\end{entry}
