

\calculusName{Intuitionistic Sequent Calculus} 
\calculusAcronym{\LJ}  
\calculusLogic{Intuitionistic Logic}
\calculusLogicOrder{First-Order}
\calculusType{Sequent Calculus}
\calculusYear{1935} 
\calculusAuthor{Gerhard Karl Erich Gentzen} 

\entryTitle{Intuitionistic Sequent Calculus \LJ}
\entryAuthor{Giselle Reis} 

\etag{Two-Sided Sequents}
\etag{Multiset Cedents}
\etag{Single-Conclusion Succedent}


\maketitle

\begin{entry}{GentzenLJ}  

\begin{calculus}

\[
\begin{array}{cc}
\infer{A \vdash A}{}
&
\infer[cut]{\Gamma, \Delta \vdash \Theta}{\Gamma \vdash A & A, \Delta \vdash \Theta}
\\[8pt]
\infer[w_l]{A, \Gamma \vdash \Theta}{\Gamma \vdash \Theta}
&
\infer[w_r]{\Gamma \vdash A}{\Gamma \vdash}
\\[8pt]
\infer[e_l]{\Gamma, A, B, \Delta \vdash \Theta}{\Gamma, B, A, \Delta \vdash \Theta}
&
\infer[c_l]{A, \Gamma \vdash \Theta}{A, A, \Gamma \vdash \Theta}
\\[8pt]
\infer[\neg_l]{\neg A, \Gamma \vdash }{\Gamma \vdash A}
&
\infer[\neg_r]{\Gamma \vdash \neg A}{A, \Gamma \vdash }
\\[8pt]
\infer[\wedge_{l}]{A_1 \wedge A_2, \Gamma \vdash \Theta}{A_i, \Gamma \vdash \Theta}
&
\infer[\wedge_r]{\Gamma \vdash A \wedge B}{\Gamma \vdash A & \Gamma \vdash B}
\\[8pt]
\infer[\vee_l]{A \vee B, \Gamma \vdash \Theta}{A, \Gamma \vdash \Theta & B, \Gamma \vdash \Theta}
&
\infer[\vee_{r}]{\Gamma \vdash A_1 \vee A_2}{\Gamma \vdash A_i}
\\[8pt]
\infer[\rightarrow_l]{A \rightarrow B, \Gamma, \Delta \vdash \Theta}{\Gamma \vdash A & B, \Delta \vdash \Theta}
&
\infer[\rightarrow_r]{\Gamma \vdash A \rightarrow B}{A, \Gamma \vdash B}
\\[8pt]
\infer[\exists_l]{\exists x.A[x], \Gamma \vdash \Theta}{A[\alpha], \Gamma \vdash \Theta}
&
\infer[\exists_r]{\Gamma \vdash \exists x.A[x]}{\Gamma \vdash A[t]}
\\[8pt]
\infer[\forall_l]{\forall x.A[x], \Gamma \vdash \Theta}{A[t], \Gamma \vdash \Theta}
&
\infer[\forall_r]{\Gamma \vdash \forall x.A[x]}{\Gamma \vdash A[\alpha]}
\\
\end{array}
\]

\centering
The eigenvariable $\alpha$ should not occur in $\Gamma$, $\Theta$ or $A[x]$. \\ 
The term $t$ should not contain variables bound in $A[t]$.
\end{calculus}


\begin{clarifications}
Gentzen introduced the sequent calculi \LK~\iref{GentzenLK} and \LJ for
classical and intuitionistic logics respectively. The rules in both systems
have the same shape, but in \LJ they may have at most one formula in the
succedent (right side of $\vdash$).
%
This restriction is equivalent to
forbidding the axiom of excluded middle in natural deduction.
%It is common to consider \LJ without the exchange rule $e_l$ just by
%interpreting $\Gamma$ and $\Theta$ as multi-sets of formulas instead of lists.
\end{clarifications}

% \begin{history}
% \end{history}

\newcommand{\LHJ}{\ensuremath{\mathbf{LHJ}}\xspace}

\begin{technicalities}
The cut rule is eliminable (\emph{Hauptsatz}~\cite{Gentzen1935}), and hence
intuitionistic predicate logic is consistent and its propositional fragment is
decidable~\cite{Gentzen1935a}. \LJ is complete relative to \NJ~\iref{GentzenNJ}
and sound relative to the Hilbert-style calculus
\LHJ~\irefmissing{ToDo}~\cite{Gentzen1935a}.
\end{technicalities}


\end{entry}
