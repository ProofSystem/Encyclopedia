
\calculusName{Synthetic Tableaux}   
\calculusAcronym{}     
\calculusLogic{Classical Logic}  
\calculusLogicOrder{Propositional}
\calculusType{Tableau}   
\calculusYear{2000}   
\calculusAuthor{Mariusz Urba\'nski} 


\entryTitle{Synthetic Tableaux}     
\entryAuthor{Dorota Leszczy\'nska-Jasion}     

\maketitle



\begin{entry}{SyntheticTableaux}

\begin{calculus}

The synthesizing rules are:

\begin{center}
\begin{tabular}{ccccccc}
&&&&$A~~~~$&&  \\

$$
\infer[\mathbf{r^1_{\rightarrow}}]{A \rightarrow B}{\lnot A}
$$
&~~~~&
$$
\infer[\mathbf{r^2_{\rightarrow}}]{A \rightarrow B}{B}
$$
&~~~~&
$$
\infer[\mathbf{r^3_{\rightarrow}}]{\lnot (A \rightarrow B)}{\lnot B}
$$

&&\\
&&&&&&\\

&&&&$\lnot A~~~~$&& \\

$$
\infer[\mathbf{r^1_{\lor}}]{A \lor B}{A}
$$
&&
$$
\infer[\mathbf{r^2_{\lor}}]{A \lor B}{B}
$$
&&
$$
\infer[\mathbf{r^3_{\lor}}]{\lnot (A \lor B)}{\lnot B}
$$

&&\\
&&&&&&\\

&&&&$A~~~~$&& \\

$$
\infer[\mathbf{r^1_{\land}}]{\lnot (A \land B)}{\lnot A}
$$
&&
$$
\infer[\mathbf{r^2_{\land}}]{\lnot (A \land B)}{\lnot B}
$$
&&
$$
\infer[\mathbf{r^3_{\land}}]{A \land B}{B}
$$

&&

$$
\infer[\mathbf{r_{\neg}}]{\lnot \lnot A}{A}
$$
\\

\end{tabular}
\end{center}

\smallskip

The premises of rules $\mathbf{r^3_{\rightarrow}}$, $\mathbf{r^3_{\lor}}$, $\mathbf{r^3_{\land}}$ may occur in any order.

\smallskip

The branching rule:

\begin{center}

\Tree[.{} {$p_i$} {$\lnot p_i$} ]

\end{center}

\end{calculus}

\begin{clarifications}
A Synthetic Tableau for a formula $A$ is a finite tree with the following properties: the tree is generated by the above rules (the root is empty), each formula labelling a node of the tree is a subformula of $A$ or the negation of a subformula of $A$, each leaf is labelled with $A$ or $\lnot A$. The tableau is a proof of $A$ if each leaf is labelled with $A$.\end{clarifications}

\begin{history}
The method has been first presented in \cite{Urbanski2001a}, \cite{Urbanski2001b}, \cite{Urbanski2002a}. In \cite{Urbanski2002a}, \cite{Urbanski2002b} and \cite{Urbanski2004} it is also presented for some extensional many-valued logics and for some paraconsistent logics.
\end{history}

\begin{technicalities}
The method is sound and complete with respect to Classical Propositional Logic and constitutes a decision procedure for CPL. The same holds with respect to the non-classical logics for which the method has been described, see \cite{Urbanski2002a}, \cite{Urbanski2002b}, \cite{Urbanski2004}.
\end{technicalities}













\end{entry}
