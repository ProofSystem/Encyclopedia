% If the calculus has an acronym, define it.
% (e.g. \newcommand{\LK}{\ensuremath{\mathbf{LK}}\xspace})

\calculusName{Sequent Systems for Negative Modalities}         % The name of the calculus
\calculusAcronym{}          % The acronym if defined above, or empty otherwise.
\calculusLogic{Negative Modal Logics}        % Specify the logic (e.g. Classical Logic, Intuitionistic Logic, ...) for which this calculus is intended.
\calculusLogicOrder{Propositional}   % Specify the order of the logic (e.g. Propositional, Quantified Propositional, First-Order, Higher-Order, ...).
\calculusType{Sequent Calculus}         % Specify the calculus type (e.g. Tableau, Sequent Calculus, Hyper-Sequent Calculus, Natural Deduction, ...)
\calculusYear{2016}         % The year when the calculus was published.

\calculusAuthor{Ori Lahav}       % The name(s) of the author(s) of the calculus.
\calculusAuthor{Jo\~ao Marcos}
\calculusAuthor{Yoni Zohar}


\entryTitle{Sequent Systems for Negative Modalities}     % Title of the entry (usually coincides with the name of the calculus).
\entryAuthor{Yoni Zohar}
%\entryAuthor{ToDo:FullNameAuthor3}

% The encyclopedia's peer-reviewing policy is described here:
% http://proofsystem.github.io/Encyclopedia/
%
% Reviewers of this entry will be acknowledged in the following lines:
% \entryReviewer{Reviewer 1's name}
% \entryReviewer{Reviewer 2's name}
% \entryReviewer{Reviewer 3's name}
%
% The lines above will be filled by the coordinators.
% If you would like to indicate people
% who could review your entry, contact the coordinators.


% If you wish, use tags to give any other information
% that might be helpful for classifying and grouping this entry:
% e.g. \etag{Two-Sided Sequents}
% e.g. \etag{Multiset Cedents}
% e.g. \etag{List Cedents}
% You are free to invent your own tags.
% The Encyclopedia's coordinator will take care of
% merging semantically similar tags in the future.


\maketitle


% If your files are called "MyProofSystem.tex" and "MyProofSystem.bib",
% then you should write "\begin{entry}{MyProofSystem}" in the line below
\begin{entry}{negmodal}

% Define here any newcommands you may need:
% e.g. \newcommand{\necessarily}{\Box}
% e.g. \newcommand{\possibly}{\Diamond}
\makeatletter
\newcommand\incircbin
{%
  \mathpalette\@incircbin
}
\newcommand\@incircbin[2]
{%
  \mathbin%
  {%
    \ooalign{\hidewidth$#1#2$\hidewidth\crcr$#1\bigcirc$}%
  }%
}
\newcommand{\oland}{\incircbin{\land}}
\makeatother
\newcommand{\G}{{\bf G}}
\newcommand{\name}[1]{{\bf (#1)}}
\newcommand{\su}{\supset}
\newcommand{\ssrul}[2]{\begin{array}{c}#1\\ \hline #2\end{array}}
\newcommand{\ddrul}[3]{\begin{array}{c}#1\hspace{2em}#2\\
\hline #3\end{array}}
\newcommand{\Ga}{\Gamma}
\newcommand{\De}{\Delta}
\newcommand{\Ra}{\Rightarrow}
\newcommand{\srul}[4]{\ssrul{\Ga #1\Ra\De #2}{\Ga
#3\Ra\De #4}}
\newcommand{\drul}[6]{\ddrul{\Ga #1\Ra\De #2}{\Ga #3\Ra\De #4}{\Ga
#5\Ra\De #6}}
\newcommand{\w}{\wedge}

\newcommand{\ineg}{{\smallsmile}}
\newcommand{\uneg}{{\smallfrown}}
%\newcommand{\wsmile}{\scalebox{1}{$\mathrlap{\Circle}\textcolor{black}{\smallsmile}$}}
%\newcommand{\wfrown}{\scalebox{1}{$\mathrlap{\Circle}\textcolor{black}{\smallfrown}$}}
%\newcommand*\circled[1]{\tikz[baseline=(char.base)]{   \node[shape=circle,draw,inner sep=1pt] (char) {#1};}}
%\newcommand{\wsmile}{{\circled{\ineg}}}
%\newcommand{\wfrown}{{\circled{\uneg}}}
%\newcommand{\wsmile}{{\textcircled{\ineg}}}
%\newcommand{\wfrown}{{\textcircled{\uneg}}}
\newcommand{\wsmile}{{\incircbin{\ineg}}}
\newcommand{\wfrown}{{\incircbin{\uneg}}}

\newcommand{\GPK}{{\rm {PK}}}
\newcommand{\GLK}{{\rm {LK}}}
\newcommand{\GPKB}{{\rm {PKB}}}
\newcommand{\GPKDB}{{\rm {PKDB}}}
\newcommand{\GPKD}{{\rm {PKD}}}
\newcommand{\GPKT}{{\rm {PKT}}}
\newcommand{\GPKF}{{\rm {PKF}}}
\newcommand{\GPKFour}{{\rm {PK4}}}
\newcommand{\GPKFourD}{{\rm {PKD4}}}
\newcommand{\KFour}{{\bf 4}}
\newcommand{\g}{\Gamma}
\renewcommand{\d}{\Delta}
\newcommand{\B}{{\bf B}}
\newcommand{\K}{{\bf K}}
\newcommand{\T}{{\bf T}}
\newcommand{\FUNC}{{\bf Fun}}
\newcommand{\Di}{{\bf D}}
\newcommand{\mostbasic}{{\rm {P}}}

\begin{calculus}

% Add the inference rules of your proof system here.
% The "proof.sty" and "bussproofs.sty" packages are available.
% If you need any other package, please contact the coordinator (Bruno Woltzenlogel Paleo <bruno.wp@gmail.com>)

\renewcommand{\arraystretch}{2}
\centering\footnotesize
%\begin{tabular}{ccc@{\hspace{2em}}cc}
\begin{tabular}{ccc@{\hspace{1em}}ccc@{\hspace{1em}}ccc@{\hspace{1em}}cc}
~&&&
${[{\ineg}{\Ra}]}$
& $\ssrul{\g\Ra\varphi, \d}{\uneg\d,\ineg\varphi\Ra \ineg\g} $
&&
${[{\Ra}{\uneg}]} $
& $\ssrul{\g,\varphi\Ra \d}{\uneg\d\Ra \uneg\varphi,\ineg\g}$
\\
${[{\Ra}{{\wsmile}}]}$
& $\ssrul{\g,\varphi,\ineg\varphi\Ra \d}{\g\Ra{\wsmile}\varphi,\d}$
%\\%[1mm]
&&
${[{{\wsmile}}{\Ra}]}$
& $\ssrul{\g\Ra\varphi, \d\ \ \ \g\Ra\ineg\varphi,\d}
{\g,{\wsmile}\varphi\Ra\d}$
&&
${[{\Ra}{\wfrown}]}$
& $\ssrul{\g,\varphi\Ra \d\ \ \ \g,\uneg\varphi\Ra\d}{\g\Ra\wfrown\varphi,\d}$
&&
${[{{\wfrown}}{\Ra}]}$
& $\ssrul{\g\Ra\varphi,\uneg\varphi,\d}
{\g,\wfrown\varphi\Ra\d}$
\\
$[\Di]$
& $\ssrul{\Gamma\Ra\Delta}{\uneg\Delta\Ra\ineg\Gamma}$
 &&
 $[\FUNC]$
 &$\ssrul{\Gamma\Ra\Delta}{\ineg\Delta\Ra\ineg\Gamma}$
%\\
&&
$[\T_{1}]$
& $\ssrul{\Gamma,\varphi\Ra\Delta}{\Gamma\Ra\ineg\varphi,\Delta}$
&&
 $[\T_{2}]$
 & $\ssrul{\Gamma\Ra\varphi,\Delta}{\Gamma,\uneg\varphi\Ra\Delta}$
\\
&&&
$[\B_{1}]$
& $\ssrul{\Gamma,\ineg\Gamma',\varphi\Ra\Delta,\uneg\Delta'}{\uneg\Delta,\Delta'\Ra\uneg\varphi,\ineg\Gamma,\Gamma'}$
&&
 $[\B_{2}]$
 & $\ssrul{\Gamma,\ineg\Gamma'\Ra\varphi,\Delta,\uneg\Delta'}
 {\uneg\Delta,\Delta',\ineg\varphi\Ra\ineg\Gamma,\Gamma'}$
 \\
 &&&
$[\KFour_{1}]$
& $\ssrul{\uneg\Gamma,\Gamma',\varphi\Ra\ineg\Delta,\Delta'}{\uneg\Gamma,\uneg\Delta'\Ra\uneg\varphi,\ineg\Delta,\ineg\Gamma'}$
&&
$[\KFour_{2}]$
& $\ssrul{\uneg\Gamma,\Gamma'\Ra\varphi,\ineg\Delta,\Delta'}{\uneg\Gamma,\uneg\Delta',\ineg\varphi\Ra\ineg\Delta,\ineg\Gamma'}$
\\
&&&
$[\Di_{B}]$
& $\ssrul{\ineg\Gamma',\Gamma\Ra\Delta,\uneg\Delta'}{\Delta',\uneg\Delta\Ra\ineg\Gamma,\Gamma'}$
&&
$[\Di_{4}]$
& $\ssrul{\uneg\Gamma',\Gamma\Ra\Delta,\ineg\Delta'}{\uneg\Gamma',\uneg\Delta\Ra\ineg\Gamma,\ineg\Delta'}$
\\
\end{tabular}
\normalsize



\end{calculus}

% The following sections ("clarifications", "history",
% "technicalities") are optional. If you use them,
% be very concise and objective. Nevertheless, do write full sentences.
% Try to have at most one paragraph per section, because line breaks
% do not look nice in a short entry.

 \begin{clarifications}
 $\mostbasic$ is obtained from propositional positive $\LK$
 by the addition of the rules for $\wsmile$ and $\wfrown$.
 {$\GPK=\mostbasic+[\uneg\Ra]+[\ineg\Ra]$,}
 {$\GPKD=\GPK+[\Di]$,}
 {$\GPKT=\GPK+[\T_{1}]+[\T_{2}]$,}
 {$\GPKF=\mostbasic+[\FUNC]+[\ineg=\uneg]$,}
 {$\GPKB=\mostbasic+[\B_{1}]+[\B_{2}]$,}
 {$\GPKFour=\mostbasic+[\KFour_{1}]+[\KFour_{2}]$,}
 {$\GPKDB=\GPKB+[\Di_{\B}]$,} and
 {$\GPKFourD=\GPKFour+[\Di_{\KFour}]$.}

 \end{clarifications}

 \begin{history}
All systems enjoy the replacement property,
 which is missing in various logics that employ a non-classical negation,
 such as Kleene three-valued logic, and da Costa's $C_{1}$.
$\GPK$ was introduced in \cite{dod:mar:ENTCS2013}. All other systems were introduced in \cite{Lahav2017}.
 \end{history}

 \begin{technicalities}
All systems enjoy a generalized subformula property,
in which $\ineg\varphi$ is considered a subformula of $\wsmile\varphi$
(and $\uneg\varphi$ is a subformula of $\wfrown\varphi$).
All systems that do not contain $\B$ also enjoy
cut-admissibility.
These facts were proven semantically, using
the general mechanism of \cite{Lahav:2013:USF:2555591.2528930}.
 \end{technicalities}


% Leave an empty line above "\end{entry}".

\end{entry}
