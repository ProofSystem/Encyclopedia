
\calculusName{Proper Multi-type Display Calculus for semi De Morgan Logic MtD.SDM}
\calculusAcronym{MtDSDM}
\calculusLogic{Semi De Morgan Logic}
\calculusLogicOrder{Propositional}
\calculusType{Multi-type Sequent Calculus}
\calculusYear{2017}
\calculusAuthor{Giuseppe Greco} \calculusAuthor{Fei Liang} \calculusAuthor{M. Andrew Moshier} \calculusAuthor{Alessandra Palmigiano}

\entryTitle{Proper Multi-type Display Calculus for semi De Morgan Logic MtD.SDM}
\entryAuthor{Fei Liang}

\etag{Two-Sided Sequents}
\etag{Multi-type Cedents}
\etag{Multi-Premise Cedent}
\etag{Multi-Conclusion Cedent}
\etag{Proper Multi-type Dispaly Calculus}

\def\fCenter{{\mbox{$\ \vdash\ $}}}
\def\ffCenter{{\mbox{$\overline{\vdash}$}}}
\def\fcenter{{\mbox{\ }}}
\EnableBpAbbreviations

\def\fns{\footnotesize}
\def\mc{\multicolumn}

\def\aol{\rule[0.5865ex]{1.38ex}{0.1ex}}
\def\pdra{\mbox{$\,>\mkern-8mu\raisebox{-0.065ex}{\aol}\,$}}
\def\pdla{\mbox{\rotatebox[origin=c]{180}{$\,>\mkern-8mu\raisebox{-0.065ex}{\aol}\,$}}}


%Distributive Lattice
%Operational Connectives
\newcommand{\xtop}{\ensuremath{\top}\xspace}
\newcommand{\xbot}{\ensuremath{\bot}\xspace}
\newcommand{\xand}{\ensuremath{\wedge}\xspace}
\newcommand{\xor}{\ensuremath{\vee}\xspace}
\newcommand{\xrarr}{\ensuremath{\rightarrow}\xspace}
\newcommand{\xcrarr}{\ensuremath{\,{>\mkern-7mu\raisebox{-0.065ex}{\rule[0.5865ex]{1.38ex}{0.1ex}}}\,}\xspace}
%Structural Connectives
\newcommand{\XTOP}{\ensuremath{\hat{\top}}\xspace}
\newcommand{\XBOT}{\ensuremath{\check{\bot}}\xspace}
\newcommand{\XAND}{\ensuremath{\:\hat{\wedge}\:}\xspace}
\newcommand{\XOR}{\ensuremath{\:\check{\vee}\:}\xspace}
\newcommand{\XRARR}{\ensuremath{\:\check{\rightarrow}\:}\xspace}
\newcommand{\XCRARR}{\ensuremath{\hat{{\:{>\mkern-7mu\raisebox{-0.065ex}{\rule[0.5865ex]{1.38ex}{0.1ex}}}\:}}}\xspace}


%De Morgan
%Operational Connectives
\newcommand{\dneg}{\ensuremath{{\sim}}\xspace}
\newcommand{\rdneg}{\ensuremath{\neg}\xspace}
\newcommand{\dtop}{\ensuremath{1}\xspace}
\newcommand{\dbot}{\ensuremath{0}\xspace}
\newcommand{\dand}{\ensuremath{\cap}\xspace}
\newcommand{\dor}{\ensuremath{\cup}\xspace}
\newcommand{\drarr}{\ensuremath{\,{\raisebox{-0.065ex}{\rule[0.5865ex]{1.38ex}{0.1ex}}\mkern-5mu\supset}\,}\xspace}
\newcommand{\dcrarr}{\ensuremath{\,{\supset\mkern-5.5mu\raisebox{-0.065ex}{\rule[0.5865ex]{1.38ex}{0.1ex}}}\,}\xspace}
%Structural Connectives
\newcommand{\DNEG}{\ensuremath{\:\tilde{*}}\xspace}
\newcommand{\ADNEG}{\ensuremath{\circledast}\xspace}
\newcommand{\DTOP}{\ensuremath{\hat{1}}\xspace}
\newcommand{\DBOT}{\ensuremath{\check{0}}\xspace}
\newcommand{\DAND}{\ensuremath{\:\hat{\cap}\:}\xspace}
\newcommand{\DOR}{\ensuremath{\:\check{\cup}\:}\xspace}
\newcommand{\DRARR}{\ensuremath{\check{\,{\raisebox{-0.065ex}{\rule[0.5865ex]{1.38ex}{0.1ex}}\mkern-5mu\supset}\,}}\xspace}

\newcommand{\DCRARR}{\ensuremath{\hat{\,{\supset\mkern-5.5mu\raisebox{-0.065ex}{\rule[0.5865ex]{1.38ex}{0.1ex}}}\,}}\xspace}

%Heterogeneous connectives
%Operational connectives
\newcommand{\wbox}{\ensuremath{\Box}\xspace}
\newcommand{\wdia}{\ensuremath{\Diamonddot}\xspace}
\newcommand{\bcirc}{\ensuremath{{\bullet}}\xspace}
\newcommand{\bbox}{\ensuremath{\boxdot}\xspace}
\newcommand{\bdia}{\ensuremath{\Diamondblack}\xspace}
%Structural connectives
\newcommand{\WBOX}{\ensuremath{\:\check{\Box}\:}\xspace}
\newcommand{\WDIA}{\ensuremath{\:\hat{\Diamond}}\xspace}
\newcommand{\BBOX}{\ensuremath{\:\check{\blacksquare}}\xspace}
\newcommand{\BDIA}{\ensuremath{\:\hat{\Diamondblack}}\xspace}
\newcommand{\WCIR}{\ensuremath{\:\tilde{{\circ}}}\xspace}
\newcommand{\BCIR}{\ensuremath{\:\tilde{{\bullet}}}\xspace}


%%%Additional packages

%%%

\maketitle

\begin{entry}{MtDSDM}

\begin{calculus}
Identity and Cut rules:
\[
\AXC{Id}
\noLine
\UI$p \fCenter p$
\DisplayProof
\qquad
\AX$X \fCenter A$
\AX$A \fCenter Y$
\RightLabel{Cut}
\BI$X \fCenter Y$
\DisplayProof
\qquad
\AX$\Gamma \fCenter \alpha$
\AX$\alpha \fCenter \Delta$
\RightLabel{Cut}
\BI$\Gamma \fCenter \Delta$
\DisplayProof
\]
Display postulates and structural rule for pure $\mathsf{De Morgan}$ type
\[
\AX $\DNEG \Gamma \fCenter \Delta$
\LeftLabel{\scriptsize adj}
\doubleLine
\UI$\DNEG \Delta \fCenter \Gamma$
\DisplayProof
\quad
\AX $\Gamma \fCenter \DNEG \Delta$
\RightLabel{\scriptsize adj}
\doubleLine
\UI  $ \Delta \fCenter \DNEG\Gamma$
\DisplayProof
\qquad
\AX$\Gamma \fCenter  \Delta$
\RightLabel{\scriptsize cont}
\doubleLine
\UI$\DNEG \Delta \fCenter \DNEG \Gamma$
\DisplayProof				
\]


%%%

Multi-type display postulates and structural rules
\[
\AX $X  \fCenter \WBOX \Gamma$
\LeftLabel{\scriptsize adj}
\doubleLine
\UI$ \BDIA  X \fCenter \Gamma$
\DisplayProof
\quad
\AX $\WCIR X \fCenter \Gamma $
\RightLabel{\scriptsize adj}
\doubleLine
\UI$X \fCenter \BCIR \Gamma$
\DisplayProof
\]
\[
\AX $X  \fCenter Y$
\LeftLabel{\scriptsize \WCIR}
\UI$\WCIR X \fCenter \WCIR Y$
\DisplayProof
\quad
\AX $\BCIR \Gamma  \fCenter \BCIR \Delta$
\LeftLabel{\scriptsize \BCIR}
\UI$\Gamma \fCenter \Delta$
\DisplayProof
\quad
\AX $\Gamma  \fCenter \WCIR\WBOX\Delta$
\LeftLabel{\scriptsize {\WCIR}\WBOX}
\doubleLine
\UI $\Gamma \fCenter \Delta$
\DisplayProof
\quad
\AX $\DTOP \fCenter \Gamma$
\LeftLabel{\scriptsize {\BDIA}\DTOP}
\UI$\BDIA\XTOP \fCenter \Gamma$
\DisplayProof
\quad
\AX $X  \fCenter \WBOX\DBOT$
\LeftLabel{\scriptsize {\WBOX}\DBOT}
\UI$X \fCenter \XBOT$
\DisplayProof
\]
\end{calculus}

\begin{clarifications}
The language $\mathcal{L}_\mathrm{MT}(\mathcal{F}, \mathcal{G})$ of  MtD.SDM consists of  {\em logical}   and {\em structural terms} in the types $\mathsf{T}_1:  = \mathsf{De Morgan}$ and $\mathsf{T}_2: = \mathsf{Distributive}$. Following the notation of \iref{MtSC}, the set  of logical terms takes as parameters: 1) a denumerable  set of atomic terms $\mathsf{At}(\mathsf{Distributive})$, elements of which are denoted $p$, possibly with indexes; 2)  sets of connectives $\mathcal{F}: = \mathcal{F}_{\mathsf{De Morgan}}\uplus\mathcal{F}_{\mathsf{Distributive}}\uplus\mathcal{F}_{\mathrm{MT}}$ and $\mathcal{G}: = \mathcal{G}_{\mathsf{De Morgan}}\uplus\mathcal{G}_{\mathsf{Distributive}}\uplus\mathcal{G}_{\mathrm{MT}}$ defined as follows:
$\mathcal{F}_{\mathsf{De Morgan}}: = \{\cap, 1, \dneg\}$, $\mathcal{F}_{\mathsf{Distributive}}: = \{\wedge, \top \}$, $\mathcal{F}_{\mathrm{MT}}: = \{\wdia\}$, where  $n_1 = n_\top = 0$, $n_{\dneg} = n_{\wdia} = 1$, and $n_\cap = n_\wedge = 2$, and $\varepsilon_{\dneg}(1)=\partial$, and $\varepsilon_{\wdia}(1) = \varepsilon_{\cap}(i) = \varepsilon_{\wedge}(i) = 1$ for every $i\in \{1, 2\}$, and
$\mathcal{G}_{\mathsf{De Morgan}}: = \{\cup, 0, \rdneg\}$, $\mathcal{G}_{\mathsf{Distributive}}: = \{\vee, \bot \}$, $\mathcal{G}_{\mathrm{MT}}: = \{\wbox, \bbox\}$, where $n_0 = n_\bot = 0$, $n_\cup = n_\vee = 2$, and $n_{\rdneg} = n_{\wbox} = n_{\bbox} = 1$, and $\varepsilon_{\rdneg}(1)=\partial$, $\varepsilon_{\wbox}(1) =\varepsilon_{\bbox}(1) = \varepsilon_{\cup}(i) = \varepsilon_{\vee}(i) = 1$ for every $i\in \{1, 2\}$. The functional type of the heterogeneous connective $\wbox$ is $\mathsf{De Morgan}\rightarrow \mathsf{Distributive}$, while functional type of the heterogeneous connectives $\wdia$ and $\bbox$ is $\mathsf{Distributive}\rightarrow \mathsf{De Morgan}$.

The structural terms are built by means of structural connectives, taking logical terms as atomic structures. The set of structural connectives includes  $\XTOP,\, \XBOT,\, \XAND,\, \XOR,\, \DTOP,\, \DBOT, \, \DAND,\, \DOR,\,  \WBOX $   which are the structural counterparts of $\xtop, \xbot,  \xand, \xor,   \dtop, \dbot,    \dand, \dor, \wbox$, respectively.  It also includes $\DNEG$ as the structural counterpart of both $\dneg$ (when occurring in antecedent position) and $\rdneg$ (when occurring in succedent position) and $\WCIR$ as structural counterpart of both $\wdia$ (when occurring in antecedent position) and $\bbox$ (when occurring in succedent position). Finally, it includes  $\BDIA,\, \BCIR,\, \XCRARR,\, \XRARR,\, \DCRARR ,\, \DRARR $, where $\BDIA$ is the left adjoint of $\WBOX$, and $\XCRARR$ and $\DCRARR$ are the left  residuals of $\XOR$ and $\DOR$ respectively, and $\XRARR$ and $\DRARR$ are the right residuals of $\XAND$ and $\DAND$ respectively, and $\BCIR$ is right and left adjoint of $\WCIR$. Hence, the functional type of $\BCIR$ is  $\mathsf{De Morgan}\rightarrow \mathsf{Distributive}$, while the functional type of  $\BDIA$ is $\mathsf{Distributive}\rightarrow \mathsf{De Morgan}$.

Summing up, the well formed terms of MtD.SDM are generated by simultaneous induction as follows:
\begin{center}
{\small
\setlength{\tabcolsep}{0.4em}
\begin{tabular}{@{}lcl@{}}
$\mathsf{De Morgan}$ &  & $\mathsf{Distributive}$\\
%&&\vspace{-4pt}\\
$\alpha ::=  \,\dtop  \mid \dbot \mid \wdia A \mid \bbox A \mid   \dneg\alpha \mid \rdneg\alpha \mid \alpha \dand \alpha \mid \alpha \dor \alpha$ & & $A ::= \,p \mid \xtop \mid \xbot \mid \wbox \alpha  \mid A \xand A \mid A \xor A$ \\
% & &\vspace{-4pt} \\
$\Gamma ::= \, \DTOP  \mid \DBOT \mid \BDIA X \mid \WCIR X \mid \DNEG \Gamma \mid \Gamma \DAND \Gamma  \mid \Gamma \DOR \Gamma \mid \Gamma \DCRARR \Gamma  \mid \Gamma \DRARR \Gamma$ & & $X ::= \,\XTOP \mid \XBOT \mid \WBOX\Gamma \mid \BCIR\Gamma \mid  X \XAND X  \mid X \XOR X \mid X \XCRARR X \mid X \XRARR X$ \\
\end{tabular}
 }
\end{center}

The introduction rules instantiate the general template described in \iref{MtSC}, and hence are omitted. Also, the pure-type structural rules are the standard ones capturing negation-free classical propositional logic (for $\mathsf{Distributive}$) and De Morgan logic (for $\mathsf{De Morgan}$) and are also omitted.
\end{clarifications}


\begin{history}
Semi De Morgan logic, introduced in an algebraic setting by H.P.~Sankappanavar \cite{sankappanavar1987semi}, is designed to capture the salient features of intuitionistic negation in a paraconsistent setting. In \cite{ma2016sequent}  a cut-free G3-style sequent calculus is introduced for semi De Morgan logic, which includes introduction rules under the scope of structural negation. In \cite{greco2017multi}, the algebraic semantics for semi De Morgan logic is recast in a multi-type algebraic framework which provides the guidelines for the design of the proper multi-type display  calculus MtD.SDM.
\end{history}

\newcommand{\LHK}{\ensuremath{\mathbf{LHK}}\xspace}
\newcommand{\NK}{\ensuremath{\mathbf{NK}}\xspace}


\begin{technicalities}
%In semi De Morgan algebras, the equations $\neg a = \neg\neg\neg a$ and $\neg\neg(a \wedge b) = \neg\neg a \wedge \neg\neg b$ are not analytic inductive.
Every known axiomatization of semi De Morgan logic is not analytic inductive according to the definition of \cite{greco2016unified}. Hence, semi De Morgan logic cannot be captured by a single-type proper display calculus on the basis of any known axiomatization. The calculus above is sound and complete w.r.t. the equivalent heterogeneous presentation of semi De Morgan algebras introduced  in \cite{greco2017multi}; it is conservative, and enjoys the cut elimination and subformula property as immediate consequences of the general theory of multi-type calculi.
\end{technicalities}

\end{entry} 