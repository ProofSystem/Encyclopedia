
% If the calculus has an acronym, define it.
\newcommand{\EK}{\mathbf{E}^\mathsf{K}}

\calculusName{Socratic Proofs for Modal Propositional K}   
\calculusAcronym{\EK}     
\calculusLogic{Modal Logics}  
\calculusLogicOrder{Propositional}
\calculusType{Socratic Proof System}   
\calculusYear{2004}   
\calculusAuthor{Dorota Leszczy\'nska-Jasion} 


\entryTitle{Socratic Proofs for Modal Propositional K}     
\entryAuthor{Dorota Leszczy\'nska-Jasion}     


\etag{Right-Sided Sequents}
\etag{Sequence Cedents}
\etag{Labelled System}


\maketitle

\begin{entry}{ModalSocraticProofsK}  

\begin{calculus}
The rules of calculus $\EK$:

$$
\infer[\mathbf{R}_\alpha]{?(\Phi ~;~ \vdash S ~'~ (\alpha_1)^{\phi(i)} ~'~ T ~;~ \vdash S ~'~ (\alpha_2)^{\phi(i)} ~'~ T ~;~ \Psi)}{?(\Phi ~;~ \vdash S ~'~ (\alpha)^{\phi(i)} ~'~ T ~;~ \Psi)}
$$

\vspace{-0.3cm}

\[
\begin{array}{ccc}
\infer[\mathbf{R}_\beta]{?(\Phi ~;~ \vdash S ~'~ (\beta_1)^{\phi(i)} ~'~(\beta_2)^{\phi(i)} ~'~ T ~;~ \Psi)}{?(\Phi ~;~ \vdash S ~'~ (\beta)^{\phi(i)} ~'~ T ~;~ \Psi)}
&~~~~~&
\infer[\mathbf{R}_{\lnot \lnot}]{?(\Phi ~;~ \vdash S ~'~ (A)^{\phi(i)} ~'~ T ~;~ \Psi)}{?(\Phi ~;~ \vdash S ~'~ (\lnot \lnot A)^{\phi(i)} ~'~ T ~;~ \Psi)}
\end{array}
\]

\vspace{-0.4cm}

\[
\begin{array}{ccc}
\infer[\mathbf{R}_\mu]{?(\Phi ~;~ \vdash S ~'~ (\mu_0)^{\phi(i),j} ~'~ T ~;~ \Psi)}{?(\Phi ~;~ \vdash S ~'~ (\mu)^{\phi(i)} ~'~ T ~;~ \Psi)}
&~~~~~~~~~~&
\infer[\mathbf{R}_\pi]{?(\Phi ~;~ \vdash S ~'~ (\pi)^{\phi(i)} ~'~ (\pi_0)^{j} ~'~ T ~;~ \Psi)}{?(\Phi ~;~ \vdash S ~'~ (\pi)^{\phi(i)} ~'~ T ~;~ \Psi)}
\end{array}
\]

where:

\begin{center}
	\begin{tabular}{ccc|cccc|cc|cc}
			\hline 
		$\alpha$ & $\alpha_{1}$ & $\alpha_{2}$ & $\beta$ & $\beta_{1}$ & $\beta_{2}$ & $\beta^{*}_{1}$ & $\mu$ & $\mu_0$ & $\pi$ & $\pi_0$ \\ 
			\hline
		$\textit{A}\wedge B$ & $\textit{A}$ & $\textit{B}$ & $\neg(A\wedge B)$ & $\neg A$ & $\neg B$ & $A$ & $\Box A$ & $A$ & $\lnot \Box A$ & $\lnot A$ \\
			
		$\neg(A\vee B)$ & $\neg A$ & $\neg B$ & $\textit{A}\vee B$ & $\textit{A}$ & $\textit{B}$ & $\neg A$ & $\lnot \Diamond A$ & $\lnot A$ & $\Diamond A$ & $A$ \\
			
		$\neg(A \rightarrow B)$ & $\textit{A}$ & $\neg B$ & $\textit{A} \rightarrow B$ & $\neg A$ & $\textit{B}$ & $A$ \\
			\hline 
	\end{tabular}

\bigskip

$\phi(i)$ is a finite sequence of numerals ending with $i$ (an index of a formula)

$\phi(i), j$ is a concatenation of $\phi(i)$ and $\langle j \rangle$
\end{center}

\bigskip

In $\mathbf{R}_\mu$, numeral $j$ must be new with respect to the sequent distinguished in the premise. In $\mathbf{R}_\pi$, the pair $\langle i, j \rangle$ is present in the premise sequent.

\end{calculus}

\begin{clarifications}
The method of Socratic proofs is a method of transforming questions but with a
clear proof-theoretic interpretation (see
also~\iref{SocraticProofsCPL},~\iref{SocraticProofsFOL}). The rules act upon
right-sided sequents, with sequences of \textit{indexed} formulas in the
succedents. The indices store the semantic information. A Socratic proof starts
with a question concerning $?(\vdash (A)^1)$ and ends with a question based on a
sequence of basic sequents, where a \textit{basic sequent} is a sequent
containing indexed formulas of the forms $B^{\phi(i)}$, $(\lnot B)^{\psi(i)}$.
\end{clarifications}

\begin{history}
The proof system has been presented in~\cite{DLJ:2004}, the completeness proof may be found in~\cite{DLJ:2007}.
\end{history}

\begin{technicalities}
A sequent $\vdash (A)^1$ has a Socratic proof in $\EK$ iff $A$ is $\mathsf{K}$-valid.
\end{technicalities}

\end{entry}
