\calculusName{Pure Type Systems}   
\calculusAcronym{}    
\calculusLogic{Type Theory}  
\calculusLogicOrder{Higher-Order}
\calculusType{Natural Deduction}   
\calculusYear{1989}   
\calculusAuthor{Berardi} \calculusAuthor{Terlouw} \calculusAuthor{Barendregt} \calculusAuthor{Geuvers} 


\entryTitle{Pure Type Systems}     
\entryAuthor{Ali Assaf}     

\maketitle

\begin{entry}{PureTypeSystems}



\newcommand{\sorts}{\mathcal{S}}
\newcommand{\axioms}{\mathcal{A}}
\newcommand{\rules}{\mathcal{R}}
\newcommand{\variables}{\mathcal{V}}
\newcommand{\constants}{\mathcal{C}}
\newcommand{\terms}{\mathcal{T}}

\newcommand{\union}{\cup}

\newcommand{\Prop}{\ast}
\newcommand{\Type}{\square}
\newcommand{\Kind}{\triangle}

\newcommand{\entails}{\vdash}

\newcommand{\negvspace}{\vspace{-1ex}}

\begin{calculus}

\negvspace
\[ \infer[axiom\ ((c : s) \in \axioms)]{\entails c : s}{ } \qquad
   \infer[start\ (x \not\in \Gamma)]{\Gamma, x : A \entails x : A}{\Gamma \entails A : s} \]
\[
 \infer[weakening\ (x\not\in\Gamma)]{\Gamma, x : A \entails M : B}{\Gamma \entails M : B \qquad \Gamma \entails A : s} \qquad
 \infer[product\ ((s_1,s_2,s_3)\in\rules)]{\Gamma \entails \Pi x : A. B : s_3}{\Gamma \entails A : s_1 \qquad \Gamma, x : A \entails B : s_2} \]
\[ \infer[abstraction]{\Gamma \entails \lambda x: A. M : \Pi x : A. B}{\Gamma, x : A \entails M : B \qquad \Gamma \entails \Pi x : A. B : s} \qquad
\infer[application]{\Gamma \entails M~N : B[x \coloneqq N]}{\Gamma \entails M : \Pi x : A. B \qquad \Gamma \entails N : A} \]
\[ \infer[conversion]{\Gamma \entails M : B}{\Gamma \entails M : A \qquad \Gamma \entails B : s \qquad A \equiv_\beta B} \]

\vspace{5pt}

\end{calculus}

\begin{clarifications}
\emph{Pure type systems} (PTS) are a general class of typed $\lambda$ calculus. They represent logical systems through the Curry-Howard correspondence and the "propositions as types" interpretation. The syntax is given by the grammar:
\negvspace\[
  \terms \Coloneqq \variables \mid \constants \mid \Pi \variables : \terms. \terms \mid \lambda \variables : \terms. \terms \mid \terms~\terms
\negvspace\]
where $\variables$ is a set of variables and $\constants$ is a set of constants.
A PTS is parameterized by a \emph{specification} $(\sorts, \axioms, \rules)$ where
$\sorts \subseteq \constants$ is the set of \emph{sorts},
$\axioms \subseteq \constants \times \sorts$ is the set of \emph{axioms}, and
$\rules \subseteq \sorts \times \sorts \times \sorts$ is the set of \emph{rules}.
\end{clarifications}

\begin{history}
Pure type systems were independently introduced by Berardi and Terlouw as a generalization of systems of the $\lambda$ cube, and further developed and popularized by Barendregt, Geuvers, Nederhof \cite{barendregt_introduction_1991,geuvers_modular_1991,barendregt_lambda_1992,geuvers_logics_1993}. Many important systems can be expressed as PTSs, including simply typed $\lambda$ calculus ($\lambda{\rightarrow}$), $\lambda\Pi$ calculus \iref{LuoLF} ($\lambda{P}$), system F \iref{fc} ($\lambda{2}$), and the calculus of constructions ($\lambda{C}$):
\negvspace\[
  \begin{array}{cc}
    \sorts = \{\Prop, \Type\} \qquad \axioms = \{(\Prop, \Type)\} \qquad \rules_{\rightarrow} = \{(\Prop, \Prop, \Prop)\} \\[1ex]
    \rules_{P} = \rules_{\rightarrow} \union \{(\Prop,\Type,\Type)\} \qquad
    \rules_{2} = \rules_{\rightarrow} \union \{(\Type,\Prop,\Prop)\} \qquad
    \rules_{C} = \rules_{P} \union \rules_{2} \union \{(\Type,\Type,\Type)\}
  \end{array}
\negvspace\]
as well as intuitionistic higher-order logic ($\lambda{HOL}$). Pure type systems form the basis of many proof assistants such as Automath, Lego, Coq, Agda, and Matita.
\end{history}

\begin{technicalities}
Soundness and decidability of type checking in PTSs are closely related to \emph{strong normalization} (SN), i.e.~the property that all well-typed terms terminate. Not all pure type systems are SN. Examples of PTSs that are \emph{not} SN (and are therefore inconsistent) are Girard's system U and the universal PTS $\lambda{\Prop}$:
\negvspace\[
  \sorts = \{\Prop\} \qquad \axioms = \{(\Prop, \Prop)\} \qquad \rules = \{(\Prop, \Prop, \Prop)\}
\negvspace\]
\end{technicalities}



\end{entry}
