
% If the calculus has an acronym, define it.
% (e.g. \newcommand{\LK}{\ensuremath{\mathbf{LK}}\xspace})

\calculusName{Signed Analytic Calculi for Finite-Valued Logics} % The name of the calculus
\calculusAcronym{}          % The acronym if defined above, or empty otherwise. 
\calculusLogic{Finite-Valued Logic}        % Specify the logic (e.g. Classical Logic, Intuitionistic Logic, ...) for which this calculus is intended.
\calculusLogicOrder{Propositional}   % Specify the order of the logic (e.g. Propositional, Quantified Propositional, First-Order, Higher-Order, ...).
\calculusType{Sequent Calculus}         % Specify the calculus type (e.g. Tableau, Sequent Calculus, Hyper-Sequent Calculus, Natural Deduction, ...)
\calculusYear{1990}         % The year when the calculus was published.

\calculusAuthor{Reiner H\"ahnle}       % The name(s) of the author(s) of the calculus.
%\calculusAuthor{ToDo:FullNameAuthor2}
%\calculusAuthor{ToDo:FullNameAuthor3}

\entryTitle{Signed Analytic Calculi for Finite-Valued Logics}     % Title of the entry (usually coincides with the name of the calculus).
\entryAuthor{Reiner H\"ahnle}     
%\entryAuthor{ToDo:FullNameAuthor2}
%\entryAuthor{ToDo:FullNameAuthor3}

% The encyclopedia's peer-reviewing policy is described here: 
% http://proofsystem.github.io/Encyclopedia/
%
% Reviewers of this entry will be acknowledged in the following lines:
\entryReviewer{Bruno Woltzenlogel Paleo}
% \entryReviewer{Reviewer 2's name}
% \entryReviewer{Reviewer 3's name}
%
% The lines above will be filled by the coordinators. 
% If you would like to indicate people 
% who could review your entry, contact the coordinators.


% If you wish, use tags to give any other information 
% that might be helpful for classifying and grouping this entry:
% e.g. \tag{Two-Sided Sequents}
% e.g. \tag{Multiset Cedents}
% e.g. \tag{List Cedents}
% You are free to invent your own tags. 
% The Encyclopedia's coordinator will take care of 
% merging semantically similar tags in the future.


\maketitle


% If your files are called "MyProofSystem.tex" and "MyProofSystem.bib", 
% then you should write "\begin{entry}{MyProofSystem}" in the line below
\begin{entry}{SignedLogics}  

% Define here any newcommands you may need:
% e.g. \newcommand{\necessarily}{\Box}
% e.g. \newcommand{\possibly}{\Diamond}

\begin{calculus}

% Add the inference rules of your proof system here.
% The "proof.sty" and "bussproofs.sty" packages are available.
% If you need any other package, please contact the coordinator (Bruno Woltzenlogel Paleo <bruno.wp@gmail.com>)

\[
  \infer[\text{Axiom\quad where }\bigcup_{k=0}^lS_k=N]{\vdash S_1{:}\varphi,\ldots S_l{:}\varphi,\,\Delta}{}\bigskip
\qquad\quad
  \infer[(S{:}C)_r]{\vdash
    S{:}C(\varphi_1,\ldots,\varphi_r),\,\Delta}{\{\vdash S_1^i{:}\varphi_1,\ldots,S_r^i{:}\varphi_r,\,\Delta\}_i}
\]

\end{calculus}

% The following sections ("clarifications", "history", 
% "technicalities") are optional. If you use them, 
% be very concise and objective. Nevertheless, do write full sentences. 
% Try to have at most one paragraph per section, because line breaks 
% do not look nice in a short entry.

\begin{clarifications}
% ToDo: write here short remarks that may help the reader to understand 
% the inference rules of the proof system.
  A one-sided sequent calculus for a generic finite-valued logic with
  truth value set $N=\{0,\ldots,n-1\}$ for any $n\geq2$. Let $C$ be an
  $r$-ary propositional connective with matrix $M_C:N^r\rightarrow N$
  and $S\subseteq N$.  We call $S$ a \emph{sign} and---for any
  propositional formula $\varphi$ over propositional variables
  $\Sigma$ and connectives $C_1,\ldots,C_m$---the expression
  $S{:}\varphi$ a \emph{signed formula}.  Interpretations
  $I:\Sigma\rightarrow N$ are given the usual homomorphic extension to
  propositional formulas.
  %
  Sequents have only a succedent which is a finite multiset of signed
  formulas. The calculus is designed such that $\vdash S{:}\varphi$ is
  derivable iff $I(\varphi)\in S$ for all~$I$. This generalizes
  two-valued validity, where $N=\{0,1\}$ and $S=\{1\}$.
  %
  The calculus has only two (generic) rules.  Let
  $\{\overline{S}^i\}_i=\{S_1^i,\ldots,S_r^i\}_i$ be a finite family
  of sets, each over $r$ many signs.  A generic rule for a connective
  $C$ and sign $S$ has $|\{\overline{S}^i\}_i|=m\geq1$ many premises.
  %
  % Further let
  % $\{s_1^i,\ldots,s_r^i\}=\overline{s}^i\in\overline{S}^i$
  % be a shorthand for $s_j^i\in S_j^i$ for all $1\leq j\leq r$.
  Any rule where
  $\emptyset\neq\{\overline{s}\mid M_C(\overline{s})\in
  S\}=\bigcap_i\,\left(\bigcup_{j=1}^r\,(N\times\cdots\times N\times
    S_j^i\times N\times\cdots\times N)\right) $
  is admissible. (If
  $\{\overline{s}\mid M_C(\overline{s})\in S\}=\emptyset$, then no
  rule is defined.)
  %
  The intuition: the intersection of all premises must contain those
  tuples in $\overline{s}\in N^r$ such that the range of $M_C$ is in
  $S$. Hence, each $\overline{s}$ must be contained in at least one
  $N\times\cdots\times N\times S_j^i\times N\times\cdots\times N$. The
  second rule is a generic axiom that detects tautologies of the form
  $N{:}\varphi$.
  
  An example is classical binary conjunction $\wedge$, where $n=2$ and
  $M_\wedge=\min$.  Conjunction on the right in signed logic becomes
  $\{1\}:\varphi_1\wedge\varphi_2,\,\Delta$. We must characterize
  $\{\overline{s}\mid M_\wedge(\overline{s})\in\{1\}\}=\{(1,1)\}$.
  This is achieved by $(\{1\}\times N)\cap(N\times\{1\})$. Hence, an
  admissible rule has two premises: $\{1\}{:}\varphi_1,\,\Delta$ and
  $\{1\}{:}\varphi_2,\,\Delta$. Now consider a three-valued logic
  ($n=3$) and a unary connective $d$ such that $d(2)=2$ and $d(s)=0$
  for $s\neq2$. One admissible rule for $\{0,2\}{:}d(\varphi)$ has a
  single premise $N{:}\varphi$. There is no rule for unsatisfiable
  formulas like $\{1\}{:}d(\varphi)$. One admissible---but not the
  simplest possible---rule for $\{2\}{:}d(\varphi)$ has the two
  premises $\{0,2\}{:}\varphi,\,\Delta$ and
  $\{1,2\}{:}\varphi,\,\Delta$.

\end{clarifications}

\begin{history}
% ToDo: write here short historical remarks about this proof system,
% especially if they relate to other proof systems. 
% Use "\iref{OtherProofSystem}" to refer to another proof system 
% in the Encyclopedia (where "OtherProofSystem" is its ID). 
% Use "\irefmissing{SuggestedIDForOtherProofSystem}" to refer to 
% another proof system that is not yet available in the encyclopedia.
  The idea to describe generic calculi for finite-valued logics with
  the help of formulas that are signed with truth value sets appears
  first in \cite{Haehnle90} for a tableau system. Predecessors with
  only single truth values as signs were described by Sucho\'n,
  Carnielli, and others. The concept was re-discovered independently
  by Baaz \& Fermueller, Doherty, and Murray \& Rosenthal.  Details
  and references are in \cite{BFS01,Haehnle01}.
\end{history}

\begin{technicalities}
% ToDo: write here remarks about soundness, completeness, decidability...
  The calculus is sound and complete for any finite-valued logic. A
  decision procedure in NP (for fixed $n$) is obtained in a standard
  manner.  The meta theory of finite truth value sets can be
  formulated in classical propositional logic and is therefore
  applicable to virtually any proof system. It also works for certain
  infinite-valued logics (e.g., {\L}ukasiewicz logic) and can be
  lifted to certain first-order quantifiers.
\end{technicalities}


% General Instructions:
% =====================

% The preferred length of an entry is 1 page. 
% Do the best you can to fit your proof system in one page.
%
% If you are finding it hard to fit what you want in one page, remember:
%
%   * Your entry needs to be neither self-contained nor fully understandable
%     (the interested reader may consult the cited full paper for details)
%
%   * If you are describing several proof systems in one entry, 
%     consider splitting your entry.
%
%   * You may reduce the size of your entry by ommitting inference rules
%     that are already described in other entries.
%
%   * Cite parsimoniously (see detailed citation instructions below).
%
% 
% If you do not manage to fit everything in one page, 
% it is acceptable for an entry to have 2 pages.
%
% For aesthetic reasons, it is preferable for an entry to have
% 1 full page or 2 full pages, in order to avoid unused blank space.



% Citation Instructions:
% ======================

% Please cite the original paper where the proof system was defined.
% To do so, you may use the \cite command within 
% one of the optional environments above,
% or use the \nocite command otherwise.

% You may also cite a modern paper or book where the 
% proof system is explained in greater depth or clarity.
% Cite parsimoniously.

% Do not cite related work. Instead, use the "\iref" or "\irefmissing" 
% commands to make an internal reference to another entry, 
% as explained within the "history" environment above.

% You do not need to create the "References" section yourself. 
% This is done automatically.


% Remove all instruction comments before submitting.


% Leave an empty line above "\end{entry}".

\end{entry}


%%% Local Variables:
%%% mode: latex
%%% TeX-master: "../main"
%%% End:
