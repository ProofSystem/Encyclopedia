\calculusName{Signed Analytic Calculi for Finite-Valued Logics}
\calculusAcronym{}
\calculusLogic{Finite-Valued Logic}
\calculusLogicOrder{Propositional}
\calculusType{Sequent Calculus}
\calculusYear{1990}

\calculusAuthor{Reiner H\"ahnle}

\entryTitle{Signed Analytic Calculi for Finite-Valued Logics}
\entryAuthor{Reiner H\"ahnle}     

\entryReviewer{Bruno Woltzenlogel Paleo}

\maketitle

\begin{entry}{SignedLogics}  

\begin{calculus}

\[
  \infer[\text{Axiom\quad where }\bigcup_{k=0}^lS_k=N]{\vdash S_1{:}\varphi,\ldots S_l{:}\varphi,\,\Delta}{}\bigskip
\qquad\quad
  \infer[(S{:}C)_r]{\vdash
    S{:}C(\varphi_1,\ldots,\varphi_r),\,\Delta}{\{\vdash S_1^i{:}\varphi_1,\ldots,S_r^i{:}\varphi_r,\,\Delta\}_i}
\]

\end{calculus}

\begin{clarifications}
  A one-sided sequent calculus for a generic finite-valued logic with
  truth value set $N=\{0,\ldots,n-1\}$ for any $n\geq2$. Let $C$ be an
  $r$-ary propositional connective with matrix $M_C:N^r\rightarrow N$
  and $S\subseteq N$.  We call $S$ a \emph{sign} and---for any
  propositional formula $\varphi$ over propositional variables
  $\Sigma$ and connectives $C_1,\ldots,C_m$---the expression
  $S{:}\varphi$ a \emph{signed formula}.  Interpretations
  $I:\Sigma\rightarrow N$ are given the usual homomorphic extension to
  propositional formulas.
  %
  Sequents have only a succedent which is a finite multiset of signed
  formulas. The calculus is designed such that $\vdash S{:}\varphi$ is
  derivable iff $I(\varphi)\in S$ for all~$I$. This generalizes
  two-valued validity, where $N=\{0,1\}$ and $S=\{1\}$.
  %
  The calculus has only two (generic) rules.  Let
  $\{\overline{S}^i\}_i=\{S_1^i,\ldots,S_r^i\}_i$ be a finite family
  of sets, each over $r$ many signs.  A generic rule for a connective
  $C$ and sign $S$ has $|\{\overline{S}^i\}_i|=m\geq1$ many premises.
  %
  % Further let
  % $\{s_1^i,\ldots,s_r^i\}=\overline{s}^i\in\overline{S}^i$
  % be a shorthand for $s_j^i\in S_j^i$ for all $1\leq j\leq r$.
  Any rule where
  $\emptyset\neq\{\overline{s}\mid M_C(\overline{s})\in
  S\}=\bigcap_i\,\left(\bigcup_{j=1}^r\,(N\times\cdots\times N\times
    S_j^i\times N\times\cdots\times N)\right) $
  is admissible. (If
  $\{\overline{s}\mid M_C(\overline{s})\in S\}=\emptyset$, then no
  rule is defined.)
  %
  The intuition: the intersection of all premises must contain those
  tuples in $\overline{s}\in N^r$ such that the range of $M_C$ is in
  $S$. Hence, each $\overline{s}$ must be contained in at least one
  $N\times\cdots\times N\times S_j^i\times N\times\cdots\times N$. The
  second rule is a generic axiom that detects tautologies of the form
  $N{:}\varphi$.
  
  An example is classical binary conjunction $\wedge$, where $n=2$ and
  $M_\wedge=\min$.  Conjunction on the right in signed logic becomes
  $\{1\}:\varphi_1\wedge\varphi_2,\,\Delta$. We must characterize
  $\{\overline{s}\mid M_\wedge(\overline{s})\in\{1\}\}=\{(1,1)\}$.
  This is achieved by $(\{1\}\times N)\cap(N\times\{1\})$. Hence, an
  admissible rule has two premises: $\{1\}{:}\varphi_1,\,\Delta$ and
  $\{1\}{:}\varphi_2,\,\Delta$. Now consider a three-valued logic
  ($n=3$) and a unary connective $d$ such that $d(2)=2$ and $d(s)=0$
  for $s\neq2$. One admissible rule for $\{0,2\}{:}d(\varphi)$ has a
  single premise $N{:}\varphi$. There is no rule for unsatisfiable
  formulas like $\{1\}{:}d(\varphi)$. One admissible---but not the
  simplest possible---rule for $\{2\}{:}d(\varphi)$ has the two
  premises $\{0,2\}{:}\varphi,\,\Delta$ and
  $\{1,2\}{:}\varphi,\,\Delta$.

\end{clarifications}

\begin{history}
  The idea to describe generic calculi for finite-valued logics with
  the help of formulas that are signed with truth value sets appears
  first in~\cite{Haehnle90} for a tableau system. Predecessors with
  only single truth values as signs were described by Sucho\'n,
  Carnielli, and others. The concept was re-discovered independently
  by Baaz \& Fermueller, Doherty, and Murray \& Rosenthal.  Details
  and references are in~\cite{BFS01,Haehnle01}.
\end{history}

\begin{technicalities}
  The calculus is sound and complete for any finite-valued logic. A
  decision procedure in NP (for fixed $n$) is obtained in a standard
  manner.  The meta theory of finite truth value sets can be
  formulated in classical propositional logic and is therefore
  applicable to virtually any proof system. It also works for certain
  infinite-valued logics (e.g., {\L}ukasiewicz logic) and can be
  lifted to certain first-order quantifiers.
\end{technicalities}

\end{entry}

