
% If the calculus has an acronym, define it.
% (e.g. \newcommand{\LK}{\ensuremath{\mathbf{LK}}\xspace})

\calculusName{Refinement of Structured Specifications}         % The name of the calculus
\calculusAcronym{\SP_{{\vdash}}}          % The acronym if defined above, or empty otherwise. 
\calculusLogic{Institution}        % Specify the logic (e.g. Classical Logic, Intuitionistic Logic, ...) for which this calculus is intended.
\calculusLogicOrder{General}   % Specify the order of the logic (e.g. Propositional, Quantified Propositional, First-Order, Higher-Order, ...).
\calculusType{Hilbert-Style}         % Specify the calculus type (e.g. Tableau, Sequent Calculus, Hyper-Sequent Calculus, Natural Deduction, ...)
\calculusYear{1991}         % The year when the calculus was published.

\calculusAuthor{Martin Wirsing}       % The name(s) of the author(s) of the calculus.


\entryTitle{Refinement of Structured Specifications}     % Title of the entry (usually coincides with the name of the calculus).
\entryAuthor{Rolf Hennicker}     
\entryAuthor{Donald Sannella}     
\entryAuthor{Andrzej Tarlecki}
\entryAuthor{Martin Wirsing}

% The encyclopedia's peer-reviewing policy is described here: 
% http://proofsystem.github.io/Encyclopedia/
%
% Reviewers of this entry will be acknowledged in the following lines:
% \entryReviewer{Reviewer 1's name}
% \entryReviewer{Reviewer 2's name}
% \entryReviewer{Reviewer 3's name}
%
% The lines above will be filled by the coordinators. 
% If you would like to indicate people 
% who could review your entry, contact the coordinators.

\maketitle

% If your files are called "MyProofSystem.tex" and "MyProofSystem.bib", 
% then you should write "\begin{entry}{MyProofSystem}" in the line below
\begin{entry}{RefineSpec}  

% Define here any newcommands you may need:
% e.g. \newcommand{\necessarily}{\Box}
% e.g. \newcommand{\possibly}{\Diamond}

\newcommand{\ang}[1]{\langle{#1}\rangle}
\newcommand{\Family}[2]{\ang{#1}_{#2}}
\newcommand{\SP}{\mathit{SP}}
\newcommand{\SPSig}[1]{{\mathit{Sig}}[#1]}
\newcommand{\SPMod}[1]{{\mathit{Mod}}[#1]}
\newcommand{\INS}{{\bf INS}}
\newcommand{\Sign}{{\bf Sign}}
\newcommand{\Sen}{{\bf Sen}}
\newcommand{\Set}{{\bf Set}}
\newcommand{\Mod}{{\bf Mod}}
\newcommand{\Cat}{{\bf Cat}}
\newcommand{\op}{\mathit{op}}
%\newcommand{\pow}{{\mathscr{P}}}
\newcommand{\pow}{{\mathit{Pow}}}
\newcommand{\translate}[2]{#1\textbf{\upshape{} with }#2}
\newcommand{\derive}[2]{#1\textbf{\upshape{} hide via }#2}
\newcommand{\union}[2]{#1 \cup #2}

\begin{calculus}

% Add the inference rules of your proof system here.
% The "proof.sty" and "bussproofs.sty" packages are available.
% If you need any other package, please contact the coordinator (Bruno Woltzenlogel Paleo <bruno.wp@gmail.com>)

\[
\infer{\SP \vdash \ang{\SPSig{\SP},\Phi}}
      { \textrm{for all } \varphi \in \Phi, \SP \vdash \varphi}
\]
\[
\infer{\SP \vdash \union{\SP_1}{\SP_2}}
      {\SP \vdash \SP_1 \qquad \SP \vdash \SP_2}
\]
\[
\infer{\SP' \vdash \translate{\SP}{\sigma}}
      {\derive{\SP'}{\sigma} \vdash \SP}
\]
\[
\infer[\renewcommand{\arraystretch}{0.8}%
        \begin{array}{@{}l@{}}
           \sigma\colon\SP\rightarrow\widehat{\SP}
           \textrm{ admits} \\
           \hfill \textrm{model expansion}
        \end{array}]{\SP \vdash \derive{\SP'}{\sigma}}
                    {\widehat{\SP} \vdash \SP'}
\]

\end{calculus}

 \begin{clarifications}
$\INS = \ang{\Sign,\; \Sen\colon\Sign\to\Set,\; \Mod\colon\Sign^\op\to\Cat,\;
      \ang{{\models_\Sigma}\subseteq|\Mod(\Sigma)|\times\Sen(\Sigma)}_{\Sigma\in|\Sign|}}$
is an institution that defines the logical system used for specifications, and
$\SP$, $\SP_1$, $\SP_2$, $\SP'$ and $\widehat{\SP}$ are structured specifications over $\INS$.
Structured specifications in $\INS$ are built from basic specifications
$\ang{\Sigma, \Phi}$ where $\Sigma\in|\Sign|$ and $\Phi\subseteq\Sen(\Sigma)$,
the union of $\Sigma$-specifications $\union{\SP_1}{\SP_2}$,
the translation ``$\translate{\SP}{\sigma}$'' of $\SP$
    along a signature morphism $\sigma\colon\Sigma\to\Sigma'$,
and hiding ``$\derive{\SP}{\sigma}$'' for hiding the symbols in $\SP$
    not occurring in the image of $\sigma\colon\Sigma'\to\Sigma$.
$\SPSig{\SP}$ is the signature of $\SP$ and
$\SPMod{\SP}\subseteq|\Mod(\SPSig{\SP})|$ is the class of models of $\SP$.
A signature morphism $\sigma\colon\SPSig{\SP}\to\SPSig{\SP'}$
is a specification morphism $\sigma\colon\SP\to\SP'$
if for every $M'\in\SPMod{\SP'}$, $\Mod(\sigma)(M')\in\SPMod{\SP}$.
Then $\sigma$ admits model expansion
if $\Mod(\sigma)\colon\SPMod{\SP'}\to\SPMod{\SP}$ is surjective.
The judgement $\SP\vdash\varphi$ is entailment for structured specifications
which is required to be sound:
$\SP\vdash\varphi$ implies $M\models_{\SPSig{\SP}}\varphi$
for every $M\in\SPMod{\SP}$.

The judgement $\SP\vdash\SP'$ is meant to capture that
$\SP$ refines (or entails) $\SP'$, that is,
$\SPSig{\SP} = \SPSig{\SP'}$ and $\SPMod{\SP}\subseteq\SPMod{\SP'}$.
\end{clarifications}

\begin{history}
The first proof systems for refinement of structured specifications were given
by Farr\'es-Casals \cite{Far89} and Wirsing \cite{Wir91}.
The above presentation can be found in \cite{ST12}, Sect.~9.3.
\end{history}

\begin{technicalities}
The calculus is sound; it is complete if the underlying entailment system for
structured specifications is complete \cite{Wir91, ST12}. 
 \cite{RH97} provides additional rules for observability operators to support refinement by observational abstraction.
\end{technicalities}
%References: [Wir91], [Far89], [ST12]  
\end{entry}
