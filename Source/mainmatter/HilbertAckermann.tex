\calculusName{Hilbert and Ackermann's Calculus}
\calculusAcronym{}
\calculusLogic{Classical Logic}
\calculusLogicOrder{First-Order}
\calculusType{Axiomatic}
\calculusYear{1928}

\calculusAuthor{David Hilbert}
\calculusAuthor{Wilhelm Ackermann}

\entryTitle{Hilbert and Ackermann's Calculus}
\entryAuthor{Richard Zach}     

\maketitle

\begin{entry}{HilbertAckermann}  

\begin{calculus}
Axioms:
\begin{align*}
\text{a)} \quad & X \lor X \rightarrow X \\
\text{b)} \quad & X \rightarrow X \lor Y \\
\text{c)} \quad & X \lor Y \rightarrow Y \lor X\\
\text{d)} \quad & (X \rightarrow Y) \rightarrow (Z \lor X \rightarrow Z \lor Y)\\
\text{e)} \quad & (x)F(x) \rightarrow F(y)\\
\text{f)} \quad & F(y) \rightarrow (Ex)F(x)
\end{align*}
Rules:
\begin{enumerate}
\item[$\alpha$.] Substitution for object, propositional, and predicate variables
\item[$\beta$.] Generalization rules, i.e.,
\[
\infer{\mathfrak{A} \rightarrow (x)\mathfrak{B}(x)}{\mathfrak{A} \rightarrow \mathfrak{B}(x)}
\qquad
\infer{(Ex)\mathfrak{B}(x) \rightarrow \mathfrak{A}}{\mathfrak{B}(x) \rightarrow \mathfrak{A}}
\]
where $x$ must not occur in~$\mathfrak{A}$.
\item[$\gamma$.] Modus ponens, i.e.,
\[
\infer{\mathfrak{A}}{\mathfrak{A} & \mathfrak{A} \rightarrow \mathfrak{B}}
\]
\end{enumerate}

\end{calculus}

\begin{clarifications}
  As in~\iref{Hilbert}, the system is formulated in a language that distinguishes
  between propositional and predicate constants and variables, but theorems with
  free variables are now allowed. Fraktur letters are used as schematic
  metavariables. The substitution rule allows the replacement of individual
  variables by variables or constants, propositional variables by any formula, and
  predicate variables with $n$ arguments $x_1$, \dots, $x_n$ by a formula in which
  $x_1$, \dots, $x_n$ occur free.
\end{clarifications}

\begin{history}
  The system was published in~\cite{HilbertAckermann1928}. It was the
  systems for which the problem of providing a completeness proof was
  first raised.
\end{history}

\begin{technicalities}
  This system was proved complete in~\cite{Godel1930}.
\end{technicalities}

\end{entry}
