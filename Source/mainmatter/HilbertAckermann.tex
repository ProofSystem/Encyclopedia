
% If the calculus has an acronym, define it.
% (e.g. \newcommand{\LK}{\ensuremath{\mathbf{LK}}\xspace})

\calculusName{Hilbert and Ackermann's Calculus}         % The name of the calculus
\calculusAcronym{}          % The acronym if defined above, or empty otherwise. 
\calculusLogic{Classical Logic}        % Specify the logic (e.g. Classical Logic, Intuitionistic Logic, ...) for which this calculus is intended.
\calculusLogicOrder{First-Order}   % Specify the order of the logic (e.g. Propositional, Quantified Propositional, First-Order, Higher-Order, ...).
\calculusType{Axiomatic}         % Specify the calculus type (e.g. Tableau, Sequent Calculus, Hyper-Sequent Calculus, Natural Deduction, ...)
\calculusYear{1928}         % The year when the calculus was published.

\calculusAuthor{David Hilbert}       % The name(s) of the author(s) of the calculus.
\calculusAuthor{Wilhelm Ackermann}
%\calculusAuthor{ToDo:FullNameAuthor3}


\entryTitle{Hilbert and Ackermann's Calculus}     % Title of the entry (usually coincides with the name of the calculus).
\entryAuthor{Richard Zach}     
%\entryAuthor{ToDo:FullNameAuthor2}
%\entryAuthor{ToDo:FullNameAuthor3}

% The encyclopedia's peer-reviewing policy is described here: 
% http://proofsystem.github.io/Encyclopedia/
%
% Reviewers of this entry will be acknowledged in the following lines:
% \entryReviewer{Reviewer 1's name}
% \entryReviewer{Reviewer 2's name}
% \entryReviewer{Reviewer 3's name}
%
% The lines above will be filled by the coordinators. 
% If you would like to indicate people 
% who could review your entry, contact the coordinators.


% If you wish, use tags to give any other information 
% that might be helpful for classifying and grouping this entry:
% e.g. \etag{Two-Sided Sequents}
% e.g. \etag{Multiset Cedents}
% e.g. \etag{List Cedents}
% You are free to invent your own tags. 
% The Encyclopedia's coordinator will take care of 
% merging semantically similar tags in the future.


\maketitle

\begin{entry}{HilbertAckermann}  

\begin{calculus}
Axioms:
\begin{align*}
\text{a)} \quad & X \lor X \rightarrow X \\
\text{b)} \quad & X \rightarrow X \lor Y \\
\text{c)} \quad & X \lor Y \rightarrow Y \lor X\\
\text{d)} \quad & (X \rightarrow Y) \rightarrow (Z \lor X \rightarrow Z \lor Y)\\
\text{e)} \quad & (x)F(x) \rightarrow F(y)\\
\text{f)} \quad & F(y) \rightarrow (Ex)F(x)
\end{align*}
Rules:
\begin{enumerate}
\item[$\alpha$.] Substitution for object, propositional, and predicate variables
\item[$\beta$.] Generalization rules, i.e.,
\[
\infer{\mathfrak{A} \rightarrow (x)\mathfrak{B}(x)}{\mathfrak{A} \rightarrow \mathfrak{B}(x)}
\qquad
\infer{(Ex)\mathfrak{B}(x) \rightarrow \mathfrak{A}}{\mathfrak{B}(x) \rightarrow \mathfrak{A}}
\]
where $x$ must not occur in~$\mathfrak{A}$.
\item[$\gamma$.] Modus ponens, i.e.,
\[
\infer{\mathfrak{A}}{\mathfrak{A} & \mathfrak{A} \rightarrow \mathfrak{B}}
\]
\end{enumerate}

\end{calculus}

\begin{clarifications}
As in \iref{Hilbert}, the system is formulated in a language that
distinguishes between propositional and predicate constants and
variables, but theorems with free variables are now allowed. Fraktur letters are used as schematic metavariables. The
substitution rule allows the replacement of individual variables by
variables or constants, propositional variables by any formula, and
predicate variables with $n$ arguments $x_1$, \dots, $x_n$ by a
formula in which $x_1$, \dots, $x_n$ occur free.
\end{clarifications}

\begin{history}
The system was published in \cite{HilbertAckermann1928}. It was the
systems for which the problem of providing a completeness proof was
first raised.
\end{history}

\begin{technicalities}
This system was proved complete by \cite{Godel1930}.
\end{technicalities}

\end{entry}
