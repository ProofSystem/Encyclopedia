\calculusName{Bernays's Propositional Calculus}
\calculusAcronym{}
\calculusLogic{Classical Logic}
\calculusLogicOrder{Propositional}
\calculusType{Axiomatic}
\calculusYear{1918}

\calculusAuthor{Paul Bernays}

\entryTitle{Bernays's Propositional Calculus}
\entryAuthor{Richard Zach}

\maketitle

\begin{entry}{Bernays}  

\begin{calculus}
Axioms:
\begin{align*}
1) \quad & \overline{XX}X &
1) \quad & X \lor X \rightarrow X\\
2) \quad & \overline{X}(XY) &
2) \quad & X \rightarrow X \lor X\\
3) \quad & \overline{XY}(YX) &
3) \quad & X \lor Y \rightarrow Y \lor X \\
4) \quad & \overline{X(YZ)}((XY)Z) &
4) \quad & X \lor (Y \lor Z) \rightarrow (X \lor Y) \lor Z \\
5) \quad & \overline{(\overline{X}Y)}(\overline{ZX}(ZY)) &
5) \quad & (X \rightarrow Y) \rightarrow (Z \lor X \rightarrow Z \lor Y) 
\end{align*}
Rules:
\begin{enumerate}
\item[a.] Substitution for propositional variables
\item[b.] Modus ponens, i.e.,
\[
\infer{\beta}{\alpha & \alpha \rightarrow \beta}
\]
\end{enumerate}

\end{calculus}

\begin{clarifications}
  Bernays used juxtaposition for disjunction $\lor$, $+$ for conjunction $\land$,
  and overlining $\overline{\phantom{X}}$ for negation. The axioms on the left are
  official, those on the right use the abbreviation $\alpha \rightarrow \beta$ for
  $\overline{\alpha}\beta$, i.e., $\overline{\alpha} \lor \beta$. The substitution
  rule allows the replacement of propositional variables by any expression.
\end{clarifications}

\begin{history}
  The axioms are a slight variation on the propositional fragment of Whitehead and
  Russell's \emph{Principia Mathematica}~\irefmissing{PrincipiaMathematica} due to
  Paul Bernays~\cite{Bernays1918}.  The system is noteworthy since Bernays was the
  first to prove completeness relative to standard truth value semantics, as well
  as decidability; see~\cite{Zach1999}. Bernays investigated systems in which
  axioms are replaced by rules, e.g., $\alpha \vdash \alpha \lor \beta$, including
  rules that operate on parts of a formula such as $\gamma(\alpha \lor \beta)
  \vdash \gamma(\beta \lor \alpha)$. He showed that a system with six rules and $X
  \rightarrow X$ as the only axiom is complete. He also showed that axiom (4) is
  provable from the others and the rest are independent (published
  in~\cite{Bernays1926}).  
\end{history}


\end{entry}
