
% If the calculus has an acronym, define it.
% (e.g. \newcommand{\LK}{\ensuremath{\mathbf{LK}}\xspace})

\calculusName{Bernays's Propositional Calculus}         % The name of the calculus
\calculusAcronym{}          % The acronym if defined above, or empty otherwise. 
\calculusLogic{Classical Logic}        % Specify the logic (e.g. Classical Logic, Intuitionistic Logic, ...) for which this calculus is intended.
\calculusLogicOrder{Propositional}   % Specify the order of the logic (e.g. Propositional, Quantified Propositional, First-Order, Higher-Order, ...).
\calculusType{Axiomatic}         % Specify the calculus type (e.g. Tableau, Sequent Calculus, Hyper-Sequent Calculus, Natural Deduction, ...)
\calculusYear{1918}         % The year when the calculus was published.

\calculusAuthor{Paul bernays}       % The name(s) of the author(s) of the calculus.
%\calculusAuthor{ToDo:FullNameAuthor2}
%\calculusAuthor{ToDo:FullNameAuthor3}


\entryTitle{Bernays's Propositional Calculus}     % Title of the entry (usually coincides with the name of the calculus).
\entryAuthor{Richard Zach}     
%\entryAuthor{ToDo:FullNameAuthor2}
%\entryAuthor{ToDo:FullNameAuthor3}

% The encyclopedia's peer-reviewing policy is described here: 
% http://proofsystem.github.io/Encyclopedia/
%
% Reviewers of this entry will be acknowledged in the following lines:
% \entryReviewer{Reviewer 1's name}
% \entryReviewer{Reviewer 2's name}
% \entryReviewer{Reviewer 3's name}
%
% The lines above will be filled by the coordinators. 
% If you would like to indicate people 
% who could review your entry, contact the coordinators.


% If you wish, use tags to give any other information 
% that might be helpful for classifying and grouping this entry:
% e.g. \etag{Two-Sided Sequents}
% e.g. \etag{Multiset Cedents}
% e.g. \etag{List Cedents}
% You are free to invent your own tags. 
% The Encyclopedia's coordinator will take care of 
% merging semantically similar tags in the future.


\maketitle

\begin{entry}{Bernays}  

\begin{calculus}
Axioms:
\begin{align*}
1)\quad & \overline{XX}X &
1)\quad & X \lor X \rightarrow X\\
2) \quad & \overline{X}(XY) &
2) \quad & X \rightarrow X \lor X\\
3) \quad & \overline{XY}(YX) &
3) \quad & X \lor Y \rightarrow Y \lor X \\
4) \quad & \overline{X(YZ)}((XY)Z) &
4) \quad & X \lor (Y \lor Z) \rightarrow (X \lor Y) \lor Z \\
5) \quad & \overline{(\overline{X}Y)}(\overline{ZX}(ZY)) &
5) \quad & (X \rightarrow Y) \rightarrow (Z \lor X \rightarrow Z \lor Y) 
\end{align*}
Rules:
\begin{enumerate}
\item[a.] Substitution for propositional variables
\item[b.] Modus ponens, i.e.,
\[
\infer{\beta}{\alpha & \alpha \rightarrow \beta}
\]
\end{enumerate}

\end{calculus}

\begin{clarifications}
Bernays used juxtaposition for disjunction $\lor$, $+$ for conjunction
$\land$, and overlining $\overline{\phantom{X}}$ for
negation. The axioms on the left are official, those on the right use the abbreviation $\alpha \rightarrow \beta$ for
$\overline{\alpha}\beta$, i.e., $\overline{\alpha} \lor \beta$. The substitution
rule allows the replacement of propositional variables by any
expression.
\end{clarifications}

\begin{history}
The axioms are a slight variation on the propositional fragment of
Whitehead and Russell's \emph{Principia
Mathematica} \irefmissing{PrincipiaMathematica} due to Paul
Bernays \cite{Bernays1918}.  The system is noteworthy since Bernays
was the first to prove completeness relative to standard truth value
semantics, as well as decidability; see \cite{Zach1999}. Bernays investigated systems in which axioms are replaced
by rules, e.g., $\alpha \vdash \alpha \lor \beta$, including rules
that operate on parts of a formula such as
$\gamma(\alpha \lor \beta) \vdash \gamma(\beta \lor \alpha)$. He showed
that a system with six rules and $X \rightarrow X$ as the only axiom
is complete. He also
showed that axiom (4) is provable from the others and the rest are
independent (published in \cite{Bernays1926}). 
\end{history}

% \begin{technicalities}
% ToDo: write here remarks about soundness, completeness, decidability...
% \end{technicalities}

\end{entry}
