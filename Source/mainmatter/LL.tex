

\renewcommand{\LL}{\ensuremath{\mathbf{LL}}\xspace}

\calculusName{Linear Sequent Calculus}   
\calculusAcronym{\LL}     
\calculusLogic{Linear Logics}  
\calculusLogicOrder{Propositional}
\calculusType{Sequent Calculus}   
\calculusYear{1987}   
\calculusAuthor{Jean-Yves Girard} 


\entryTitle{Linear Sequent Calculus LL}     
\entryAuthor{Olivier Laurent}     



\etag{One-Sided Sequents}
\etag{List Cedents}

\maketitle



\begin{entry}{LL}  




\begin{calculus}
% Add the inference rules of your proof system here.
% The "proof.sty" and "bussproofs.sty" packages are available.
% If you need any other package, please contact the editor (bruno@logic.at)
\begin{center}
\AxiomC{}
\UnaryInfC{$\seq A^\bot,A$}
\DisplayProof
\qquad
\AxiomC{$\seq \Gamma,A$}
\AxiomC{$\seq \Delta,A^\bot$}
\BinaryInfC{$\seq \Gamma,\Delta$}
\DisplayProof
\qquad
\AxiomC{$\seq \Gamma$}
\UnaryInfC{$\seq \sigma(\Gamma)$}
\DisplayProof
\\[2ex]
\AxiomC{$\seq \Gamma,A$}
\AxiomC{$\seq \Delta,B$}
\BinaryInfC{$\seq \Gamma,\Delta,A\otimes B$}
\DisplayProof
\qquad
\AxiomC{$\seq \Gamma,A,B$}
\UnaryInfC{$\seq \Gamma,A\parr B$}
\DisplayProof
\qquad
\AxiomC{}
\UnaryInfC{$\seq 1$}
\DisplayProof
\qquad
\AxiomC{$\seq \Gamma$}
\UnaryInfC{$\seq \Gamma,\bot$}
\DisplayProof
\\[2ex]
\AxiomC{$\seq \Gamma,A$}
\UnaryInfC{$\seq \Gamma,A\oplus B$}
\DisplayProof
\qquad
\AxiomC{$\seq \Gamma,B$}
\UnaryInfC{$\seq \Gamma,A\oplus B$}
\DisplayProof
\qquad
\AxiomC{$\seq \Gamma,A$}
\AxiomC{$\seq \Gamma,B$}
\BinaryInfC{$\seq \Gamma,A\with B$}
\DisplayProof
\qquad
\AxiomC{}
\UnaryInfC{$\seq \Gamma,\top$}
\DisplayProof
\\[2ex]
\AxiomC{$\seq \Gamma,A$}
\UnaryInfC{$\seq \Gamma,\wn A$}
\DisplayProof
\qquad
\AxiomC{$\seq \wn\Gamma,A$}
\UnaryInfC{$\seq \wn\Gamma,\oc A$}
\DisplayProof
\qquad
\AxiomC{$\seq \Gamma,\wn A,\wn A$}
\UnaryInfC{$\seq \Gamma,\wn A$}
\DisplayProof
\qquad
\AxiomC{$\seq \Gamma$}
\UnaryInfC{$\seq \Gamma,\wn A$}
\DisplayProof
\\[2ex]
\[
\begin{array}{c@{\quad}c@{\quad}c}
& (A\otimes B)^\bot = A^\bot\parr B^\bot
& 1^\bot = \bot
\\
(X^\bot)^\bot = X
& (A\parr B)^\bot = A^\bot\otimes B^\bot
& \bot^\bot = 1
\\
(\oc A)^\bot = \wn (A^\bot)
& (A\oplus B)^\bot = A^\bot\with B^\bot
& 0^\bot = \top
\\
(\wn A)^\bot = \oc (A^\bot)
& (A\with B)^\bot = A^\bot\oplus B^\bot
& \top^\bot = 0
\end{array}
\]
$\Gamma$ and $\Delta$ are lists of formulas.\\
$\sigma$ is a permutation.
\end{center}
\end{calculus}



\begin{clarifications}
% ToDo: write here short remarks that may help the reader to understand 
% the inference rules of the proof system.
If $\Gamma=A_1,\dots,A_n$ then $\wn\Gamma=\wn A_1,\dots,\wn A_n$.
Negation is not a connective. It is defined using De Morgan's laws so that $(A^\bot)^\bot=A$.
The linear implication can be defined as $A\multimap B = A^\bot\parr B$.
\end{clarifications}

\begin{history}
% ToDo: write here short historical remarks about this proof system,
% especially if they relate to other proof systems. 
% Use "\iref{OtherProofSystem}" to refer to another proof system 
% in the Encyclopedia (where "OtherProofSystem" is its ID). 
% Use "\irefmissing{SuggestedIDForOtherProofSystem}" to refer to 
% another proof system that is not yet available in the encyclopedia.
Linear Logic and its sequent calculus \LL~\cite{ll} come from the analysis of intuitionistic logic through Girard's decomposition of the intuitionistic implication into the linear implication: $A\imp B = \oc{A}\multimap B$.
\end{history}

\begin{technicalities}
% ToDo: write here remarks about soundness, completeness, decidability...
Cut elimination holds.
\LL{} is sound and complete with respect to phase semantics~\cite{ll}.
\LL{} is not decidable~\cite{Lincoln1992}.
Sequent calculi \LK\iref{GentzenLK} and \LJ\iref{GentzenLJ} can be translated into \LL.
\end{technicalities}













\end{entry}



%%% Local Variables: 
%%% mode: latex
%%% TeX-master: "main"
%%% End: 
