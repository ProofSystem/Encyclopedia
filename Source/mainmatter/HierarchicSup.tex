


\calculusName{Hierarchic Superposition}   
\calculusAcronym{}     
\calculusLogic{Classical Logic}  
\calculusLogicOrder{First-Order}
\calculusType{Superposition}   
\calculusYear{1992/2013}   
\calculusAuthor{Leo Bachmair} 
\calculusAuthor{Harald Ganzinger} 
\calculusAuthor{Uwe Waldmann} 


\entryTitle{Hierarchic Superposition}     
\entryAuthor{Uwe Waldmann}     





\maketitle



\begin{entry}{HierarchicSup}




\begin{calculus}

% Add the inference rules of your proof system here.
% The "proof.sty" and "bussproofs.sty" packages are available.
% If you need any other package, please contact the editor (bruno@logic.at)

Abstraction
\[
\infer[\textit{Abstraction}]
{C[x] \lor \neg x \approx t}{C[t]\vphantom{[]}}
\]
applied exhaustively until no literal contains operator symbols
from both $\Sigma_{\mathrm{Base}}$ and $\Sigma_{\mathrm{Ext}}$,
followed by saturation under
\[
\infer[\textit{Constraint Refutation}]
{\bot}{M & M \models_{\mathrm{Base}} \bot}
\]
and the rules of the standard superposition calculus~\iref{Superposition},
where the latter are restricted in such a way that only extension
literals participate in inferences and that all unifying substitutions
must be simple.

$C$ is an equational clause,
$t$ is a term,
$x$ is a fresh variable,
$M$ is a finite set of clauses over $\Sigma_{\mathrm{Base}}$.
\end{calculus}



\begin{clarifications}
Hierarchic superposition is a refutational saturation calculus for
first-order clauses with equality
modulo a base specification
(e.\,g., some kind of arithmetic),
for which a decision procedure is available
that can be used as a ``black-box''
in the \textit{Constraint Refutation} rule.
The inference rules are supplemented by a redundancy criterion
that permits to delete clauses that are unnecessary for
deriving a contradiction during the saturation, see~\iref{SaturationWithRed}.
\end{clarifications}

\begin{history}
The hierarchic superposition calculus~\cite{BachmairGanzingerWaldmann1992ALP,BachmairGanzingerWaldmann1994AAECC}
works in the framework of hierarchic specifications
consisting of a base part and an extension,
where the models of the hierarchic specification
are those models of the extension clauses that
are conservative extensions of some base model.
The calculus is refutationally complete,
provided that the set of clauses is sufficiently
complete after abstraction and that the base specification is compact.
An improved variant of the calculus
was given in~\cite{BaumgartnerWaldmann2013CADE};
this calculus uses a weaker form of abstraction that is
guaranteed to preserve sufficient completeness
\looseness=-1
but requires an additional abstraction step after each inference.

\end{history}


\end{entry}
