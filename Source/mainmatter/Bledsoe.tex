


\calculusName{Bledsoe's Natural Deduction}
\calculusAcronym{\Bledsoe}
\calculusLogic{Classical Logic}
\calculusLogicOrder{First-Order}
\calculusType{Natural Deduction}
\calculusYear{1973-1978}
\calculusAuthor{Woody W. Bledsoe}


\entryTitle{Bledsoe's Natural Deduction - Prover}
\entryAuthor{Dominique Pastre}

\etag{Natural Methods}

\maketitle

\begin{entry}{Bledsoe}  

\begin{calculus}

\textbf {SPLIT: basic rules of Natural Deduction}(see~\iref{GentzenNJ}),
for example\\
To prove $A\land B$, prove $A$ and prove $B$\\
To prove $p \rightarrow  A \land B$, prove $(p \rightarrow A) \land (p \rightarrow B)$\\
To prove $p \ \lor q \rightarrow  A $, prove $(p \rightarrow A) \land (q \rightarrow A)$\\
To prove $\exists x P(x) \rightarrow D$, prove $P(y) \rightarrow D$, where $y$ is a new variable\\
\textbf {REDUCE: conversion rules}, for example\\
To prove $x \in A \cap B$, prove   $x \in A \land x \in B$ \\
To prove $S \in {\cal  P} (A)$, prove $S \subset A \land S \in  {\cal U}$\\
To prove $x \in \sigma F$, prove $\exists y (y \in F \land x \in y)$

\textbf {DEFINITIONS},  example\\
$A \subset B$ is defined by $\forall x(x \in A \rightarrow x \in B)$ and is 
replaced by $x \in A \rightarrow x \in B$ or by 
$x_o \in A \rightarrow x_o \in B$, depending on the position 
of the formula in the theorem.\\
\textbf {IMPLY: }  
in addition to SPLIT and REDUCE rules, \\
- search for substitutions which unify some hypotheses and a conclusion and \\
compose them until obtaining the empty substitution (theorem proved) or failing\\
- forward chaining : if $P$ and $P'$ are unified by $\theta$ ($P \theta$=$P' \theta$), then a hypothesis  
$P' \land (P \rightarrow Q)$ is converted into 
$P' \land (P \rightarrow Q) \land Q \theta$\\
- PEEK forward chaining : if $P\theta$=$P'\theta$ and $A$ has the definition 
$(P \rightarrow Q)$, 
then a hypothesis $P' \land A$ is converted into $P'\land A \land Q \theta$ \\
- backward chaining : if $A \rightarrow D$ and $D \theta = C \theta$, 
replace the conclusion $C$ by $A \theta$
\end{calculus}

\begin{clarifications}
Bledsoe's natural deduction may be seen as both an extension and 
a restriction of formal natural deduction~\iref{GentzenNJ}. 
In SPLIT and REDUCE, there is reduction but not expansion. 
Some subroutines convert expressions into forms convenient 
for applying the rules. 
The notions of hypothesis and conclusion are privileged.
\end{clarifications}

\begin{history}
After having applied the rules of IMPLY and REDUCE, the first version of
{\sc \Bledsoe}~\cite{bledsoe:1971} called a resolution program if necessary. 
Then, in~\cite{bledsoe:1972}, these calls to resolution are completely 
replaced by IMPLY. 
\Bledsoe has been working in set theory, limit theorems, topology and 
program verification.
\end{history}

\begin{technicalities}
The system is sound but not complete.
Bledsoe emphasizes the fact that, with these methods, provers may succeed 
because they proceed in a natural human-like way~\cite{bledsoe:1977}.
\end{technicalities}



\end{entry}

