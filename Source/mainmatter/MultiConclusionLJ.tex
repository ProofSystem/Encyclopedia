
\calculusName{Multi-Conclusion Intuitionistic Sequent Calculus} 
\calculusAcronym{\LJmc}
\calculusLogic{Intuitionistic Logic}
\calculusLogicOrder{First-Order}
\calculusType{Sequent Calculus}
\calculusYear{1954} 
\calculusAuthor{Sh\^{o}ji Maehara} 

\entryTitle{Multi-Conclusion Sequent Calculus \LJmc}
\entryAuthor{Giselle Reis} 
\entryAuthor{Valeria de Paiva}

\etag{Two-Sided Sequents}
\etag{Multiset Cedents}
\etag{Multi-Conclusion Succedent}

\maketitle


\begin{entry}{MultiConclusionLJ}  


\begin{calculus}
%
\[
\begin{array}{cc}
\infer{A \vdash A}{}
&
\infer[cut]{\Gamma, \Delta \vdash \Theta, \Lambda}{\Gamma \vdash
\Theta, A & A, \Delta \vdash \Lambda}
\\[5pt]
\infer[\wedge_l]{A_1 \wedge A_2, \Gamma \vdash \Theta}{A_i, \Gamma \vdash \Theta}
&
\infer[\wedge_r]{\Gamma \vdash \Theta, A \wedge B}{\Gamma \vdash
\Theta, A & \Gamma \vdash \Theta, B}
\\[5pt]
\infer[\vee_l]{A \vee B, \Gamma \vdash \Theta}{A, \Gamma \vdash \Theta & B, \Gamma \vdash \Theta}
&
\infer[\vee_r]{\Gamma \vdash \Theta, A_1 \vee A_2}{\Gamma \vdash
\Theta, A_i}
\\[5pt]
\infer[\rightarrow_l]{A \rightarrow B, \Gamma, \Delta \vdash \Theta, \Lambda}{\Gamma
\vdash \Theta, A & B, \Delta \vdash \Lambda}
&
\infer[\rightarrow_r]{\Gamma \vdash A \rightarrow B}{A, \Gamma \vdash B}
\\
\end{array}
\]
\[
\begin{array}{cccc}
\infer[\exists_l]{\exists x.Ax, \Gamma \vdash \Theta}{A\alpha, \Gamma \vdash \Theta}
&
\infer[\exists_r]{\Gamma \vdash \Theta, \exists x.Ax}{\Gamma \vdash
\Theta, At}
&
\infer[\forall_l]{\forall x.Ax, \Gamma \vdash \Theta}{At, \Gamma \vdash \Theta}
&
\infer[\forall_r]{\Gamma \vdash \forall x.Ax}{\Gamma \vdash A\alpha}
\\[5pt]
\infer[\neg_l]{\neg A, \Gamma \vdash \Theta}{\Gamma \vdash \Theta, A}
&
\infer[\neg_r]{\Gamma \vdash \neg A}{A, \Gamma \vdash }
&
\infer[e_l]{\Gamma, A, B, \Delta \vdash \Theta}{\Gamma, B, A, \Delta \vdash \Theta}
&
\infer[e_r]{\Gamma \vdash \Theta, A, B, \Lambda}{\Gamma \vdash
\Theta, B, A, \Lambda}
\\[5pt]
\infer[c_l]{A, \Gamma \vdash \Theta}{A, A, \Gamma \vdash \Theta}
&
\infer[c_r]{\Gamma \vdash \Theta, A}{\Gamma \vdash \Theta, A, A}
&
\infer[w_l]{A, \Gamma \vdash \Theta}{\Gamma \vdash \Theta}
&
\infer[w_r]{\Gamma \vdash A}{\Gamma \vdash}
\\
\end{array}
\]
%
\centering
The eigenvariable $\alpha$ should not occur in $\Gamma$, $\Theta$ or $A[x]$. \\ 
The term $t$ should not contain variables bound in $A[t]$.
\end{calculus}


\begin{clarifications}
While \LJ \iref{GentzenLJ} is defined by restricting \LK \iref{GentzenLK} to
single conclusion, in \LJmc only the rules $\neg_r$, $\rightarrow_r$ and
$\forall_r$ have this restriction.
\end{clarifications}

\begin{history}
\LJmc was proposed in \cite{Maehara1954} and used to prove the completeness of
\LJ \iref{GentzenLJ} in \cite{takeuti87}. 
It also appears in \cite{dragalin88} (as GHPC) and \cite{dummett77} (as L').
%It is also the sequent calculus system
%for intuitionistic logic in \cite{dragalin88}, as GHPC, and an alternative for
%the single conclusion system in \cite{dummett77}, as L'.
\end{history}

\begin{technicalities}
\LJmc is equivalent to \LJ, and this is established by translating
sequents of the form $\Gamma \vdash A_1, ..., A_n$ into 
sequents of the form $\Gamma \vdash A_1 \vee ... \vee A_n$. 
Cut can be eliminated by using a combination of the rewriting
rules for cut-elimination in \LJ and \LK and permutation of inferences, as shown
by Schellinx \cite{Schellinx91} and Reis \cite{GisellePhD}.
\end{technicalities}

\end{entry}
