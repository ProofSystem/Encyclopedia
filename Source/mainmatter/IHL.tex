\calculusName{Intuitionistic Hybrid Logic}   
\calculusAcronym{\IHL}     
\calculusLogic{Intuitionistic} 
\calculusLogicOrder{Propositional}
\calculusType{Natural Deduction}   
\calculusYear{2003}   
\calculusAuthor{Valeria de Paiva}
\calculusAuthor{Torben Bra\"uner}

\entryTitle{Intuitionistic Hybrid Logic (IHL)}
\entryAuthor{Valeria de Paiva}
\entryAuthor{Harley Eades III}
\entryAuthor{Torben Bra\"uner}

\maketitle

\begin{entry}{IHL}  

\newcommand{\llimp}[0]{\leftharpoonup}
\newcommand{\rlimp}[0]{\rightharpoonup}    
  
\begin{calculus}
  \[ \setlength{\arraycolsep}{-1px}
  \begin{array}{c}
    \begin{array}{cccccccccccc}
    \infer[\land I]{@_a (A \land B)}{@_a A & @_a B}
    & \quad &
    \infer[\land E_1]{@_a A}{@_a (A \land B)}
    & \quad &
    \infer[\land E_2]{@_a B}{@_a (A \land B)}
    & \quad &
    \infer[\lor I_1]{@_a A \lor B}{@_a A}
    \\[6pt]        
    \infer[\lor I_2]{@_a A \lor B}{@_a B}
    & \quad &
    \infer[\lor E]{C}{@_a A \lor B & \infer*{C}{[@_a A]} & \infer*{C}{[@_a B]}}
    & \quad &
    \infer[\rightarrow I]{@_a (A \rightarrow B)}{\infer*{@_a B}{[@_a A]}}
    & \quad &
    \infer[\rightarrow E]{@_a B}{@_a (A \rightarrow B) & @_a A}    
    \end{array}   
    \\ \\
    \setlength{\arraycolsep}{-1px}
  \begin{array}{cccccccccccccc}
    \infer[\perp E]{C}{@_a \perp}
    & \quad &
    \infer[@ I]{@_c @_a A}{@_a A}
    & \quad &
    \infer[\text{Ref}]{a : a}{}
    & \quad &
    \infer[\text{Nom}_1]{c : A}{a : c & a : A}
    & \quad &
    \infer[\text{Nom}_2]{c : \Diamond b}{a : c & a : \Diamond b}
    \\[6pt]
    \infer[@ E]{@_a A}{@_c @_a A}
    & \quad &
    \infer[\Diamond I]{@_a \Diamond A}{@_e A & @_a \Diamond e}
    & \quad &
    \infer[\Diamond E]{C}{@_a \Diamond A & \infer*{C}{[@_c A][@_a \Diamond c]}}
    & \quad &
    \infer[\Box I]{@_a \Box A}{\infer*{@_c A}{[@_a \Diamond c]}}
    & \quad &
    \infer[\Box E]{@_e A}{@_a \Box A & @_a \Diamond e}
  \end{array}
  \end{array}
  \]
\end{calculus}

\begin{clarifications}
  Formulas of IHL are defined by the following grammar:
  \begin{center} 
    \begin{math}
      \begin{array}{lll}
        A,B,C ::= p \mid a \mid A \land B \mid A \lor B \mid A \rightarrow B \mid \perp \mid \Box A \mid \Diamond A \mid @_a A
      \end{array}
    \end{math}
  \end{center}
  where $a$ ranges over nominals and $p$ propositional symbols. In the
  rule $\Diamond E$, $c$ does not occur in $@_a \lozenge A $, in $C$,
  or in any undischarged assumptions other than the specified
  occurrences of $@_c A$ and $@_a \lozenge c $.  Furthermore, in $\Box
  I$, $c$ does not occur in $@_a \square A $ or in any undischarged
  assumptions other than the specified occurrences of $@_a \lozenge
  c$.  In the rule $\text{Nom}_1$, $A$, is any proposition (ordinary
  or nominal).
\end{clarifications}

\begin{history}
%% ToDo: write here short historical remarks about this proof system,
%% especially if they relate to other proof systems. 
%% Use "\iref{OtherProofSystem}" to refer to another proof system 
%% in the Encyclopedia (where "OtherProofSystem" is its ID). 
%% Use "\irefmissing{SuggestedIDForOtherProofSystem}" to refer to 
%% another proof system that is not yet available in the encyclopedia.
The Natural Deduction system for IHL was originally suggested in
\cite{braunerdepaiva2003} and developed in \cite{braunerdepaiva2006}.
This system adds nominals and satisfaction operators to a version of
Intuitionistic Modal Logic described using labelled deduction, in the
style of Simpson \cite{simpson1994}.  Axioms, or a Hilbert-style
calculus version of the system, were provided in \cite{brauner2006}.
Some of the properties of the intuitionistic system, as well as a
discussion of some of its applications to type systems in computing,
appeared in \cite{brauner2011}.
\end{history}

%% \begin{technicalities}
%% ...
%% \end{technicalities}

\end{entry}
