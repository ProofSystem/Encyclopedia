\calculusName{Intuitionistic Hybrid Logic}   
\calculusAcronym{\IHL}     
\calculusLogic{Intuitionistic Logics} 
\calculusLogicOrder{Propositional}
\calculusType{Natural Deduction}   
\calculusYear{2017}   
\calculusAuthor{Valeria de Paiva}
\calculusAuthor{Torben Bra\"uner}

\entryTitle{Intuitionistic Hybrid Logic (IHL)}
\entryAuthor{Valeria de Paiva}
\entryAuthor{Harley Eades III}

\maketitle

\begin{entry}{IHL}  

\newcommand{\llimp}[0]{\leftharpoonup}
\newcommand{\rlimp}[0]{\rightharpoonup}    
  
\begin{calculus}
  \[
  \begin{array}{c}
    \begin{array}{cccccccccccc}
    \infer[\land I]{a @ A \land B}{a @ A & a @ B}
    & \quad &
    \infer[\land E_1]{a @ A}{a @ A \land B}
    & \quad &
    \infer[\land E_2]{a @ B}{a @ A \land B}
    & \quad &
    \infer[\lor I_1]{a @ A \lor B}{a @ A}
    \\[6pt]        
    \infer[\lor I_2]{a @ A \lor B}{a @ B}
    & \quad &
    \infer[\lor E]{C}{a @ A \lor B & \infer*{C}{[a @ A]} & \infer*{C}{[a @ B]}}
    & \quad &
    \infer[\rightarrow I]{a @ (A \rightarrow B)}{\infer*{a @ B}{[a @ A]}}
    & \quad &
    \infer[\rightarrow E]{a @ B}{a @ (A \rightarrow B) & a @ A}
    \\[6pt]            
    \infer[\perp E]{C}{a @ \perp}
    & \quad &
    \infer[@ I]{c @ a @ A}{a @ A}
    \\[6pt]
  \end{array}
  \\[6pt]
  \begin{array}{cccccccccccccc}
    \infer[@ E]{a @ A}{c @ a @ A}
    & \quad &
    \infer[\Diamond I]{a @ \Diamond A}{e @ A & a @ \Diamond e}
    & \quad &
    \infer[\Diamond E]{C}{a @ \Diamond A & \infer*{C}{[c @ A][a @ \Diamond c]}}
    & \quad &
    \infer[\Box I]{a @ \Box A}{\infer*{c @ A}{[a @ \Diamond c]}}
    & \quad &
    \infer[\Box E]{e @ A}{a @ \Box A & a @ \Diamond e}
  \end{array}
  \end{array}
  \]
\end{calculus}

\begin{clarifications}
  Formulas of IHL are defined by the following grammar:
  \begin{center} 
    \begin{math}
      \begin{array}{lll}
        A,B,C ::= p \mid a \mid A \land B \mid A \lor B \mid A \to B \mid \perp \mid \Box A \mid \Diamond A \mid a @ A
      \end{array}
    \end{math}
  \end{center}
  where $a$ ranges over nominals and $p$ propositional symbols. In the
  rule $\Diamond E$, $c$ does not occur in $a@ \lozenge A $, in $C$,
  or in any undischarged assumptions other than the specified
  occurrences of $c@ A$ and $a@ \lozenge c $.  Furthermore, in $\Box
  I$, $c$ does not occur in $a@\square A $ or in any undischarged
  assumptions other than the specified occurrences of $a@ \lozenge c$.
\end{clarifications}

\begin{history}
%% ToDo: write here short historical remarks about this proof system,
%% especially if they relate to other proof systems. 
%% Use "\iref{OtherProofSystem}" to refer to another proof system 
%% in the Encyclopedia (where "OtherProofSystem" is its ID). 
%% Use "\irefmissing{SuggestedIDForOtherProofSystem}" to refer to 
%% another proof system that is not yet available in the encyclopedia.
  
The Natural Deduction system for IHL was originally suggested in
\cite{braunerdepaiva2003} and developed in \cite{braunerdepaiva2006}.
This system adds nominals and satisfaction operators to a version of
Intuitionistic Modal Logic described using labelled deduction, in the
style of Simpson \cite{simpson1994}.  Axioms, or a Hilbert-style
calculus version of the system, were provided in \cite{brauner2006}.
Some of the properties of the intuitionistic system, as well as a
discussion of some of its applications to type systems in computing,
appeared in \cite{brauner2011}.
\end{history}

%% \begin{technicalities}
%% ...
%% \end{technicalities}

\end{entry}
