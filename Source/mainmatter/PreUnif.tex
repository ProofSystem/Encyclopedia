\calculusName{Higher-Order Pre-Unification}   
\calculusAcronym{}    
\calculusLogic{Unification}  
\calculusLogicOrder{Higher-Order}
\calculusType{Unification}   
\calculusYear{1975}   
\calculusAuthor{Gérard Pierre Huet} 


\entryTitle{Higher-Order Pre-Unification}
\entryAuthor{Tomer Libal}     

\entryReviewer{Bruno Woltzenlogel Paleo}


\maketitle

\begin{entry}{PreUnif}



\newcommand{\upair}[2]{\langle#1, #2\rangle}
\newcommand{\spair}[2]{\langle\langle#1, #2\rangle\rangle}
\newcommand{\vco}[2]{#1_1,\ldots,#1_{#2}}
\newcommand{\vc}[2]{\overline{#1_{#2}}}

\begin{calculus}

% Add the inference rules of your proof system here.
% The "proof.sty" and "bussproofs.sty" packages are available.
% If you need any other package, please contact the editor (bruno@logic.at)

\[
\infer[\mathit{delete}] {S}
               {\{\upair u u\} \cup S}
\qquad
\infer[\mathit{varelim}]{\{\spair z {{\lambda \vc x k}.v}\} \cup \sigma(S)\downarrow_\beta}
               {\{\upair {{\lambda\vc x k.z(\vc x k)}} {\lambda\vc x k.v}\} \cup S}
\]\\[-1em]
\[
\infer[\mathit{decomp}] {\{\upair {{\lambda\vc x k}.v_1} {{\lambda\vc x k}.u_1} ,\ldots,
                  \upair {{\lambda\vc x k}.v_n} {{\lambda\vc x k}.u_n}\}  \cup S}
               {\{\upair {{\lambda\vc x k}.a(\vc v n)}
                         {{\lambda\vc x k}.a(\vc u n}\} \cup S}
\]\\[-1em]
\[
\infer[\mathit{imitate}]{\{\upair {y\uparrow_\eta} {t\uparrow_\eta}, \upair {\lambda\vc x k.y(\vc u n)}
                 {\lambda\vc x k.b(\vc v m)}\} \cup S}
               {\{\upair {\lambda\vc x k.y(\vc u n)}
               {\lambda\vc x k.b(\vc v m)}\} \cup S}
\qquad
\infer[\mathit{project}]{\{\upair {y\uparrow_\eta} {s\uparrow_\eta}, \upair {\lambda\vc x k.y(\vc u n)}
                 {\lambda\vc x k.a(\vc v m)}\} \cup S}
               {\{\upair {\lambda\vc x k.y(\vc u n)}
               {\lambda\vc x k.a(\vc v m)}\} \cup S}
\]\\
Where $a\in\Sigma$ or $a\in{\vc x k}$; $b\in\Sigma$; $z$ does not occur in $v$;
$\sigma = [\lambda\vc x k.v/x]$; $t = \lambda\vc x n.b(\overline{y_m(\vc x n)})$;
$s = \lambda\vc x n.x_i(\overline{y_l(\vc x n)})$ for $0<i\leq n$ and $l = \texttt{ty}(x_i)$.
\end{calculus}

\begin{clarifications}
  $\Sigma$ is the term signature. $\vc o p = \vco o p$. $z,x,\vc x k$ and $y,\vc y m$ are variables.
  $v,\vc v n$ and $u,\vc u n$ are terms.
  $\upair v u$ and $\spair v u$ are unsolved and solved, respectively, pairs of $\lambda$-terms.
  $S$ is a set of such pairs and $\sigma$ is a substitution.
  $\downarrow_\beta$ denotes $\beta$-normalization and $\uparrow_\eta$ denotes $\eta$-expansion.
  $\texttt{ty}(a) = n$ for a symbol $a$ of type $\beta_1\rightarrow\ldots\rightarrow \beta_n\rightarrow \gamma$.
  The set $S$ must originaly contain terms in $\beta$-normalized and $\eta$-expanded form.
\end{clarifications}

\begin{history}
   In contrast to the first-order case,
   the question whether higher-order terms are unifiable is
   undecidable already in the second-order case~\cite{goldfarb81tcs}.
   The first complete procedure for higher-order unification was given by
   Jensen and Pietrzykowski~\cite{jensen76tcs}.
   The use of pre-unifiers, introduced by Huet, enabled the search to
   be less redundant and more efficient.
\end{history}

\begin{technicalities}
  Huet~\cite{huet75tcs} introduced the procedure without assuming the axiom of
  functional extensionality and showed that assuming this axiom makes the
  procedure non-redundant.  The above set of rules assumes extensionality.  The
  set $S$ is considered pre-solved if it contains only solved or ``flex-flex''
  pairs where a ``flex'' term is a term whose head is a free variable.  The
  solved pairs in $S'$ are the substitution components~\cite{Robinson1965JACM}
  of a pre-unifier of $S$ which can always be extended into a unifier.  The
  application of $\mathit{imitate}$ and $\mathit{project}$ is a ``don't-know''
  non-determinism, while the choice of $\mathit{delete}$, $\mathit{varelim}$ or
  $\mathit{decomp}$ is a ``don't-care'' non-determinism.  Nevertheless, by
  choosing an appropriate strategy, the application of the above rules always
  terminates on a unifiable set of pairs of terms and enumerates (with
  extensionality) a complete and minimal set of pre-unifiers.
\end{technicalities}

\end{entry}
