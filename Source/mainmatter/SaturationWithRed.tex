


\calculusName{Saturation With Redundancy}   
\calculusAcronym{}     
\calculusLogic{Classical Logic}  
\calculusType{Meta-Calculus}   
\calculusYear{1990}   
\calculusAuthor{Leo Bachmair} \calculusAuthor{Harald Ganzinger} 


\entryTitle{Saturation With Redundancy}     
\entryAuthor{Uwe Waldmann}     





\maketitle



\begin{entry}{SaturationWithRed}  




\begin{calculus}

% Add the inference rules of your proof system here.
% The "proof.sty" and "bussproofs.sty" packages are available.
% If you need any other package, please contact the editor (bruno@logic.at)

\textbf{Primary Rules}

\[
\infer[\textit{Deduction}]
{N \cup \{C\}}{N & N \models C}
\]
\[
\infer[\textit{Deletion}]
{N}{N \cup \{C\} & C \textrm{~$\mathcal{R}$-redundant w.\,r.\,t.~} N}
\]

\textbf{Derived Rules}

\[
\infer[\textit{Simplification}]
{N \cup M}{N \cup \{C\} & N \cup \{C\} \models M & C \textrm{~$\mathcal{R}$-redundant w.\,r.\,t.~} N \cup M}
\]
\mbox{}\quad is a shorthand for
\[
\infer[\textit{Deletion}]
{N \cup M}{\infer[\textit{Deduction}^+]
           {N \cup \{C\} \cup M}{N \cup \{C\} & N \cup \{C\} \models M}
           & C \textrm{~$\mathcal{R}$-redundant w.\,r.\,t.~} N \cup M}
\]

$N$ and $M$ are finite sets of formulas, $C$ is a formula.
\end{calculus}



\begin{clarifications}
This is a meta-inference system for refutational calculi
that is parameterized by
(1) an entailment relation~$\models$, (2) an inference system $\mathcal{I}$
and (3) a redundancy criterion $\mathcal{R}$ for formulas and inferences,
such that $\mathcal{I}$-inferences are sound w.\,r.\,t.~$\models$,
and such that $\mathcal{I}$-inferences whose conclusion is contained in
$N$ are $\mathcal{R}$-redundant w.\,r.\,t.~$N$.
Note that the \textit{Deduction} rule is not restricted to adding the conclusions
of $\mathcal{I}$-inferences from $N$;
fairness, however, requires that
every $\mathcal{I}$-inference from persisting formulas must become
$\mathcal{R}$-redundant at some point
(for instance, by adding its conclusion).
\end{clarifications}

\begin{history}
In theorem proving calculi with a redundancy concept,
closure under the inference rules can be replaced by
a refined notion of saturation that allows to
alternate between derivation of new formulas and
elimination of irrelevant formulas
(e.\,g., tautologies and subsumed formulas).
The system was introduced by
Bachmair and Gan\-zin\-ger~\cite{BachmairGanzinger1990CTRS}
for superposition~\iref{Superposition};
it can be used for most other super\-posi\-tion-like calculi,
such as constraint superposition~\iref{ConstraintSup},
superposition modulo theories~\iref{CancellativeSup}, or
hierarchic superposition~\iref{HierarchicSup}, with appropriate choices for
\looseness=-1
$\models$, $\mathcal{I}$, and $\mathcal{R}$.
\end{history}

% \begin{technicalities}
% ToDo: write here remarks about soundness, completeness, decidability...
% \end{technicalities}













\end{entry}
