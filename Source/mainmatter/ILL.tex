


\calculusName{Intuitionistic Linear Logic}
\calculusAcronym{\ILL}
%\calculusLogic{Intuitionistic Linear Logic}
\calculusLogic{Linear Logics}
\calculusLogicOrder{Propositional}
\calculusType{Sequent Calculus}
\calculusYear{1987}

\calculusAuthor{Jean-Yves Girard}
\calculusAuthor{Yves Lafont}


\entryTitle{Intuitionistic Linear Logic (ILL)}
\entryAuthor{Joseph Boudou} \entryAuthor{Sergei Soloviev}


\maketitle


\begin{entry}{ILL} 

\newcommand{\llaconj}{\binampersand}
\newcommand{\lladisj}{\oplus}
\newcommand{\llimp}{\multimap}
\newcommand{\llmconj}{\otimes}
\newcommand{\llmdisj}{\bindnasrepma}
\newcommand{\llzero}{0}
\newcommand{\llone}{1}

\newcommand{\sepproof}{\hskip 2em plus 6em\relax}
\newcommand{\sepseq}{\quad}
\newcommand{\sepline}{\]\[}

\newenvironment{infruleset}[1]{%
  \sc{#1} \vspace{-1ex} \[ %
}{%
  \] %
}

\begin{calculus}

\begin{infruleset}{Structural}
  \infer{A \vdash A}{}
  \sepproof
  \infer[\mbox{\textit{(cut)}}]{\Gamma, \Delta \vdash B}{\Gamma \vdash A \sepseq A, \Delta \vdash B}
  \sepproof
  \infer{\Gamma, B, A, \Delta \vdash C}{\Gamma, A, B, \Delta \vdash C}
\end{infruleset}

\begin{infruleset}{Multiplicative}
  \infer{\vdash \llone}{}
  \sepproof
  \infer{\Gamma, \llone \vdash A}{\Gamma \vdash A}
  \sepproof
  \infer{\Gamma, \Delta \vdash A \llmconj B}{\Gamma \vdash A \sepseq \Delta \vdash B}
  \sepproof
  \infer{\Gamma, A \llmconj B \vdash C}{\Gamma, A, B \vdash C}
  \sepline
%
  \infer{\Gamma \vdash A \llimp B}{\Gamma, A \vdash B}
  \sepproof
  \infer{\Gamma, A \llimp B, \Delta \vdash C}{\Gamma \vdash A \sepseq B, \Delta \vdash C}
\end{infruleset}

\begin{infruleset}{Additive}
  \infer{\Gamma \vdash \top}{}
  \sepproof
  \infer{\Gamma \vdash A \llaconj B}{\Gamma \vdash A \sepseq \Gamma \vdash B}
  \sepproof
  \infer{\Gamma, A_1 \llaconj A_2 \vdash B}{\Gamma, A_i \vdash B}
  \sepline
%
  \infer{\Gamma, \llzero \vdash A}{}
  \sepproof
  \infer{\Gamma \vdash A_1 \lladisj A_2}{\Gamma \vdash A_i}
  \sepproof
  \infer{\Gamma, A \lladisj B \vdash C}{\Gamma, A \vdash C \sepseq \Gamma, B \vdash C}
\end{infruleset}

\begin{infruleset}{Exponential}
  \infer{!\Gamma \vdash !A}{!\Gamma \vdash A}
  \sepproof
  \infer{\Gamma, !A \vdash B}{\Gamma \vdash B}
  \sepproof
  \infer{\Gamma, !A \vdash B}{\Gamma, A \vdash B}
  \sepproof
  \infer{\Gamma, !A \vdash B}{\Gamma, !A, !A \vdash B}
\end{infruleset}

\vspace{-1em}


\end{calculus}

\begin{clarifications}
  Succedents are single formulas.
  Antecedents are ordered list of formulas.
  If $\Gamma$ is the list $A_1, \ldots, A_n$ of formulas, $!\Gamma$ denotes the list $!A_1, \ldots, !A_n$.
  First order quantifiers can be added with rules similar to \LJ~\iref{GentzenLJ}.
  Conversely, removing the exponential rules leads to the intuitionistic multiplicative additive linear logic (IMALL).
  And by further removing the additive rules, the intuitionistic multiplicative linear logic (IMLL)~\cite{mints1977closed} is obtained.
\end{clarifications}

\begin{history}
  Introduced by Girard and Lafont in~\cite{lafont1987tapsoft} as intuitionistic variant of \LL~\iref{LL}.
  \ILL has multiple applications in categorical logic.
\end{history}

\begin{technicalities}
  Enjoys cut elimination~\cite{lafont1987tapsoft}.
\end{technicalities}


\end{entry}
