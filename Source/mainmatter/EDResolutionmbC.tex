
% If the calculus has an acronym, define it.
\newcommand{\EmbC}{\mathbf{E}^\mathsf{mbC}_{res}}

\calculusName{Erotetic Dual Resolution for mbC}   
\calculusAcronym{\EmbC}     
\calculusLogic{Paraconsistent Logics}  
\calculusLogicOrder{Propositional}
\calculusType{Resolution}   
\calculusYear{2014}   

\calculusAuthor{Szymon Chlebowski}
\calculusAuthor{Dorota Leszczy\'nska-Jasion} 



\entryTitle{Erotetic Dual Resolution for mbC}     
\entryAuthor{Dorota Leszczy\'nska-Jasion}     


\etag{resolution}
\etag{non-clausal resolution}
\etag{dual resolution}

\maketitle



\begin{entry}{EDResolutionmbC}  




\begin{calculus}

% Add the inference rules of your proof system here.
% The "proof.sty" and "bussproofs.sty" packages are available.
% If you need any other package, please contact the editor (bruno@logic.at)
The rules of $\Eres$ (see~\iref{EDResolutionCPL}) and the following rules (`$\lnot$' is used for the classical negation and `$\sim$' for the paraconsistent one):

\vspace{-0.3cm}

$$
\quad
\infer[\textbf{R}_{\sim }]
{?(\Phi;\:\dashv S\:'\:\neg A\:' T;\:\dashv S\:'\:\chi\sim A\:'\:T;\Psi)}
{?(\Phi;\:\dashv S\:' \sim A\:'\:T;\Psi)}
\qquad
\infer[\textbf{R}_{\neg\sim}]
{?(\Phi;\:\dashv S\:'\:A\:'\:\neg\chi\sim A\:'\:T;\Psi)}
{?(\Phi;\:\dashv S\:'\:\neg\sim A\:'\:T;\Psi)}$$

\vspace{-0.5cm}

$$\infer[\textbf{R}_{\circ}]
{?(\Phi\:;\:\dashv S\:'\:\neg A\:'\:\chi\!\circ A\:'\:T\:;\:\dashv S\:'\:\neg\sim A\:'\:\chi\!\circ A\:'\:T\:;\:\Psi)}
{?(\Phi\:;\:\dashv S\:'\:\circ A\:'\:T\:;\:\Psi)}$$

\vspace{-0.3cm}

$$\infer[\textbf{R}_{\neg\circ}]
{?(\Phi\:;\:\dashv S\:'\:A\:'\:\sim A\:'\:T\:;\:\dashv S\:'\:\neg\chi\!\circ A\:'\:T\:;\:\Psi)}
{?(\Phi\:;\:\dashv S\:'\:\neg\circ A\:'\:T\:;\:\Psi)}$$
\end{calculus}



\begin{clarifications}
See~\iref{EDResolutionCPL} for notational conventions.

The calculus is worded in a language being an extension of the language of $\mathsf{mbC}$, where the role of the additional $\chi$ operator is to syntactically express the fact that certain formulas (e.g. of the form `$\sim A$', `$\circ A$') may have a logical value independent of the value of $A$.
\end{clarifications}

\begin{history}
The calculus $\EmbC$ has been presented in~\cite{SzChDLJ:LFI} together with similar calculi for $\mathsf{CLuN}$, $\mathsf{CLuNs}$ and for Classical Propositional Logic. The idea to use $\chi$ operator was taken from~\cite{WVL:2005}, where the authors presented erotetic calculi for logics $\mathsf{CLuN}$ and $\mathsf{CLuNs}$ in a non-resolution account.
\end{history}

\begin{technicalities}
A formula $A$ is $\mathsf{mbC}$-valid iff $\dashv A$ has a Socratic refutation in $\EmbC$. A formula $A$ is $\mathsf{CLuN}$-valid iff $\dashv A$ has a Socratic refutation constructed without the use of rules $\mathbf{R}_\circ$, $\mathbf{R}_{\lnot \circ}$. Similar results are obtained with respect to $\mathsf{CLuNs}$ and the classical case.
\end{technicalities}













\end{entry}
