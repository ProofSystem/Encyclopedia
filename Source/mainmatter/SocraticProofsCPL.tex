
\newcommand{\ESTAR}{\mathbf{E}^*}

\calculusName{Socratic Proofs for CPL}   
\calculusAcronym{\ESTAR}     
\calculusLogic{Classical Logic}  
\calculusLogicOrder{Propositional}
\calculusType{Socratic Proof System}   
\calculusYear{2003}   
\calculusAuthor{Andrzej Wi\'{s}niewski} 


\entryTitle{Socratic Proofs for CPL}     
\entryAuthor{Dorota Leszczy\'nska-Jasion}     


\etag{Two-Sided Sequents}
\etag{Single Conclusion}
\etag{Sequence Cedents}
\etag{Invertible Rules}

\maketitle


\begin{entry}{SocraticProofsCPL}  

\begin{calculus}

\[
\begin{array}{ccc}

\infer[\textbf{L}_\alpha]{?(\Phi ~;~  S ~'~ \alpha_{1} ~'~ \alpha_{2} ~'~ T\:\vdash\:C ~;~ \Psi)}{?(\Phi ~;~ S ~'~ \alpha ~'~ T\:\vdash\: C ~;~ \Psi)}

&~~~~~~&

\infer[\textbf{R}_{\alpha}]{?(\Phi ~;~  S\:\vdash\:\alpha_{1} ~;~ S\:\vdash\: \alpha_{2} ~;~ \Psi)}{?(\Phi ~;~ S\:\vdash\: \alpha ~;~ \Psi)}

\\
\end{array}
\]

$$
\infer[\textbf{L}_{\beta}]{?(\Phi ~;~  S ~'~ \beta_{1} ~'~ T\:\vdash\: C ~;~ S ~'~ \beta_{2} ~'~ T\:\vdash\: C ~;~ \Psi)}{?(\Phi ~;~ S ~'~ \beta ~'~ T\:\vdash\: C ~;~ \Psi)}
$$

\[
\begin{array}{ccc}

\infer[\textbf{R}_{\beta}]{?(\Phi ~;~  S ~'~ \beta^{*}_{1}\:\vdash\:\beta_{2} ~;~ \Psi)}{?(\Phi ~;~    S\:\vdash\:\beta ~;~ \Psi)}

&~~~~~~&

\infer[\textbf{L}_{\lnot \lnot}]{?(\Phi ~;~  S ~'~ A ~'~ T\:\vdash\: C ~;~ \Psi)}{?(\Phi ~;~  S ~'~ \lnot \lnot A ~'~ T\:\vdash\: C ~;~ \Psi)}

\\
\end{array}
\]

Where:

\begin{center}
	\begin{tabular}{ccc|cccc}
			\hline 
		$\alpha$ & $\alpha_{1}$ & $\alpha_{2}$ & $\beta$ & $\beta_{1}$ & $\beta_{2}$ & $\beta^{*}_{1}$  \\ 
			\hline
		$\textit{A}\wedge B$ & $\textit{A}$ & $\textit{B}$ & $\neg(A\wedge B)$ & $\neg A$ & $\neg B$ & $A$ \\
			
		$\neg(A\vee B)$ & $\neg A$ & $\neg B$ & $\textit{A}\vee B$ & $\textit{A}$ & $\textit{B}$ & $\neg A$ \\
			
		$\neg(A \rightarrow B)$ & $\textit{A}$ & $\neg B$ & $\textit{A} \rightarrow B$ & $\neg A$ & $\textit{B}$ & $A$ \\
	\end{tabular}
\end{center}

\end{calculus}

\begin{clarifications}
The method of Socratic proofs is a method of transforming questions, but these are based on sequences of two-sided, single-conclusion sequents with sequences of formulas in both cedents. $\Phi$, $\Psi$ are finite (possibly empty) sequences of sequents. $S$, $T$ are finite (possibly empty) sequences of formulas. The semicolon `;' is the concatenation sign for sequences of sequents, whereas `$'$' is the concatenation sign for sequences of formulas. A Socratic proof of sequent `$S \vdash A$' in $\ESTAR$ is a finite sequence of questions guided by the rules of $\ESTAR$, starting with `$? (S \vdash A)$' and ending with a question based on a sequence of basic sequents, where a \textit{basic sequent} is a sequent containing the same formula in both of its cedents or containing a formula and its negation in the antecedent.
\end{clarifications}

\begin{history}
The method has been first presented in \cite{AW:2004}. Calculus $\ESTAR$ is called \textit{erotetic} calculus since it is a calculus of questions (\textit{erotema} means \textit{question} in Greek). Proof-theoretically, it may be viewed as a calculus of hypersequents with `;' understood conjunctively. It is grounded in Inferential Erotetic Logic (cf. \cite{AW:2013}).
\end{history}

\begin{technicalities}
A sequent `$S \vdash A$' has a Socratic proof in $\ESTAR$ iff $A$ is CPL-entailed by the set of terms of $S$. The rules are invertible.
\end{technicalities}












\end{entry}
