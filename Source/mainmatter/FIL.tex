


\calculusName{Full Intuitionistic Logic}   
\calculusAcronym{\FIL}     
\calculusLogic{Intuitionistic Logic}  
\calculusLogicOrder{Propositional}
\calculusType{Sequent Calculus}   
\calculusYear{1995} 
\calculusAuthor{Valeria de Paiva} \calculusAuthor{Luiz Pereira} 

\entryTitle{Full Intuitionistic Logic (FIL)}        
\entryAuthor{Harley Eades III} \entryAuthor{Valeria de Paiva}  


\etag{Two-Sided Sequents}
\etag{Dependency Tracking}


\maketitle



\begin{entry}{FIL}  



\begin{calculus}
\[
\begin{array}{ccc}
  \infer[ax]{A(n) \Rightarrow A/\{n\}}{}
  & \quad &
  \infer[\perp \Rightarrow]{\perp (n) \Rightarrow A_1/\{n\},\ldots,A_k/\{n\}}{}\\
  \\[-5pt]
  \infer[cut]{\Gamma_1,\Gamma_1 \Rightarrow \Delta_1,\Delta^*_1}{\Gamma_1 \Rightarrow \Delta_1,A/S & A(n),\Gamma_1 \Rightarrow \Delta_1}
  & \quad &
  \infer[perm \Rightarrow]{\Gamma_1,B(n),A(m),\Gamma_1 \Rightarrow \Delta}{\Gamma_1,A(m),B(n),\Gamma_1 \Rightarrow \Delta}\\
  \\[-5pt]
  \infer[\Rightarrow perm]{\Gamma \Rightarrow \Delta_1,B/S_1,A/S_1,\Delta_1}{\Gamma \Rightarrow \Delta_1,A/S_1,B/S_1,\Delta_1}
  & \quad &
  \infer[weak \Rightarrow]{A(n),\Gamma \Rightarrow \Delta^*}{\Gamma \Rightarrow \Delta}\\
  \\[-0.5pt]
  \infer[\Rightarrow weak]{\Gamma \Rightarrow \Delta,A/\{\}}{\Gamma \Rightarrow \Delta}
  & \quad &
  \infer[cont \Rightarrow]{\Gamma,A(k) \Rightarrow \Delta^*}{\Gamma,A(n),A(m) \Rightarrow \Delta}\\
  \\[-0.5pt]
  \infer[\Rightarrow cont]{\Gamma \Rightarrow \Delta,A/S_1 \cup S_1}{\Gamma \Rightarrow \Delta,A/S_1,A/S_1}
  & \quad &
  \infer[\lor \Rightarrow]{\Gamma_1,\Gamma_1,(A \lor B)(k) \Rightarrow \Delta^*_1,\Delta^*_1}{\Gamma_1,A(n) \Rightarrow \Delta_1 & \Gamma_1,B(m) \Rightarrow \Delta_1}\\
  \\[-0.5pt]
  \infer[\Rightarrow \lor]{\Gamma \Rightarrow \Delta,(A \lor B)/S_1 \cup S_1}{\Gamma \Rightarrow \Delta,A/S_1,B/S_1}
  & \quad &
  \infer[\land \Rightarrow]{\Gamma,(A \land B)(k) \Rightarrow \Delta^*}{\Gamma,A(n),B(m) \Rightarrow \Delta}\\
  \\[-0.5pt]
  \infer[\Rightarrow \land]{\Gamma \Rightarrow \Delta,(A \land B)/S_1 \cup S_1}{\Gamma \Rightarrow \Delta,A/S_1 & \Gamma \Rightarrow \Delta,B/S_1}
  & \quad &
  \infer[\to \Rightarrow]{(A \to B)(n),\Gamma_1,\Gamma_1 \Rightarrow \Delta_1,\Delta^*_1}{\Gamma_1 \Rightarrow \Delta_1,A/S & B(n),\Gamma_1 \Rightarrow \Delta_1}\\
  \\[-0.5pt]
  \infer[\Rightarrow \to]{\Gamma \Rightarrow \Delta, (A \to B)/S - \{n\}}{\Gamma, A(n) \Rightarrow \Delta,B/S}\\
\end{array}
\]
\end{calculus}



\begin{clarifications}
Sequents are of the form $\Gamma \Rightarrow \Delta$ where $\Gamma$ is
a multiset of pairs of formulas and natural number indicies, and
$\Delta$ is a multiset of pairs of formulas and sets of natural number
indicies.  The set of natural number indicies for a particular
conclusion, formula on the right, indicates which hypotheses the
conclusion depends on.  This dependency tracking is used to enforce
intuitionism in the rule $\Rightarrow \to$.  See \cite{dePaiva:2005}
for more details.
\end{clarifications}

\begin{history}
%% ToDo: write here short historical remarks about this proof system,
%% especially if they relate to other proof systems. 
%% Use "\iref{OtherProofSystem}" to refer to another proof system 
%% in the Encyclopedia (where "OtherProofSystem" is its ID). 
%% Use "\irefmissing{SuggestedIDForOtherProofSystem}" to refer to 
%% another proof system that is not yet available in the encyclopedia.
The system FIL was announced in the abstract \cite{dePaiva:1995} but
only published officially ten years later in \cite{dePaiva:2005}.  The
system was conceived after the remark in the paper describing
FILL \iref{FILL} that intuitionism is about proofs that resemble
functions, not about a cardinality constraint in the sequent
calculus. The system shows we can use a notion of {\em{dependency
between formulae}} to enforce the constructive character of
derivations. This is similar to an impoverished Curry-Howard term
assignment.
\end{history}












\end{entry}
