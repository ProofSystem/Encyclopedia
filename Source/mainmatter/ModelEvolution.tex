
% If the calculus has an acronym, define it.
% (e.g. \newcommand{\LK}{\ensuremath{\mathbf{LK}}\xspace})
% \newcommand{\ME}{\ensuremath{\mathbf{ME}}\xspace}

\calculusName{Model Evolution}   % The name of the calculus
%\calculusAcronym{\ME}    % The acronym if defined above, or empty otherwise. 
\calculusLogic{Classical Logic}  % Specify the logic (e.g. classical, intuitionistic, ...) for which this calculus is intended.
\calculusLogicOrder{First-Order}
\calculusType{Hybrid}   
% \calculusType{Instance-Based}   
\calculusYear{2003}   % The year when the calculus was invented.
\calculusAuthor{\iCA{Peter Baumgartner}, \iCA{Cesare Tinelli}} % The name(s) of the author(s) of the calculus.


\entryTitle{Model Evolution}     % Title of the entry (usually coincides with the name of the calculus).
\entryAuthor{Peter Baumgartner}     


% If you wish, use tags to give any other information 
% that might be helpful for classifying and grouping this entry:
% e.g. \tag{Two-Sided Sequents}
% e.g. \tag{Multiset Cedents}
% e.g. \tag{List Cedents}
% You are free to invent your own tags. 
% The Encyclopedia's coordinator will take care of 
% merging semantically similar tags in the future.


\maketitle


% If your files are called "MyProofSystem.tex" and "MyProofSystem.bib", 
% then you should write "\begin{entry}{MyProofSystem}" in the line below
\begin{entry}{ModelEvolution}

% Define here any newcommands you may need:
% e.g. \newcommand{\necessarily}{\Box}
% e.g. \newcommand{\possibly}{\Diamond}


\begin{calculus}

% Add the inference rules of your proof system here.
% The "proof.sty" and "bussproofs.sty" packages are available.
% If you need any other package, please contact the editor (bruno@logic.at)

\[
  \begin{array}{cc}
\multicolumn{2}{c}{
\infer[\textit{Split}]
{\Lambda, L\sigma\: \vdash\: \Phi,\: L \lor C\qquad \Lambda, (\overline{L}\sigma)^{\mathrm{sko}}\: \vdash\:  \Phi,\: L \lor C}
{\Lambda\: \vdash\:  \Phi,\: L \lor C}
}
    \\[3ex]
\multicolumn{2}{c}{
\infer[\textit{Commit}]
{{\Lambda, K, L, L\sigma}\:\vdash\:{\Phi} \qquad {\Lambda, K, L, \overline{L}\sigma}\:\vdash\:{\Phi}}
{{\Lambda, K, L}\:\vdash\:{\Phi}}
} \\[3ex]
\infer[\textit{Assert}]
{{\Lambda, L}\:\vdash\:{\Phi,\: L}}
{{\Lambda}\:\vdash\:{\Phi,\: L}}
&
\infer[\textit{Close}]
{{\Lambda}\:\vdash\:{\Box}}
{{\Lambda}\:\vdash\:{\Phi,\: C}}
  \end{array}
\]
\end{calculus}

% The following sections ("clarifications", "history", 
% "technicalities") are optional. If you use them, 
% be very concise and objective. Nevertheless, do write full sentences. 
% Try to have at most one paragraph per section, because line breaks 
% do not look nice in a short entry.

\begin{clarifications}
Model Evolution is a refutationally complete calculus for first-order clause
logic. The inference rules operate on sequents of the form $\Lambda\:\vdash\:\Phi$ where $\Lambda$ is a
set of literals and $\Phi$ is a clause set. Derivation trees are constructed top-down
and start with the sequent $\neg v\:\vdash\:\psi$, where $\neg v$ is a pseudo-literal, representing
the set of all negative literals, and $\psi$ is the given clause set. The calculus
derives (in the limit) a sequent $\Lambda\:\vdash\:\Phi$ such that the interpretation induced by $\Lambda$ is a model
of $\Phi$ unless $\psi$ is unsatisfiable. All inference rules above are subject
to certain applicability conditions,
see~\cite{Baumgartner:Tinelli:ModelEvolutionCalculus:CADE:2003}.
Additional optional inference rules, not shown here, help improve performance in practice.
\end{clarifications}

\begin{history}
Model Evolution~\cite{Baumgartner:Tinelli:ModelEvolutionCalculus:CADE:2003}
is an improved version of the earlier FDPLL
calculus~\cite{Baumgartner:FDPLL:CADE:00}, a lifiting of the core of the
propositional DPLL method to the first-order level. Model Evolution has been extended by
ordered paramodulation inference rules for equality
reasoning~\cite{Baumgartner:Tinelli:ModelEvolutionCalculusEquality:CADE:2005}, by lemma
learning techniques inspired by modern CDCL SAT
solvers~\cite{Baumgartner:etal:ModelEvolutionLearning:LPAR:2006} and by 
reasoning modulo background
theories~\cite{Baumgartner:Fuchs:Tinelli:MELIA:LPAR:2008,Baumgartner:Tinelli:MEET:CADE:2011}. 
It has been combined with the superposition calculus in~\cite{Baumgartner:Waldmann:MESUP:CADE:2009}.
\end{history}

% \begin{technicalities}
% ToDo: write here remarks about soundness, completeness, decidability...
% \end{technicalities}


% General Instructions:
% =====================

% The preferred length of an entry is 1 page. 
% Do the best you can to fit your proof system in one page.
%
% If you are finding it hard to fit what you want in one page, remember:
%
%   * Your entry needs to be neither self-contained nor fully understandable
%     (the interested reader may consult the cited full paper for details)
%
%   * If you are describing several proof systems in one entry, 
%     consider splitting your entry.
%
%   * You may reduce the size of your entry by ommitting inference rules
%     that are already described in other entries.
%
%   * Cite parsimoniously (see detailed citation instructions below).
%
% 
% If you do not manage to fit everything in one page, 
% it is acceptable for an entry to have 2 pages.
%
% For aesthetical reasons, it is preferable for an entry to have
% 1 full page or 2 full pages, in order to avoid unused blank space.



% Citation Instructions:
% ======================

% Please cite the original paper where the proof system was defined.
% To do so, you may use the \cite command within 
% one of the optional environments above,
% or use the \nocite command otherwise.

% You may also cite a modern paper or book where the 
% proof system is explained in greater depth or clarity.
% Cite parsimoniously.

% Do not cite related work. Instead, use the "\iref" or "\irefmissing" 
% commands to make an internal reference to another entry, 
% as explained within the "history" environment above.

% You do not need to create the "References" section yourself. 
% This is done automatically.




% Leave an empty line above "\end{entry}".

\end{entry}
