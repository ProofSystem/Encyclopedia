

% \newcommand{\ME}{\ensuremath{\mathbf{ME}}\xspace}

\calculusName{Model Evolution}   
%\calculusAcronym{\ME}     
\calculusLogic{Classical Logic}  
\calculusLogicOrder{First-Order}
\calculusType{Hybrid}   
% \calculusType{Instance-Based}   
\calculusYear{2003}   
\calculusAuthor{Peter Baumgartner} \calculusAuthor{Cesare Tinelli} 


\entryTitle{Model Evolution}     
\entryAuthor{Peter Baumgartner}     





\maketitle



\begin{entry}{ModelEvolution}




\begin{calculus}

% Add the inference rules of your proof system here.
% The "proof.sty" and "bussproofs.sty" packages are available.
% If you need any other package, please contact the editor (bruno@logic.at)

\[
  \begin{array}{cc}
\multicolumn{2}{c}{
\infer[\textit{Split}]
{\Lambda, L\sigma\: \vdash\: \Phi,\: L \lor C\qquad \Lambda, (\overline{L}\sigma)^{\mathrm{sko}}\: \vdash\:  \Phi,\: L \lor C}
{\Lambda\: \vdash\:  \Phi,\: L \lor C}
}
    \\[3ex]
\multicolumn{2}{c}{
\infer[\textit{Commit}]
{{\Lambda, K, L, L\sigma}\:\vdash\:{\Phi} \qquad {\Lambda, K, L, \overline{L}\sigma}\:\vdash\:{\Phi}}
{{\Lambda, K, L}\:\vdash\:{\Phi}}
} \\[3ex]
\infer[\textit{Assert}]
{{\Lambda, L}\:\vdash\:{\Phi,\: L}}
{{\Lambda}\:\vdash\:{\Phi,\: L}}
&
\infer[\textit{Close}]
{{\Lambda}\:\vdash\:{\Box}}
{{\Lambda}\:\vdash\:{\Phi,\: C}}
  \end{array}
\]
\end{calculus}



\begin{clarifications}
Model Evolution is a refutationally complete calculus for first-order clause
logic. The inference rules operate on sequents of the form $\Lambda\:\vdash\:\Phi$ where $\Lambda$ is a
set of literals and $\Phi$ is a clause set. Derivation trees are constructed top-down
and start with the sequent $\neg v\:\vdash\:\psi$, where $\neg v$ is a pseudo-literal, representing
the set of all negative literals, and $\psi$ is the given clause set. The calculus
derives (in the limit) a sequent $\Lambda\:\vdash\:\Phi$ such that the interpretation induced by $\Lambda$ is a model
of $\Phi$ unless $\psi$ is unsatisfiable. All inference rules above are subject
to certain applicability conditions,
see~\cite{Baumgartner:Tinelli:ModelEvolutionCalculus:CADE:2003}.
Additional optional inference rules, not shown here, help improve performance in practice.
\end{clarifications}

\begin{history}
Model Evolution~\cite{Baumgartner:Tinelli:ModelEvolutionCalculus:CADE:2003}
is an improved version of the earlier FDPLL
calculus~\cite{Baumgartner:FDPLL:CADE:00}, a lifiting of the core of the
propositional DPLL method to the first-order level. Model Evolution has been extended by
ordered paramodulation inference rules for equality
reasoning~\cite{Baumgartner:Tinelli:ModelEvolutionCalculusEquality:CADE:2005}, by lemma
learning techniques inspired by modern CDCL SAT
solvers~\cite{Baumgartner:etal:ModelEvolutionLearning:LPAR:2006} and by 
reasoning modulo background
theories~\cite{Baumgartner:Fuchs:Tinelli:MELIA:LPAR:2008,Baumgartner:Tinelli:MEET:CADE:2011}. 
It has been combined with the superposition calculus in~\cite{Baumgartner:Waldmann:MESUP:CADE:2009}.
\end{history}

% \begin{technicalities}
% ToDo: write here remarks about soundness, completeness, decidability...
% \end{technicalities}













\end{entry}
