


\calculusName{Paramodulation}   
\calculusAcronym{}     
\calculusLogic{Classical Logic}  
\calculusLogicOrder{First-Order}
\calculusType{Resolution}   
\calculusYear{1969}   
\calculusAuthor{George Robinson} \calculusAuthor{Larry Wos} 


\entryTitle{Paramodulation}     
\entryAuthor{Uwe Waldmann}     





\maketitle



\begin{entry}{Paramodulation}  




\begin{calculus}

% Add the inference rules of your proof system here.
% The "proof.sty" and "bussproofs.sty" packages are available.
% If you need any other package, please contact the editor (bruno@logic.at)

\[
\infer[\textit{Paramodulation}]
{(D \lor C \lor L[u'])\sigma}{D \lor u \approx u'
& C \lor L[v]}
\]
\[
\infer[\textit{Resolution}]
{(D \lor C)\sigma}{D \lor B\vphantom{[]}
& C \lor \neg A}
\]
\[
\infer[\textit{Factoring}]
{(C \lor L_1)\sigma}{C \lor L_1 \lor \dots \lor L_n}
\]
\[
\infer[\textit{Reflexivity}]
{x \approx x\vphantom{[]}}{\vphantom{[]}}
\]

\medskip

$C,D$ are (possibly empty) equational clauses,
$L,L_1,\dots,L_n$ are literals,
$A,B$ are atoms,
$u,u',v$ are terms;
$u$ and $v$, $A$ and $B$, or $L_1,\dots,L_n$, respectively,
are unifiable with most general unifier~$\sigma$.

\end{calculus}



\begin{clarifications}
Paramodulation is a refutational saturation calculus for
first-order clauses (disjunctions of possibly negated atoms)
with equality (denoted by~$\approx$).
It works on a set $N$ of clauses that is saturated
by successively computing inferences
with premises in $N$ and adding the conclusion of the inference to $N$,
until the empty clause (i.\,e., false) is derived.
\end{clarifications}

\begin{history}
Handling the equality axioms in the resolution calculus~\iref{Resolution}
is impractical due to the huge search space generated in particular
by the transitivity axiom.
The paramodulation calculus
developed by Robinson and Wos~\cite{RobinsonWos1969} extends
resolution by specific inference rules that
render explicit inferences with the equality axioms unnecessary.
The original completeness proof also assumed the
presence of so-called functional-reflexive axioms of the
form
$f(x_1,\dots,x_n) \approx f(x_1,\dots,x_n)$;
this was later shown to be superfluous by Brand~\cite{Brand1975SIAMJC}.
Many refinements were developed in the sequel,
aiming in particular at reducing the number of possible inferences,
see~\iref{Superposition}.

\end{history}

\begin{technicalities}
The paramodulation calculus is refutationally complete for
first-order logic with equality.
\end{technicalities}













\end{entry}
