\calculusName{Sequent Calculus for Logic of Partial Quasiary Predicates}
\calculusAcronym{QE-calculus}
\calculusLogic{Extended Logic of Partial Quasiary Predicates}
\calculusLogicOrder{First-order}
\calculusType{Sequent Calculus}
\calculusYear{2017}

\calculusAuthor{Mykola Nikitchenko}
\calculusAuthor{Stepan Skilniak}

\entryTitle{Sequent Calculus for Logic of Partial Quasiary Predicates}
\entryAuthor{Mykola Nikitchenko}     
\entryAuthor{Stepan Skilniak}

\maketitle

\begin{entry}{QE-calculus}  

\begin{calculus}

Sequent rules for propositional compositions: 
\begin{center}
\begin{minipage}[b]{0.45\textwidth}
\raisebox{5pt}{$\vee_L$} \infer[;]
{\Phi \vee \Psi,\Gamma\to\Delta}{\Phi,\Gamma \to \Delta && \Psi,\Gamma\to \Delta}
\end{minipage}
\begin{minipage}[t]{0.45\textwidth}
\raisebox{5pt}{$\vee_R$} \infer[;]
{\Gamma\to \Phi \vee \Psi,\Delta}{\Gamma\to\Phi,\Psi,\Delta}
\end{minipage}


\begin{minipage}[b]{0.45\textwidth}
\raisebox{5pt}{$\neg_L$} \infer[;]
{\neg \Phi,\Gamma\to\Delta}{\Gamma\to\Phi,\Delta}
\end{minipage}
\begin{minipage}[t]{0.45\textwidth}
\raisebox{5pt}{$\neg_R$} \infer[.]
{\Gamma\to\neg\Phi,\Delta}{\Phi,\Gamma\to\Delta}
\end{minipage}
\end{center}

Sequent rules for renomination compositions ($C_{{\rm R}\exists}$ is $ u \in
fu(R^{\bar v}_{\bar x}(\exists y \Phi))$, $ C_{\rm RU}$ is $y \in fu(\Phi)$):
\begin{center}
\begin{minipage}[b]{0.45\textwidth}
\raisebox{5pt}{R${\vee_L}$} \infer[;]
{R^{\bar v}_{\bar x}(\Phi\vee\Psi),\Gamma\to\Delta}
{R^{\bar v}_{\bar x}(\Phi)\vee R^{\bar v}_{\bar x}(\Psi),\Gamma\to\Delta}
\end{minipage}
\begin{minipage}[t]{0.45\textwidth}
\raisebox{5pt}{R${\vee_R}$} \infer[;]
{\Gamma\to R^{\bar v}_{\bar x}(\Phi\vee\Psi),\Delta}
{\Gamma\to R^{\bar v}_{\bar x}(\Phi)\vee R^{\bar v}_{\bar x}(\Psi),\Delta}
\end{minipage}


\begin{minipage}[b]{0.45\textwidth}
\raisebox{5pt}{R${\neg_L}$} \infer[;]
{R^{\bar v}_{\bar x}(\neg\Phi),\Gamma\to\Delta}
{\neg R^{\bar v}_{\bar x}(\Phi),\Gamma\to\Delta}
\end{minipage}
\begin{minipage}[t]{0.45\textwidth}
\raisebox{5pt}{R${\neg_R}$} \infer[;]
{\Gamma\to R^{\bar v}_{\bar x}(\neg\Phi),\Delta}
{\Gamma\to\neg R^{\bar v}_{\bar x}(\Phi),\Delta}
\end{minipage}


\begin{minipage}[b]{0.45\textwidth}
\raisebox{5pt}{RR$_{L}$} \infer[;]
{R^{\bar v}_{\bar x}(R^{\bar w}_{\bar y}(\Phi)),\Gamma\to\Delta}
{R^{\bar v}_{\bar x}\circ^{\bar w}_{\bar y}(\Phi),\Gamma\to\Delta}
\end{minipage}
\begin{minipage}[t]{0.45\textwidth}
\raisebox{5pt}{RR$_{R}$} \infer[;]
{\Gamma\to R^{\bar v}_{\bar x}(R^{\bar w}_{\bar y}(\Phi)),\Delta}
{\Gamma\to R^{\bar v}_{\bar x}\circ^{\bar w}_{\bar y}(\Phi),\Delta}
\end{minipage}


\begin{minipage}[b]{0.45\textwidth}
\raisebox{5pt}{R$\exists_{L}$} \infer[, C_{{\rm R}\exists};]
{R^{\bar v}_{\bar x}(\exists y \Phi),\Gamma\to\Delta}
{\exists u R^{\bar v}_{\bar x}R^{y}_{u}(\Phi),\Gamma\to\Delta}
\end{minipage}
\begin{minipage}[t]{0.45\textwidth}
\raisebox{5pt}{R$\exists_{R}$} \infer[, C_{{\rm R}\exists};]
{\Gamma\to R^{\bar v}_{\bar x}(\exists y \Phi),\Delta}
{\Gamma\to \exists u R^{\bar v}_{\bar x}R^{y}_{u}(\Phi),\Delta}
\end{minipage}


\begin{minipage}[b]{0.45\textwidth}
\raisebox{5pt}{R$\varepsilon s_L$} \infer[, z \notin {\bar v};]
{R^{\bar v}_{\bar x}(\varepsilon z),\Gamma\to\Delta}{\varepsilon z, \Gamma\to\Delta}
\end{minipage}
\begin{minipage}[b]{0.45\textwidth}
\raisebox{5pt}{R$\varepsilon s_R$} \infer[, z \notin {\bar v};]
{ \Gamma\to R^{\bar v}_{\bar x}(\varepsilon z),\Delta}{\Gamma\to\varepsilon z,\Delta }
\end{minipage}



\begin{minipage}[b]{0.45\textwidth}
\raisebox{5pt}{R$\varepsilon_L$} \infer[;]
{R^{\bar v,z}_{\bar x, y}(\varepsilon z),\Gamma\to\Delta}{\varepsilon y, \Gamma\to\Delta}
\end{minipage}
\begin{minipage}[b]{0.45\textwidth}
\raisebox{5pt}{R$\varepsilon_R$} \infer[;]
{ \Gamma\to R^{\bar v,z}_{\bar x,y}(\varepsilon z),\Delta}{\Gamma\to\varepsilon y,\Delta}
\end{minipage}

\begin{minipage}[b]{0.45\textwidth}
\raisebox{5pt}{R$_L$} \infer[;]
{R(\Phi),\Gamma\to\Delta}{\Phi,\Gamma\to\Delta}
\end{minipage}
\begin{minipage}[t]{0.45\textwidth}
\raisebox{5pt}{R$_R$} \infer[;]
{\Gamma\to R(\Phi),\Delta}{\Gamma\to\Phi,\Delta}
\end{minipage}

\begin{minipage}[b]{0.45\textwidth}
\raisebox{5pt}{RI$_L$} \infer[;]
{R^{z,\bar v}_{z,\bar x}(\Phi),\Gamma\to\Delta}{R^{\bar v}_{\bar x}(\Phi),\Gamma\to\Delta}
\end{minipage}
\begin{minipage}[t]{0.45\textwidth}
\raisebox{5pt}{RI$_R$} \infer[;]
{\Gamma\to R^{z,\bar v}_{z,\bar x}(\Phi),\Delta}{\Gamma\to R^{\bar v}_{\bar x}(\Phi),\Delta}
\end{minipage}

\begin{minipage}[b]{0.45\textwidth}
\raisebox{5pt}{RU$_L$} \infer[, C_{\rm RU};]
{R^{y,\bar v}_{z,\bar u}(\Phi),\Gamma\to\Delta}{R^{\bar v}_{\bar u}(\Phi),\Gamma\to\Delta}
\end{minipage}
\begin{minipage}[t]{0.45\textwidth}
\raisebox{5pt}{RU$_R$} \infer[, C_{\rm RU}.]
{\Gamma\to R^{y,\bar v}_{z,\bar u}(\Phi),\Delta}{\Gamma\to R^{\bar v}_{\bar u}(\Phi),\Delta}
\end{minipage}
\end{center}

\smallskip
Sequent rules for   quantification compositions:
\begin{center}
\begin{minipage}[b]{0.45\textwidth}
\raisebox{5pt}{$\exists$E$_L$} \infer[,z \in fu(\exists x \Phi,\Gamma,\Delta);]
{\exists x \Phi,\Gamma\to\Delta}{R^x_z(\Phi),\Gamma\to\varepsilon z,\Delta}
\end{minipage}
\begin{minipage}[t]{0.45\textwidth}
\raisebox{5pt}{$\exists$E1$_R$} \infer[;]
{\Gamma\to \exists x \Phi,\varepsilon y,\Delta}{ \Gamma\to R^x_y(\Phi),\exists x \Phi,\varepsilon y,   \Delta}
\end{minipage}


\begin{minipage}[t]{0.90\textwidth}
\raisebox{5pt}{$\exists$E2$_R$} \infer[,\varepsilon (\Delta)=\emptyset,z \in fu(\exists x \Phi,\Gamma,\Delta);]
{\Gamma\to \exists x \Phi,\Delta}{ \Gamma\to R^x_z(\Phi),\varepsilon z,\exists x \Phi,\Delta   }
\end{minipage}


\begin{minipage}[t]{0.90\textwidth}
\raisebox{5pt}{$\exists$E3$_R$} \infer[, y\in uns(\Gamma\to\Delta).]
{\Gamma\to \exists x \Phi,\Delta}{\varepsilon y, \Gamma\to\exists x \Phi,\Delta && \Gamma \to  R^x_y(\Phi),\varepsilon y,\exists x \Phi,\Delta}
\end{minipage}
\end{center}
Rule $\exists$E1$_R$  is applied when at least one variable is assigned.
Rule $\exists$E2$_R$ is applied when there are no assigned variables (in this
case a fresh unassigned variables is assigned). This means  the first
application of quantification elimination (therefore $\varepsilon
(\Delta)=\emptyset$).  Rule $\exists$E3$_R$ is applied when an unspecified
variable is involved into quantifier elimination. In this case two branches
appear:  with this variable being unassigned and assigned.  

\end{calculus}


\begin{clarifications}
  Quasiary predicates are predicates over partial assignments of variables. Such
  predicates do not have fixed arity and generalize $n$-ary predicates. This
  specific feature of quasiary predicates requires additional operations
  (compositions): renomination $R_{x_{1} ,...,x_{n} }^{v_{1} ,...,v_{n} }$
  ($R^{\bar v}_{\bar x}$) and  variable unassignment  predicate $\varepsilon x$.
  Thus, the language of the logic is defined by Kleene-like compositions:
  disjunction $\vee$, negation $\neg$, existential quantifier $\exists x$,
  $R^{\bar v}_{\bar x}$, and $\varepsilon x$. Quasiary predicates are sensitive to
  unassigned variables therefore rules of sequent calculus classify variables as
  assigned, unassigned, and unspecified. To make correct transformations of
  formulas from $\Gamma$, unessential variables (analog of fresh variables in
  classical logic) are required; their set is denoted $fu(\Gamma)$.      
\end{clarifications}

\begin{history}
  Quasiary predicates (and functions) represent semantics of programs and their
  components.  Thus, logics of quasiary predicates are program-oriented logics.
  Various types of such logics are described in~\cite{NikShk2008,NikShk2013}. The
  presented QE-calculus, proof of its soundness and completeness can be found
  in~\cite{NikShk2017}. Hoare-like program logic based on partial quasiary predicates
  is described in~\cite{KNS2013}. Satisfiability-preserving translation of
  formulas of quasiary logic to formulas of classical first-order logic is
  formulated in~\cite{NikTym2012}.
\end{history}


\end{entry}
