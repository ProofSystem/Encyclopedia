\calculusName{Natural Deduction for Modal Logic \textbf{K}}
\calculusAcronym{\NDK}
\calculusLogic{Intuitionistic}
\calculusLogicOrder{Higher-Order}  
\calculusType{Natural Deduction}       
\calculusYear{2015}         

\calculusAuthor{Bruno Woltzenlogel Paleo}   

\entryTitle{Modal Natural Deduction}  
\entryAuthor{Bruno Woltzenlogel Paleo}     


\maketitle

\begin{entry}{ModalNaturalDeduction}  

\newcommand{\s}{\qquad}
\newcommand{\nec}{\Box} % necessarily
\newcommand{\pos}{\Diamond} % possibly

\begin{calculus}

\s\s\s\s\s\s
\infer[\nec_I]{\nec A}{\omega: \fbox{\infer*{A}{}} }
\s\s\s\s
\infer[\nec_E]{w: \fbox{ \infer*{}{A} } }{\nec A}

\vspace{2em}

\s\s\s\s\s\s
\infer[\pos_I]{\pos A}{w: \fbox{\infer*{A}{}} }
\s\s\s\s
\infer[\pos_E]{\omega: \fbox{ \infer*{}{A} } }{\pos A}

\vspace{1em}


\begin{center}
  \textbf{eigen-box condition:}\\ 
  a modal inference is said to \emph{access}\\
   the box immediately above or below it. \\
  $\nec_I$ and $\pos_E$ are \emph{strong} modal rules: \\
  $\omega$ must be a fresh name for the box they access \\ 
  (in analogy to the eigen-variable condition for strong quantifier rules). \\
  Every box must be accessed by \emph{exactly one} strong modal inference. \\
  \vspace{0.5em}
  \textbf{boxed assumption condition:} \\
  assumptions should be discharged within the box where they are created.
\end{center}

\end{calculus}


\begin{clarifications}
  This calculus extends any natural deduction calculus (e.g.~\iref{GentzenNJ}) by
  surrounding derivations by boxes and by adding the four modal rules shown above,
  which introduce and eliminate modal operators, moving formulas in and out of
  boxes.
\end{clarifications}

\begin{history}
  An embedding of (a higher-order version of) this calculus in the \textsc{Coq}
  proof assistant was mentioned in~\cite{CSR} and the calculus was studied in more
  depth in~\cite{StudiaLogica}, where it was used to formalize G\"odel's
  ontological argument.
\end{history}

\begin{technicalities}
  In~\cite{StudiaLogica}, this calculus was shown to be sound and complete
  relative to a natural deduction calculus extended with a necessitation rule and
  the modal axiom $K$. Consequently, the calculus is sound and complete for rigid
  modal logic \textbf{K}.
\end{technicalities}


\end{entry}
