\preface
\vspace{-150pt}
\section*{2$^{nd}$ Edition}
\vspace{130pt}

In December 2014, I had the honor to submit one of the first entries to the
\textbf{Encyclopedia of Proof Systems} at the request of Bruno Woltzenlogel
Paleo.
%
Less than one year later, the Encyclopedia already counted 64 entries, which
were presented at a poster session during CADE-25.
%
After this successful event, Bruno has kindly invited me to co-organize with him
a workshop during the Brasilia Spring on Automated Deduction, formed by the
conferences TABLEAUX, ITP (Interactive Theorem Proving) and FroCoS (Frontiers of
Combining Systems), in September 2017.

The EPS workshop comprised of presentations of new entries by the authors, an
open discussion about the Encyclopedia (suggestion of improvements and long-term
goals), and a hands-on session for active contributions.
%
The workshop was accompanied by a poster session where the newest entries were
displayed.
%
We would like to thank
Katalin Bimb\'{o},
Serenella Cerrito,
Clare Dixon,
Reiner H\"{a}hnle,
Rolf Hennicker,
Ullrich Hustadt,
Bj\"{o}rn Lellmann,
Jo\~{a}o Marcos,
Renate Schmidt, and
Yoni Zohar
for participating in the workshop and contributing to the discussions.
%
There was a wide variety of interesting and accessible talks about proof systems
in different areas, and many suggestions of new entries and features for
the Encyclopedia.
%
We would also like to thank Cl\'{a}udia Nalon for all her support with the
logistics of the workshop and for organizing a great conference.

In total, 34 new entries by 32 authors were submitted to the Encyclopedia of
Proof Systems.
%
Once again, a wide range of calculi is represented, such as resolution, sequent,
axiomatic, display, and natural deduction.
%
In addition to different logics (e.g., temporal, paraconsistent, hybrid,
epistemic, etc.), there are calculi for different systems as well, such as
unification and structured specifications.
%
We are particularly happy to include in this new edition Hilbert's, Bernay's and
Ackermann's calculi, thanks to Richard Zach.
%
Many people had expressed that those historically important systems deserved an
entry in the Encyclopedia. Now they finally have a place here.

This second edition of the Encyclopedia of Proof Systems book extends the first
edition with the 34 new entries.
%
Additionally, with the aim of encouraging practical applications of proof systems, a new meta-data tag for implementations or formalizations of an entry
is now available.

New proof systems are proposed each day, so the Encyclopedia will always be open
for new contributions. 
%
With almost 100 entries on the most diverse systems, this effort of knowledge
organization can only succeed as a joint effort of the community.
%
We are grateful for the support we have received so far and hope the
Encyclopedia continues to grow in the years to come.

\vspace{\baselineskip}
\begin{flushright}\noindent
December 2017 \hfill {\it Giselle Reis}
\end{flushright}


