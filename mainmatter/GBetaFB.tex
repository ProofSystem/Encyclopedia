% If the calculus has an acronym, define it.
% (e.g. \newcommand{\LK}{\ensuremath{\mathbf{LK}}\xspace})

\newcommand{\typearrow}{\shortrightarrow}

\def\tps{\sc Tps}
\def\leo{{\sc Leo}}
\def\Vars{{\cV}}                      %set of variables
\def\Signat{{\Sigma}}                 %Signature
\def\free{{free}}
\def\thdash{\null\mathrel{\vdash\!\!\!\vdash}}
\def\semtrue{{\tt T}}
\def\semfalse{{\tt F}}
\def\semival{\upsilon}
\def\follof{\longleftrightarrow}         % Logische Aequivalenz
\def\implies{\rightarrow}
\def\Leibq{\mbox{\bf Q}}
\def\Leibeq{\doteq}
\def\Andrewseq{\stackrel{..}{=}}
\def\Metaeq{\equiv}
\def\Primeq{=}
\def\Primq{\bf{q}}
\def\Domeq{\sf{q}}
\def\deq{{\;\colon\kern-.2em=\;}}
\def\ov#1{{\overline{#1}}}
\def\sdot{\rule{.3ex}{.3ex}\hskip.1ex}
\def\lamdot#1{\lambda {#1}}
\def\alldot#1{\forall {#1}}
\def\exdot#1{\exists{#1}\sdot}
\def\setdivider{\;\bigl|\;}
\def\lden{[\kern-0.18em[}    %oeffnende Denotatklammern
\def\rden{]\kern-0.18em]}  %schliessende Denotatklammern
\def\quotientspace#1#2{{#1/_{\kern-.5ex{#2}}}}
\def\restrict#1#2{{#1\bigr|_{#2}}}
\def\Funcs{{\mbox{$\cF$}}}
\def\to{\longrightarrow}
\def\appo{@}
\def\termstructure#1#2{\cT\kern-.3ex\cE_{#1}(#2)}
\def\Termstructure#1{{\termstructure{#1}\Signat}}
\def\rulename#1{{\it {#1}}}
\def\algname#1#2{\rulename{#1(#2)}}
\def\bnormform#1{\left.#1\hspace*{-.4ex}\right\downarrow_\beta}
\def\benormform#1{\left.#1\hspace*{-.4ex}\right\downarrow_{\beta\eta}}
\def\Bnormform#1{\left.{#1}\hspace*{-.4ex}\right\downarrow_\beta}
\def\Benormform#1{\left.{#1}\hspace*{-.4ex}\right\downarrow_{\beta\eta}}
\def\ambnormform#1{{#1}\hspace*{-1.1ex}\downarrow_{\kern-.2em\scriptscriptstyle *}} % ``ambiguous'' normal form
\def\eqb{{\Metaeq_{\beta}}}
\def\eqe{{\Metaeq_{\eta}}}
\def\eqbe{{\Metaeq_{\beta\eta}}}
\def\convarrow{\rightarrow}
\newcommand{\convb}{\convarrow_{\beta}}
\newcommand{\conve}{\convarrow_{\eta}}
\def\Types{{\cT}}                     %set of type symbols
\def\ar{\rightarrow}
\def\typea{\alpha}
\def\typeb{\beta}
\def\typec{\gamma}
\def\typed{\delta}
\def\typebool{o}
\def\typeind{\iota}
\def\abtype#1{_{#1}}
\def\abtypea{\abtype\typea}
\def\abtypebool{\abtype\typebool}
\def\wff#1{\hbox{\it wff}_{#1}}
\def\Wff#1{\hbox{\it wff}_{#1}(\Signat)}
\def\cWff#1{{c\Wff{#1}}}
\def\cWffind{\cWff\typeind}
\def\cwffa{\cwff\typea}
\def\cWffa{\cWff\typea}
\def\Wffa{\Wff\typea}
\def\Wffb{\Wff{\beta}}
\def\Wffbool{\Wff\typebool}
\def\Wffall{\Wff{}}
\def\cWffbool{\cWff\typebool}
\def\cWffall{\cWff{}}
\def\Cwff#1{\hbox{\it cwff}_{#1}(\Signat)}
\def\cwff#1{\hbox{\it cwff}_{#1}}
\def\Wffcl#1{\Cwff{#1}}
\def\Wffclbool{\Cwff{\typebool}}
\def\Wffclall{\Wffcl{}}
\def\HOL{{\cH\kern-.4ex\cO\kern-.4ex\cL}}  % Higher-order Logic
\def\falseconst{\bF\abtypebool}
\def\trueconst{\bT\abtypebool}
\def\Calc{{\mathfrak T}}
\def\AndCalc{{\Calc_\beta}}
\def\NCalc{{\mathfrak N\kern -.1em\mathfrak K}}
\def\Ndcalc{{\NCalc_\beta}}
\def\NdcalcFB{{\NCalc_{\beta\propf\propb}}}
\def\allNdcalc{{\NCalc_\ast}}
\def\ndcalcbeta{\algname\NCalc{\beta}}
\def\ndcalceta{\algname\NCalc{\propeta}}
\def\ndcalcxi{\algname\NCalc{\propxi}}
\def\ndcalcf{\algname\NCalc{\propf}}
\def\ndcalcb{\algname\NCalc{\propb}}
\def\ndcalceqr{\algname\NCalc{=^r}}
\def\ndcalceql{\algname\NCalc{=^l}}
\def\ndcalchyp{\algname\NCalc{Hyp}}
\def\ndcalccontr{\algname\NCalc{Contr}}
\def\ndcalcnegi{\algname\NCalc{\neg I}}
\def\ndcalcnege{\algname\NCalc{\neg E}}
\def\ndcalcoril{\algname\NCalc{\lor I_L}}
\def\ndcalcorir{\algname\NCalc{\lor I_R}}
\def\ndcalcore{\algname\NCalc{\lor E}}
\def\ndcalcpii{\algname\NCalc{\Pi I}}
\def\ndcalcpie{\algname\NCalc{\Pi E}}
\def\premdivider{{\mbox{$\quad$}}}
\def\@lineskipamount{4pt}
\def\lowerhalf#1{\hbox{\raise -0.8\baselineskip\hbox{#1}}}
\def\inruleanhelp#1#2#3{\setbox\tempa=\hbox{$\displaystyle{\mathstrut #2}$}%
                        \setbox\tempd=\hbox{$\; #3$}%
                        \setbox\tempb=\vbox{\vskip 2pt\halign{##\cr
        \mud{#1}\cr
        \noalign{\vskip\the\lineskip}%
        \noalign{\hrule height 0pt}%
        \rig{\vbox to 0pt{\vss\hbox to 0pt{\copy\tempd \hss}\vss}}\cr
        \noalign{\hrule}%
        \noalign{\vskip\the\lineskip}%
        \mud{\copy\tempa}\cr}}%
                      \tempc=\wd\tempb
                      \advance\tempc by \wd\tempa
                      \divide\tempc by 2 }
\def\inrulean#1#2#3{{\inruleanhelp{#1}{#2}{#3}%
                     \hbox to \wd\tempa{\hss \box\tempb \hss}}}
\def\inrulebn#1#2#3#4{\inrulean{#1\premdivider #2}{#3}{#4}}
\def\inrulecn#1#2#3#4#5{\inrulean{#1\premdivider #2\premdivider #3}{#4}{#5}}
\def\ian#1#2#3{{\lineskip\@lineskipamount\inrulean{#1}{#2}{#3}}} %premise, conc, name
\def\ibn#1#2#3#4{{\lineskip\@lineskipamount\inrulebn{#1}{#2}{#3}{#4}}}  %prem1, prem2, conc, name
\def\ianc#1#2#3{{\lineskip\@lineskipamount\lowerhalf{\inruleanhelp{#1}{#2}{#3}%
                   \box\tempb\hskip\wd\tempd}}}
\def\ibnc#1#2#3#4{{\lineskip\@lineskipamount\ianc{#1\premdivider #2}{#3}{#4}}}
\def\icnc#1#2#3#4#5{{\lineskip\@lineskipamount\ianc{#1\premdivider #2\premdivider #3}{#4}{#5}}}
\def\Rulespacing{\renewcommand{\arraystretch}{4}%
 \arraycolsep 0em}
\newbox\tempa
\newbox\tempb
\newdimen\tempc
\newbox\tempd
\def\mud#1{\hfil $\displaystyle{#1}$\hfil}
\def\rig#1{\hfil $\displaystyle{#1}$}
\def\propf{\mathfrak{f}}
\def\propn{{\hspace*{-0.01cm}f\hspace*{-0.07cm}r}}
\def\propeta{\eta}
\def\propxi{\xi}
\def\propq{{\mathfrak{q}}}
\def\propb{{\mathfrak{b}}}
\def\PD{{\mbox{\hspace*{-0.02cm}}\beta}}
\def\PETA{{\mbox{\hspace*{-0.02cm}}\beta\mbox{\hspace*{-0.02cm}}\propeta}}
\def\PXI{{\mbox{\hspace*{-0.02cm}}\beta\mbox{\hspace*{-0.02cm}}\propxi}}
\def\PF{{\mbox{\hspace*{-0.02cm}}\beta\mbox{\hspace*{-0.02cm}}\propf}}
\def\PB{{\mbox{\hspace*{-0.02cm}}\beta\mbox{\hspace*{-0.02cm}}\propb}}
\def\PETAB{{\mbox{\hspace*{-0.02cm}}\beta\mbox{\hspace*{-0.02cm}}\propeta\mbox{\hspace*{-0.02cm}}\propb}}
\def\PXIB{{\mbox{\hspace*{-0.02cm}}\beta\mbox{\hspace*{-0.02cm}}\propxi\mbox{\hspace*{-0.02cm}}\propb}}
\def\PFB{{\mbox{\hspace*{-0.02cm}}\beta\mbox{\hspace*{-0.02cm}}\propf\mbox{\hspace*{-0.02cm}}\propb}}
\def\MOD{{\mathfrak{M}}}
\def\MODD{\MOD_\PD}
\def\MODETA{\MOD_\PETA}
\def\MODXI{\MOD_\PXI}
\def\MODF{\MOD_\PF}
\def\MODB{\MOD_\PB}
\def\MODETAB{\MOD_\PETAB}
\def\MODXIB{\MOD_\PXIB}
\def\MODFB{\MOD_\PFB}
\def\HE{{\mathfrak{H}}}
\def\STM{{\mathfrak{ST}}}
\def\MODALL{\MOD_*}
\def\Leibext#1{{\mbox{\sc EXT}_{\Leibeq}^{#1}}}
\def\Primext#1{{\mbox{\sc EXT}_{\Primeq}^{#1}}}
\def\Primexta{\Primext{\typea\ar\typeb}}
\def\Primexto{\Primext\typebool}
\def\Leibexta{\Leibext{\typea\ar\typeb}}
\def\Leibexto{\Leibext\typebool}
\def\Leibextf{\Leibext{\ar}}
\def\Primextf{\Primext{\ar}}
\def\simml{\stackrel{.}{\sim}}
\def\acc{\Gamma_{\mbox{\hspace*{-0.1cm}}\Sigma}}
\def\ACC#1{\mathfrak{Acc}_{#1}}
\def\ACCstar{{\ACC{*}}}
\def\ACCMODD{\ACC\PD}
\def\ACCMODETA{\ACC\PETA}
\def\ACCMODXI{\ACC\PXI}
\def\ACCMODF{\ACC\PF}
\def\ACCMODB{\ACC\PB}
\def\ACCMODETAB{\ACC\PETAB}
\def\ACCMODXIB{\ACC\PXIB}
\def\ACCMODFB{\ACC\PFB}
\def\absstar{\nabla_{\mbox{\hspace*{-0.08cm}}*}}
\def\absc{\nabla_{\mbox{\hspace*{-0.08cm}}c}}
\def\abssat{\nabla_{\mbox{\hspace*{-0.08cm}}sat}}
\def\absneg{\nabla_{\mbox{\hspace*{-0.08cm}}\neg}}
\def\absbe{\nabla_{\mbox{\hspace*{-0.08cm}}\propeta}}
\def\absxi{\nabla_{\mbox{\hspace*{-0.08cm}}\propxi}}
\def\absbeta{\nabla_{\mbox{\hspace*{-0.08cm}}\beta}}
\def\absor{\nabla_{\mbox{\hspace*{-0.08cm}}\vee}}
\def\absand{\nabla_{\mbox{\hspace*{-0.08cm}}\wedge}}
\def\absforall{\nabla_{\mbox{\hspace*{-0.08cm}}\forall}}
\def\absexists{\nabla_{\mbox{\hspace*{-0.08cm}}\exists}}
\def\absf{\nabla_{\mbox{\hspace*{-0.08cm}}\propf}}
\def\absb{\nabla_{\mbox{\hspace*{-0.08cm}}\propb}}
\def\abser{\nabla^{r}_{\mbox{\hspace*{-0.08cm}}\Primeq}}
\def\abses{\nabla^s_{\mbox{\hspace*{-0.08cm}}\Primeq}}
\def\abspl{\nabla^{\Leibeq}_{\mbox{\hspace*{-0.08cm}}\Primeq}}
\def\absplb{\nabla^{\Leibeq^-}_{\mbox{\hspace*{-0.08cm}}\Primeq^-}}
\def\abslp{\nabla^{\Primeq}_{\mbox{\hspace*{-0.08cm}}\Leibeq}}
\def\abslpb{\nabla^{\Primeq^-}_{\mbox{\hspace*{-0.08cm}}\Leibeq^-}}
\def\absrb{\nabla_{\mbox{\hspace*{-0.08cm}}\Leibeq}^{r}}
\def\absbb{\nabla_{\mbox{\hspace*{-0.08cm}}\Leibeq}^o}
\def\absfb{\nabla_{\mbox{\hspace*{-0.08cm}}\Leibeq}^\ar}
\def\absappb{\nabla_{\mbox{\hspace*{-0.08cm}}\Leibeq}^{App}}
\def\hintcstar{\stackrel{\leftarrow}{\nabla}_{\mbox{\hspace*{-0.08cm}}*}}
\def\hintdstar{\vec{\nabla}_{\mbox{\hspace*{-0.08cm}}*}}
\def\hintdc{\vec{\nabla}_{\mbox{\hspace*{-0.08cm}}c}}
\def\hintdsat{\vec{\nabla}_{\mbox{\hspace*{-0.08cm}}sat}}
\def\hintdneg{\vec{\nabla}_{\mbox{\hspace*{-0.08cm}}\neg}}
\def\hintdbe{\vec{\nabla}_{\mbox{\hspace*{-0.08cm}}\propeta}}
\def\hintdbeta{\vec{\nabla}_{\mbox{\hspace*{-0.08cm}}\beta}}
\def\hintdor{\vec{\nabla}_{\mbox{\hspace*{-0.08cm}}\vee}}
\def\hintdand{\vec{\nabla}_{\mbox{\hspace*{-0.08cm}}\wedge}}
\def\hintdforall{\vec{\nabla}_{\mbox{\hspace*{-0.08cm}}\forall}}
\def\hintdexists{\vec{\nabla}_{\mbox{\hspace*{-0.08cm}}\exists}}
\def\hintxi{\vec{\nabla}_{\mbox{\hspace*{-0.08cm}}\propxi}}
\def\hintdf{\vec{\nabla}_{\mbox{\hspace*{-0.08cm}}\propf}}
\def\hintdb{\vec{\nabla}_{\mbox{\hspace*{-0.08cm}}\propb}}
\def\hintdpl{\vec{\nabla}^{\Leibeq}_{\mbox{\hspace*{-0.08cm}}\Primeq}}
\def\hintder{\vec{\nabla}^r_{\mbox{\hspace*{-0.08cm}}\Primeq}}
\def\hintdleibrefl{\vec{\nabla}^{r}_{\mbox{\hspace*{-0.08cm}}\Leibeq}}
\def\hintdleibapp{\vec{\nabla}^{\ar}_{\mbox{\hspace*{-0.08cm}}\Leibeq}}
\def\hintdleibtrans{\vec{\nabla}^{tr}_{\mbox{\hspace*{-0.08cm}}\Leibeq}}
\def\hintsatneg{\overline{\nabla}_{\mbox{\hspace*{-0.08cm}}\neg}}
\def\hintsator{\overline{\nabla}_{\mbox{\hspace*{-0.08cm}}\vee}}
\def\hintsatforall{\overline{\nabla}_{\mbox{\hspace*{-0.08cm}}\forall}}
\def\hintsatforallbeta{\overline{\nabla}_{\mbox{\hspace*{-0.08cm}}\forall}^\beta}
\def\hintsatb{\overline{\nabla}_{\mbox{\hspace*{-0.08cm}}\propb}}
\def\hintsatxi{\overline{\nabla}_{\mbox{\hspace*{-0.08cm}}\propxi}}
\def\hintsatxibeta{\overline{\nabla}_{\mbox{\hspace*{-0.08cm}}\propxi}^\beta}
\def\hintsateta{\overline{\nabla}_{\mbox{\hspace*{-0.08cm}}\propeta}}
\def\hintsatf{\overline{\nabla}_{\mbox{\hspace*{-0.08cm}}\propf}}
\def\hintsatfbeta{\overline{\nabla}_{\mbox{\hspace*{-0.08cm}}\propf}^\beta}
\def\hintsatrefl{\overline{\nabla}_{\mbox{\hspace*{-0.08cm}}r}}
\def\hintpl{\overline{\nabla}^{\Leibeq}_{\mbox{\hspace*{-0.08cm}}\Primeq}}
\def\Scom{{\bf{S}}}
\def\Kcom{{\bf{K}}}
\def\HINT#1{\mathfrak{Hint}_{#1}}
\def\HINTstar{{\HINT{*}}}
\def\HINTMODD{\HINT\PD}
\def\HINTMODETA{\HINT\PETA}
\def\HINTMODXI{\HINT\PXI}
\def\HINTMODF{\HINT\PF}
\def\HINTMODB{\HINT\PB}
\def\HINTMODETAB{\HINT\PETAB}
\def\HINTMODXIB{\HINT\PXIB}
\def\HINTMODFB{\HINT\PFB}
\def\valprop{\mathfrak{L}}
\def\valnot#1{\valprop_\neg(#1)}
\def\valor#1{\valprop_\lor(#1)}
\def\valand#1{\valprop_\land(#1)}
\def\valimplies#1{\valprop_\implies(#1)}
\def\valequiv#1{\valprop_\follof(#1)}
\def\valpi#1#2{\valprop^{#1}_\forall(#2)}
\def\valsigma#1#2{\valprop^{#1}_\exists(#2)}
\def\valeq#1#2{\valprop^{#1}_\Primeq(#2)}
\newcommand{\bA}{{\mathbf{A}}}
\newcommand{\bB}{{\mathbf{B}}}
\newcommand{\bC}{{\mathbf{C}}}
\newcommand{\bD}{{\mathbf{D}}}
\newcommand{\bE}{{\mathbf{E}}}
\newcommand{\bF}{{\mathbf{F}}}
\newcommand{\bG}{{\mathbf{G}}}
\newcommand{\bH}{{\mathbf{H}}}
\newcommand{\bI}{{\mathbf{I}}}
\newcommand{\bL}{{\mathbf{L}}}
\newcommand{\bM}{{\mathbf{M}}}
\newcommand{\bN}{{\mathbf{N}}}
\newcommand{\bO}{{\mathbf{O}}}
\newcommand{\bP}{{\mathbf{P}}}
\newcommand{\bQ}{{\mathbf{Q}}}
\newcommand{\bT}{{\mathbf{T}}}
\newcommand{\bU}{{\mathbf{U}}}
\newcommand{\bW}{{\mathbf{W}}}
\newcommand{\bX}{{\mathbf{X}}}
\def\phi{\varphi}
\def\cA{{\mathcal{A}}}
\def\cB{{\mathcal{B}}}
\def\cC{{\mathcal{C}}}
\def\cD{{\mathcal{D}}}
\def\cE{{\mathcal{E}}}
\def\cF{{\mathcal{F}}}
\def\cG{{\mathcal{G}}}
\def\cR{{\mathcal{R}}}
\def\cH{{\mathcal{H}}}
\def\cI{{\mathcal{I}}}
\def\cJ{{\mathcal{J}}}
\def\cL{{\mathcal{L}}}
\def\cM{{\mathcal{M}}}
\def\cO{{\mathcal{O}}}
\def\cP{{\mathcal{P}}}
\def\cS{{\mathcal{S}}}
\def\cT{{\mathcal{T}}}
\def\cV{{\mathcal{V}}}
\def\sema{{\sf a}}
\def\semA{{\sf A}}
\def\semb{{\sf b}}
\def\semB{{\sf B}}
\def\semc{{\sf c}}
\def\semd{{\sf d}}
\def\seme{{\sf e}}
\def\semf{{\sf f}}
\def\semg{{\sf g}}
\def\semh{{\sf h}}
\def\semH{{\sf H}}
\def\semG{{\sf G}}
\def\semi{{\sf i}}
\def\semk{{\sf k}}
\def\semn{{\sf n}}
\def\semp{{\sf p}}
\def\semP{{\sf P}}
\def\semq{{\sf q}}
\def\semr{{\sf r}}
\def\semw{{\sf w}}
\def\defins#1{{\defemph{#1}s}}
\def\defin#1{{\defemph{#1}}}
\def\defemph#1{{\em #1}}
\def\TFuncs{\Funcs_\Types}
\def\sigcard{{\aleph_s}}
\def\my@ref#1#2#3{#1~\ref{#2:#3}}
%% \newenvironment{myfig}[2]%
%% {\begin{figure}[tb]\def\myfiglabel{#1}\def\myfigcaption{{#2}}\begin{center}}
%% {\caption{\myfigcaption}\label{fig:\myfiglabel}\end{center}\end{figure}}
%% \def\myfigref#1{\my@ref{Figure}{fig}{#1}}
\def\thmnl{\strut\par\noindent}

%\def\absitem#1{\item[{\makebox[0.4cm][l]{$#1$}}]}
\def\absitem#1{\item[$#1$]}
\def\Andrewseq{\stackrel{..}{=}}


\def\funcint#1#2{\funcinto(#1,#2)}
\def\funcinto{\Upsilon}
\def\diffarg#1#2{\diffargo(#1,#2)}
\def\diffargo{\overline{\Upsilon}}
\def\diffarga#1#2{\diffargoa(#1,#2)}
\def\diffargoa{\Upsilon_1}
\def\diffargb#1#2{\diffargob(#1,#2)}
\def\diffargob{\Upsilon_2}
\def\ord{\mbox{order-of}}

\def\defins#1{{\defemph{#1}s}}
\def\defin#1{{\defemph{#1}}}
\def\defemph#1{{\em #1}}

\def\propeta{\eta}

\def\sindex#1#2{}
\def\MATH{\vspace*{-0.2cm}\[}
\def\EMATH{\vspace*{-0.2cm}\]}
\def\df#1{{\bf #1}}
\def\dfi#1{{\bf #1}}

\def\TFuncs{\Funcs_\Types}
\def\Cwff#1{\hbox{\it cwff}_{#1}(\Signat)}
\def\cwff#1{\hbox{\it cwff}_{#1}}
\def\Wffcl#1{\Cwff{#1}}
\def\wffcl#1{\cwff{#1}}
\def\Wffclbool{\Cwff{\typebool}}
\def\Wffclall{\Wffcl{}}
\def\wffclbool{\wwff{\typebool}}
\def\wffclall{\wffcl{}}
\def\bnormform#1{#1\hspace*{-.4ex}\bigl\downarrow_\beta}
\def\benormform#1{#1\hspace*{-.4ex}\bigl\downarrow_{\beta\eta}}
\def\Bnormform#1{\left.{#1}\hspace*{-.4ex}\right\downarrow_\beta}
\def\Benormform#1{\left.{#1}\hspace*{-.4ex}\right\downarrow_{\beta\eta}}

\def\sigcard{{\aleph_s}}

\def\nUnif#1{unif_{#1}(\Sigma)}
\def\nunif{\not=}

\def\absu#1{\nabla^{\nunif}_{\mbox{\hspace*{-0.08cm}}#1}}
\def\absieunif#1{\nabla^{\typeind}_{\mbox{\hspace*{-0.08cm}}#1 u}}
\def\absbeunif#1{\nabla^{\typebool}_{\mbox{\hspace*{-0.08cm}}#1 u}}
\def\absfeunif#1{\nabla^{\ar}_{\mbox{\hspace*{-0.08cm}}#1 u}}
\def\absur{\nabla^{r}_{\mbox{\hspace*{-0.08cm}}\nunif}}

\def\hintdur{\vec{\nabla}^{r}_{\mbox{\hspace*{-0.08cm}}\nunif}}

\def\bconv{\Metaeq_\beta}

\def\Geneq{\approx}

\def\absgef#1{\nabla^\ar_{\mbox{\hspace*{-0.08cm}}#1}}
\def\absgeb#1{\nabla^\typebool_{\mbox{\hspace*{-0.08cm}}#1}}

\def\absfu{\nabla^{\propf}_{\mbox{\hspace*{-0.08cm}}\nunif}}
\def\absbu{\nabla^{\propb}_{\mbox{\hspace*{-0.08cm}}\nunif}}
\def\absbeu{\nabla^{\propeta}_{\mbox{\hspace*{-0.08cm}}\nunif}}
\def\absbetau{\nabla^{\beta}_{\mbox{\hspace*{-0.08cm}}\nunif}}

%% already defined below
%% \def\absdec#1{\nabla_{\mbox{\hspace*{-0.08cm}}dec}^{#1}}
%% \def\absfundec{\absdec{f}}%{\nabla_{\mbox{\hspace*{-0.08cm}}fdec}}
%% \def\abskdec{\nabla_{\mbox{\hspace*{-0.08cm}}k}}
%% \def\absmate{\nabla_{\mbox{\hspace*{-0.08cm}}m}}

\def\ACCE#1#2{\mathfrak{Acc}_{#1}^{#2}}
\def\ACCstarE#1{\ACCE{*}{#1}}
\def\ACCMODDE#1{\ACCE{\PD}{#1}}
\def\ACCMODETAE#1{\ACCE{\PETA}{#1}}
\def\ACCMODFE#1{\ACCE{\PF}{#1}}
\def\ACCMODBE#1{\ACCE{\PB}{#1}}
\def\ACCMODETABE#1{\ACCE{\PETAB}{#1}}
\def\ACCMODFBE#1{\ACCE{\PFB}{#1}}
\def\ACCstarG{\ACCE{*}{\Geneq}}
\def\ACCMODDG{\ACCE{\PD}{\Geneq}}
\def\ACCMODETAG{\ACCE{\PETA}{\Geneq}}
\def\ACCMODFG{\ACCE{\PF}{\Geneq}}
\def\ACCMODBG{\ACCE{\PB}{\Geneq}}
\def\ACCMODETABG{\ACCE{\PETAB}{\Geneq}}
\def\ACCMODFBG{\ACCE{\PFB}{\Geneq}}

%  Macros not appearing in submit.tex - Chad - 7/02
\def\above#1#2{\normalspacing\begin{array}[b]{c}\relax #1\\\relax #2\end{array}}
\def\abovel#1#2{\normalspacing\begin{array}[b]{l}\relax #1\\[.2cm]\relax #2\end{array}}
\def\abovec#1#2{\normalspacing\begin{array}{c}\relax #1\\\relax #2\end{array}}
\def\abovecdots#1#2{\normalspacing\begin{array}[b]{c}\relax #1\\\relax\vdots\\\relax #2\end{array}}
\def\TPS{{\sc Tps}}
\def\powerset{{\mathcal P}}

\def\ambnormform#1{{#1}\hspace*{-1.1ex}\downarrow_{\kern-.2em\scriptscriptstyle *}} % ``ambiguous'' normal form

\def\starstructure#1#2{{\cF_{*}^{#2}}}
\def\fbstructure#1{{\cF_{\propf\propb}^{#1}}}
\def\fstructure#1{{\cF_{\propf}^{#1}}}
\def\xibstructure#1{{\cF_{\propxi\propb}^{#1}}}
\def\xistructure#1{{\cF_{\propxi}^{#1}}}
\def\compatstructure#1{{\cF_{c}^{#1}}}

\def\Wffn#1{\hbox{\it wff}_{#1}}
\def\enormform#1{{#1}\hspace*{-1.1ex}\downarrow^{\kern-.2em\scriptscriptstyle\eta}}
\def\absdec#1{\nabla_{\mbox{\hspace*{-0.08cm}}d}^{#1}}
%\def\absxidec{\absdec\propxi}%{\nabla_{\mbox{\hspace*{-0.08cm}}\propxi dec}}
%\def\absfundec{\absdec{f}}%{\nabla_{\mbox{\hspace*{-0.08cm}}fdec}}
%\def\absdecd{\absdec{1}}%{\nabla_{\mbox{\hspace*{-0.08cm}}1 dec}}
%\def\absdece{\absdec{2}}%{\nabla_{\mbox{\hspace*{-0.08cm}}2 dec}}
%\def\absdecb{\absdec{3}}%{\nabla_{\mbox{\hspace*{-0.08cm}}3 dec}}
%\def\absdecc{\absdec{4}}%{\nabla_{\mbox{\hspace*{-0.08cm}}4 dec}}
\def\absdecd{\nabla_{\mbox{\hspace*{-0.08cm}}d}}
\def\absdece{\nabla_{\mbox{\hspace*{-0.08cm}}cd}}
\def\absdecb{\nabla^{f}_{\mbox{\hspace*{-0.08cm}}d}}
\def\absdecc{\nabla^{\lambda}_{\mbox{\hspace*{-0.08cm}}d}}
\def\abskdec{\nabla_{\mbox{\hspace*{-0.08cm}}k}}
\def\absmate{\nabla_{\mbox{\hspace*{-0.08cm}}m}}
\def\hintdec#1{\vec\nabla_{\mbox{\hspace*{-0.08cm}}d}^{#1}}
%\def\hintxidec{\vec\nabla_{\mbox{\hspace*{-0.08cm}}\propxi dec}}
%\def\hintfundec{\vec\nabla_{\mbox{\hspace*{-0.08cm}}fdec}}
%\def\hintdecd{\hintdec{1}}%\def\hintdecb{\vec\nabla_{\mbox{\hspace*{-0.08cm}}1 dec}}
%\def\hintdece{\hintdec{2}}%\def\hintdecc{\vec\nabla_{\mbox{\hspace*{-0.08cm}}2 dec}}
%\def\hintdecb{\hintdec{3}}%\def\hintdecb{\vec\nabla_{\mbox{\hspace*{-0.08cm}}3 dec}}
%\def\hintdecc{\hintdec{4}}%\def\hintdecc{\vec\nabla_{\mbox{\hspace*{-0.08cm}}4 dec}}
\def\hintdecd{\vec\nabla_{\mbox{\hspace*{-0.08cm}}d}}
\def\hintdece{\vec\nabla_{\mbox{\hspace*{-0.08cm}}cd}}
\def\hintdecb{\vec\nabla^{f}_{\mbox{\hspace*{-0.08cm}}d}}
\def\hintdecc{\vec\nabla^{\lambda}_{\mbox{\hspace*{-0.08cm}}d}}
\def\hintkdec{\vec\nabla_{\mbox{\hspace*{-0.08cm}}k}}
\def\hintmate{\vec\nabla_{\mbox{\hspace*{-0.08cm}}m}}
\def\prel{\sim} % PER on possible values semantics
\def\possl{\left\lceil} % possible values left brack
\def\possr{\right\rceil} % possible values right brack

\def\propf{\mathfrak{f}}

%%%%%%%%%%%%%%%%%%%%%%%%%%%%%%%%%%%%%%%%%%%%%%%%%%%%%%%%
%%  sequent calculus
%%%%%%%%%%%%%%%%%%%%%%%%%%%%%%%%%%%%%%%%%%%%%%%%%%%%%%%%

\def\SEQCALC{\cG}
\def\SEQCALCIstar{\cG^{-}_{*}}
\def\SEQCALCID{\SEQCALC^{-}_\PD}
\def\SEQCALCIETA{\SEQCALC^{-}_{\PETA}}
\def\SEQCALCIXI{\SEQCALC^{-}_{\PXI}}
\def\SEQCALCIF{\SEQCALC^{-}_{\PF}}
\def\SEQCALCIB{\SEQCALC^{-}_{\PB}}
\def\SEQCALCIETAB{\SEQCALC^{-}_{\PETAB}}
\def\SEQCALCIXIB{\SEQCALC^{-}_{\PXIB}}
\def\SEQCALCIFB{\SEQCALC^{-}_{\PFB}}

\def\SEQCALCstar{\cG_{*}}
\def\SEQCALCD{\SEQCALC_\PD}
\def\SEQCALCETA{\SEQCALC_{\PETA}}
\def\SEQCALCXI{\SEQCALC_{\PXI}}
\def\SEQCALCF{\SEQCALC_{\PF}}
\def\SEQCALCB{\SEQCALC_{\PB}}
\def\SEQCALCETAB{\SEQCALC_{\PETAB}}
\def\SEQCALCXIB{\SEQCALC_{\PXIB}}
\def\SEQCALCFB{\SEQCALC_{\PFB}}
\def\SEQCALCQB{\SEQCALC_{\PQB}}
\def\accseq#1{\acc^{#1}}
\def\accSEQCALCstar{\acc^{\cG_{*}}}
\def\accSEQCALCD{\acc^{\SEQCALC_\PD}}
\def\accSEQCALCETA{\acc^{\SEQCALC_{\PETA}}}
\def\accSEQCALCXI{\acc^{\SEQCALC_{\PXI}}}
\def\accSEQCALCF{\acc^{\SEQCALC_{\PF}}}
\def\accSEQCALCQ{\acc^{\SEQCALC_{\PQ}}}
\def\accSEQCALCB{\acc^{\SEQCALC_{\PB}}}
\def\accSEQCALCETAB{\acc^{\SEQCALC_{\PETAB}}}
\def\accSEQCALCXIB{\acc^{\SEQCALC_{\PXIB}}}
\def\accSEQCALCFB{\acc^{\SEQCALC_{\PFB}}}
\def\accSEQCALCQB{\acc^{\SEQCALC_{\PQB}}}
\def\accMODstar{\acc^{\MOD_{*}}}
\def\accMODD{\acc^{\MOD_{\PD}}}
\def\accMODETA{\acc^{\MOD_{\PETA}}}
\def\accMODXI{\acc^{\MOD_{\PXI}}}
\def\accMODF{\acc^{\MOD_{\PF}}}
\def\accMODB{\acc^{\MOD_{\PB}}}
\def\accMODETAB{\acc^{\MOD_{\PETAB}}}
\def\accMODXIB{\acc^{\MOD_{\PXIB}}}
\def\accMODFB{\acc^{\MOD_{\PFB}}}



\def\seqweak{\algname\SEQCALC{weak}}
\def\seqcut{\algname\SEQCALC{cut}}
\def\seqbetainv{\algname\SEQCALC{\beta^\downarrow}}
\def\seqbeinv{\algname\SEQCALC{\beta\eta^\downarrow}}
\def\seqneginv{\algname\SEQCALC{Inv^\neg}}
\def\seqlorl{\algname\SEQCALC{\lor_-}}
\def\seqlorr{\algname\SEQCALC{\lor_+}}
\def\seqneg{\algname\SEQCALC{\neg}}
\def\seqnegl{\algname\SEQCALC{\neg_-}}
\def\seqnegr{\algname\SEQCALC{\neg_+}}
\def\seqpilg#1{\algname\SEQCALC{\Pi_-^{#1}}}
\def\seqpil{\algname\SEQCALC{\Pi_-^l}}
\def\seqpirg#1{\algname\SEQCALC{\Pi_+^{#1}}}
\def\seqpir{\algname\SEQCALC{\Pi_+^c}}
\def\seqbeta{\algname\SEQCALC{\beta}}
\def\seqbetal{\algname\SEQCALC{\beta_L}}
\def\seqbetar{\algname\SEQCALC{\beta_R}}
\def\seqinit{\algname\SEQCALC{init}}
\def\seqbe{\algname\SEQCALC{\beta\eta}}
\def\seqbel{\algname\SEQCALC{\beta\eta_L}}
\def\seqber{\algname\SEQCALC{\beta\eta_R}}
\def\seqinitleib{\algname\SEQCALC{Init^{\Leibeq}}}
\def\seqxi{\algname\SEQCALC{\xi}}
\def\seqpropf{\algname\SEQCALC{\propf}}
\def\seqpropb{\algname\SEQCALC{\propb}}
\def\seqdec{\algname\SEQCALC{d}}
\def\seqdech{\algname\SEQCALC{d^h}}
\def\seqxidec{\algname\SEQCALC{\xi dec}}
\def\seqfdec{\algname\SEQCALC{\propf dec}}
\def\seqdecb{\algname\SEQCALC{fd}}
\def\seqdecc{\algname\SEQCALC{\lambda d}}
\def\seqk{\algname\SEQCALC{k}}
\def\seqcut{\algname\SEQCALC{cut}}
\def\seqcuta{\algname\SEQCALC{cut^\bA}}
\def\seqaxiomb{\algname\SEQCALC{\cB}}
\def\axiomf{\cF_{\kern-.3em\typea\kern-.1em\typeb}}
\def\axiomb{\cB_\typebool}
\def\seqaxiomf{\algname\SEQCALC\axiomf}

\def\RESCALC{\cR}
\def\resweak{\algname\RESCALC{weak}}
\def\resdetectchoice{\algname\RESCALC{DetectChoice}}
\def\reschoice{\algname\RESCALC{Choice}}
\def\resres{\algname\RESCALC{Res}}
\def\resfac{\algname\RESCALC{Fac}}
\def\restriv{\algname\RESCALC{Triv}}
\def\ressubst{\algname\RESCALC{Subst}}
\def\resprimsubst{\algname\RESCALC{PrimSubst}}
\def\resflexrigid{\algname\RESCALC{FlexRigid}}
\def\rescut{\algname\RESCALC{cut}}
\def\resbetainv{\algname\RESCALC{\beta^\downarrow}}
\def\resbeinv{\algname\RESCALC{\beta\eta^\downarrow}}
\def\resneginv{\algname\RESCALC{Inv^\neg}}
\def\reslorl{\algname\RESCALC{\lor_\false}}
\def\reslorr{\algname\RESCALC{\lor_\true}}
\def\resneg{\algname\RESCALC{\neg}}
\def\resnegl{\algname\RESCALC{\neg_\false}}
\def\resnegr{\algname\RESCALC{\neg_\true}}
\def\respilg#1{\algname\RESCALC{\Pi_\false^{#1}}}
\def\respil{\algname\RESCALC{\Pi_\false^l}}
\def\respirg#1{\algname\RESCALC{\Pi_\true^{#1}}}
\def\respir{\algname\RESCALC{\Pi_\true^c}}
\def\resbeta{\algname\RESCALC{\beta}}
\def\resbetal{\algname\RESCALC{\beta_L}}
\def\resbetar{\algname\RESCALC{\beta_R}}
\def\resinit{\algname\RESCALC{init}}
\def\resbe{\algname\RESCALC{\beta\eta}}
\def\resbel{\algname\RESCALC{\beta\eta_L}}
\def\resber{\algname\RESCALC{\beta\eta_R}}
\def\resinitleib{\algname\RESCALC{Init^{\Leibeq}}}
\def\resxi{\algname\RESCALC{\xi}}
\def\respropf{\algname\RESCALC{\propf}}
\def\respropb{\algname\RESCALC{\propb}}
\def\respropfr{\algname\RESCALC{\propf_\true}}
\def\respropbr{\algname\RESCALC{\propb_\true}}
\def\respropfl{\algname\RESCALC{\propf_\false}}
\def\respropbl{\algname\RESCALC{\propb_\false}}
\def\resdec{\algname\RESCALC{d}}
\def\resdech{\algname\RESCALC{d^h}}
\def\resxidec{\algname\RESCALC{\xi dec}}
\def\resfdec{\algname\RESCALC{\propf dec}}
\def\resdecb{\algname\RESCALC{fd}}
\def\resdecc{\algname\RESCALC{\lambda d}}
\def\resk{\algname\RESCALC{k}}
\def\rescut{\algname\RESCALC{cut}}
\def\rescuta{\algname\RESCALC{cut^\bA}}
\def\resaxiomb{\algname\RESCALC{\cB}}
\def\resaxiomf{\algname\RESCALC\axiomf}



\def\cHstar{\cH_{*}}
\def\cHFstar{\cH_{\propf *}}
\def\cHXIstar{\cH_{\propxi *}}
\def\cHD{\cH_\PD}
\def\cHF{\cH_{\PF}}
\def\cHXI{\cH_{\PXI}}
\def\cHETA{\cH_{\PETA}}
\def\cHB{\cH_{\PB}}
\def\cHETAB{\cH_{\PETAB}}
\def\cHFB{\cH_{\PFB}}
\def\cHXIB{\cH_{\PXIB}}
\def\cHQB{\cH_{\PQB}}

%\def\hwcomp{{\parallel{\mbox{\hspace*{-.27cm}}}\scriptscriptstyle^w}\ }
\def\hwcomp{\between}
\def\hcomp{{\parallel}}
\def\hfcomp{{\parallel{\mbox{\hspace*{-.25cm}}}\scriptstyle^\propf}\ }
\def\hxicomp{{\parallel{\mbox{\hspace*{-.25cm}}}\scriptstyle_\propxi}\ }
\def\hfxicomp{{\parallel{\mbox{\hspace*{-.25cm}}}\scriptstyle^\propf{\mbox{\hspace*{-.12cm}}}\scriptstyle_\propxi}\ }
\def\hbcomp{{\parallel{\mbox{\hspace*{-.25cm}}}\scriptstyle^\propb}\ }
\def\phiofseq#1#2{\Phi_{#1\rightarrow #2}}
%\def\phiofseq#1{\Phi_{#1}} % if 1-sided

\def\perdom#1#2#3{\overline{#1^{#2}_{#3}}}
\def\perd#1#2#3{#1^{#2}_{#3}}
\def\repfns#1#2#3{\cR\cF^{#1}_{#2\ar #3}}

\def\possbools#1#2{{\bf\cB}_{#1}^{#2}}

\def\perpropconst{\partial^\Sigma}
\def\perpropv{\partial^\semival}
\def\perpropvsurj{\partial^{\semtrue\semfalse}}
%\def\perproptot{\partial^{\typebool,\typeind}}
\def\perproptota#1{\partial^{tot}_{#1}}
\def\perpropneg{\partial^\neg}
\def\perpropor{\partial^\lor}
\def\perpropall{\partial^\Pi}
\def\perpropq{\partial^\propq}
\def\perpropb{\partial^\propb}
\def\perpropf{\partial^\propf}
\def\perpropxi{\partial^\propxi}
\def\perpropeta{\partial^\propeta}
%\def\perpropcong{\partial^c}
\def\perpropcong{\partial^{\appo}}
\def\perpropsubst{\partial^{sub}}

\def\compatrefl{\sharp^r}
\def\compatcons{\sharp^c}
\def\compatfns{\sharp^\ar}
\def\compatbeta{\sharp^\beta}

\def\diseqn{\nunif}


\def\perrep#1{#1^{\star}}

\def\semt{{\sf t}}

\def\andrewsarbval{\diamond}
\def\tracers#1{{\mathcal T}(#1)}
\def\pvalretr{{\mathcal R}}
\def\pvfst#1{\lfloor #1\rfloor_{_{\mbox{\bf\footnotesize 1}}}}
\def\pvsnd#1{\lfloor #1\rfloor_{_{\mbox{\bf\footnotesize 2}}}}
%\def\pvfst#1{\lfloor #1\rfloor_{_1}}
%\def\pvsnd#1{\lfloor #1\rfloor_{_2}}

\newcommand{\bK}{{\mathbf{K}}}
\def\dersize#1{size(#1)}



\calculusName{HO Sequent Calculi $\SEQCALCD$ and $\SEQCALCFB$}   % The name of the calculus
\calculusAcronym{GBetaFB}    % The acronym if defined above, or empty otherwise. 
\calculusLogic{classical higher-order logic}  % Specify the logic (e.g. classical, intuitionistic, ...) for which this calculus is intended.
\calculusType{resolution}   % Specify the calculus type (e.g. Frege-Hilbert style, tableau, sequent calculus, hypersequent calculus, natural deduction, ...)
\calculusYear{2003-2009}   % The year when the calculus was invented.
\calculusAuthor{Christoph Benzm{\"u}ller \and Chad Brown \and Michael Kohlhase} % The name(s) of the author(s) of the calculus.


\entryTitle{HO Sequent Calculi $\SEQCALCD$ and $\SEQCALCFB$}     % Title of the entry (usually coincides with the name of the calculus).
\entryAuthor{Christoph Benzm{\"u}ller}    % Your name(s). Separate multiple names with "\and".


% If you wish, use tags to give any other information 
% that might be helpful for classifying and grouping this entry:
% e.g. \tag{Two-Sided Sequents}
% e.g. \tag{Multiset Cedents}
% e.g. \tag{List Cedents}
% You are free to invent your own tags. 
% The Encyclopedia's coordinator will take care of 
% merging semantically similar tags in the future.



\maketitle


% If your files are called "MyProofSystem.tex" and "MyProofSystem.bib", 
% then you should write "\begin{entry}{MyProofSystem}" in the line below
\begin{entry}{GBetaFB}  

% Define here any newcommands you may need:
% e.g. \newcommand{\necessarily}{\Box}
% e.g. \newcommand{\possibly}{\Diamond}


\begin{calculus}

% Add the inference rules of your proof system here.
% The "proof.sty" and "bussproofs.sty" packages are available.
% If you need any other package, please contact the editor (bruno@logic.at)
\textbf{Basic Rules} \hfill
 \ianc{\Delta,s}{\Delta,\neg\neg s}{\seqneg} \quad
  \ibnc{\Delta,\neg s}{\Delta,\neg t}{\Delta,\neg( s\lor t)}{\seqlorl} \quad
  \ianc{\Delta,s,t}
       {\Delta,( s\lor t)}
       {\seqlorr} \\[.5em]
  \phantom{b} \hfill
  \ibnc{\Delta,{\neg\bnormform{(s l)}}}
       {l_\typea \textrm{ closed term}}
       {\Delta,\neg\Pi^\typea s}
       {\seqpil} \quad 
  \ibnc{\Delta,\bnormform{(s c)}}
       {c_\typed \textrm{ new symbol}}
       {\Delta,\Pi^\typea s}
       {\seqpir} \\[.5em]
\textbf{Initialization} \hfill
  \ianc{s{\mbox{ atomic (and $\beta$-normal)}}}{\Delta,s,\neg
    s}{\seqinit} \hfill
  \ibnc{\Delta,( s\Leibeq^\typebool t)}{ s {\mbox{,}} t {\mbox{ atomic}}}
       {\Delta,\neg s,t}
       {\seqinitleib} \\[.5em]
\textbf{Extensionality} \hfill
  \ianc{\Delta,\bnormform{(\alldot {X_\typea} s X\Leibeq^{\typeb} t X)}}
       {\Delta,( s\Leibeq^{\typea\ar\typeb} t)}
       {\seqpropf} \quad
  \ibnc{\Delta,\neg s,t}
       {\Delta,\neg t,s}
       {\Delta,( s\Leibeq^\typebool t)}
       {\seqpropb} \\[.5em]
\textbf{Decomposition} \hfill
  \ibnc{\Delta,( s^1\Leibeq^{\typea_1} t^1)\;\cdots\;\Delta,( s^n\Leibeq^{\typea_n} t^n)}
       {\begin{array}{c} n\geq 1,\typeb\in\{\typebool,\typeind\}, h_{\ov{\typea^n}\ar\typeb}\in\Signat\end{array}}
       {\Delta, (h\ov{ s^n}\Leibeq^{\typeb} h\ov{ t^n})}
       {\seqdec}\\[1em]

One-sided sequent calculus $\SEQCALCD$ is defined by the rules
$\seqinit$, $\seqneg$, $\seqlorl$, $\seqlorr$, $\seqpil$ and $\seqpir$.

Calculus $\SEQCALCFB$ extends $\SEQCALCD$ by the additional
rules $\seqpropb$, $\seqpropf$, $\seqdec$,  and $\seqinitleib$.
\end{calculus}

% The following sections ("clarifications", "history", 
% "technicalities") are optional. If you use them, 
% be very concise and objective. Nevertheless, do write full sentences. 
% Try to have at most one paragraph per section, because line breaks 
% do not look nice in a short entry.

 \begin{clarifications}
   $\Delta$ and $\Delta'$ are finite sets of $\beta$-normal closed
   formulas of classical higher-order logic (HOL; Church's Type Theory)
   \cite{sep-type-theory-church}. $\Delta,s$ denotes the set
   $\Delta\cup\{s\}$. Let
   $\alpha,\beta,o \in {T}$. HOL \emph{terms} are defined by
   the grammar ($c_{\alpha}$ denotes typed constants and $X_{\alpha}$
   typed variables distinct from $c_\alpha$):
   $ s,t ::= c_{\alpha} \mid X_{\alpha} \mid (\lambda
   X_{\alpha}s_{\beta})_{\alpha \typearrow \beta} \mid (s_{\alpha
     \typearrow \beta}\, t_\alpha)_{\beta} \mid (\neg_{o \typearrow
     o}\;s_{o})_o \mid (s_o \vee_{o\typearrow o \typearrow o} t_o)_o
   \mid (\Pi_{(\alpha \typearrow o)\typearrow o}\; s_{\alpha
     \typearrow o})_o$.
   \emph{Leibniz equality} $\Leibeq^\alpha$ at type $\alpha$ is
   defined as
   $ s_\typea \Leibeq^\alpha t_\typea := \forall P_{\typea\typearrow
     o} (\neg P s \vee P t)$.
   For each simply typed $\lambda$-term $s$ there is a unique
   \emph{$\beta$-normal form} (denoted $\Bnormform{s}$).  HOL formulas
   are defined as terms of type $o$. A \emph{non-atomic formula} is
   any formula whose {$\beta$-normal form} is of the form
   $[c\ov{\bA^n}]$ where $c$ is a logical constant.  An \emph{atomic
     formula} is any other formula.

   Theorem proving in these calculi works as follows: In order to
   prove that a (closed) conjecture formula $c$ logically follows from
   a (possibly empty) set of (closed) axioms $\{a^1,\ldots,a^n\}$, we
   start from the initial sequent
   $\Delta:=\{c,\neg a^1,\ldots,\neg a^n\}$ and reason backwards by
   applying the respective calculus rules.
\end{clarifications}


 \begin{history}
The calculi have been presented in \cite{J18}.
Earlier (two-sided) versions and further related sequent calculi for
HOL have been  presented in \cite{R19} and \cite{BrownPhD}.

% ToDo: write here short historical remarks about this proof system,
% especially if they relate to other proof systems. 
% Use "\iref{OtherProofSystem}" to refer to another proof system 
% in the Encyclopedia (where "OtherProofSystem" is its ID). 
% Use "\irefmissing{SuggestedIDForOtherProofSystem}" to refer to 
% another proof system that is not yet available in the encyclopedia.
\end{history}

\begin{technicalities} 
  $\SEQCALCD$ is sound and complete for elementary type theory
  ($\SEQCALCD$ is thus also sound for HOL).  $\SEQCALCFB$ is sound and
  complete for HOL. Moreover, both calculi are cut-free and they do not
  admit cut-simulation \cite{J18}.
\end{technicalities}


% General Instructions:
% =====================

% The preferred length of an entry is 1 page. 
% Do the best you can to fit your proof system in one page.
%
% If you are finding it hard to fit what you want in one page, remember:
%
%   * Your entry needs to be neither self-contained nor fully understandable
%     (the interested reader may consult the cited full paper for details)
%
%   * If you are describing several proof systems in one entry, 
%     consider splitting your entry.
%
%   * You may reduce the size of your entry by ommitting inference rules
%     that are already described in other entries.
%
%   * Cite parsimoniously (see detailed citation instructions below).
%
% 
% If you do not manage to fit everything in one page, 
% it is acceptable for an entry to have 2 pages.
%
% For aesthetical reasons, it is preferable for an entry to have
% 1 full page or 2 full pages, in order to avoid unused blank space.



% Citation Instructions:
% ======================

% Please cite the original paper where the proof system was defined.
% To do so, you may use the \cite command within 
% one of the optional environments above,
% or use the \nocite command otherwise.

% You may also cite a modern paper or book where the 
% proof system is explained in greater depth or clarity.
% Cite parsimoniously.

% Do not cite related work. Instead, use the "\iref" or "\irefmissing" 
% commands to make an internal reference to another entry, 
% as explained within the "history" environment above.

% You do not need to create the "References" section yourself. 
% This is done automatically.




% Leave an empty line above "\end{entry}".

\end{entry}