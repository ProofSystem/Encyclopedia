


\calculusName{Classical Sequent Calculus}   % The name of the calculus
\calculusAcronym{\LK}    % The acronym if defined above, or empty otherwise. 
\calculusLogic{Classical Predicate Logic}
\calculusType{Sequent Calculus}
\calculusYear{1935}   % The year when the calculus was invented.
\calculusAuthor{Gerhard Gentzen} % The name(s) of the author(s) of the calculus.

\entryTitle{Classical Sequent Calculus \LK}
\entryAuthor{Martin Riener}    % Your name(s). Separate multiple names with "\and"

\maketitle


\begin{entry}{GentzenLK}  

% new commands %
\newcommand{\lkproves}{\ensuremath{\vdash}}
\renewcommand{\fCenter}{\lkproves}

\newcommand{\twocolumns}[2]{
\begin{minipage}[b]{0.5\linewidth}
\centering
#1
\end{minipage}
\hspace{0.5cm}
\begin{minipage}[b]{0.5\linewidth}
\centering
#2
\end{minipage}
}

\newcommand{\threecolumns}[3]{
\begin{minipage}[b]{0.31\linewidth}
\centering
#1
\end{minipage}
\hspace{0.01\linewidth}
\begin{minipage}[b]{0.31\linewidth}
\centering
#2
\end{minipage}
\hspace{0.01\linewidth}
\begin{minipage}[b]{0.31\linewidth}
\centering
#3
\end{minipage}\\
}

\newcommand{\fourcolumns}[4]{
\begin{minipage}[b]{0.23\linewidth}
\centering
#1
\end{minipage}
\hspace{0.01\linewidth}
\begin{minipage}[b]{0.23\linewidth}
\centering
#2
\end{minipage}
\hspace{0.01\linewidth}
\begin{minipage}[b]{0.23\linewidth}
\centering
#3
\end{minipage}
\hspace{0.01\linewidth}
\begin{minipage}[b]{0.23\linewidth}
\centering
#4
\end{minipage}
}



\newcommand{\LKAX}[2]{\AxiomC{\ensuremath{#1} \fCenter \ensuremath{#2}}}
\newcommand{\LKUI}[2]{\UnaryInfC{\ensuremath{#1} \fCenter \ensuremath{#2}}}
\newcommand{\LKBI}[2]{\BinaryInfC{\ensuremath{#1} \fCenter \ensuremath{#2}}}
\newcommand{\LKLL}[1]{\LeftLabel{\footnotesize \ensuremath{#1}}}
\newcommand{\LKRL}[1]{\RightLabel{\footnotesize \ensuremath{#1}}}
\newcommand{\LKRLN}[1]{\RightLabel{#1}}

\newcommand{\SALLL  }{\LKRL{\forall_l}}
\newcommand{\SALLR  }{\LKRL{\forall_r\,(*)}}
\newcommand{\SEXL   }{\LKRL{\exists_l\,(*)}}
\newcommand{\SEXR   }{\LKRL{\exists_r}}
\newcommand{\SANDL  }{\LKRL{\land_l}}
\newcommand{\SANDR  }{\LKRL{\land_r}}
\newcommand{\SORL   }{\LKRL{\lor_l}}
\newcommand{\SORR   }{\LKRL{\lor_r}}
\newcommand{\SIMPL  }{\LKRL{\imp_l}}
\newcommand{\SIMPR  }{\LKRL{\imp_r}}
\newcommand{\SNEGL  }{\LKRL{\neg_l}}
\newcommand{\SNEGR  }{\LKRL{\neg_r}}
\newcommand{\SWEAKL }{\LKRL{w_l}}
\newcommand{\SWEAKR }{\LKRL{w_r}}
\newcommand{\SCONTRL}{\LKRL{c_l}}
\newcommand{\SCONTRR}{\LKRL{c_r}}
\newcommand{\SEXCHL }{\LKRL{e_l}}
\newcommand{\SEXCHR }{\LKRL{e_r}}
\newcommand{\SCUT   }{\LKRL{cut}}
\newcommand{\SDEF   }{\LKRL{def}}

\newcommand{\ALLL}     [3]{\SALLL \LKUI{#2}{#3} }
\newcommand{\ALLR}     [3]{\SALLR \LKUI{#2}{#3} }
\newcommand{\EXL}      [3]{\SEXL  \LKUI{#2}{#3} }
\newcommand{\EXR}      [3]{\SEXR  \LKUI{#2}{#3} }
\newcommand{\ANDL}     [2]{\SANDL \LKUI{#1}{#2} }
\newcommand{\ANDR}     [2]{\SANDR \LKBI{#1}{#2} }
\newcommand{\ORL}      [2]{\SORL  \LKBI{#1}{#2} }
\newcommand{\ORR}      [2]{\SORR  \LKUI{#1}{#2} }
\newcommand{\IMPL}     [2]{\SIMPL \LKBI{#1}{#2}}
\newcommand{\IMPR}     [2]{\SIMPR \LKUI{#1}{#2}}
\newcommand{\NEGL}     [2]{\SNEGL \LKUI{#1}{#2}}
\newcommand{\NEGR}     [2]{\SNEGR \LKUI{#1}{#2}}
\newcommand{\EQL}      [2]{\SEQL  \LKBI{#1}{#2}}
\newcommand{\EQR}      [2]{\SEQR  \LKBI{#1}{#2}}
\newcommand{\WEAKL}    [2]{\SWEAKL \LKUI{#1}{#2}}
\newcommand{\WEAKR}    [2]{\SWEAKR \LKUI{#1}{#2}}
\newcommand{\EXCHL}    [2]{\SEXCHL \LKUI{#1}{#2}}
\newcommand{\EXCHR}    [2]{\SEXCHR \LKUI{#1}{#2}}
\newcommand{\CONTRL}   [2]{\SCONTRL \LKUI{#1}{#2}}
\newcommand{\CONTRR}   [2]{\SCONTRR \LKUI{#1}{#2}}
\newcommand{\CUT}      [2]{\SCUT    \LKBI{#1}{#2}}


\begin{calculus}

\[
\begin{array}{cc}
\infer{A \vdash A}{}
&
\infer[cut]{\Gamma, \Delta \vdash \Lambda, \Theta}{\Gamma \vdash \Lambda, A & A, \Delta \vdash \Theta}
\\[8pt]
\infer[w_l]{A, \Gamma \vdash \Theta}{\Gamma \vdash \Theta}
&
\infer[w_r]{\Gamma \vdash \Theta, A}{\Gamma \vdash \Theta}
\\[8pt]
\infer[e_l]{\Gamma, A, B, \Delta \vdash \Theta}{\Gamma, B, A, \Delta \vdash \Theta}
\quad%&
\infer[c_l]{A, \Gamma \vdash \Theta}{A, A, \Gamma \vdash \Theta}
\quad
&%\\[8pt]
\quad
\infer[e_r]{\Gamma \vdash \Theta, A, B, \Delta}{\Gamma \vdash \Theta, B, A, \Delta}
\quad%&
\infer[c_r]{\Gamma \vdash \Theta, A}{\Gamma \vdash \Theta, A, A}
\\[8pt]
\infer[\neg_l]{\neg A, \Gamma \vdash \Theta}{\Gamma \vdash \Theta, A}
&
\infer[\neg_r]{\Gamma \vdash \Theta, \neg A}{A, \Gamma \vdash \Theta}
\\[8pt]
\infer[\wedge_{l}]{A_1 \wedge A_2, \Gamma \vdash \Theta}{A_i, \Gamma \vdash \Theta}
&
\infer[\wedge_r]{\Gamma \vdash \Theta, A \wedge B}{\Gamma \vdash \Theta, A & \Gamma \vdash \Theta,  B}
\\[8pt]
\infer[\vee_l]{A \vee B, \Gamma \vdash \Theta}{A, \Gamma \vdash \Theta & B, \Gamma \vdash \Theta}
&
\infer[\vee_{r}]{\Gamma \vdash \Theta, A_1 \vee A_2}{\Gamma \vdash \Theta, A_i}
\\[8pt]
\infer[\rightarrow_l]{A \rightarrow B, \Gamma, \Delta \vdash \Lambda, \Theta}{\Gamma \vdash \Lambda, A & B, \Delta \vdash \Theta}
&
\infer[\rightarrow_r]{\Gamma \vdash \Theta, A \rightarrow B}{A, \Gamma \vdash \Theta, B}
\\[8pt]
\infer[\exists_l]{\exists x.A[x], \Gamma \vdash \Theta}{A[\alpha], \Gamma \vdash \Theta}
\quad
\infer[\forall_l]{\forall x.A[x], \Gamma \vdash \Theta}{A[t], \Gamma \vdash \Theta}
\ \
&
\ \
\infer[\forall_r]{\Gamma \vdash \Theta, \forall x.A[x]}{\Gamma \vdash \Theta, A[\alpha]}
\quad
\infer[\exists_r]{\Gamma \vdash \Theta, \exists x.A[x]}{\Gamma \vdash \Theta, A[t]}
\\
\end{array}
\]

\centering
The eigenvariable $\alpha$ should not occur in $\Gamma$, $\Theta$ or $A[x]$. \\ 
The term $t$ should not contain variables bound in $A[t]$.
\end{calculus}


\begin{history}
This is a modern presentation of Gentzen's original \LK calculus\cite{lk:Gentzen1935}, using modern notations and rule names. Originally, terms could only be object variables (either free or bound); there were no constant and function symbols.
\end{history}

\newcommand{\LHK}{\ensuremath{\mathbf{LHK}}\xspace}
\newcommand{\NK}{\ensuremath{\mathbf{NK}}\xspace}


\begin{technicalities}
\LK is complete relative to \NK (i.e. \NJ \iref{GentzenNJ} with the axiom of excluded middle) and sound relative to a Hilbert-style calculus \LHK \cite{lk:Gentzen1935a}. Cut is eliminable (\emph{Hauptsatz} \cite{lk:Gentzen1935}), and hence classical predicate logic is consistent. Any \emph{prenex} cut-free proof may be further transformed into a shape with only propositional inferences above and only quantifier and structural inferences below a \emph{midsequent} \cite{lk:Gentzen1935a}.
\end{technicalities}

\end{entry}
% undefine for new commands
\let\LKAX\undefined
\let\LKUI\undefined
\let\LKBI\undefined
\let\LKLL\undefined
\let\LKRL\undefined
\let\LKRLN\undefined

\let\SALLL  \undefined
\let\SALLR  \undefined
\let\SEXL   \undefined
\let\SEXR   \undefined
\let\SANDL  \undefined
\let\SANDR  \undefined
\let\SORL   \undefined
\let\SORR   \undefined
\let\SIMPL  \undefined
\let\SIMPR  \undefined
\let\SNEGL  \undefined
\let\SNEGR  \undefined
\let\SWEAKL \undefined
\let\SWEAKR \undefined
\let\SCONTRL\undefined
\let\SCONTRR\undefined
\let\SEXCHL \undefined
\let\SEXCHR \undefined
\let\SCUT   \undefined
\let\SDEF   \undefined

\let\ALLL  \undefined
\let\ALLR  \undefined
\let\EXL   \undefined
\let\EXR   \undefined
\let\ANDL  \undefined
\let\ANDR  \undefined
\let\ORL   \undefined
\let\ORR   \undefined
\let\IMPL  \undefined
\let\IMPR  \undefined
\let\NEGL  \undefined
\let\NEGR  \undefined
\let\WEAKL \undefined
\let\WEAKR \undefined
\let\CONTRL\undefined
\let\CONTRR\undefined
\let\EXCHL \undefined
\let\EXCHR \undefined
\let\CUT   \undefined
\let\DEF   \undefined