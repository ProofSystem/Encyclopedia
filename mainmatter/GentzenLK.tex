
\newcommand{\ND}{\ensuremath{\mathbf{ND}}\xspace}
\newcommand{\LK}{\ensuremath{\mathbf{LK}}\xspace}
\newcommand{\LJ}{\ensuremath{\mathbf{LJ}}\xspace}

\calculusName{Classical Sequent Calculus}   % The name of the calculus
\calculusAcronym{\LK}    % The acronym if defined above, or empty otherwise. 
\calculusLogic{Classical Predicate Logic}
\calculusType{Sequent Calculus}
\calculusYear{1935}   % The year when the calculus was invented.
\calculusAuthor{Gerhard Gentzen} % The name(s) of the author(s) of the calculus.

\entryTitle{Classical Sequent Calculus \LK}
\entryAuthor{Martin Riener}    % Your name(s). Separate multiple names with "\and"

\maketitle


\begin{entry}{GentzenLK}  

% new commands %
\newcommand{\lkproves}{\ensuremath{\vdash}}
\renewcommand{\fCenter}{\lkproves}

\newcommand{\twocolumns}[2]{
\begin{minipage}[b]{0.5\linewidth}
\centering
#1
\end{minipage}
\hspace{0.5cm}
\begin{minipage}[b]{0.5\linewidth}
\centering
#2
\end{minipage}
}

\newcommand{\threecolumns}[3]{
\begin{minipage}[b]{0.31\linewidth}
\centering
#1
\end{minipage}
\hspace{0.01\linewidth}
\begin{minipage}[b]{0.31\linewidth}
\centering
#2
\end{minipage}
\hspace{0.01\linewidth}
\begin{minipage}[b]{0.31\linewidth}
\centering
#3
\end{minipage}\\
}

\newcommand{\fourcolumns}[4]{
\begin{minipage}[b]{0.23\linewidth}
\centering
#1
\end{minipage}
\hspace{0.01\linewidth}
\begin{minipage}[b]{0.23\linewidth}
\centering
#2
\end{minipage}
\hspace{0.01\linewidth}
\begin{minipage}[b]{0.23\linewidth}
\centering
#3
\end{minipage}
\hspace{0.01\linewidth}
\begin{minipage}[b]{0.23\linewidth}
\centering
#4
\end{minipage}
}



\newcommand{\LKAX}[2]{\AxiomC{\ensuremath{#1} \fCenter \ensuremath{#2}}}
\newcommand{\LKUI}[2]{\UnaryInfC{\ensuremath{#1} \fCenter \ensuremath{#2}}}
\newcommand{\LKBI}[2]{\BinaryInfC{\ensuremath{#1} \fCenter \ensuremath{#2}}}
\newcommand{\LKLL}[1]{\LeftLabel{\footnotesize \ensuremath{#1}}}
\newcommand{\LKRL}[1]{\RightLabel{\footnotesize \ensuremath{#1}}}
\newcommand{\LKRLN}[1]{\RightLabel{#1}}

\newcommand{\SALLL  }{\LKRL{\forall:l}}
\newcommand{\SALLR  }{\LKRL{\forall:r\,(*)}}
\newcommand{\SEXL   }{\LKRL{\exists:l\,(*)}}
\newcommand{\SEXR   }{\LKRL{\exists:r}}
\newcommand{\SANDL  }{\LKRL{\land:l}}
\newcommand{\SANDR  }{\LKRL{\land:r}}
\newcommand{\SORL   }{\LKRL{\lor:l}}
\newcommand{\SORR   }{\LKRL{\lor:r}}
\newcommand{\SIMPL  }{\LKRL{\imp:l}}
\newcommand{\SIMPR  }{\LKRL{\imp:r}}
\newcommand{\SNEGL  }{\LKRL{\neg:l}}
\newcommand{\SNEGR  }{\LKRL{\neg:r}}
\newcommand{\SWEAKL }{\LKRL{w:l}}
\newcommand{\SWEAKR }{\LKRL{w:r}}
\newcommand{\SCONTRL}{\LKRL{c:l}}
\newcommand{\SCONTRR}{\LKRL{c:r}}
\newcommand{\SEXCHL }{\LKRL{e:l}}
\newcommand{\SEXCHR }{\LKRL{e:r}}
\newcommand{\SCUT   }{\LKRL{cut}}
\newcommand{\SDEF   }{\LKRL{def}}

\newcommand{\ALLL}     [3]{\SALLL \LKUI{#2}{#3} }
\newcommand{\ALLR}     [3]{\SALLR \LKUI{#2}{#3} }
\newcommand{\EXL}      [3]{\SEXL  \LKUI{#2}{#3} }
\newcommand{\EXR}      [3]{\SEXR  \LKUI{#2}{#3} }
\newcommand{\ANDL}     [2]{\SANDL \LKUI{#1}{#2} }
\newcommand{\ANDR}     [2]{\SANDR \LKBI{#1}{#2} }
\newcommand{\ORL}      [2]{\SORL  \LKBI{#1}{#2} }
\newcommand{\ORR}      [2]{\SORR  \LKUI{#1}{#2} }
\newcommand{\IMPL}     [2]{\SIMPL \LKBI{#1}{#2}}
\newcommand{\IMPR}     [2]{\SIMPR \LKUI{#1}{#2}}
\newcommand{\NEGL}     [2]{\SNEGL \LKUI{#1}{#2}}
\newcommand{\NEGR}     [2]{\SNEGR \LKUI{#1}{#2}}
\newcommand{\EQL}      [2]{\SEQL  \LKBI{#1}{#2}}
\newcommand{\EQR}      [2]{\SEQR  \LKBI{#1}{#2}}
\newcommand{\WEAKL}    [2]{\SWEAKL \LKUI{#1}{#2}}
\newcommand{\WEAKR}    [2]{\SWEAKR \LKUI{#1}{#2}}
\newcommand{\EXCHL}    [2]{\SEXCHL \LKUI{#1}{#2}}
\newcommand{\EXCHR}    [2]{\SEXCHR \LKUI{#1}{#2}}
\newcommand{\CONTRL}   [2]{\SCONTRL \LKUI{#1}{#2}}
\newcommand{\CONTRR}   [2]{\SCONTRR \LKUI{#1}{#2}}
\newcommand{\CUT}      [2]{\SCUT    \LKBI{#1}{#2}}


\begin{calculus}

% Add the inference rules of your proof system here.
% The "proof.sty" and "bussproofs.sty" packages are available.
% If you need any other package, please contact the editor (bruno@logic.at)
%\textbf{Structural rules:}\\
\twocolumns{
  \AxiomC{}
  \LKRL{Axiom}
  \LKUI{D}{D}
  \DisplayProof
}{
  \LKAX{\Gamma}{\Theta,D}
  \LKAX{\Gamma,D}{\Theta}
  \CUT{\Gamma}{\Theta}
  \DisplayProof
}
\vspace{1mm}
\threecolumns{
  \LKAX{\Gamma}{\Theta}
  \WEAKL{D, \Gamma}{\Theta}
  \DisplayProof
}{
  \LKAX{\Gamma}{\Theta}
  \WEAKR{\Gamma}{\Theta, D}
  \DisplayProof
}{
  \LKAX{D,D,\Gamma}{\Theta}
  \CONTRL{D, \Gamma}{\Theta}
  \DisplayProof
}
\vspace{1mm}
\threecolumns{
  \LKAX{\Gamma,D,E,\Delta}{\Theta}
  \EXCHL{\Gamma, E, D, \Delta}{\Theta}
  \DisplayProof
}{
  \LKAX{\Gamma}{\Theta,D,E, \Lambda}
  \EXCHR{\Gamma}{\Theta, E, D, \Lambda}
  \DisplayProof
 
}{
  \LKAX{\Gamma}{\Theta,D,D}
  \CONTRR{\Gamma}{\Theta, D}
  \DisplayProof
}
\vspace{1mm}
%\textbf{Logical rules:}\\
\threecolumns{
\LKAX{A,\Gamma}{\Theta}
\ANDL{A \land B,\Gamma}{\Theta}
\DisplayProof
}{
\LKAX{B,\Gamma}{\Theta}
\ANDL{A \land B,\Gamma}{\Theta}
\DisplayProof
}{
\LKAX{\Gamma}{\Theta,A}
\ORR{\Gamma}{\Theta,A\lor B}
\DisplayProof
}
\vspace{1mm}
\threecolumns{
\LKAX{\Gamma}{\Theta,A}
\NEGL{\neg A,\Gamma}{\Theta}
\DisplayProof
}{
\LKAX{A,\Gamma}{\Theta}
\NEGR{\Gamma}{\Theta, \neg A}
\DisplayProof
}{
\LKAX{\Gamma}{\Theta,B}
\ORR{\Gamma}{\Theta,A\lor B}
\DisplayProof
}
\vspace{1mm}
\twocolumns{
\LKAX{A,  \Gamma}{\Theta}
\LKAX{B,  \Gamma}{\Theta}
\ORL{A \lor B,\Gamma}{\Theta}
\DisplayProof
}{
\LKAX{\Gamma}{\Theta,A}
\LKAX{\Gamma}{\Theta,B}
\ANDR{\Gamma}{\Theta,A \land B}
\DisplayProof
}
\vspace{1mm}
\twocolumns{
\LKAX{\Gamma}{\Theta, A}
\LKAX{B,  \Delta}{\Lambda}
\IMPL{A \imp B,\Gamma}{\Theta}
\DisplayProof
}{
\LKAX{A, \Gamma}{\Theta, B}
\IMPR{\Gamma}{\Theta,A \imp B}
\DisplayProof
}
\vspace{1mm}
\twocolumns{
  \LKAX{F\,a, \Gamma}{\Theta}
  \ALLL{}{\forall x\,F x, \Gamma}{\Theta}
  \DisplayProof
}{
  \LKAX{\Gamma}{\Theta,F\,a}
  \EXR{}{\Gamma}{\Theta,\exists x\,F x}
  \DisplayProof
}
\vspace{1mm}
\twocolumns{
  \LKAX{F\,a, \Gamma}{\Theta}
  \EXL{}{\exists x\,F x, \Gamma}{\Theta}
  \DisplayProof
}{
  \LKAX{\Gamma}{\Theta,F\,a}
  \ALLR{}{\Gamma}{\Theta,\forall x\,F x}
  \DisplayProof
}

$(*)$: Eigenvariable condition: $x$ does not occur in $\Gamma$, $\Delta$ and $\forall x\, F x$/$\exists x\, F x$.

%} % end centering
\end{calculus}

% The following environments ("clarifications", "history", 
% "technicalities") are optional. If you do use them, 
% be very concise and objective.

 \begin{clarifications}
   In all rules, $A,B,D,E,F$ are formulas, $\Gamma,\Theta,\Delta,\Lambda$ are lists of formulas, $a$ is a free and $x$ a bound variable. Within the quantifier rules, $F a$ is obtained from $F x$ by applying the substitution $\{ a/x \} $.

 \end{clarifications}

 \begin{history}
This is Gentzen's original formulation\cite{lk:Gentzen1935} where the term language has only free and bound variables. Takeuti's version\cite{lk:Takeuti1975} adds individual and function constants. An invertible variant of \LK is $\LK'$\irefmissing{LKprime}, an intuitionistic version is \LJ\iref{SequentCalculusLJ}. 
\end{history}

\begin{technicalities}
% ToDo: write here remarks about soundness, completeness, decidability...
Soundness is obtained by a translation to \ND\irefmissing{ND}, completeness by cut-elimination i.e. Gentzen's Hauptsatz\cite{lk:Gentzen1935}.
\end{technicalities}

\end{entry}
% undefine for new commands
\let\LKAX\undefined
\let\LKUI\undefined
\let\LKBI\undefined
\let\LKLL\undefined
\let\LKRL\undefined
\let\LKRLN\undefined

\let\SALLL  \undefined
\let\SALLR  \undefined
\let\SEXL   \undefined
\let\SEXR   \undefined
\let\SANDL  \undefined
\let\SANDR  \undefined
\let\SORL   \undefined
\let\SORR   \undefined
\let\SIMPL  \undefined
\let\SIMPR  \undefined
\let\SNEGL  \undefined
\let\SNEGR  \undefined
\let\SWEAKL \undefined
\let\SWEAKR \undefined
\let\SCONTRL\undefined
\let\SCONTRR\undefined
\let\SEXCHL \undefined
\let\SEXCHR \undefined
\let\SCUT   \undefined
\let\SDEF   \undefined

\let\ALLL  \undefined
\let\ALLR  \undefined
\let\EXL   \undefined
\let\EXR   \undefined
\let\ANDL  \undefined
\let\ANDR  \undefined
\let\ORL   \undefined
\let\ORR   \undefined
\let\IMPL  \undefined
\let\IMPR  \undefined
\let\NEGL  \undefined
\let\NEGR  \undefined
\let\WEAKL \undefined
\let\WEAKR \undefined
\let\CONTRL\undefined
\let\CONTRR\undefined
\let\EXCHL \undefined
\let\EXCHR \undefined
\let\CUT   \undefined
\let\DEF   \undefined