

\calculusName{Intuitionistic Sequent Calculus}   % The name of the calculus
\calculusAcronym{\LJ}    % The acronym if defined above, or empty otherwise. 
\calculusLogic{Intuitionistic Predicate Logic}
\calculusType{Sequent Calculus}
\calculusYear{1935}   % The year when the calculus was invented.
\calculusAuthor{Gerhard Karl Erich Gentzen} % The name(s) of the author(s) of the calculus.

\entryTitle{Intuitionistic Sequent Calculus \LJ}
\entryAuthor{Giselle Reis}    % Your name(s). Separate multiple names with "\and"

\maketitle


% If your files are called "<ID>.tex" and "<ID>.bib", 
% then you should write "\begin{entry}{<ID>}" in the line below
\begin{entry}{GentzenLJ}  

% Define here any newcommands you may need:
% e.g. \newcommand{\necessarily}{\Box}
% e.g. \newcommand{\possibly}{\Diamond}

\begin{calculus}

% Add the inference rules of your proof system here.
% The "proof.sty" and "bussproofs.sty" packages are available.
% If you need any other package, please contact the editor (bruno@logic.at)
\[
\begin{array}{cc}
\infer{A \vdash A}{}
&
\infer[cut]{\Gamma, \Delta \vdash \Theta}{\Gamma \vdash A & A, \Delta \vdash \Theta}
\\[8pt]
\infer[\neg_l]{\neg A, \Gamma \vdash }{\Gamma \vdash A}
&
\infer[\neg_r]{\Gamma \vdash \neg A}{A, \Gamma \vdash }
\\[8pt]
\infer[\wedge_{l}]{A_1 \wedge A_2, \Gamma \vdash \Theta}{A_i, \Gamma \vdash \Theta}
&
\infer[\wedge_r]{\Gamma \vdash A \wedge B}{\Gamma \vdash A & \Gamma \vdash B}
\\[8pt]
\infer[\vee_l]{A \vee B, \Gamma \vdash \Theta}{A, \Gamma \vdash \Theta & B, \Gamma \vdash \Theta}
&
\infer[\vee_{r}]{\Gamma \vdash A_1 \vee A_2}{\Gamma \vdash A_i}
\\[8pt]
\infer[\rightarrow_l]{A \rightarrow B, \Gamma, \Delta \vdash \Theta}{\Gamma \vdash A & B, \Delta \vdash \Theta}
&
\infer[\rightarrow_r]{\Gamma \vdash A \rightarrow B}{A, \Gamma \vdash B}
\\[8pt]
\infer[\exists_l]{\exists x.A[x], \Gamma \vdash \Theta}{A[\alpha], \Gamma \vdash \Theta}
&
\infer[\exists_r]{\Gamma \vdash \exists x.A[x]}{\Gamma \vdash A[t]}
\\[8pt]
\infer[\forall_l]{\forall x.A[x], \Gamma \vdash \Theta}{A[t], \Gamma \vdash \Theta}
&
\infer[\forall_r]{\Gamma \vdash \forall x.A[x]}{\Gamma \vdash A[\alpha]}
\\[8pt]
\infer[e_l]{\Gamma, A, B, \Delta \vdash \Theta}{\Gamma, B, A, \Delta \vdash \Theta}
&
\infer[c_l]{A, \Gamma \vdash \Theta}{A, A, \Gamma \vdash \Theta}
\\[8pt]
\infer[w_l]{A, \Gamma \vdash \Theta}{\Gamma \vdash \Theta}
&
\infer[w_r]{\Gamma \vdash A}{\Gamma \vdash}
\\
\end{array}
\]
\end{calculus}

% The following environments ("clarifications", "history", 
% "technicalities") are optional. If you do use them, 
% be very concise and objective.

\begin{clarifications}
% ToDo: write here short remarks that may help the reader to understand 
% the inference rules of the proof system.
In all rules, $A$, $A_i$ and $B$ are arbitrary formulas and $\Theta$ is a set
with at most one formula. In rules $\exists_l$
and $\forall_r$, $\alpha$ is a variable not contained in $A$, $\Gamma$ or
$\Theta$. In rules $\exists_r$ and $\forall_l$, $t$ does not contain variables
bound in $A$.
It is common to consider \textbf{LJ} without the exchange rule $e_l$ just by
interpreting $\Gamma$ and $\Theta$ as multi-sets of formulas instead of lists.
Also, the conjunction $\&$ is usually denoted by $\wedge$.
\end{clarifications}

\begin{history}
Proposed by Gentzen in \cite{Gentzen1935} by restricting the
succedent of sequents in \irefmissing{GentzenLK} to have at most one
formula. In the original paper, he notes that this restriction is equivalent to
removing the principle of excluded middle from the classical 
natural deduction system \irefmissing{GentzenNK} 
in order to obtain \NJ \iref{GentzenNJ}.
\end{history}

\begin{technicalities}
Soundness and completeness of \LJ can be proved using a translation of \LJ
derivations into \NJ \iref{GentzenNJ}.
Decidability of the propositional fragment and consistency of intuitionistic
logic follows from cut admissibility in this calculus (\emph{Hauptsatz}).
\end{technicalities}


\end{entry}
