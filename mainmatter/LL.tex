
% If the calculus has an acronym, define it.
% (e.g. \newcommand{\LK}{\ensuremath{\mathbf{LK}}\xspace})

\calculusName{Linear Logic Sequent Calculus}   % The name of the calculus
\calculusAcronym{LL}    % The acronym if defined above, or empty otherwise. 
\calculusLogic{LL}  % Specify the logic (e.g. classical, intuitionistic, ...) for which this calculus is intended.
\calculusType{sequent calculus}   % Specify the calculus type (e.g. Frege-Hilbert style, tableau, sequent calculus, hypersequent calculus, natural deduction, ...)
\calculusYear{1987}   % The year when the calculus was invented.
\calculusAuthor{Jean-Yves Girard} % The name(s) of the author(s) of the calculus.


\entryTitle{Linear Logic Sequent Calculus}     % Title of the entry (usually coincides with the name of the calculus).
\entryAuthor{Joseph Boudou \and Sergei Soloviev}    % Your name(s). Separate multiple names with "\and".


% If you wish, use tags to give any other information 
% that might be helpful for classifying and grouping this entry:
% e.g. \tag{Two-Sided Sequents}
% e.g. \tag{Multiset Cedents}
% e.g. \tag{List Cedents}
% You are free to invent your own tags. 
% The Encyclopedia's coordinator will take care of 
% merging semantically similar tags in the future.


\maketitle


% If your files are called "MyProofSystem.tex" and "MyProofSystem.bib", 
% then you should write "\begin{entry}{MyProofSystem}" in the line below
\begin{entry}{LL} 

% Define here any newcommands you may need:
% e.g. \newcommand{\necessarily}{\Box}
% e.g. \newcommand{\possibly}{\Diamond}
\newcommand{\llaconj}{\binampersand}
\newcommand{\lladisj}{\oplus}
\newcommand{\llimp}{\multimap}
\newcommand{\llmconj}{\otimes}
\newcommand{\llmdisj}{\bindnasrepma}
\newcommand{\llneg}[1]{{#1}^\bot}
\newcommand{\llzero}{0}
\newcommand{\llone}{1}

\newcommand{\sepproof}{\hskip 2em plus 6em\relax}
\newcommand{\sepseq}{\quad\vdash}
\newcommand{\sepline}{\]\[}

\newenvironment{infruleset}[1]{%
  \sc{#1} \[ %
}{%
  \] \vspace{-1em} %
}
\newcommand{\axiom}[2][axiom]{\infer[\mbox{\textit{(#1)}}]{\ensuremath{\vdash #2 }}{}}
\newcommand{\infrule}[3]{\infer[\mbox{\textit{(#1)}}]{\ensuremath{\vdash #3 }}{\ensuremath{\vdash #2 }}}

\begin{calculus}

\begin{infruleset}{Structural}
  \axiom[identity]{A, \llneg{A}}
  \sepproof
  \infrule{cut}{\Gamma, A \sepseq \llneg{A}, \Delta}{\Gamma, \Delta}
  \sepline
  \infrule{permutation}{\Gamma, A, B, \Delta}{\Gamma, B, A, \Delta}
\end{infruleset}

\begin{infruleset}{Logical}
  \axiom[one]{\llone}
  \sepproof
  \axiom[true]{\Gamma, \top}
  \sepproof
  \infrule{false}{\Gamma}{\Gamma, \bot}
  \sepline
  \infrule{times}{\Gamma, A \sepseq B, \Delta}{\Gamma, A \llmconj B, \Delta}
  \sepproof
  \infrule{par}{\Gamma, A, B}{\Gamma, A \llmdisj B}
  \sepline
  \infrule{with}{\Gamma, A \sepseq \Gamma, B}{\Gamma, A \llaconj B}
  \sepproof
  \infrule{left plus}{\Gamma, A}{\Gamma, A \lladisj B}
  \sepproof
  \infrule{right plus}{\Gamma, B}{\Gamma, A \lladisj B}
  \sepline
  \infrule{weakening}{\Gamma}{\Gamma, ?A}
  \sepproof
  \infrule{of course}{?\Gamma, A}{?\Gamma, !A}
  \sepline
  \infrule{contraction}{\Gamma, ?A, ?A}{\Gamma, ?A}
  \sepproof
  \infrule{dereliction}{\Gamma, A}{\Gamma, ?A}
\end{infruleset}

% Add the inference rules of your proof system here.
% The "proof.sty" and "bussproofs.sty" packages are available.
% If you need any other package, please contact the editor (bruno@logic.at)


\end{calculus}

% The following sections ("clarifications", "history", 
% "technicalities") are optional. If you use them, 
% be very concise and objective. Nevertheless, do write full sentences. 
% Try to have at most one paragraph per section, because line breaks 
% do not look nice in a short entry.

\begin{clarifications}
  Sequents are ordered lists of formulas.
  If $\Gamma$ is a list $A_1, \ldots, A_n$ of formulas, $?\Gamma$ denotes the list $?A_1, \ldots, ?A_n$.
  Formulas are contructed from propositional variables $p, q, \ldots$
  and their dual $\llneg{p}, \llneg{q}, \ldots$, using the constructs
  $\llzero, \top, \bot, \llone, \mathord{\llmconj}, \mathord{\llmdisj},
  \mathord{\llaconj}, \mathord{\lladisj}, \mathord{!}$ and $\mathord{?}$.
  Negation is \emph{defined} by:
  \[
  \begin{array}{rlrl}
    \llneg{(p)} &= \llneg{p} ~&~ \llneg{(\llneg{p})} &= p \\
    \llneg{\llzero} &= \top ~&~ \llneg{\top} &= \llzero \\
    \llneg{\bot} &= \llone ~&~ \llneg{\llone} &= \bot \\
    \llneg{(A \llmconj B)} &= \llneg{A} \llmdisj \llneg{B} ~&~
    \llneg{(A \llmdisj B)} &= \llneg{A} \llmconj \llneg{B} \\
    \llneg{(A \llaconj B)} &= \llneg{A} \lladisj \llneg{B} ~&~
    \llneg{(A \lladisj B)} &= \llneg{A} \llaconj \llneg{B} \\
    \llneg{(!A)} &= ?\llneg{A} ~&~ \llneg{(?A)} &= !\llneg{A} \\
  \end{array}
  \]
  Similarly, the linear implication is defined by $A \llimp B = \llneg{A} \llmdisj B$.
  The multiplicatives $\mathord{\llmconj}, \mathord{\llmdisj}, \mathord{\llimp}$,
  additives $\mathord{\llaconj}, \mathord{\lladisj}$
  and exponentials $\mathord{!}, \mathord{?}$ of linear logic
  are inspired by linear and bilinear aspects in the behaviour of standard connectives,
  that may be seen, in particular, in the treatment of contexts and structural rules 
  in intuitionistic, classical and modal systems.
\end{clarifications}

\begin{history}
  Introduced by Girard in~\cite{girard1987tcs}.
\end{history}


% \begin{technicalities}
% ToDo: write here remarks about soundness, completeness, decidability...
% \end{technicalities}


% General Instructions:
% =====================

% The preferred length of an entry is 1 page. 
% Do the best you can to fit your proof system in one page.
%
% If you are finding it hard to fit what you want in one page, remember:
%
%   * Your entry needs to be neither self-contained nor fully understandable
%     (the interested reader may consult the cited full paper for details)
%
%   * If you are describing several proof systems in one entry, 
%     consider splitting your entry.
%
%   * You may reduce the size of your entry by ommitting inference rules
%     that are already described in other entries.
%
%   * Cite parsimoniously (see detailed citation instructions below).
%
% 
% If you do not manage to fit everything in one page, 
% it is acceptable for an entry to have 2 pages.
%
% For aesthetical reasons, it is preferable for an entry to have
% 1 full page or 2 full pages, in order to avoid unused blank space.



% Citation Instructions:
% ======================

% Please cite the original paper where the proof system was defined.
% To do so, you may use the \cite command within 
% one of the optional environments above,
% or use the \nocite command otherwise.

% You may also cite a modern paper or book where the 
% proof system is explained in greater depth or clarity.
% Cite parsimoniously.

% Do not cite related work. Instead, use the "\iref" or "\irefmissing" 
% commands to make an internal reference to another entry, 
% as explained within the "history" environment above.

% You do not need to create the "References" section yourself. 
% This is done automatically.




% Leave an empty line above "\end{entry}".

\end{entry}
